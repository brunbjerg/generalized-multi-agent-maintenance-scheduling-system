\section{Goals of the External Stay}
The most significant goals of the external stay are:

\begin{itemize}
	\item Assess the feasibility of implementing my Ph.D. project in an industry setting.
	% \item Gauge whether my Ph.D. project can be implemented in practice. 
	\item Integrate my application into a test environment at Decision Brain, if applicable.
	\item Get expert feedback on my scheduling methodology. 
	\item Learn best practices for implementing scheduling solutions in industry.
\end{itemize}

It is my belief that I how developed a scalable approach to modelling a generic 
maintenance scheduling system (see section~\ref{sec:technical}). I am 
modelling something that is similar to what is described in \citet{palmerMaintenancePlanningScheduling2019} 
which is a source that has a more practical orientation than most academic works. 

My code operates on backend SAP tables and user inputs so I believe that there may be a possibility of integrating my 
code into a system at Decision Brain if this is deemed valuable. Integrating the system at Decision 
Brain would allow us to evaluate its potential financial value and determine whether pursuing a
full implementation could be mutually beneficial.

\newpage
\section{Possible Setup of the External Stay}
I propose to have a dedicated contact at Decision Brain during my external stay, with who I can discuss ideas and seek help from.

\begin{itemize}
	% TODO: This is not that great should be changed
	\item First month: Determine if I can integrate my application in a relevant project at Decision Brain.
	\item Second month: Work on implementing the scheduling system in Decision Brain with weekly or biweekly feedback.
	\item Third month: Assess the feasibility of the project and possible course corrections.
\end{itemize}

The main incentive for Decision Brain to partake in this project would be to get a new perspective on how to develop scheduling solutions,
a hiring opportunity, getting a partial implementation of scheduling code if that is deemed relevant.

