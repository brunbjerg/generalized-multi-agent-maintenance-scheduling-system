\newpage
\section*{Appendix}
This appendix contains the pseudo-code and mathematical models associated with each metaheuristic implementation. 

The mathematical models here are only guiding, as everything is implemented as metaheuristics a one-to-one mapping
between the mathematical models and the actual implementation can lose some nuances.

\subsection*{Pseudo Code}

All implemented algorithms are based on a variant of the Large Neighborhood Search metaheuristic with some modification to enable both message
passing and solution state shared through atomic pointer swapping. 

\begin{figure}[H]
	\newcommand{\MessageQueue}{Q}
\newcommand{\Solution}{X}
\newcommand{\ProblemInstance}{P}
\newcommand{\SharedSolution}{S}

\begin{algorithm}[H]
\caption{Actor-based Large Neighborhood Search} 
\begin{algorithmic}[1]
\State \textbf{Input} $\MessageQueue$    = message queue
\State \textbf{Input} $\ProblemInstance$ = problem instance
\State \textbf{Input} $\Solution$        = initial schedule
\State \textbf{Input} $\SharedSolution$  = shared solution
% \State $\Solution^b = \Solution$
\Repeat \label{alg:actor-based-large-neighborhood-search:main-optimization-loop}
	\State $\Solution^t = clone(\Solution)$ \label{alg:actor-based-large-neighborhood-search:clone-solution}
	\While{$\MessageQueue.has\_message()$} \label{alg:actor-based-large-neighborhood-search:message-loop}
        % \State $m = queue.pop()$
        % \State $m.destruct(\Solution^b)$
		\State $\ProblemInstance.update(\SharedSolution, m)$
        \State $\Solution^t.destruct(\SharedSolution, m)$
    \EndWhile
	
    \State $\Solution^t.repair(\SharedSolution)$ \label{alg:actor-based-large-neighborhood-search:repair}

    \If{accept($\Solution^t, \Solution$)} \label{alg:actor-based-large-neighborhood-search:accept}                       
        \State $\Solution.update(\Solution^t)$
    \EndIf                                         

    \If{$c(\Solution^t) < c(\Solution)$} \label{alg:actor-based-large-neighborhood-search:better-solution}                             
        \State $\Solution.update(\Solution^t)$
		\State $\SharedSolution.atomic\_pointer\_swap(\Solution)$
    \EndIf                                           
	\State $\MessageQueue.push(m)$ \label{alg:actor-based-large-neighborhood-search:schedule-message}
\Until
\end{algorithmic}
\label{alg:actor-based-large-neighborhood-search}
\end{algorithm}


\end{figure}

I have decided to include the mathematical models of the four different metaheuristics. It is not meant to provide a through understanding
but show how the larger process can be modelled through a series of smaller models. \textbf{Notice:} the "meta variables" sections are
the decision variables (and time $\tau$) from other models that are being used in the respective model as a parameters (meaning they are 
not variables in that specific model and that model cannot then change the value of the variable).

\newpage
\subsection*{Strategic Model: A Knapsack Variant}

\newpage
\begin{alignat}{2}
	& \text{\rule{\linewidth}{0.4pt}} \notag\\
	& \textbf{Meta variables:} \notag\\
	& \ElementScheduler \in \SetScheduler \\
	& \VarTacticalWork{}{} \\ 
	& \tau \in [0, \infty] \\
	& \text{\rule{\linewidth}{0.4pt}} \notag\\
	& \textbf{Minimize:} \notag                                                                                                                                                        \\
	& \sum_{\ElementWorkOrder \in \SetWorkOrder{}} \sum_{\ElementPeriod \in \SetPeriod} \ParStrategicValue \cdot \VarStrategicWorkOrderAssignment{\ElementWorkOrder}{\ElementPeriod}  \notag\\ 
	& + \sum_{\ElementPeriod \in \SetPeriod} \sum_{\ElementResource \in \SetResource} \ParStrategicPenalty \cdot \VarStrategicExcess     \notag                                              \\
	& + \sum_{\ElementPeriod \in \SetPeriod} \sum_{\ElementWorkOrder1 \in \SetWorkOrder{}} \sum_{\ElementWorkOrder2 \in \SetWorkOrder{}} 	 \quad \ParClusteringValue \cdot \VarStrategicWorkOrderAssignment{\ElementWorkOrder1}{\ElementPeriod} \cdot \VarStrategicWorkOrderAssignment{\ElementWorkOrder2}{\ElementPeriod}  \\
	& \text{\rule{\linewidth}{0.4pt}} \notag\\
	& \textbf{Subject to:} \notag                                                                                                                                                      \\
	& \sum_{\ElementWorkOrder \in \SetWorkOrder{}} \ParStrategicWorkOrderWeight \cdot \VarStrategicWorkOrderAssignment{\ElementWorkOrder}{\ElementPeriod} \leq \ \ParStrategicResource + \VarStrategicExcess                                                                           \quad \forall \ElementPeriod \in \SetPeriod \quad \forall \ElementResource \in \SetResource                                                                                      \\
	& \sum_{\ElementWorkOrder \in \SetWorkOrder{}} \VarStrategicWorkOrderAssignment{\ElementWorkOrder}{\ElementPeriod} = 1              \quad \forall \ElementPeriod \in \SetPeriod                                                                                                                                      \\
	& \VarStrategicWorkOrderAssignment{\ElementWorkOrder}{\ElementPeriod} = 0                                                            \quad \forall (\ElementWorkOrder, \ElementPeriod) \in \ParStrategicExclude                                                                                                       \\
	& \VarStrategicWorkOrderAssignment{\ElementWorkOrder}{\ElementPeriod} = 1                                                            \quad \forall (\ElementWorkOrder, \ElementPeriod) \in \ParStrategicInclude                                                                                                       \\
	& \VarStrategicWorkOrderAssignment{\ElementWorkOrder}{\ElementPeriod} \in \{0, 1\}                                                   \quad \forall \ElementWorkOrder \in \SetWorkOrder{} \quad \forall \ElementPeriod \in \SetPeriod                                                                                 \\ 
	& \VarStrategicExcess \in \mathbb{R}^{+}                                                                                             \quad \forall \ElementPeriod \in \SetPeriod \quad \forall \ElementResource \in \SetResource                                                                                  \\ 
	& \text{\rule{\linewidth}{0.4pt}} \notag
\end{alignat}
\newpage


\newpage
\subsection*{Tactical Model: A Resource Constrained Project Scheduling Problem Variant}
\section{The Tactical Model}
\begin{itemize}
	\item Respect precedence constraints
	\item Respect daily resource requirements for each trait
	\item Penalize exceeded daily capacity
\end{itemize}

After the strategic model has optimized its schedule the tactical agent will continue scheduling the output at a more detailed level. This means that now the tactical agent will schedule 
out on each of the days of the work orders scheduled by the strategic agent. 

The tactical model is responsible for providing an initial suggestion for a weekly schedule, below we see the model for the tactical agent.
\begin{alignat}{2}
	& \textbf{Meta variables:} \notag\\
	& \ElementScheduler = \SetScheduler \notag\\
	& \tau \in [0, \infty] \notag\\
	& \VarStrategicWorkOrderAssignment{}{} \notag\\
	& \textbf{Minimize:} \notag\\
	& \sum_{\ElementOperation \in \SetOperation{}{\VarStrategicWorkOrderAssignment{}{}}} \sum_{\ElementDays \in \SetDays{}} \ParTacticalValue!!!!!!!!!!!!! \cdot \VarTacticalWork{\ElementDays}{\ElementOperation}\notag\\  
	& + \sum_{r \in \SetResource} \sum_{\ElementDays \in \SetDays{}} \ParTacticalPenalty \cdot \VarTacticalExcess                                               \\  
	& \textbf{Subject to:}                                                          \notag                                                                   \\
	& \sum_{\ElementOperation \in \SetOperation{}{\VarStrategicWorkOrderAssignment{}{}}} \ParOperationWork{\ElementOperation} \cdot \VarTacticalWork{\ElementDays}{\ElementOperation}\notag\\
	& \quad \leq \ParTacticalResource + \VarTacticalExcess\notag\\ 
	& \quad \forall \ElementDays \in \SetDays{} \quad \forall r \in \SetResource\\ 
	& \sum_{\ElementDays = \ParEarliestStart}^{\ParLatestFinish} \VarTacticalWorkBinary{\ElementDays}{\ElementOperation} = \ParDuration \notag\\
	& \quad \forall \ElementOperation \in \SetOperation{}{\VarStrategicWorkOrderAssignment{}{}} \\
	& \sum_{\ElementDays^* \in  \SetDays{\ParDuration}} \VarTacticalWorkBinary{\ElementDays^*}{\ElementOperation} \notag\\
	& \quad = \ParDuration \cdot \VarTacticalWorkBinaryConsecutive \notag\\ 
	& \quad \forall \ElementOperation \in \SetOperation{}{\VarStrategicWorkOrderAssignment{}{}} \quad \forall \ElementDays \in \SetDays{} \\
	& \sum_{\ElementOperation \in \SetOperation{}{\VarStrategicWorkOrderAssignment{}{}}} \VarTacticalWorkBinaryConsecutive = 1, \notag\\
	& \quad \forall \ElementDays \in \SetDays{} \notag\\
	& \sum_{\ElementDays \in \SetDays{}} \ElementDays \cdot \VarTacticalWorkBinary{\ElementDays}{\ElementOperation1} + \VarTacticalOperationDifference  = \sum_{\ElementDays \in \SetDays{}} \ElementDays \cdot \VarTacticalWorkBinary{\ElementDays}{\ElementOperation2}                   \notag  \\ 
	& \quad \forall (o1, \ElementOperation2) \in \ParFinishStart                                                           \\ 
	& \sum_{\ElementDays \in \SetDays{}} \ElementDays \cdot \VarTacticalWorkBinary{\ElementDays}{\ElementOperation1} = \sum_{\ElementDays \in \SetDays{}} \ElementDays \cdot \VarTacticalWorkBinary{\ElementDays}{\ElementOperation2}  \notag                               \\ 
	& \quad \forall (o1, \ElementOperation2) \in \ParStartStart                                                       \\ 
	& \VarTacticalWork{\ElementDays}{\ElementOperation} \leq \ParNumberOfPeople \cdot \ParOperatingTime \notag                                                     \\ 
	& \quad \forall \ElementDays \in \SetDays{} \quad \forall \ElementOperation \in \SetOperation{}{\VarStrategicWorkOrderAssignment{}{}}                                                                  \\
	& \VarTacticalWork{\ElementDays}{\ElementOperation} \in \mathbb{R} \quad \notag\\
	& \quad \forall \ElementDays \in \SetDays{} \quad \forall \ElementOperation \in \SetOperation{}{\VarStrategicWorkOrderAssignment{}{}}                                         \\
	& \VarTacticalExcess \in \mathbb{R} \quad\notag\\
	& \quad \forall r \in \SetResource \quad \forall \ElementDays \in \SetDays{}                                        \\
	& \VarTacticalWorkBinary{\ElementDays}{\ElementOperation} \in \{0, 1\}\quad \notag\\
	& \quad \forall \ElementDays \in \SetDays{} \quad \forall \ElementOperation \in \SetOperation{}{\VarStrategicWorkOrderAssignment{}{}} \\
	& \VarTacticalWorkBinaryConsecutive \in \{0, 1\}\quad \notag\\
	& \quad \forall \ElementDays \in \SetDays{} \quad \forall \ElementOperation \in \SetOperation{}{\VarStrategicWorkOrderAssignment{}{}} \\
	& \VarTacticalOperationDifference \in \{0, 1\} \\
	& \quad \forall \ElementOperation \in \SetOperation{}{\VarStrategicWorkOrderAssignment{}{}}                                      \\
	& \VarMetaTime \in  [0, \infty] 
\end{alignat}


\newpage
\subsection*{Supervisor Model: An Assignment Problem Variant}
\section{The Supervisor Model}
The maintenance supervisor is considered the most central person in a maintenance scheduling system. 
All the work of the planner and scheduler should be considered a service for the supervisor.

The supervisor has multiple different responsibilities among them are: 

\begin{itemize}
	\item Assigning work orders
	\item Creating a daily schedule
	\item Keeping the schedule up-to-date
\end{itemize}


\begin{alignat}{2}
	& \textbf{Meta variables:} \notag\\
	& \tau \in [0, \infty] \notag\\
	& \ElementSupervisor \in \SetSupervisor \notag\\
	& \VarStrategicWorkOrderAssignment{}{} \notag\\
	& \VarIncludeActivity{} \notag\\
	& \textbf{Maximize:} \notag\\
	& \sum_{\ElementActivity \in \SetActivity{\VarStrategicWorkOrderAssignment{}{}}{}} \sum_{\ElementTechnician \in \SetTechnician} \ParSupervisorValue \cdot \VarSupervisorAssignment{\ElementActivity}{\ElementTechnician} \\ 
	& \textbf{Subject to:} \notag\\ 
	& \sum_{\ElementActivity \in \SetActivity{\VarStrategicWorkOrderAssignment{}{}}{\ElementOperation}} \VarActivityWork{\ElementActivity} = \ParOperationWork{\ElementOperation}   \notag\\
	& \quad \forall \ElementOperation \in \SetOperation{}{\VarStrategicWorkOrderAssignment{}{}}\\
	& \sum_{\ElementTechnician \in \SetTechnician} \sum_{\ElementActivity \in \SetActivity{\VarStrategicWorkOrderAssignment{}{}}{\ElementOperation}}\VarSupervisorAssignment{\ElementActivity}{\ElementTechnician} = \VarSupervisorAssignmentWhole \cdot \ParNumberOfPeople \notag\\
	& \quad \forall \ElementOperation \in \SetOperation{}{\VarStrategicWorkOrderAssignment{}{}}  \\
	& \sum_{\ElementOperation \in \SetOperation{\ElementWorkOrder}{\VarStrategicWorkOrderAssignment{}{}}} \VarSupervisorAssignmentWhole = !!!! |\SetOperation{\ElementWorkOrder}{\VarStrategicWorkOrderAssignment{}{}}| \notag\\ 
	& \quad \forall \ElementWorkOrder \in \SetWorkOrder{,\VarStrategicWorkOrderAssignment{}{}} \\
	& \sum_{\ElementActivity \in \SetActivity{\VarStrategicWorkOrderAssignment{}{}}{\ElementOperation}} \VarSupervisorAssignment{\ElementActivity}{\ElementTechnician} \leq 1 \notag\\
	& \quad \forall \ElementOperation \in \SetOperation{}{\VarStrategicWorkOrderAssignment{}{}} \quad \forall \ElementTechnician \in \SetTechnician \\  
	& \VarSupervisorAssignment{\ElementActivity}{\ElementTechnician} \leq \ParFeasible \notag\\
	& \quad \forall \ElementOperation \in \SetOperation{}{\VarStrategicWorkOrderAssignment{}{}} \quad \forall \ElementTechnician \in \SetTechnician \\
	& \VarSupervisorAssignment{\ElementActivity}{\ElementTechnician} \in \{0, 1\} \notag\\
	& \quad \forall \ElementOperation \in \SetOperation{}{\VarStrategicWorkOrderAssignment{}{}} \quad \forall \ElementTechnician \in \SetTechnician \\ 
	& \VarActivityWork{\ElementActivity} \in [\ParLowerActivityWork, \ParOperationWork{\ElementActivity}] \notag\\
	& \quad \forall \ElementActivity \in \SetActivity{\VarStrategicWorkOrderAssignment{}{}}{} \\
    & \VarMetaTime \in  [0, \infty] 
\end{alignat}

In the supervisor model shown in \ref{} the set $O$ and $W$ comes from the tactical algorithm
and value $v$ and the information of whether or not the operation can be assigned to a 
specific operational model comes from the operational model itself and is captured in the.

Can this be done? What should the Supervisor have here? He should have what is necessary to
handle the. 

\newpage
\subsection*{Technician Model: Single Machine Scheduling Problem Variant}
\begin{alignat}{2}
	& \text{\rule{\linewidth}{0.4pt}} \notag\\
	& \textbf{Meta variables:}                                                                                                                                                                         \notag\\
	& \ElementTechnician \in \SetTechnician                                                                                                                                                            \\
	& \VarStrategicWorkOrderAssignment{}{}                                                                                                                                                             \\
	& \VarSupervisorAssignment{}{}                                                                                                                                                                     \\
	& \tau \in [0, \infty]                                                                                                                                                                             \\
	& \text{\rule{\linewidth}{0.4pt}} \notag\\
	& \textbf{Maximize:}                                                                                                                                                                               \notag\\
	& \sum_{\ElementActivity \in \SetActivity{\VarSupervisorAssignment{}{\ElementTechnician}}{}} \sum_{\ElementWorkSegment \in \SetWorkSegment} \VarProcessingTime                                           \\
	& \text{\rule{\linewidth}{0.4pt}} \notag\\
	& \textbf{Subject to:}                                                                                                                                                                             \notag\\
    & \sum_{\ElementWorkSegment \in \SetWorkSegment} \VarProcessingTime \cdot \VarActiveSegment{\ElementActivity}{\ElementWorkSegment} = \ParActivityWork{} \cdot \VarIncludeActivity{\ElementActivity}                                                                                                                                            \quad \forall \ElementActivity \in \SetActivity{\VarSupervisorAssignment{}{\ElementTechnician}}{}                                                                                                      \\
	& \VarStartOfSegment{\ElementActivity2}{1} \geq \VarFinishOfSegment{\ElementActivity1}{last(\ElementActivity1)} + \ParPreparation                                                                    \quad \forall \ElementActivity1 \in \SetActivity{\VarSupervisorAssignment{}{\ElementTechnician}}{} \quad \forall \ElementActivity2 \in \SetActivity{\VarSupervisorAssignment{}{\ElementTechnician}}{}  \\
	& \VarStartOfSegment{\ElementActivity}{\ElementWorkSegment} \geq \VarFinishOfSegment{\ElementActivity}{\ElementWorkSegment-1} - \ParConstraintLimit \cdot (2 - \VarActiveSegment{\ElementActivity}{\ElementWorkSegment} + \VarActiveSegment{\ElementActivity}{\ElementWorkSegment-1})                                                                      \notag\\
	& \quad \forall \ElementActivity \in \SetActivity{\VarSupervisorAssignment{}{\ElementTechnician}}{} \quad\forall \ElementWorkSegment \in \SetWorkSegment                                                 \\ 
	& \VarProcessingTime = \VarFinishOfSegment{\ElementActivity}{\ElementWorkSegment} - \VarStartOfSegment{\ElementActivity}{\ElementWorkSegment}                                                                                                                                              \quad \forall \ElementActivity \in \SetActivity{\VarSupervisorAssignment{}{\ElementTechnician}}{}  \quad\forall \ElementWorkSegment \in \SetWorkSegment                                                \\
	& \VarStartOfSegment{\ElementActivity}{\ElementWorkSegment} \geq \ParEvent + \ParEventDuration - \ParConstraintLimit \cdot (1 - \VarSegmentInRelation) \notag\\ 
	& \quad \forall \ElementActivity \in \SetActivity{\VarSupervisorAssignment{}{\ElementTechnician}}{}  \quad\forall \ElementWorkSegment \in \SetWorkSegment                                           \quad \forall i \in \SetTimeInstance  \quad\forall \ElementEvent \in \SetEvent                                                                                                                         \\
	& \VarFinishOfSegment{\ElementActivity}{\ElementWorkSegment} \leq \ParEvent + \ParConstraintLimit \cdot \VarSegmentInRelation \notag\\
	& \quad \forall \ElementActivity \in \SetActivity{\VarSupervisorAssignment{}{\ElementTechnician}}{}  \quad\forall \ElementWorkSegment \in \SetWorkSegment                                         \quad \forall i \in \SetTimeInstance  \quad\forall \ElementEvent \in \SetEvent                                                                                                                         \\
	& \VarStartOfSegment{\ElementActivity}{1} \geq \ParTimeWindowStart  \quad \forall \ElementActivity \in \SetActivity{\VarSupervisorAssignment{}{\ElementTechnician}}{}                                                                                                      \\
	& \VarFinishOfSegment{\ElementActivity}{last(\ElementActivity)} \leq \ParTimeWindowFinish  \quad \forall \ElementActivity \in \SetActivity{\VarSupervisorAssignment{}{\ElementTechnician}}{}                                                                                                      \\
	& \VarActiveSegment{\ElementActivity}{\ElementWorkSegment} \in \{0, 1\}  \quad \forall \ElementActivity \in \SetActivity{\VarSupervisorAssignment{}{\ElementTechnician}}{} \quad \forall \ElementWorkSegment \in \SetWorkSegment                                                \\
	& \VarStartOfSegment{\ElementActivity}{\ElementWorkSegment} \in [\ParAvailabilityStart, \ParAvailabilityFinish]                                                                                                           \notag\\
	& \quad \forall \ElementActivity \in \SetActivity{\VarSupervisorAssignment{}{\ElementTechnician}}{} \quad \forall \ElementWorkSegment \in \SetWorkSegment                                                \\
	& \VarFinishOfSegment{\ElementActivity}{\ElementWorkSegment} \in [\ParAvailabilityStart, \ParAvailabilityFinish]                                                                                                             \notag\\
	& \quad \forall \ElementActivity \in \SetActivity{\VarSupervisorAssignment{}{\ElementTechnician}}{} \quad \forall \ElementWorkSegment \in \SetWorkSegment                                                \\
	& \VarProcessingTime \in [0, \ParOperationWork{\ElementActivity\_to\_o(\ElementActivity)} ]  \quad \forall \ElementActivity \in \SetActivity{\VarSupervisorAssignment{}{\ElementTechnician}}{} \quad \forall \ElementWorkSegment \in \SetWorkSegment                                                \\
	& \VarSegmentInRelation \in \{0 , 1\}                                                                                                                                                                                                                                                                                     \quad \forall \ElementActivity \in \SetActivity{\VarSupervisorAssignment{}{\ElementTechnician}}{} \quad \forall \ElementWorkSegment \in \SetWorkSegment  \quad \forall i \in \SetTimeInstance \quad \forall \ElementEvent \in \SetEvent                                                                                                                         \\
	& \VarIncludeActivity{\ElementActivity} \in \{0, 1\}                                                                                                                                      \quad \forall \ElementActivity \in \SetActivity{\VarSupervisorAssignment{}{\ElementTechnician}}{}                                                                                                     \\ 
	& \text{\rule{\linewidth}{0.4pt}} \notag
\end{alignat}

