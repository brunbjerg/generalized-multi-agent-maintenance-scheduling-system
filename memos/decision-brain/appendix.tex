\newpage
\section*{Appendix}
This appendix contains the pseudo-code and mathematical models associated with each metaheuristic implementation. 

The mathematical models here are only guiding, as everything is implemented as metaheuristics a one-to-one mapping
between the mathematical models and the actual implementation can lose some nuances.

\subsection*{Pseudo Code}

All implemented algorithms are based on a variant of the Large Neighborhood Search metaheuristic with some modification to enable both message
passing and solution state shared through atomic pointer swapping. 

\begin{figure}[H]
	\newcommand{\MessageQueue}{Q}
\newcommand{\Solution}{X}
\newcommand{\ProblemInstance}{P}
\newcommand{\SharedSolution}{S}

\begin{algorithm}[H]
\caption{Actor-based Large Neighborhood Search} 
\begin{algorithmic}[1]
\State \textbf{Input} $\MessageQueue$    = message queue
\State \textbf{Input} $\ProblemInstance$ = problem instance
\State \textbf{Input} $\Solution$        = initial schedule
\State \textbf{Input} $\SharedSolution$  = SharedSolution
% \State $\Solution^b = \Solution$
\Repeat
	\State $\Solution^t = clone(\Solution)$
	\While{$\MessageQueue.has\_message()$}
        % \State $m = queue.pop()$
        % \State $m.destruct(\Solution^b)$
		\State $\ProblemInstance.update(\SharedSolution, m)$
        \State $\Solution^t.destruct(\SharedSolution, m)$
    \EndWhile
	
    \State $\Solution^t.repair(\SharedSolution)$

    \If{accept($\Solution^t, \Solution$)}                       
        \State $\Solution.update(\Solution^t)$
    \EndIf                                         

    \If{$c(\Solution^t) < c(\Solution)$}                             
        \State $\Solution.update(\Solution^t)$
		\State $\SharedSolution.atomic\_pointer\_swap(\Solution)$
    \EndIf                                           
	\State $\MessageQueue.push(m)$
\Until
\end{algorithmic}
\end{algorithm}


\end{figure}

I have decided to include the mathematical models of the four different models. It is not meant to provide a through understanding
but show how the larger process can be modelled through a series of smaller models. \textbf{Notice:} the "meta variables" sections are
the decision variables (and time $\tau$) from other models that are being used in the respective model as a parameters (meaning they are 
not variables in that specific model and that model cannot then change the value of the variable).

\newpage
\subsection*{Strategic Model: A Knapsack Variant}
\section{The Strategic Model}


The Strategic Model have multiple different purposes.
\begin{itemize}
	\item Schedule Work Order out across the weekly periods
	\item Prioritize all the different released work orders
	\item Respect the available weekly hours available for each trait
\end{itemize}

The Strategic model is responsible for grouping work orders into weekly or biweekly periods depending on which kind of maintenance setup that one is running.
This kind of model closely resembles a variant of the multi-compartment multi-knapsack problem. 

\begin{alignat}{2}
	\text{Min} &\sum_{w \in W} \sum_{p \in P} v_{wp}(t) \cdot x_{wp}(t)                                                                                             \\ 
	+ & \sum_{p \in P} \sum_{\tau = 1}^T d \cdot pen_{p\tau}(t)                                                                                                     \\
	+ & \sum_{p \in P} \sum_{w1 \in W} \sum_{w2 \in W} clu_{w1, w2} \cdot x_{w1p} \cdot x_{w2p}                              \label{eqn:objective_function_strategic} \\
    & \text{subject to:} \notag                                                                                                                                       \\
	& \sum_{w \in W} c_{w\tau} \cdot x_{wp}(t) \leq \ cap_{p\tau}(t) + pen_{p\tau}(t) \notag                                                                          \\ 
	& \forall p \in P, \forall \tau \in T                                                                                    \label{eqn:capacity_constraint}          \\
	& \sum_{w \in W} x_{wp}(t) = 1                                              , \quad \forall p \in P                      \label{eqn:single_workorder_constraint}  \\
	& x_{wp}(t) = 0                                                             , \quad \forall (w, p) \in E(t)              \label{eqn:exclusion_constraint}         \\
	& x_{wp}(t) = 1                                                             , \quad \forall (w, p) \in I(t)              \label{eqn:inclusion_constraint}         \\
	& x_{wp}(t) \in \{0, 1\}                                                    , \quad \forall w \in W, \forall p \in P     \label{eqn:x_integrality_constraint}     \\ 
	& pen_{p\tau}(t) \in \mathbb{R}^{+}                                         , \quad \forall p \in P, \forall \tau \in T  \label{eqn:p_non_negativity_constraint}
    & t \in  [0, \infty] 
\end{alignat}

\newpage
\subsection*{Tactical Model: A Resource Constrained Project Scheduling Problem Variant}
\section{The Tactical Model}
\begin{itemize}
	\item Respect precedence constraints
	\item Respect daily resource requirements for each trait
	\item Penalize exceeded daily capacity
\end{itemize}

After the strategic model has optimized its schedule the tactical agent will continue scheduling the output at a more detailed level. This means that now the tactical agent will schedule 
out on each of the days of the work orders scheduled by the strategic agent. 

The tactical model is responsible for providing an initial suggestion for a weekly schedule, below we see the model for the tactical agent.
\begin{alignat}{2}
\text{Min}     & \sum_{o \in O} \sum_{d \in D} v_{do}(t) \cdot y_{do}(t)                                                      \\  
	         + & \sum_{c \in C} \sum_{d \in D} pen \cdot p_{cd}(t)                                               \\  
			                                                                                                  \\
               &\text{subject to:}                                                          \notag                                                                   \\
	           & \sum_{o \in O} w_{co} \cdot y_{do}(t)  \leq R_{dc} + p_{dc}(t)                                   \\ 
			   & \quad \quad \forall  d \in D, \forall c \in C                              \notag                                    \\ 
	           & \sum_{d \in D} d \cdot y_{do1}(t) + \delta_o  = \sum_{d \in D} d \cdot y_{do2}(t)                    \\ 
			   & \quad \quad \forall (o1, o2) \in \text{finish-start}(t)                        \notag                                   \\ 
	           & \sum_{d \in D} d \cdot y_{do1}(t) = \sum_{d \in D} d \cdot y_{do2}(t)                                \\ 
			   & \quad \quad \forall (o1, o2) \in \text{start-start}(t)                          \notag                             \\ 
			   & y_{do}(t) \leq number_o(t) * operating\_time_o                                                     \\ 
			   & \quad \quad ,\forall d \in D, \forall o \in O                              \notag                                    \\
			   & y_{do}(t) \in \{0, 1\} \quad ,\forall d \in D, o \in O                                          \\
			   & p_{cd}(t) \in \mathbb{R} \quad ,\forall c \in C, d \in D                                        \\
			   & \delta_o \in \{ duration\_lower_o(t),                                                           \\ 
			   & \quad \quad duration\_upper_o \}(t) \quad, \forall o \in O \notag                                      \\
			   & t \in  [0, \infty] 
\end{alignat}


\newpage
\subsection*{Supervisor Model: An Assignment Problem Variant}
\newif\ifincludenormal\

\pgfkeys{
	/supervisormodel/.is family, /supervisormodel,
	default/.style = {
		normal=true,
	},
	normal/.is if=includenormal,
}
\newcommand{\supervisorModel}[1][]{
	\pgfkeys{/supervisormodel, default, #1}
	\begin{alignat}{2}
		& \text{\rule{\linewidth}{0.8pt}} \notag \label{}                                                                                                                                                                                                                                                                                                                                                                     \\ 
		& \textbf{Meta variables:} \notag\\
		& \ElementSupervisor \in \SetSupervisor \\
		& \VarStrategicWorkOrderAssignment{}{} \\
		& \VarIncludeActivity{} \\
		& \tau \in [0, \infty] \\
		& \text{\rule{\linewidth}{0.4pt}} \notag\\
		& \textbf{Maximize:} \notag\\
		& \sum_{\ElementActivity \in \SetActivity{\VarStrategicWorkOrderAssignment{}{}}{}} \sum_{\ElementTechnician \in \SetTechnician} \ParSupervisorValue \cdot \VarSupervisorAssignment{\ElementActivity}{\ElementTechnician} \\ 
		& \text{\rule{\linewidth}{0.4pt}} \notag\\
		& \textbf{Subject to:} \notag\\ 
		& \sum_{\ElementActivity \in \SetActivity{\VarStrategicWorkOrderAssignment{}{}}{\ElementOperation}} \VarActivityWork{\ElementActivity} = \ParOperationWork{\ElementOperation}    \quad \forall \ElementOperation \in \SetOperation{}{\VarStrategicWorkOrderAssignment{}{}}\\
		& \sum_{\ElementTechnician \in \SetTechnician} \sum_{\ElementActivity \in \SetActivity{\VarStrategicWorkOrderAssignment{}{}}{\ElementOperation}}\VarSupervisorAssignment{\ElementActivity}{\ElementTechnician} = \VarSupervisorOperationWhole \cdot \ParNumberOfPeople  \quad \forall \ElementOperation \in \SetOperation{}{\VarStrategicWorkOrderAssignment{}{}}  \\
		& \sum_{\ElementOperation \in \SetOperation{\ElementWorkOrder}{\VarStrategicWorkOrderAssignment{}{}}} \VarSupervisorOperationWhole = |\SetOperation{\ElementWorkOrder}{\VarStrategicWorkOrderAssignment{}{}}| \cdot \VarSupervisorWorkOrderWhole  \quad \forall \ElementWorkOrder \in \SetWorkOrder{,\VarStrategicWorkOrderAssignment{}{}} \\
		& \sum_{\ElementActivity \in \SetActivity{\VarStrategicWorkOrderAssignment{}{}}{\ElementOperation}} \VarSupervisorAssignment{\ElementActivity}{\ElementTechnician} \leq 1  \quad \forall \ElementOperation \in \SetOperation{}{\VarStrategicWorkOrderAssignment{}{}} \quad \forall \ElementTechnician \in \SetTechnician \\  
		& \VarSupervisorAssignment{\ElementActivity}{\ElementTechnician} \leq \ParFeasible  \quad \forall \ElementActivity \in \SetActivity{\VarTacticalWork}{\ElementOperation} \quad \forall \ElementOperation \in \SetOperation{}{\VarStrategicWorkOrderAssignment{}{}} \quad \forall \ElementTechnician \in \SetTechnician \\
		& \VarSupervisorAssignment{\ElementActivity}{\ElementTechnician} \in \{0, 1\}  \quad \forall \ElementOperation \in \SetOperation{}{\VarStrategicWorkOrderAssignment{}{}} \quad \forall \ElementTechnician \in \SetTechnician \\ 
		& \VarSupervisorOperationWhole \in \{0, 1\}  \quad \forall \ElementOperation \in \SetOperation{}{\VarStrategicWorkOrderAssignment{}{}} \\ 
		& \VarSupervisorWorkOrderWhole \in \{0, 1\}  \quad \forall \ElementWorkOrder \in \SetWorkOrder{,\VarStrategicWorkOrderAssignment{}{}} \\ 
		& \VarActivityWork{\ElementActivity} \in [\ParLowerActivityWork, \ParOperationWork{\ElementActivity}]  \quad \forall \ElementActivity \in \SetActivity{\VarStrategicWorkOrderAssignment{}{}}{}\\ 
		& \text{\rule{\linewidth}{0.8pt}} \notag \label{}                                                                                                                                                                                                                                                                                                                                                                      
	\end{alignat}
}

\newpage
\subsection*{Technician Model: Single Machine Scheduling Problem Variant}
\section{The Operational Model}

Here the o is a single operation and o2 is another operation. It is crucial to understand here that the main
decision variable, $x$ defines an ordering of the operations that a single operational agent will do the 
operations in. 

The $\VarStartOfSegment{a}{k}$ is the start time of job $i$ in segment $k$ and $\VarFinishOfSegment{a}{k}$ is the finish time of job $i$ in segment $k$.
$\VarProcessingTime{a}{k}$ is the processing time of each segment. 
\begin{alignat}{2}
	& Max \sum_{a \in \SetActivity{\VarSupervisorAssignment}} \sum_{k \in \SetWorkSegment} \VarProcessingTime                                                         \\
	& \text{Subject to:} \notag                                                                                                                                       \\
    & \sum_{k \in \SetWorkSegment} \VarProcessingTime \cdot \VarActiveSegment{a}{k} = \ParOperationWork \cdot \VarIncludeActivity \quad \forall a \in \SetActivity{\VarSupervisorAssignment} \\
	& \VarStartOfSegment{a2}{1} \geq \VarFinishOfSegment{a1}{last(a1)} + \ParPreparation \notag                                                                       \\ 
	& \quad \forall a1 \in \SetActivity{\VarSupervisorAssignment}, a2 \in \SetActivity{\VarSupervisorAssignment}                                                     \\
	& \VarStartOfSegment{a}{k} \geq \VarFinishOfSegment{a}{k-1} - \ParConstraintLimit \cdot (2 - \VarActiveSegment{a}{k} + \VarActiveSegment{a}{k-1})                \notag\\
	& \quad \forall a \in \SetActivity{\VarSupervisorAssignment}\forall k \in \SetWorkSegment \\ 
	& \VarProcessingTime = \VarFinishOfSegment{a}{k} - \VarStartOfSegment{a}{k}                                                                                       \\
	& \quad \forall a \in \SetActivity{\VarSupervisorAssignment}, k \in \SetWorkSegment \notag                                                                        \\
	& \VarStartOfSegment{a}{k} \geq \ParEvent + \ParEventDuration - \ParConstraintLimit \cdot (1 - \VarSegmentInRelation)                                             \notag\\ 
	& \quad \forall a \in \SetActivity{\VarSupervisorAssignment}, k \in \SetWorkSegment, i \in \SetTimeInstance, e \in \SetEvent                                \\
	& \VarFinishOfSegment{a}{k} \leq \ParEvent + \ParConstraintLimit \cdot \VarSegmentInRelation                                                                      \notag\\ 
	& \quad \forall a \in \SetActivity{\VarSupervisorAssignment}, k \in \SetWorkSegment, i \in \SetTimeInstance, e \in \SetEvent                                \\
	& \VarStartOfSegment{a}{1} \geq \ParTimeWindowStart \forall a \in \SetActivity{\VarSupervisorAssignment}                                                          \\
	& \VarFinishOfSegment{a}{last(a)} \leq \ParTimeWindowFinish \forall a \in \SetActivity{\VarSupervisorAssignment}                                                  \\
	& \VarActiveSegment{a}{k} \in \{0, 1\} \quad \forall a \in \SetActivity{\VarSupervisorAssignment}, k \in \SetWorkSegment                                          \\
	& \VarStartOfSegment{a}{k} \in [\ParAvailabilityStart, \ParAvailabilityFinish] \notag\\
	& \quad \forall a \in \SetActivity{\VarSupervisorAssignment}, k \in \SetWorkSegment  \\
	& \VarFinishOfSegment{a}{k} \in [\ParAvailabilityStart, \ParAvailabilityFinish] \notag\\
	& \quad \forall a \in \SetActivity{\VarSupervisorAssignment}, k \in \SetWorkSegment \\
	& \VarProcessingTime \in [0, \ParOperationWork] \quad \forall a \in \SetActivity{\VarSupervisorAssignment}, k \in \SetWorkSegment                                               \\
	& \VarSegmentInRelation \in \{0, 1\}                                                                                                                              \\ 
	& \quad \forall a \in \SetActivity{\VarSupervisorAssignment}, k \in \SetWorkSegment, i \in \SetTimeInstance, e \in \SetEvent \notag                               \\
	& \theta_i \in \{0, 1\} \quad \forall a \in \SetActivity{\VarSupervisorAssignment}                                                                                \\
\end{alignat}

