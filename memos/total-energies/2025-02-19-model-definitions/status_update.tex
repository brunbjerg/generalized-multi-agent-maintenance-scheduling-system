\section{Status Update}

\begin{itemize}
	\item Clustering in the strategic model
	\item Latest system architecture
	\item Next steps
\end{itemize}

\section{Clustering}
% \pgfkeys{
	/hexagon/.is family, /hexagon,
	default/.style = {
	},
}
\newcommand{\drawClusteringFunctionalLocation}[1][]{
	\pgfkeys{/hexagon, default, #1}

	\begin{tikzpicture}[font=\footnotesize, scale=0.5, line width=1.05]
		\draw (0,0) node[circle] {A};
	\end{tikzpicture}
}

% \drawClusteringFunctionalLocation[]

Hi Brian and Valentin,

A while back, we discussed clustering work orders within the
algorithm. My notes indicate that clustering based on either \textbf{functional location}
or \textbf{platform location} appeared to be the most promising approach (as also highlighted in \citep{palmerMaintenancePlanningScheduling2019weeklyscheduling}). Although
this wasn’t part of the priorities outlined in my previous memo, I ended up
implementing this feature in the code for my latest research paper.

To give you a quick overview, I'll explain the general concept focusing on the
functional location aspect. The calculations and methodology are based on the
diagram in figure~\ref{fig:functional-location} that you provided.

\begin{figure}[H]
	\centering
	\includegraphics[width=10cm]{../../../figures/total-figures/functional_location_total.png}\label{fig:functional-location}
	\caption{Functional location hierarchy for clustering calculation}
\end{figure}

Equation~\ref{eqn:clustering} defines a similarity measure between work orders based
on their functional location as detailed below:

\newcommand{\Highlight}[2]{\colorbox{#2}{#1}}
\begin{equation}
	\begin{aligned}
		clus&tering\_value_{wo1, wo2}=  \\
		      \ \ & if \ \ SITE_{wo1}      & == & \ \  SITE_{wo2}      &\ \ \{ +20 \} \\ 
		    + \ \ & if \ \ SECTOR_{wo1}    & == & \ \  SECTOR_{wo2}    &\ \ \{ +10 \} \\ 
		    + \ \ & if \ \ SYSTEM_{wo1}    & == & \ \  SYSTEM_{wo2}    &\ \ \{ +10 \} \\ 
		    + \ \ & if \ \ SUBSYSTEM_{wo1} & == & \ \  SUBSYSTEM_{wo2} &\ \ \{ +10 \} \\ 
		    + \ \ & if \ \ TAG_{wo1}       & == & \ \  TAG_{wo2}       &\ \ \{ +50 \} \\ 
	\end{aligned}
	\label{eqn:clustering}
\end{equation}

This $clustering\_value_{wo1, wo2}$ in equation~\ref{eqn:clustering} is then calculated for all \textbf{released} work orders
for a given asset on which algorithm is initialized. In table~\ref{tab:work-order-functional-location} will now go through a small
example to show how this could look like.

\begin{table}[H]
	\centering
	\begin{tabular}{lrrrrr}
		\toprule 
		\textbf{Functional Location} & \textbf{Site} & \textbf{Sector} & \textbf{System} & \textbf{Subsystem} & \textbf{Tag} \\
		\midrule  
		\textbf{2100000001} & DF & B & 3 & 6 & DFBA-FA-4004\\
		\textbf{2100000002} & DF & B & 3 & 6 & DFBA-FA-4004\\
		\textbf{2100000003} & DF & B & 2 & 3 & DFFA-LIT-500106\\
		\bottomrule
	\end{tabular}
	\caption{\parbox{0.6\textwidth}{Functional locations for three different example work orders}}
	\label{tab:work-order-functional-location}

\end{table}

Using the data in Table~\ref{tab:work-order-functional-location} we can
calculate the matrix in Table~\ref{tab:clustering-matrix} to obtain a similarity
between the different work orders.

\begin{table}[ht]
	\centering
	\label{tab:clustering-matrix}
	\begin{tabular}{|l|c|c|c|}
		\toprule
		\textbf{Work Orders}& \textbf{2100000001}        & \textbf{2100000002}         & \textbf{2100000003} \\
		\midrule
		\textbf{2100000001} & -                          & \Highlight{100}{dtu-green}  & \Highlight{30}{dtu-yellow} \\
		\midrule
		\textbf{2100000002} & \Highlight{100}{dtu-green} & -                           & \Highlight{30}{dtu-yellow} \\
		\midrule
		\textbf{2100000003} & \Highlight{30}{dtu-yellow} & \Highlight{30}{dtu-yellow}  & - \\
		\bottomrule
	\end{tabular}
	\caption{\parbox{0.8\textwidth}{Clustering values between work orders based on functional location. Yellow values are
		calculated as: 20 + 10 = \Highlight{30}{dtu-yellow} (same SITE and SECTOR), and the green values have been calculated as:
		20 + 10 + 10 + 10 + 50 = \Highlight{100}{dtu-green} (same SITE, SECTOR, SYSTEM, SUBSYSTEM, and TAG)}}
\end{table}

To see where this goes into the strategic model, refer to \textbf{red} part in the model in Section~\ref{sec:strategic-model}.

\section{Latest Architecture}
I have updated the overall architecture extensively with around 5000 lines of
code changes. This is not something that you will see the benefit of directly
but it is crucial for the program to perform as intended if scaled up. This
change means that we have gone from using message-passing when the algorithms
have to share state to using atomic pointer swaps (the purple \textbf{Shared
Solution} in Figure~\ref{fig:hexagon:simplified}).

\documentclass{standalone}
\usepackage{tikz}
\usetikzlibrary {positioning}

<<<<<<< HEAD
\newcommand{\drawHexagon}[5][draw=black]{
	\draw[#1, fill=#4, ] (#2,#3) ++(30:2) -- ++(90:2) -- ++(150:2) -- ++(210:2) -- ++(270:2) -- ++(330:2) -- cycle;
	\node at (#2,#3+2) {#5};
||||||| 2d4327a
\newcommand{\drawHexagon}[4]{
	\draw[fill=#3] (#1,#2) ++(30:2) -- ++(90:2) -- ++(150:2) -- ++(210:2) -- ++(270:2) -- ++(330:2) -- cycle;
	\node at (#1,#2+2) {#4};
=======
\begin{document}
\newcommand{\drawHexagon}[4]{
	\draw[fill=#3] (#1,#2) ++(30:2) -- ++(90:2) -- ++(150:2) -- ++(210:2) -- ++(270:2) -- ++(330:2) -- cycle;
	\node at (#1,#2+2) {#4};
>>>>>>> 023797133fb426c1bb01f920d8f5635c343d11a6
}

\newif\ifuserinterfacelayer
\newif\ifpersistencelayer
\newif\ifmetaheuristicslayer
\newif\ifatomicpointerswaplayer

\pgfkeys{
	/hexagon/.is family, /hexagon,
	default/.style = {
		persistence=false,
		userinterface=false,
		metaheuristics=true,
	},
	persistence/.is if=persistencelayer,
	userinterface/.is if=userinterfacelayer,
	metaheuristics/.is if=metaheuristicslayer,
}
\newcommand{\drawModelSetupHexagon}[1][]{
	\pgfkeys{/hexagon, default, #1}

	\begin{tikzpicture}[scale=0.6, line width=1.05]
	
	\ifuserinterfacelayer
		\drawHexagon{ 2                      }{ 2}{dtu-yellow}{UI}
		\drawHexagon{{6 - 2 * (2 - sqrt(3)) }}{ 2}{dtu-yellow}{UI}
		\drawHexagon{{4 - 1 * (2 - sqrt(3)) }}{-1}{dtu-yellow}{UI}
		\drawHexagon{{0 + 1 * (2 - sqrt(3)) }}{-1}{dtu-yellow}{UI}
		\drawHexagon{{8 - 3 * (2 - sqrt(3)) }}{-1}{dtu-yellow}{UI}

		\drawHexagon{{2 - 0 * (2 - sqrt(3)) }}{-4}{dtu-yellow}{UI}
		\drawHexagon{{6 - 2 * (2 - sqrt(3)) }}{-4}{dtu-yellow}{UI}

		\drawHexagon{{10 - 4 * (2 - sqrt(3)) }}{-4}{dtu-yellow}{UI}
		\drawHexagon{{-2 + 2 * (2 - sqrt(3)) }}{-4}{dtu-yellow}{UI}

		\drawHexagon{{12 - 5 * (2 - sqrt(3)) }}{-1}{dtu-yellow}{UI}
		\drawHexagon{{-4 + 3 * (2 - sqrt(3)) }}{-1}{dtu-yellow}{UI}
	\fi

	\ifpersistencelayer
		\drawHexagon[draw=none]{ 2                      }{ 2}{dtu-blue}{}
		\drawHexagon[draw=none]{{6 - 2 * (2 - sqrt(3)) }}{ 2}{dtu-blue}{}
		\drawHexagon[draw=none]{{4 - 1 * (2 - sqrt(3)) }}{-1}{dtu-blue}{Database}
		\drawHexagon[draw=none]{{0 + 1 * (2 - sqrt(3)) }}{-1}{dtu-blue}{}
		\drawHexagon[draw=none]{{8 - 3 * (2 - sqrt(3)) }}{-1}{dtu-blue}{}

		\drawHexagon[draw=none]{{2 - 0 * (2 - sqrt(3)) }}{-4}{dtu-blue}{}
		\drawHexagon[draw=none]{{6 - 2 * (2 - sqrt(3)) }}{-4}{dtu-blue}{}

		\drawHexagon[draw=none]{{10 - 4 * (2 - sqrt(3)) }}{-4}{dtu-blue}{}
		\drawHexagon[draw=none]{{-2 + 2 * (2 - sqrt(3)) }}{-4}{dtu-blue}{}

		\drawHexagon[draw=none]{{12 - 5 * (2 - sqrt(3)) }}{-1}{dtu-blue}{}
		\drawHexagon[draw=none]{{-4 + 3 * (2 - sqrt(3)) }}{-1}{dtu-blue}{}
	\fi

	\ifmetaheuristicslayer
		\drawHexagon{ 2                      }{ 2}{dtu-blue}{Strategic}
		\drawHexagon{{6 - 2 * (2 - sqrt(3)) }}{ 2}{dtu-green}{Tactical}
		\drawHexagon{{4 - 1 * (2 - sqrt(3)) }}{-1}{dtu-red}{Supervisor}
		\drawHexagon{{0 + 1 * (2 - sqrt(3)) }}{-1}{dtu-red}{Supervisor}
		\drawHexagon{{8 - 3 * (2 - sqrt(3)) }}{-1}{dtu-red}{Supervisor}

		\drawHexagon{{2 - 0 * (2 - sqrt(3)) }}{-4}{dtu-corporate-red}{Technician}
		\drawHexagon{{6 - 2 * (2 - sqrt(3)) }}{-4}{dtu-corporate-red}{Technician}

		\drawHexagon{{10 - 4 * (2 - sqrt(3)) }}{-4}{dtu-corporate-red}{Technician}
		\drawHexagon{{-2 + 2 * (2 - sqrt(3)) }}{-4}{dtu-corporate-red}{Technician}

		\drawHexagon{{12 - 5 * (2 - sqrt(3)) }}{-1}{dtu-corporate-red}{Technician}
		\drawHexagon{{-4 + 3 * (2 - sqrt(3)) }}{-1}{dtu-corporate-red}{Technician}
	\fi

	
	\end{tikzpicture}
}

\ifstandalone
	

\definecolor{red}{HTML}{8A3F3A}
\definecolor{yellow}{HTML}{E0BB3C}
\definecolor{blue}{HTML}{4569E0}
\definecolor{green}{HTML}{17E561}
\definecolor{other}{HTML}{6A939E}

	\drawModelSetupHexagon
\fi

\end{document}

\begin{figure}[H]
	\centering
    \drawModelSetupHexagon[simplifiedtechnicallanguage=true]
	\caption{\parbox{0.8\textwidth}{
		In the system so far, we have three stakeholders, each running an algorithm
		in perpetuity to optimize their respective schedules. The \textbf{purple} dots indicate
		that the solutions for all algorithms will be shared among them, enabling
		rapid response to changes once the system is fully implemented. The \textbf{yellow}
		dots highlight points of user interaction. The idea is that the algorithms will
		handle approximately 60–80 of the work, after which the end-user will manually
		adjust aspects that the algorithms cannot manage.
	}}\label{fig:hexagon:simplified}
\end{figure}

\section{Next Steps}
The last feedback from \textbf{Brian} is highlighted in yellow:
\begin{itemize}
	\item What aspects of the program does each of you think should be prioritized now?
	\item \colorbox{yellow}{\textbf{Providing some data which can be analyzed to check the output quality.}}
	\item I believe that making this program work without any involvement of a offshore
    	  supervisor (the person responsible for assigning names to the technicians) 
		  will be very difficult. Do you agree or disagree?
	\item \colorbox{yellow}{\textbf{The end-user is interested in the results and the proof we can give that the}}\\
		  \colorbox{yellow}{\textbf{output is the best feasible solution. Offshore are not having their mindset on}}\\
		  \colorbox{yellow}{\textbf{IT solutions in their daily work. They will be able to guide in the direction of}}\\
		  \colorbox{yellow}{\textbf{how they want it presented as end-user.}}\\
\end{itemize} 

I have unfortunately only made a little further progress on the first point.
I wish I had more time to test more intensively but I have to prioritize
publishing research to finish the research project on time. I believe that if
this project is to be successfully implemented, I will need to work full time on
the system rather than splitting my focus with ongoing research.

Without an extension of the research project we have to drop the advanced parts
of the frontends. I cannot program that amount of code alone within the current
timeline of the project. I hope that you will have the imagination to see how the
system could perform if the relevant frontends were actually implemented. 

I will prioritize making the \textbf{tactical} algorithm work now in line
with the first priority that \textbf{Brian} highlighted from the last memo.
