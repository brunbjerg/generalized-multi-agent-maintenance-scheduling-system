\section{Status Update}

\begin{itemize}
	\item Writing to SAP
	\item Coding on the server side
	\item Academic Papers
	\item External Stay
	\item Problems and Issues
\end{itemize}

% FIX: MENTION THE NAME OF THE SAP PERSON.
\section{Writing to SAP}
I have had disscussions with people in Paris about writing to SAP. Specificially 
asked if it would be possible to write to table AFVC column ABLAD which is the 
column called "Unloading Point". This is posible but it will require all
my source code to be uploaded to Total Energies servers, which I are a little 
reluctant to do as I will lose control over the research project.

\section{Coding on the server side}
Since out last meeting I have done a significant amount of coding on the server side
of the application, most of it is related to the tactical model, the algorithm that schedules
on the days, to make it dynamic and provide results that are satisfactory.

To reiterate the long term goal here refer to figure~\ref{fig:hexagon:persistence}, \ref{fig:hexagon:metaheuristics}, and \ref{fig:hexagon:userinterfaces}.
Figure~\ref{fig:hexagon:persistence} holds all the data
needed to make the system function as well as capture all the changes that the end-users make based on the results of the algorithms.

\usetikzlibrary {positioning}
\newcommand{\drawHexagon}[6][draw=black]{
	\draw[#1, fill=#4] (#2,#3) ++(30:#6) -- ++(90:#6) -- ++(150:#6) -- ++(210:#6) -- ++(270:#6) -- ++(330:#6) -- cycle;
	\node[align=center] at (#2,#3+2) {#5};
}

\newif\ifpersistencelayer
\newif\ifatomicpointerswaplayer
\newif\ifmetaheuristicslayer
\newif\ifuserinterfacelayer
\newif\iforchestratorlayer
\newif\ifsimplifiedlayer

\pgfkeys{
	/hexagon/.is family, /hexagon,
	default/.style = {
		persistence=false,
		atomicpointerswap=false,
		metaheuristics=false,
		orchestrator=false,
		userinterface=false,
		simplified=false,
	},
	persistence/.is if=persistencelayer,
	atomicpointerswap/.is if=atomicpointerswaplayer,
	metaheuristics/.is if=metaheuristicslayer,
	orchestrator/.is if=orchestratorlayer,
	userinterface/.is if=userinterfacelayer,
	simplified/.is if=simplifiedlayer,
}
\newcommand{\drawModelSetupHexagon}[1][]{
	\pgfkeys{/hexagon, default, #1}

	\begin{tikzpicture}[font=\footnotesize, scale=0.5, line width=1.05]
	

	\ifpersistencelayer
		\drawHexagon[draw=none]{ 2                      }{ 2}{dtu-blue}{}{2}
		\drawHexagon[draw=none]{{6 - 2 * (2 - sqrt(3)) }}{ 2}{dtu-blue}{}{2}
		\drawHexagon[draw=none]{{4 - 1 * (2 - sqrt(3)) }}{-1}{dtu-blue}{Persistence}{2}
		\drawHexagon[draw=none]{{0 + 1 * (2 - sqrt(3)) }}{-1}{dtu-blue}{}{2}
		\drawHexagon[draw=none]{{8 - 3 * (2 - sqrt(3)) }}{-1}{dtu-blue}{}{2}

		\drawHexagon[draw=none]{{2 - 0 * (2 - sqrt(3)) }}{-4}{dtu-blue}{}{2}
		\drawHexagon[draw=none]{{6 - 2 * (2 - sqrt(3)) }}{-4}{dtu-blue}{}{2}

		\drawHexagon[draw=none]{{10 - 4 * (2 - sqrt(3)) }}{-4}{dtu-blue}{}{2}
		\drawHexagon[draw=none]{{-2 + 2 * (2 - sqrt(3)) }}{-4}{dtu-blue}{}{2}

		\drawHexagon[draw=none]{{12 - 5 * (2 - sqrt(3)) }}{-1}{dtu-blue}{}{2}
		\drawHexagon[draw=none]{{-4 + 3 * (2 - sqrt(3)) }}{-1}{dtu-blue}{}{2}
		% Legend for each layer
		\drawHexagon{{14.0  }}{+3.0}{dtu-blue}{}{0.75}
		\node[align=right, anchor=west] at ({15.0}, +3.75) {Persistence};
		\drawHexagon{{14.0  }}{+1.5}{dtu-white}{}{0.75}
		\node[align=right, anchor=west] at ({15.0}, +2.25) {Atomic Pointer};
		\drawHexagon{{14.0  }}{+0.0}{dtu-white}{}{0.75}
		\node[align=right, anchor=west] at ({15.0}, +0.75) {Metaheuristics};
		\drawHexagon{{14.0  }}{-1.5}{dtu-white}{}{0.75}
		\node[align=right, anchor=west] at ({15.0}, -0.75) {Orchestration};
		\drawHexagon{{14.0  }}{-3.0}{dtu-white}{}{0.75}
		\node[align=right, anchor=west] at ({15.0}, -2.25) {User interfaces};
	\fi


	\ifatomicpointerswaplayer
		\drawHexagon[]{ 2                      }{ 2}{dtu-green}{Shared\\solution\\pointer}{2}
		\drawHexagon[]{{6 - 2 * (2 - sqrt(3)) }}{ 2}{dtu-green}{Shared\\solution\\pointer}{2}
		\drawHexagon[]{{4 - 1 * (2 - sqrt(3)) }}{-1}{dtu-green}{Shared\\solution\\pointer}{2}
		\drawHexagon[]{{0 + 1 * (2 - sqrt(3)) }}{-1}{dtu-green}{Shared\\solution\\pointer}{2}
		\drawHexagon[]{{8 - 3 * (2 - sqrt(3)) }}{-1}{dtu-green}{Shared\\solution\\pointer}{2}

		\drawHexagon[]{{2 - 0 * (2 - sqrt(3)) }}{-4}{dtu-green}{Shared\\solution\\pointer}{2}
		\drawHexagon[]{{6 - 2 * (2 - sqrt(3)) }}{-4}{dtu-green}{Shared\\solution\\pointer}{2}

		\drawHexagon[]{{10 - 4 * (2 - sqrt(3)) }}{-4}{dtu-green}{Shared\\solution\\pointer}{2}
		\drawHexagon[]{{-2 + 2 * (2 - sqrt(3)) }}{-4}{dtu-green}{Shared\\solution\\pointer}{2}

		\drawHexagon[]{{12 - 5 * (2 - sqrt(3)) }}{-1}{dtu-green}{Shared\\solution\\pointer}{2}
		\drawHexagon[]{{-4 + 3 * (2 - sqrt(3)) }}{-1}{dtu-green}{Shared\\solution\\pointer}{2}
		% Legend for each layer
		\drawHexagon{{14.0  }}{+3.0}{dtu-white}{}{0.75}
		\node[align=right, anchor=west] at ({15.0}, +3.75) {Persistence};
		\drawHexagon{{14.0  }}{+1.5}{dtu-green}{}{0.75}
		\node[align=right, anchor=west] at ({15.0}, +2.25) {Atomic Pointer};
		\drawHexagon{{14.0  }}{+0.0}{dtu-white}{}{0.75}
		\node[align=right, anchor=west] at ({15.0}, +0.75) {Metaheuristics};
		\drawHexagon{{14.0  }}{-1.5}{dtu-white}{}{0.75}
		\node[align=right, anchor=west] at ({15.0}, -0.75) {Orchestration};
		\drawHexagon{{14.0  }}{-3.0}{dtu-white}{}{0.75}
		\node[align=right, anchor=west] at ({15.0}, -2.25) {User interfaces};
	\fi

	\ifsimplifiedlayer

		\node[align=right, anchor=west] at ({-5.5}, +3.75) {};
		\drawHexagon{{+2 + 0 * (2 - sqrt(3)) }}{ 2}{dtu-green}{Scheduler}{2}
		\drawHexagon{{+4 - 1 * (2 - sqrt(3)) }}{-1}{dtu-red}{Supervisor}{2}
		\drawHexagon{{+0 + 1 * (2 - sqrt(3)) }}{-1}{dtu-red}{Supervisor}{2}
		\drawHexagon{{+2 - 0 * (2 - sqrt(3)) }}{-4}{dtu-corporate-red}{Technician}{2}
		\drawHexagon{{+6 - 2 * (2 - sqrt(3)) }}{-4}{dtu-corporate-red}{Technician}{2}
		\drawHexagon{{-2 + 2 * (2 - sqrt(3)) }}{-4}{dtu-corporate-red}{Technician}{2}
		\drawHexagon{{+8 - 3 * (2 - sqrt(3)) }}{-1}{dtu-corporate-red}{Technician}{2}
		\drawHexagon{{-4 + 3 * (2 - sqrt(3)) }}{-1}{dtu-corporate-red}{Technician}{2}

		% Scheduler
		\draw[thin, fill=dtu-yellow] (2, 5) circle (0.35);
		\draw[thin, fill=dtu-purple] (2, 3) circle (0.35);
		% Supervisor 1
		\draw[thin, fill=dtu-yellow] ({+4 - 1 * (2 - sqrt(3)) }, 02) circle (0.35);
		\draw[thin, fill=dtu-purple] ({+4 - 1 * (2 - sqrt(3)) }, -0) circle (0.35);
		% Supervisor 2
		\draw[thin, fill=dtu-yellow] ({+0 + 1 * (2 - sqrt(3)) }, 02) circle (0.35);
		\draw[thin, fill=dtu-purple] ({+0 + 1 * (2 - sqrt(3)) }, -0) circle (0.35);
		% Technician 1
		\draw[thin, fill=dtu-yellow] ({+2 - 0 * (2 - sqrt(3)) }, -1) circle (0.35);
		\draw[thin, fill=dtu-purple] ({+2 - 0 * (2 - sqrt(3)) }, -3) circle (0.35);
		% Technician 2
		\draw[thin, fill=dtu-yellow] ({+6 - 2 * (2 - sqrt(3)) }, -1) circle (0.35);
		\draw[thin, fill=dtu-purple] ({+6 - 2 * (2 - sqrt(3)) }, -3) circle (0.35);
		% Technician 3
		\draw[thin, fill=dtu-yellow] ({-2 + 2 * (2 - sqrt(3)) }, -1) circle (0.35);
		\draw[thin, fill=dtu-purple] ({-2 + 2 * (2 - sqrt(3)) }, -3) circle (0.35);
		% Technician 4
		\draw[thin, fill=dtu-yellow] ({+8 - 3 * (2 - sqrt(3)) }, 02) circle (0.35);
		\draw[thin, fill=dtu-purple] ({+8 - 3 * (2 - sqrt(3)) }, -0) circle (0.35);
		% Technician 5
		\draw[thin, fill=dtu-yellow] ({-4 + 3 * (2 - sqrt(3)) }, 02) circle (0.35);
		\draw[thin, fill=dtu-purple] ({-4 + 3 * (2 - sqrt(3)) }, -0) circle (0.35);

		% Legend for each layer
		\node[align=right, anchor=west] at ({12.0}, +3.75) {Atomic Pointer};
		\draw[fill=dtu-purple] (11.0,  +3.75) circle (0.5);

		\node[align=right, anchor=west] at ({12.0}, +2.25) {Scheduler Metaheuristic};
		\drawHexagon{{11.0  }}{+1.75}{dtu-green}{}{0.5}
		\node[align=right, anchor=west] at ({12.0}, +0.75) {Supervisor Metaheuristic};
		\drawHexagon{{11.0  }}{+0.25}{dtu-red}{}{0.5}
		\node[align=right, anchor=west] at ({12.0}, -0.75) {Technician Metaheuristic};
		\drawHexagon{{11.0  }}{-1.25}{dtu-corporate-red}{}{0.5}
		\node[align=right, anchor=west] at ({12.0}, -2.25) {User interfaces (Message Passing)};
		\draw[fill=dtu-yellow] (11.0, -2.25) circle (0.5);
	\fi

	\ifmetaheuristicslayer
		\drawHexagon{ 2                      }{ 2}{dtu-blue}{Strategic}{2}
		\drawHexagon{{6 - 2 * (2 - sqrt(3)) }}{ 2}{dtu-green}{Tactical}{2}
		\drawHexagon{{4 - 1 * (2 - sqrt(3)) }}{-1}{dtu-red}{Supervisor}{2}
		\drawHexagon{{0 + 1 * (2 - sqrt(3)) }}{-1}{dtu-red}{Supervisor}{2}
		\drawHexagon{{8 - 3 * (2 - sqrt(3)) }}{-1}{dtu-red}{Supervisor}{2}

		\drawHexagon{{2 - 0 * (2 - sqrt(3)) }}{-4}{dtu-corporate-red}{Technician}{2}
		\drawHexagon{{6 - 2 * (2 - sqrt(3)) }}{-4}{dtu-corporate-red}{Technician}{2}

		\drawHexagon{{10 - 4 * (2 - sqrt(3)) }}{-4}{dtu-corporate-red}{Technician}{2}
		\drawHexagon{{-2 + 2 * (2 - sqrt(3)) }}{-4}{dtu-corporate-red}{Technician}{2}

		\drawHexagon{{12 - 5 * (2 - sqrt(3)) }}{-1}{dtu-corporate-red}{Technician}{2}
		\drawHexagon{{-4 + 3 * (2 - sqrt(3)) }}{-1}{dtu-corporate-red}{Technician}{2}

		% Legend for each layer
		\drawHexagon{{14.0  }}{+3.0}{dtu-white}{}{0.75}
		\node[align=right, anchor=west] at ({15.0}, +3.75) {Persistence};
		\drawHexagon{{14.0  }}{+1.5}{dtu-white}{}{0.75}
		\node[align=right, anchor=west] at ({15.0}, +2.25) {Atomic Pointer};
		\drawHexagon{{14.0  }}{+0.0}{dtu-corporate-red}{}{0.75}
		\node[align=right, anchor=west] at ({15.0}, +0.75) {Metaheuristics};
		\drawHexagon{{14.0  }}{-1.5}{dtu-white}{}{0.75}
		\node[align=right, anchor=west] at ({15.0}, -0.75) {Orchestration};
		\drawHexagon{{14.0  }}{-3.0}{dtu-white}{}{0.75}
		\node[align=right, anchor=west] at ({15.0}, -2.25) {User interfaces};
	\fi

	\iforchestratorlayer
		\drawHexagon{ 2                      }{ 2}{dtu-orange}{}{2}
		\drawHexagon{{6 - 2 * (2 - sqrt(3)) }}{ 2}{dtu-orange}{}{2}
		\drawHexagon{{4 - 1 * (2 - sqrt(3)) }}{-1}{dtu-orange}{Orche-\\strator}{2}
		\drawHexagon{{0 + 1 * (2 - sqrt(3)) }}{-1}{dtu-orange}{}{2}
		\drawHexagon{{8 - 3 * (2 - sqrt(3)) }}{-1}{dtu-orange}{}{2}

		\drawHexagon{{2 - 0 * (2 - sqrt(3)) }}{-4}{dtu-orange}{}{2}
		\drawHexagon{{6 - 2 * (2 - sqrt(3)) }}{-4}{dtu-orange}{}{2}

		\drawHexagon{{10 - 4 * (2 - sqrt(3)) }}{-4}{dtu-orange}{}{2}
		\drawHexagon{{-2 + 2 * (2 - sqrt(3)) }}{-4}{dtu-orange}{}{2}

		\drawHexagon{{12 - 5 * (2 - sqrt(3)) }}{-1}{dtu-orange}{}{2}
		\drawHexagon{{-4 + 3 * (2 - sqrt(3)) }}{-1}{dtu-orange}{}{2}
		% Legend for each layer
		\drawHexagon{{14.0  }}{+3.0}{dtu-white}{}{0.75}
		\node[align=right, anchor=west] at ({15.0}, +3.75) {Persistence};
		\drawHexagon{{14.0  }}{+1.5}{dtu-white}{}{0.75}
		\node[align=right, anchor=west] at ({15.0}, +2.25) {Atomic Pointer};
		\drawHexagon{{14.0  }}{+0.0}{dtu-white}{}{0.75}
		\node[align=right, anchor=west] at ({15.0}, +0.75) {Metaheuristics};
		\drawHexagon{{14.0  }}{-1.5}{dtu-orange}{}{0.75}
		\node[align=right, anchor=west] at ({15.0}, -0.75) {Orchestration};
		\drawHexagon{{14.0  }}{-3.0}{dtu-white}{}{0.75}
		\node[align=right, anchor=west] at ({15.0}, -2.25) {User interfaces};
	\fi

	
	\ifuserinterfacelayer
		\drawHexagon{ 2                      }{ 2}{dtu-yellow}{UI}{2}
		\drawHexagon{{6 - 2 * (2 - sqrt(3)) }}{ 2}{dtu-yellow}{UI}{2}
		\drawHexagon{{4 - 1 * (2 - sqrt(3)) }}{-1}{dtu-yellow}{UI}{2}
		\drawHexagon{{0 + 1 * (2 - sqrt(3)) }}{-1}{dtu-yellow}{UI}{2}
		\drawHexagon{{8 - 3 * (2 - sqrt(3)) }}{-1}{dtu-yellow}{UI}{2}

		\drawHexagon{{2 - 0 * (2 - sqrt(3)) }}{-4}{dtu-yellow}{UI}{2}
		\drawHexagon{{6 - 2 * (2 - sqrt(3)) }}{-4}{dtu-yellow}{UI}{2}

		\drawHexagon{{10 - 4 * (2 - sqrt(3)) }}{-4}{dtu-yellow}{UI}{2}
		\drawHexagon{{-2 + 2 * (2 - sqrt(3)) }}{-4}{dtu-yellow}{UI}{2}

		\drawHexagon{{12 - 5 * (2 - sqrt(3)) }}{-1}{dtu-yellow}{UI}{2}
		\drawHexagon{{-4 + 3 * (2 - sqrt(3)) }}{-1}{dtu-yellow}{UI}{2}
		% Legend for each layer
		\drawHexagon{{14.0  }}{+3.0}{dtu-white}{}{0.75}
		\node[align=right, anchor=west] at ({15.0}, +3.75) {Persistence};
		\drawHexagon{{14.0  }}{+1.5}{dtu-white}{}{0.75}
		\node[align=right, anchor=west] at ({15.0}, +2.25) {Atomic Pointer};
		\drawHexagon{{14.0  }}{+0.0}{dtu-white}{}{0.75}
		\node[align=right, anchor=west] at ({15.0}, +0.75) {Metaheuristics};
		\drawHexagon{{14.0  }}{-1.5}{dtu-white}{}{0.75}
		\node[align=right, anchor=west] at ({15.0}, -0.75) {Orchestration};
		\drawHexagon{{14.0  }}{-3.0}{dtu-yellow}{}{0.75}
		\node[align=right, anchor=west] at ({15.0}, -2.25) {User interfaces};
	\fi
	
	\end{tikzpicture}
}

\begin{figure}[H]
	\centering
    \drawModelSetupHexagon[userinterface=false, persistence=true, metaheuristics=false]
	\caption{Common database layer for all algorithms in the system. The database layer is initialized from
	the \textbf{SAP}, and the also receives inputs from the user for \textbf{Resources} and \textbf{Timehorizons}.}
	\label{fig:hexagon:persistence}
\end{figure}

\begin{figure}[H]
	\centering
    \drawModelSetupHexagon[userinterface=false, persistence=false, metaheuristics=true]
	\caption{The proposed model setup. The server creates one \textbf{Strategic} algorithm as shown in section~\ref{sec:model:strategic}, 
	on \textbf{Tactical} algorithm as shown in section~\ref{sec:model:tactical}, and 
	one for each \textbf{Supervisor} as shown in section~\ref{sec:model:supervisor}, 
	finally there one model for each \textbf{Technician} as shown in section~\ref{sec:model:technician}}
	\label{fig:hexagon:metaheuristics}
\end{figure}
Figure~\ref{fig:hexagon:metaheuristics} 
shows the setup where each hexagon is a mathematical algorithm that is optimizing
a certain part of the scheduling process. Here you both have seen a little of the \textbf{Strategic}
and the \textbf{Tactical}. These models will never be able of their own to model the scheduling process therefore user-interfaces are 
created for each of these stakeholders which is illustrated in figure~\ref{fig:hexagon:userinterfaces}.

\begin{figure}[H]
	\centering
    \drawModelSetupHexagon[userinterface=true, persistence=false, metaheuristics=false]
	\caption{Every algorithm has its own user interface. This means that there will be a view into the system that is 
	unique for each kind of stakeholder.}
	\label{fig:hexagon:userinterfaces}
\end{figure}

There are more than these three layers, they serve to provide a high level visualization of what the code is aiming to do.

\subsection{Separating Responsibilites}
Brian mentioned at the previous meeting that for this kind of system it will be crucial that
each stakeholder (Scheduler, Supervisor, Technician) that we want to be part of the system has 
clearly defined boundaries for what they can and cannot change. This means that there are internally
in the server clear boundaries for what each user of the system can change.

\section{Academic Papers}
I unfortunately have to write academic papers even though we do not have solid results yet.
The first paper
is titled \textbf{Actor-based Large Neighborhood Search}, I am trying to complete it as 
soon as possible that we can get back to testing.

\section{External Stay}
I am going of external stay in Paris at a company called \textbf{Decision Brain} it is a 
company that creates maintenance scheduling software that resembles the kind of system 
that we are trying to develop. I am going there with the intention of learning how to
best proceed with implementing the system that we are working on here at Total.
