\documentclass[portrait, a0paper]{tikzposter}

% Packages
\usepackage{amsmath}
\usepackage{graphicx}
\usepackage{multicol}
\usepackage{enumitem}
\usepackage{amssymb}
\usepackage{fontspec}

% Set the main document font to Arial
\setmainfont{FiraCode Nerd Font}

% Title, Author, Institute
\title{Multi-model Maintenance Scheduling}
\author{Christian Brunbjerg Jespersen}
\institute{Technical University of Denmark}

% Define custom colors (optional)
\definecolor{maincolor}{cmyk}{0, 91, 72, 23} % A shade of blue
\definecolor{dtu}{RGB}{255, 102, 0} % A shade of orange

% \colorlet{titlebgcolor}{maincolor}
% \colorlet{titletitlefg}{maincolor}
% \colorlet{titleauthorfg}{maincolor}

\colorlet{blockbodycolor}{maincolor}
\colorlet{blocktitlecolor}{dtu}
\colorlet{blockbodybg}{maincolor}
\colorlet{blocktitlefg}{dtu}
% Theme and color style
% \usetheme{Basic}
% \usecolorstyle{Default}

% Sets
\newcommand{\SetWorkOrder}[1]{W(\VarMetaTime, \VarStrategicWorkOrderAssignment{}{})}
\newcommand{\SetPeriod}{P(\VarMetaTime)}
\newcommand{\SetResource}{R(\VarMetaTime)} %RECONSIDER | RELEVANT FOR BOTH STRATEGIC AND TACTICAL
\newcommand{\SetOperation}[2]{O_{#1}(\VarMetaTime, #2 )}

\newcommand{\SetDays}[1]{D_{#1}(\VarMetaTime)}
\newcommand{\SetActivity}[2]{A_{#2}(\VarMetaTime, #1)}
\newcommand{\SetTechnician}{T(\VarMetaTime)}
\newcommand{\SetWorkSegment}{K(\VarSupervisorAssignment{}{})}

\newcommand{\SetTimeInstance}{I(\VarMetaTime)}
\newcommand{\SetEvent}{E(\VarMetaTime)}

% Parameters
\newcommand{\ParStrategicValue}{strategic\_value_{wp}(\VarMetaTime)}
\newcommand{\ParStrategicPenalty}{strategic\_penalty}
\newcommand{\ParClusteringValue}{clustering\_value_{w1, w2}}
\newcommand{\ParStrategicResource}{resource_{pr}(\VarMetaTime)}

\newcommand{\ParStrategicWorkOrderWeight}{work\_order\_work_{wr}}
\newcommand{\ParStrategicInclude}{include(\VarMetaTime)}
\newcommand{\ParStrategicExclude}{exclude(\VarMetaTime)}
\newcommand{\ParTacticalValue}{tactical\_value_{do}(\VarMetaTime)}

\newcommand{\ParTacticalPenalty}{tactical\_penalty}
\newcommand{\ParOperationWork}[1]{work_{#1}(\VarMetaTime)}
\newcommand{\ParTacticalResource}{tactical\_resource_{dr}(\VarMetaTime)}
\newcommand{\ParStartStart}{start\_start_{o1, o2}}

\newcommand{\ParFinishStart}{finish\_start_{o1, o2}}
\newcommand{\ParNumberOfPeople}{number_{o}(\VarMetaTime)}
\newcommand{\ParOperatingTime}{operating\_time_{o}}
\newcommand{\ParDurationLower}{duration\_lower_{o}(\VarMetaTime)}

\newcommand{\ParDurationUpper}{duration\_upper_{o}(\VarMetaTime)}
\newcommand{\ParSupervisorValue}{supervisor\_value_{at}(\VarMetaTime)} % We should determine if the supervisor should assign for each operation or for each activity. My guts say that it should be for each activity
\newcommand{\ParFeasible}{feasible_{at}(\VarIncludeActivity{})}
\newcommand{\ParOperationsForWorkOrder}{work\_order\_to\_operations_{w}}

\newcommand{\ParOperationsInWorkOrder}{operations\_in\_work\_order_{w}}
\newcommand{\ParActivitiesForOperation}{activities\_for\_operation_{o}}
\newcommand{\ParLowerActivityWork}{lower\_activity\_work_{a}(\VarMetaTime)}
\newcommand{\ParActivityWork}[1]{activity\_work_{a}(\VarMetaTime, \VarActivityWork{#1})}
\newcommand{\ParPreparation}{preparation_{a1, a2}}

\newcommand{\ParEvent}{event_{ie}}
\newcommand{\ParEventDuration}{duration_{ie}}
\newcommand{\ParConstraintLimit}{constraint\_limit}
\newcommand{\ParTimeWindowStart}{time\_window\_start_{a}(\VarTacticalWork{}{})}

\newcommand{\ParTimeWindowFinish}{time\_window\_finish_{a}(\VarTacticalWork{}{})}
\newcommand{\ParAvailabilityStart}{availability\_start(\VarMetaTime)}
\newcommand{\ParAvailabilityFinish}{availability\_finish(\VarMetaTime)}

% Variables
\newcommand{\VarStrategicWorkOrderAssignment}[2]{\alpha_{#1#2}(\VarMetaTime)}
\newcommand{\VarStrategicExcess}{\epsilon_{pr}(\VarMetaTime)}
\newcommand{\VarTacticalWork}[2]{\beta_{#1#2}(\VarMetaTime)}
\newcommand{\VarTacticalExcess}{\mu_{rd}(\VarMetaTime)} : excess capacity for each day

\newcommand{\VarTacticalWorkBinary}[2]{\sigma_{#1#2}(\VarMetaTime)}
\newcommand{\VarTacticalWorkBinaryConsecutive}{\eta_{do}(\VarMetaTime)}
\newcommand{\VarTacticalOperationDifference}{\Delta_{o}(\VarMetaTime)}
\newcommand{\VarSupervisorAssignment}[2]{\gamma_{#1#2}(\VarMetaTime)}
\newcommand{\VarSupervisorAssignmentWhole}{\phi_{o}(\VarMetaTime)}

\newcommand{\VarActivityWork}[1]{\rho_{#1}(\VarMetaTime)}

\newcommand{\VarProcessingTime}{\delta_{ak}(\VarMetaTime)} 
\newcommand{\VarActiveSegment}[2]{\pi_{#1#2}(\VarMetaTime)}
\newcommand{\VarStartOfSegment}[2]{\lambda_{#1#2}(\VarMetaTime)}
\newcommand{\VarFinishOfSegment}[2]{\Lambda_{#1#2}(\VarMetaTime)}

\newcommand{\VarSegmentInRelation}{\omega_{akie}(\VarMetaTime)}
\newcommand{\VarIncludeActivity}[1]{\theta_{#1}(\VarMetaTime)}

% Meta variables
\newcommand{\VarMetaTime}{\tau}

\begin{document}
% Create header
% \begin{tikzpicture}[remember picture,overlay]
%     % \node[fill=maincolor, minimum width=\paperwidth, minimum height=10cm, text=white] (header) at (current page.north) {
%     %     \begin{minipage}{0.9\paperwidth}
%     %         % \centering
%     %         % \vspace{1cm}
%     %         % \huge \textbf{\@title} \\[0.5cm]
%     %         % \Large \textit{\@author} - \@institute
%     %         % \vspace{1cm}
%     %     \end{minipage}
%     % };
% \end{tikzpicture}
% Title block
% \block{Header Block}{Header}
% \begin{tikzpicture}[remember picture,overlay]
    % \node[fill=maincolor, minimum width=\paperwidth, minimum height=10cm, text=white] at (current page.north) {
    %     \begin{minipage}{0.9\paperwidth}
    %         \centering
    %         \vspace{1cm}
    %         \scalebox{1.5}{\Huge \textbf{\textsf{\@title}}} \\[0.5cm]  % Larger, bold, sans-serif title
    %         \Large \textit{\@author} \\
    %         \Large \textit{\@institute}
    %         \vspace{1cm}
    %     \end{minipage}
    % };
% \end{tikzpicture}
\maketitle

% Start of poster content
\begin{columns}

% Column 1
\column{0.60}
\block{Introduction}{
	Operation 
}

\block{Research Questions}{
  \begin{enumerate}[label=\arabic*.]
    \item How to implement a scheduling system that can coordinate in real-time
	\item 
    \item Can you coordinate metaheuristics based on different mathematical models in real-time
  \end{enumerate}
}

\block{Solution Method}{
	\usetikzlibrary {positioning}


\definecolor{red}{HTML}{8A3F3A}
\definecolor{yellow}{HTML}{E0BB3C}
\definecolor{blue}{HTML}{4569E0}
\definecolor{green}{HTML}{17E561}
\definecolor{other}{HTML}{6A939E}

\newcommand{\ModelColor}{red}
\newcommand{\UserInterfaceColor}{yellow}
\newcommand{\PersistenceColor}{blue}
\newcommand{\PointerSwapColor}{green}
\newcommand{\OrchestratorColor}{other}

\pgfkeys{
	/graph/.is family, /graph,
	default/.style = {
		show_shared_pointer = false,
		show_orchestrator = false,
		show_persistence = false,
		show_user_interface = false,
		basis/.estore in = 2cm,
	},
	show_shared_pointer/.estore in = \ShowSharedSolutionCommunication,
	show_orchestrator/.estore in = \ShowOrchestratorCommunication,
	show_persistence/.estore in = \ShowPersistenceCommunication,
	show_user_interface/.estore in = \ShowUserInterfaceCommunication,
	basis/.estore in = \basisinput,
}
\newlength{\basis}
\tikzset{
  basis/.code={\setlength{\basis}{\basisinput}}, % TikZ assignment code
  basis/.default=1cm,                   % Provide a default (\b@sis is undefined/unassigned)
  basis,                                % Set initial Value (\b@sis is defined/assigned)
 }

\newcommand{\drawGraph}[1]{
	\pgfkeys{/graph, default, #1}
	
	\begin{tikzpicture}[scale=0.75][
		% Define styles and settings
		node distance=2cm,
		block/.style={rectangle, draw, fill=blue!20, text centered, minimum height=3em},
		arrow/.style={-Stealth, thick}
		]


		\ifthenelse{\equal{\ShowOrchestratorCommunication}{true}}{
			\draw[color=other,-, ultra thick] (Strategic) -- (Orchestrator);
			\draw[color=other,-, ultra thick] (Tactical) -- (Orchestrator);
			\draw[color=other,-, ultra thick] (Supervisor) -- (Orchestrator);
			\draw[color=other,-, ultra thick] (Operational_1) -- (Orchestrator);
			\draw[color=other,-, ultra thick] (Operational_2) -- (Orchestrator);
			\draw[color=other,-, ultra thick] (Operational_3) -- (Orchestrator);
		}{}
		% \draw[help lines] (0\basis, 0\basis) grid (10\basis, 8\basis);
		\draw (5\basis,4\basis) node[minimum height=5cm,minimum width=7.0cm,rounded corners=5pt] {};

	    \draw (2.5\basis,5.5\basis) node[minimum height=1cm,minimum width=1cm,fill=\ModelColor,rounded corners=5pt] (Strategic) {Stra};
	    \draw (5.0\basis,4.0\basis) node[minimum height=1cm,minimum width=1\basis,fill=\ModelColor,rounded corners=5pt] (Supervisor) {Sup};
		\draw (7.5\basis,5.5\basis) node[minimum height=1cm,minimum width=1cm,fill=\ModelColor,rounded corners=5pt] (Tactical) {Tac};

		\draw (2.5\basis,2.5\basis) node[minimum height=1cm,minimum width=1cm,fill=\ModelColor,rounded corners=5pt] (Operational_1) {$O_{1}$};
		\draw (5.0\basis,2.5\basis) node[minimum height=1cm,minimum width=1cm,fill=\ModelColor,rounded corners=5pt] (Operational_2) {$O_{2}$};
		\draw (7.5\basis,2.5\basis) node[minimum height=1cm,minimum width=1cm,fill=\ModelColor,rounded corners=5pt,rounded corners=5pt] (Operational_3) {$O_{3}$};
	
		\draw (Strategic) edge (Tactical);
		\draw (Strategic) edge (Tactical);
		\draw (5\basis,5.5\basis) edge (Supervisor);
		\draw (Supervisor) edge (Operational_1);
		\draw (Supervisor) edge (Operational_2);
		\draw (Supervisor) edge (Operational_3);
		\draw (5.0\basis,0.5\basis)   node[minimum height=1cm,minimum width=5.0cm,            fill=\PersistenceColor,rounded corners=5pt] {SchedulingEnvironment};
		\draw (5.0\basis,7.5\basis)   node[minimum height=1cm,minimum width=5.0cm,            fill=\OrchestratorColor,rounded corners=5pt] (Orchestrator) {Orchestrator};
		\draw (0.5\basis,4.0\basis)   node[rotate=90, minimum height=1cm, minimum width=3.25cm,fill=\PointerSwapColor,rounded corners=5pt] {SharedSolution};
		\draw (9.5\basis,5.5\basis) node[rotate=90, minimum height=1cm, minimum width=1cm,fill=\UserInterfaceColor,rounded corners=5pt] {UI};
		\draw (9.5\basis,4.0\basis)   node[rotate=90, minimum height=1cm, minimum width=1cm,fill=\UserInterfaceColor,rounded corners=5pt] {UI};
		\draw (9.5\basis,2.5\basis) node[rotate=90, minimum height=1cm, minimum width=1cm,fill=\UserInterfaceColor,rounded corners=5pt] {UI};

		% Legend
		\begin{scope}[shift={(10.6\basis,5.7\basis)}]
			\node at (-0.25\basis,1\basis) [right] {Communication Type};
			\draw[color=\OrchestratorColor,fill] (0\basis,0.0\basis) rectangle (0.5cm, 0.5cm);
			\node[anchor=west] at (0.5\basis, 0.25\basis) {\scriptsize Channels};
			\draw[color=\PointerSwapColor,fill] (0\basis,-1.0\basis) rectangle(0.5cm, -0.5cm); 
			\node[anchor=west] at (0.5\basis, -0.75\basis) {\scriptsize Atomic Pointer Swap};
			\draw[color=\ModelColor,fill] (0\basis,-2.0\basis) rectangle(0.5cm, -1.5cm); 
			\node[anchor=west] at (0.5\basis, -1.75\basis) {\scriptsize Metaheurics};
			\draw[color=\PersistenceColor,fill] (0\basis,-3.0\basis) rectangle(0.5cm, -2.5cm); 
			\node[anchor=west] at (0.5\basis, -2.75\basis) {\scriptsize Mutex lock};
			\draw[color=\UserInterfaceColor,fill] (0\basis,-4.0\basis) rectangle(0.5cm, -3.5cm); 
			\node[anchor=west] at (0.5\basis, -3.75\basis) {\scriptsize Channels};
		\end{scope}
		\ifthenelse{\equal{\ShowSharedSolutionCommunication}{true}}{
			\draw[->, thick] (Strategic) -- (Orchestrator);
		}{}
		\ifthenelse{\equal{\ShowUserInterfaceCommunication}{true}}{
			\draw[->, thick] (Strategic) -- (Orchestrator);
		}{}
		\ifthenelse{\equal{\ShowPersistenceCommunication}{true}}{
			\draw[->, thick] (Strategic) -- (Orchestrator);
		}{}
		

	\end{tikzpicture}
}


	\drawGraph{basisinput=4cm}
}

\block{Case Study: Total Energies}{
	The proposed system has been developed in collaboration with Total Energies to help optimize there
	maintenace scheduling operations.
}
\block{Results}{
  \begin{itemize}
    \item \textbf{Improved Efficiency}: Achieved a 15\% reduction in total costs across the supply chain.
    \item \textbf{Stakeholder Satisfaction}: Increased satisfaction scores among all actors by 20\%.
    \item \textbf{Collaborative Strategies}: Developed joint policies that benefit all parties.
  \end{itemize}
}

\block{Future Work}{
  \begin{itemize}
    \item Extend the approach to international supply chains.
    \item Incorporate real-time data analytics for dynamic decision-making.
    \item Explore applications in other sectors like healthcare and transportation.
  \end{itemize}
}

\block{Conclusion}{

}
% Column 2
\column{0.40}
\block{Methodology}{
  \section{Strategic}
	\scalebox{0.5}{
		\begin{minipage}{0.68\textwidth}
			\section{The Strategic Model}


The Strategic Model have multiple different purposes.
\begin{itemize}
	\item Schedule Work Order out across the weekly periods
	\item Prioritize all the different released work orders
	\item Respect the available weekly hours available for each trait
\end{itemize}

The Strategic model is responsible for grouping work orders into weekly or biweekly periods depending on which kind of maintenance setup that one is running.
This kind of model closely resembles a variant of the multi-compartment multi-knapsack problem. 

\begin{alignat}{2}
	\text{Min} &\sum_{w \in W} \sum_{p \in P} v_{wp}(t) \cdot x_{wp}(t)                                                                                             \\ 
	+ & \sum_{p \in P} \sum_{\tau = 1}^T d \cdot pen_{p\tau}(t)                                                                                                     \\
	+ & \sum_{p \in P} \sum_{w1 \in W} \sum_{w2 \in W} clu_{w1, w2} \cdot x_{w1p} \cdot x_{w2p}                              \label{eqn:objective_function_strategic} \\
    & \text{subject to:} \notag                                                                                                                                       \\
	& \sum_{w \in W} c_{w\tau} \cdot x_{wp}(t) \leq \ cap_{p\tau}(t) + pen_{p\tau}(t) \notag                                                                          \\ 
	& \forall p \in P, \forall \tau \in T                                                                                    \label{eqn:capacity_constraint}          \\
	& \sum_{w \in W} x_{wp}(t) = 1                                              , \quad \forall p \in P                      \label{eqn:single_workorder_constraint}  \\
	& x_{wp}(t) = 0                                                             , \quad \forall (w, p) \in E(t)              \label{eqn:exclusion_constraint}         \\
	& x_{wp}(t) = 1                                                             , \quad \forall (w, p) \in I(t)              \label{eqn:inclusion_constraint}         \\
	& x_{wp}(t) \in \{0, 1\}                                                    , \quad \forall w \in W, \forall p \in P     \label{eqn:x_integrality_constraint}     \\ 
	& pen_{p\tau}(t) \in \mathbb{R}^{+}                                         , \quad \forall p \in P, \forall \tau \in T  \label{eqn:p_non_negativity_constraint}
    & t \in  [0, \infty] 
\end{alignat}

		\end{minipage}
	}
  \section{Tactical}
	\scalebox{0.5}{
		\begin{minipage}{0.68\textwidth}
			\section{The Tactical Model}
\begin{itemize}
	\item Respect precedence constraints
	\item Respect daily resource requirements for each trait
	\item Penalize exceeded daily capacity
\end{itemize}

After the strategic model has optimized its schedule the tactical agent will continue scheduling the output at a more detailed level. This means that now the tactical agent will schedule 
out on each of the days of the work orders scheduled by the strategic agent. 

The tactical model is responsible for providing an initial suggestion for a weekly schedule, below we see the model for the tactical agent.
\begin{alignat}{2}
\text{Min}     & \sum_{o \in O} \sum_{d \in D} v_{do}(t) \cdot y_{do}(t)                                                      \\  
	         + & \sum_{c \in C} \sum_{d \in D} pen \cdot p_{cd}(t)                                               \\  
			                                                                                                  \\
               &\text{subject to:}                                                          \notag                                                                   \\
	           & \sum_{o \in O} w_{co} \cdot y_{do}(t)  \leq R_{dc} + p_{dc}(t)                                   \\ 
			   & \quad \quad \forall  d \in D, \forall c \in C                              \notag                                    \\ 
	           & \sum_{d \in D} d \cdot y_{do1}(t) + \delta_o  = \sum_{d \in D} d \cdot y_{do2}(t)                    \\ 
			   & \quad \quad \forall (o1, o2) \in \text{finish-start}(t)                        \notag                                   \\ 
	           & \sum_{d \in D} d \cdot y_{do1}(t) = \sum_{d \in D} d \cdot y_{do2}(t)                                \\ 
			   & \quad \quad \forall (o1, o2) \in \text{start-start}(t)                          \notag                             \\ 
			   & y_{do}(t) \leq number_o(t) * operating\_time_o                                                     \\ 
			   & \quad \quad ,\forall d \in D, \forall o \in O                              \notag                                    \\
			   & y_{do}(t) \in \{0, 1\} \quad ,\forall d \in D, o \in O                                          \\
			   & p_{cd}(t) \in \mathbb{R} \quad ,\forall c \in C, d \in D                                        \\
			   & \delta_o \in \{ duration\_lower_o(t),                                                           \\ 
			   & \quad \quad duration\_upper_o \}(t) \quad, \forall o \in O \notag                                      \\
			   & t \in  [0, \infty] 
\end{alignat}


		\end{minipage}
	}
  \section{Supervisor}
	\scalebox{0.5}{
		\begin{minipage}{0.68\textwidth}
			\newif\ifincludenormal\

\pgfkeys{
	/supervisormodel/.is family, /supervisormodel,
	default/.style = {
		normal=true,
	},
	normal/.is if=includenormal,
}
\newcommand{\supervisorModel}[1][]{
	\pgfkeys{/supervisormodel, default, #1}
	\begin{alignat}{2}
		& \text{\rule{\linewidth}{0.8pt}} \notag \label{}                                                                                                                                                                                                                                                                                                                                                                     \\ 
		& \textbf{Meta variables:} \notag\\
		& \ElementSupervisor \in \SetSupervisor \\
		& \VarStrategicWorkOrderAssignment{}{} \\
		& \VarIncludeActivity{} \\
		& \tau \in [0, \infty] \\
		& \text{\rule{\linewidth}{0.4pt}} \notag\\
		& \textbf{Maximize:} \notag\\
		& \sum_{\ElementActivity \in \SetActivity{\VarStrategicWorkOrderAssignment{}{}}{}} \sum_{\ElementTechnician \in \SetTechnician} \ParSupervisorValue \cdot \VarSupervisorAssignment{\ElementActivity}{\ElementTechnician} \\ 
		& \text{\rule{\linewidth}{0.4pt}} \notag\\
		& \textbf{Subject to:} \notag\\ 
		& \sum_{\ElementActivity \in \SetActivity{\VarStrategicWorkOrderAssignment{}{}}{\ElementOperation}} \VarActivityWork{\ElementActivity} = \ParOperationWork{\ElementOperation}    \quad \forall \ElementOperation \in \SetOperation{}{\VarStrategicWorkOrderAssignment{}{}}\\
		& \sum_{\ElementTechnician \in \SetTechnician} \sum_{\ElementActivity \in \SetActivity{\VarStrategicWorkOrderAssignment{}{}}{\ElementOperation}}\VarSupervisorAssignment{\ElementActivity}{\ElementTechnician} = \VarSupervisorOperationWhole \cdot \ParNumberOfPeople  \quad \forall \ElementOperation \in \SetOperation{}{\VarStrategicWorkOrderAssignment{}{}}  \\
		& \sum_{\ElementOperation \in \SetOperation{\ElementWorkOrder}{\VarStrategicWorkOrderAssignment{}{}}} \VarSupervisorOperationWhole = |\SetOperation{\ElementWorkOrder}{\VarStrategicWorkOrderAssignment{}{}}| \cdot \VarSupervisorWorkOrderWhole  \quad \forall \ElementWorkOrder \in \SetWorkOrder{,\VarStrategicWorkOrderAssignment{}{}} \\
		& \sum_{\ElementActivity \in \SetActivity{\VarStrategicWorkOrderAssignment{}{}}{\ElementOperation}} \VarSupervisorAssignment{\ElementActivity}{\ElementTechnician} \leq 1  \quad \forall \ElementOperation \in \SetOperation{}{\VarStrategicWorkOrderAssignment{}{}} \quad \forall \ElementTechnician \in \SetTechnician \\  
		& \VarSupervisorAssignment{\ElementActivity}{\ElementTechnician} \leq \ParFeasible  \quad \forall \ElementActivity \in \SetActivity{\VarTacticalWork}{\ElementOperation} \quad \forall \ElementOperation \in \SetOperation{}{\VarStrategicWorkOrderAssignment{}{}} \quad \forall \ElementTechnician \in \SetTechnician \\
		& \VarSupervisorAssignment{\ElementActivity}{\ElementTechnician} \in \{0, 1\}  \quad \forall \ElementOperation \in \SetOperation{}{\VarStrategicWorkOrderAssignment{}{}} \quad \forall \ElementTechnician \in \SetTechnician \\ 
		& \VarSupervisorOperationWhole \in \{0, 1\}  \quad \forall \ElementOperation \in \SetOperation{}{\VarStrategicWorkOrderAssignment{}{}} \\ 
		& \VarSupervisorWorkOrderWhole \in \{0, 1\}  \quad \forall \ElementWorkOrder \in \SetWorkOrder{,\VarStrategicWorkOrderAssignment{}{}} \\ 
		& \VarActivityWork{\ElementActivity} \in [\ParLowerActivityWork, \ParOperationWork{\ElementActivity}]  \quad \forall \ElementActivity \in \SetActivity{\VarStrategicWorkOrderAssignment{}{}}{}\\ 
		& \text{\rule{\linewidth}{0.8pt}} \notag \label{}                                                                                                                                                                                                                                                                                                                                                                      
	\end{alignat}
}

		\end{minipage}
	}
  \section{Operational}
	\scalebox{0.5}{
		\begin{minipage}{0.68\textwidth}
			\section{The Operational Model}

Here the o is a single operation and o2 is another operation. It is crucial to understand here that the main
decision variable, $x$ defines an ordering of the operations that a single operational agent will do the 
operations in. 

The $\VarStartOfSegment{a}{k}$ is the start time of job $i$ in segment $k$ and $\VarFinishOfSegment{a}{k}$ is the finish time of job $i$ in segment $k$.
$\VarProcessingTime{a}{k}$ is the processing time of each segment. 
\begin{alignat}{2}
	& Max \sum_{a \in \SetActivity{\VarSupervisorAssignment}} \sum_{k \in \SetWorkSegment} \VarProcessingTime                                                         \\
	& \text{Subject to:} \notag                                                                                                                                       \\
    & \sum_{k \in \SetWorkSegment} \VarProcessingTime \cdot \VarActiveSegment{a}{k} = \ParOperationWork \cdot \VarIncludeActivity \quad \forall a \in \SetActivity{\VarSupervisorAssignment} \\
	& \VarStartOfSegment{a2}{1} \geq \VarFinishOfSegment{a1}{last(a1)} + \ParPreparation \notag                                                                       \\ 
	& \quad \forall a1 \in \SetActivity{\VarSupervisorAssignment}, a2 \in \SetActivity{\VarSupervisorAssignment}                                                     \\
	& \VarStartOfSegment{a}{k} \geq \VarFinishOfSegment{a}{k-1} - \ParConstraintLimit \cdot (2 - \VarActiveSegment{a}{k} + \VarActiveSegment{a}{k-1})                \notag\\
	& \quad \forall a \in \SetActivity{\VarSupervisorAssignment}\forall k \in \SetWorkSegment \\ 
	& \VarProcessingTime = \VarFinishOfSegment{a}{k} - \VarStartOfSegment{a}{k}                                                                                       \\
	& \quad \forall a \in \SetActivity{\VarSupervisorAssignment}, k \in \SetWorkSegment \notag                                                                        \\
	& \VarStartOfSegment{a}{k} \geq \ParEvent + \ParEventDuration - \ParConstraintLimit \cdot (1 - \VarSegmentInRelation)                                             \notag\\ 
	& \quad \forall a \in \SetActivity{\VarSupervisorAssignment}, k \in \SetWorkSegment, i \in \SetTimeInstance, e \in \SetEvent                                \\
	& \VarFinishOfSegment{a}{k} \leq \ParEvent + \ParConstraintLimit \cdot \VarSegmentInRelation                                                                      \notag\\ 
	& \quad \forall a \in \SetActivity{\VarSupervisorAssignment}, k \in \SetWorkSegment, i \in \SetTimeInstance, e \in \SetEvent                                \\
	& \VarStartOfSegment{a}{1} \geq \ParTimeWindowStart \forall a \in \SetActivity{\VarSupervisorAssignment}                                                          \\
	& \VarFinishOfSegment{a}{last(a)} \leq \ParTimeWindowFinish \forall a \in \SetActivity{\VarSupervisorAssignment}                                                  \\
	& \VarActiveSegment{a}{k} \in \{0, 1\} \quad \forall a \in \SetActivity{\VarSupervisorAssignment}, k \in \SetWorkSegment                                          \\
	& \VarStartOfSegment{a}{k} \in [\ParAvailabilityStart, \ParAvailabilityFinish] \notag\\
	& \quad \forall a \in \SetActivity{\VarSupervisorAssignment}, k \in \SetWorkSegment  \\
	& \VarFinishOfSegment{a}{k} \in [\ParAvailabilityStart, \ParAvailabilityFinish] \notag\\
	& \quad \forall a \in \SetActivity{\VarSupervisorAssignment}, k \in \SetWorkSegment \\
	& \VarProcessingTime \in [0, \ParOperationWork] \quad \forall a \in \SetActivity{\VarSupervisorAssignment}, k \in \SetWorkSegment                                               \\
	& \VarSegmentInRelation \in \{0, 1\}                                                                                                                              \\ 
	& \quad \forall a \in \SetActivity{\VarSupervisorAssignment}, k \in \SetWorkSegment, i \in \SetTimeInstance, e \in \SetEvent \notag                               \\
	& \theta_i \in \{0, 1\} \quad \forall a \in \SetActivity{\VarSupervisorAssignment}                                                                                \\
\end{alignat}

		\end{minipage}
	}

}

% Column 3
\column{0.50}

% % Contact Information
% \block{Contact Information}{
%   \begin{itemize}
%     \item \textbf{Email}: \href{mailto:your.email@institution.edu}{your.email@institution.edu}
%     \item \textbf{Phone}: +1 (234) 567-8901
%     \item \textbf{Website}: \href{http://www.yourwebsite.com}{www.yourwebsite.com}
%   \end{itemize}
% }

\end{columns}

\end{document}


% POSTER TEMPLATE AND GUIDANCE

% This document provides a guidance and set of minimum requirements for the research poster of the type made for the DTU Offshore Technology Conference. The aim is to improve the communication of the research to as wide an audience as possible.
% Size
% A0 format (841 x 1189 mm)
% Challenge
% In simple and concise terms state the following:
% •	What is the problem?
% •	Who is it a problem for?
% •	How big is the problem?

% Solution 
% •	How have the problem been solved? (if it is solved)
% •	What is the compelling unique solution?
% •	How will it be implemented?
% •	How is the problem going to be solved? (research hypothesis if not solved yet)

% Alternatives
% Why is this solution better than others?

% Connections and Context
% How does this research link to other projects?
% Is it based on a previous piece of work?
% How does this piece of work fit into the prototype as a whole?

% Other information can be useful to help the audience visualize the project. A time line of what has happened to bring you to this stage and what the proposed timeline is until completion. Again here the level of detail should be low and just show the major milestones.
% Of course, an acknowledgement of who the team is and what programme it is a part of is useful for some to put a face on the research.


