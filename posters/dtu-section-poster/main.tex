\documentclass[portrait, a0paper]{tikzposter}

% Packages
\usepackage{amsmath}
\usepackage{graphicx}
\usepackage{multicol}
\usepackage{enumitem}

% Title, Author, Institute
\title{A Multi-Actor Approach to Operations Research}
\author{Your Name}
\institute{Your Institution}

% Define custom colors (optional)
\definecolor{maincolor}{RGB}{0, 102, 204} % A shade of blue
\definecolor{secondarycolor}{RGB}{255, 102, 0} % A shade of orange

% Theme and color style
\usetheme{Basic}
\usecolorstyle{Default}

\begin{document}

% Title block
\maketitle

% Start of poster content
\begin{columns}

% Column 1
\column{0.33}
\block{Introduction}{
  Operations research (OR) traditionally focuses on optimizing processes within a single organization. However, many real-world problems involve multiple actors with diverse objectives and constraints. This poster explores a multi-actor approach to OR, emphasizing collaboration and conflict resolution among stakeholders.
}

\block{Objectives}{
  \begin{enumerate}[label=\arabic*.]
    \item Integrate multiple stakeholder perspectives into OR models.
    \item Develop methods to handle conflicting objectives.
    \item Propose collaborative optimization strategies.
  \end{enumerate}
}

% Column 2
\column{0.33}
\block{Methodology}{
  \begin{itemize}
    \item \textbf{Game Theory}: Analyze strategic interactions between actors.
    \item \textbf{Multi-Criteria Decision Making}: Evaluate alternatives based on multiple criteria.
    \item \textbf{Agent-Based Modeling}: Simulate the actions and interactions of autonomous agents.
    \item \textbf{Negotiation Models}: Facilitate agreement among parties with conflicting interests.
  \end{itemize}
}

\block{Solution Method}{
 asd 
}

\block{Case Study}{
  \textbf{Supply Chain Management} \\
  A complex supply chain involving suppliers, manufacturers, distributors, and retailers. Each actor aims to optimize its own performance metrics, which may conflict with others. The multi-actor approach seeks a globally optimal solution that considers the objectives of all stakeholders.
}

% Column 3
\column{0.33}
\block{Results}{
  \begin{itemize}
    \item \textbf{Improved Efficiency}: Achieved a 15\% reduction in total costs across the supply chain.
    \item \textbf{Stakeholder Satisfaction}: Increased satisfaction scores among all actors by 20\%.
    \item \textbf{Collaborative Strategies}: Developed joint policies that benefit all parties.
  \end{itemize}
}

\block{Future Work}{
  \begin{itemize}
    \item Incorporate real-time data analytics for dynamic decision-making.
    \item Explore applications in other sectors like healthcare and transportation.
  \end{itemize}
}

% % Contact Information
% \block{Contact Information}{
%   \begin{itemize}
%     \item \textbf{Email}: \href{mailto:your.email@institution.edu}{your.email@institution.edu}
%     \item \textbf{Phone}: +1 (234) 567-8901
%     \item \textbf{Website}: \href{http://www.yourwebsite.com}{www.yourwebsite.com}
%   \end{itemize}
% }

\end{columns}

\end{document}


% POSTER TEMPLATE AND GUIDANCE

% This document provides a guidance and set of minimum requirements for the research poster of the type made for the DTU Offshore Technology Conference. The aim is to improve the communication of the research to as wide an audience as possible.
% Size
% A0 format (841 x 1189 mm)
% Challenge
% In simple and concise terms state the following:
% •	What is the problem?
% •	Who is it a problem for?
% •	How big is the problem?

% Solution 
% •	How have the problem been solved? (if it is solved)
% •	What is the compelling unique solution?
% •	How will it be implemented?
% •	How is the problem going to be solved? (research hypothesis if not solved yet)

% Alternatives
% Why is this solution better than others?

% Connections and Context
% How does this research link to other projects?
% Is it based on a previous piece of work?
% How does this piece of work fit into the prototype as a whole?

% Other information can be useful to help the audience visualize the project. A time line of what has happened to bring you to this stage and what the proposed timeline is until completion. Again here the level of detail should be low and just show the major milestones.
% Of course, an acknowledgement of who the team is and what programme it is a part of is useful for some to put a face on the research.


