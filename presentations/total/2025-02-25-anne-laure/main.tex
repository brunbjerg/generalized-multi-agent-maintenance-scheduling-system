\documentclass{article}


\usepackage{standalone}
\usepackage{pgf}
\usepackage{gnuplottex}
\usepackage{amssymb}
\usepackage{amsmath}
\usepackage{algorithm}
\usepackage{algorithmicx}
\usepackage{algpseudocode}
\usepackage{pgfplots}
\usepackage{epstopdf}  
\usepackage[utf8]{inputenc}
\usepackage[T1]{fontenc}
\usepackage{amsmath}
\usepackage{makecell}
\usepackage{amssymb}
\usepackage{multicol}
\usepackage{natbib}
\usepackage{hyperref}
\usepackage{booktabs} % For professional quality tables
\usepackage{array}    % For better column alignment
\usepackage{tikz}
\usepackage[a4paper, margin=2cm]{geometry}
\usepackage{pgfplots}
\usepackage{epstopdf}  
\usepackage{float}
\usepackage{tikz}
\usepackage{pgf}
\usepackage{gnuplottex}
\usepackage{amsmath}
\usepackage{algorithmicx}
\usepackage{algorithm}
\usepackage{algpseudocode}
\usepackage{amsmath}
\usepackage{amssymb}
\usepackage{bibentry}
\usepackage{enumitem}
\usepackage{fancyhdr}
\usepackage{fontspec}
\usepackage{graphicx}
\usepackage{ifthen}
\usepackage{makecell}
\usepackage{multicol}
\usepackage{natbib}
\usepackage{parskip}
\usepackage{pifont}
\usepackage{pgfkeys}
\usepackage{tikz}
\usepackage{import}
\usepackage{lastpage}
\usepackage{booktabs}
\usepackage{nicematrix}
\usepackage{multicol}

% Sets
\newcommand{\SetWorkOrder}[1]{W(\VarMetaTime, \VarStrategicWorkOrderAssignment{}{})}
\newcommand{\SetPeriod}{P(\VarMetaTime)}
\newcommand{\SetResource}{R(\VarMetaTime)} %RECONSIDER | RELEVANT FOR BOTH STRATEGIC AND TACTICAL
\newcommand{\SetOperation}[2]{O_{#1}(\VarMetaTime, #2 )}

\newcommand{\SetDays}[1]{D_{#1}(\VarMetaTime)}
\newcommand{\SetActivity}[2]{A_{#2}(\VarMetaTime, #1)}
\newcommand{\SetTechnician}{T(\VarMetaTime)}
\newcommand{\SetWorkSegment}{K(\VarSupervisorAssignment{}{})}

\newcommand{\SetTimeInstance}{I(\VarMetaTime)}
\newcommand{\SetEvent}{E(\VarMetaTime)}

% Parameters
\newcommand{\ParStrategicValue}{strategic\_value_{wp}(\VarMetaTime)}
\newcommand{\ParStrategicPenalty}{strategic\_penalty}
\newcommand{\ParClusteringValue}{clustering\_value_{w1, w2}}
\newcommand{\ParStrategicResource}{resource_{pr}(\VarMetaTime)}

\newcommand{\ParStrategicWorkOrderWeight}{work\_order\_work_{wr}}
\newcommand{\ParStrategicInclude}{include(\VarMetaTime)}
\newcommand{\ParStrategicExclude}{exclude(\VarMetaTime)}
\newcommand{\ParTacticalValue}{tactical\_value_{do}(\VarMetaTime)}

\newcommand{\ParTacticalPenalty}{tactical\_penalty}
\newcommand{\ParOperationWork}[1]{work_{#1}(\VarMetaTime)}
\newcommand{\ParTacticalResource}{tactical\_resource_{dr}(\VarMetaTime)}
\newcommand{\ParStartStart}{start\_start_{o1, o2}}

\newcommand{\ParFinishStart}{finish\_start_{o1, o2}}
\newcommand{\ParNumberOfPeople}{number_{o}(\VarMetaTime)}
\newcommand{\ParOperatingTime}{operating\_time_{o}}
\newcommand{\ParDurationLower}{duration\_lower_{o}(\VarMetaTime)}

\newcommand{\ParDurationUpper}{duration\_upper_{o}(\VarMetaTime)}
\newcommand{\ParSupervisorValue}{supervisor\_value_{at}(\VarMetaTime)} % We should determine if the supervisor should assign for each operation or for each activity. My guts say that it should be for each activity
\newcommand{\ParFeasible}{feasible_{at}(\VarIncludeActivity{})}
\newcommand{\ParOperationsForWorkOrder}{work\_order\_to\_operations_{w}}

\newcommand{\ParOperationsInWorkOrder}{operations\_in\_work\_order_{w}}
\newcommand{\ParActivitiesForOperation}{activities\_for\_operation_{o}}
\newcommand{\ParLowerActivityWork}{lower\_activity\_work_{a}(\VarMetaTime)}
\newcommand{\ParActivityWork}[1]{activity\_work_{a}(\VarMetaTime, \VarActivityWork{#1})}
\newcommand{\ParPreparation}{preparation_{a1, a2}}

\newcommand{\ParEvent}{event_{ie}}
\newcommand{\ParEventDuration}{duration_{ie}}
\newcommand{\ParConstraintLimit}{constraint\_limit}
\newcommand{\ParTimeWindowStart}{time\_window\_start_{a}(\VarTacticalWork{}{})}

\newcommand{\ParTimeWindowFinish}{time\_window\_finish_{a}(\VarTacticalWork{}{})}
\newcommand{\ParAvailabilityStart}{availability\_start(\VarMetaTime)}
\newcommand{\ParAvailabilityFinish}{availability\_finish(\VarMetaTime)}

% Variables
\newcommand{\VarStrategicWorkOrderAssignment}[2]{\alpha_{#1#2}(\VarMetaTime)}
\newcommand{\VarStrategicExcess}{\epsilon_{pr}(\VarMetaTime)}
\newcommand{\VarTacticalWork}[2]{\beta_{#1#2}(\VarMetaTime)}
\newcommand{\VarTacticalExcess}{\mu_{rd}(\VarMetaTime)} : excess capacity for each day

\newcommand{\VarTacticalWorkBinary}[2]{\sigma_{#1#2}(\VarMetaTime)}
\newcommand{\VarTacticalWorkBinaryConsecutive}{\eta_{do}(\VarMetaTime)}
\newcommand{\VarTacticalOperationDifference}{\Delta_{o}(\VarMetaTime)}
\newcommand{\VarSupervisorAssignment}[2]{\gamma_{#1#2}(\VarMetaTime)}
\newcommand{\VarSupervisorAssignmentWhole}{\phi_{o}(\VarMetaTime)}

\newcommand{\VarActivityWork}[1]{\rho_{#1}(\VarMetaTime)}

\newcommand{\VarProcessingTime}{\delta_{ak}(\VarMetaTime)} 
\newcommand{\VarActiveSegment}[2]{\pi_{#1#2}(\VarMetaTime)}
\newcommand{\VarStartOfSegment}[2]{\lambda_{#1#2}(\VarMetaTime)}
\newcommand{\VarFinishOfSegment}[2]{\Lambda_{#1#2}(\VarMetaTime)}

\newcommand{\VarSegmentInRelation}{\omega_{akie}(\VarMetaTime)}
\newcommand{\VarIncludeActivity}[1]{\theta_{#1}(\VarMetaTime)}

% Meta variables
\newcommand{\VarMetaTime}{\tau}



\definecolor{red}{HTML}{8A3F3A}
\definecolor{yellow}{HTML}{E0BB3C}
\definecolor{blue}{HTML}{4569E0}
\definecolor{green}{HTML}{17E561}
\definecolor{other}{HTML}{6A939E}


\pagestyle{fancy}
\fancyhf{}
\lhead{Memo: Ordinator Status v0.2.5 \\ Date: 2025-02-25}
\rhead{Christian Brunbjerg Jespersen \\ cbrje@dtu.dk \& christian-brunbjerg.jespersen@external.totalenergies.com}

\rfoot{Recipients: \textbf{Anne-Laure Debar}}
\lfoot{page \thepage\ of\ \pageref{LastPage}}

\begin{document}

\tableofcontents

\newpage
\section{Project Overview}
\subsection{Stakeholders}

\begin{itemize}
    \item Brian Friis Nielsen (Total)
    \item Valentin Ispas      (GMC - Norspie)
    \item Anne-Laure Debar    (Total)
    \item Baptiste Dubillaud  (Total)
\end{itemize}

\subsection{Highlevel Work Flow}
\begin{figure}[H]
    \includegraphics[width=1.0\textwidth]{../../../project-plans/scipo-total/figures/scheduling-as-is-to-be.png}
    \caption{Schematic difference between the current way of doing things versus how it could be done in the
    future. Each stakeholder can immediately sees an optimized schedule based on the state in the optimization
    algorithms. This means that the moment that a \textbf{Scheduler}, \textbf{Supervisor}, or \textbf{Technician}
    sees that there is something wrong his part of the schedule it can be handled immediately. After an excel file
    has been sent in the as-is example the remaining stakeholders are working blind}
\end{figure}

\textbf{Issues:}
\begin{itemize}
    \item Manually sending email around: Manual process leads to long iteration time on schedules
    \item State duplication: Information on the same \textbf{work order} is found in many places at any one time
    \item SAP is not made for optimization workflows: You copy the current SAP state to excel and continue working from there.
    \item Ordinator is a layer between SAP and the \textbf{scheduler} and \textbf{supervisor}: Ordinator holds all state which can be manipulated and optimizes it (scheduling, optimizations etc.).
\end{itemize}

\subsection{Excel Examples}
\begin{figure}[H]
\includegraphics[width=1.0\textwidth]{../../../project-plans/scipo-total/figures/schedulers-view-into-the-scheduling-process.png}
\caption{Schedulers view of the scheduling process}
\end{figure}
\begin{figure}[H]
\includegraphics[width=1.0\textwidth]{../../../project-plans/scipo-total/figures/technician-excel.png}
\caption{Supervisors and Technicians view of the scheduling process}
\end{figure}
\section{Product Vision}

\documentclass{standalone}
\usepackage{tikz}
\usetikzlibrary {positioning}

<<<<<<< HEAD
\newcommand{\drawHexagon}[5][draw=black]{
	\draw[#1, fill=#4, ] (#2,#3) ++(30:2) -- ++(90:2) -- ++(150:2) -- ++(210:2) -- ++(270:2) -- ++(330:2) -- cycle;
	\node at (#2,#3+2) {#5};
||||||| 2d4327a
\newcommand{\drawHexagon}[4]{
	\draw[fill=#3] (#1,#2) ++(30:2) -- ++(90:2) -- ++(150:2) -- ++(210:2) -- ++(270:2) -- ++(330:2) -- cycle;
	\node at (#1,#2+2) {#4};
=======
\begin{document}
\newcommand{\drawHexagon}[4]{
	\draw[fill=#3] (#1,#2) ++(30:2) -- ++(90:2) -- ++(150:2) -- ++(210:2) -- ++(270:2) -- ++(330:2) -- cycle;
	\node at (#1,#2+2) {#4};
>>>>>>> 023797133fb426c1bb01f920d8f5635c343d11a6
}

\newif\ifuserinterfacelayer
\newif\ifpersistencelayer
\newif\ifmetaheuristicslayer
\newif\ifatomicpointerswaplayer

\pgfkeys{
	/hexagon/.is family, /hexagon,
	default/.style = {
		persistence=false,
		userinterface=false,
		metaheuristics=true,
	},
	persistence/.is if=persistencelayer,
	userinterface/.is if=userinterfacelayer,
	metaheuristics/.is if=metaheuristicslayer,
}
\newcommand{\drawModelSetupHexagon}[1][]{
	\pgfkeys{/hexagon, default, #1}

	\begin{tikzpicture}[scale=0.6, line width=1.05]
	
	\ifuserinterfacelayer
		\drawHexagon{ 2                      }{ 2}{dtu-yellow}{UI}
		\drawHexagon{{6 - 2 * (2 - sqrt(3)) }}{ 2}{dtu-yellow}{UI}
		\drawHexagon{{4 - 1 * (2 - sqrt(3)) }}{-1}{dtu-yellow}{UI}
		\drawHexagon{{0 + 1 * (2 - sqrt(3)) }}{-1}{dtu-yellow}{UI}
		\drawHexagon{{8 - 3 * (2 - sqrt(3)) }}{-1}{dtu-yellow}{UI}

		\drawHexagon{{2 - 0 * (2 - sqrt(3)) }}{-4}{dtu-yellow}{UI}
		\drawHexagon{{6 - 2 * (2 - sqrt(3)) }}{-4}{dtu-yellow}{UI}

		\drawHexagon{{10 - 4 * (2 - sqrt(3)) }}{-4}{dtu-yellow}{UI}
		\drawHexagon{{-2 + 2 * (2 - sqrt(3)) }}{-4}{dtu-yellow}{UI}

		\drawHexagon{{12 - 5 * (2 - sqrt(3)) }}{-1}{dtu-yellow}{UI}
		\drawHexagon{{-4 + 3 * (2 - sqrt(3)) }}{-1}{dtu-yellow}{UI}
	\fi

	\ifpersistencelayer
		\drawHexagon[draw=none]{ 2                      }{ 2}{dtu-blue}{}
		\drawHexagon[draw=none]{{6 - 2 * (2 - sqrt(3)) }}{ 2}{dtu-blue}{}
		\drawHexagon[draw=none]{{4 - 1 * (2 - sqrt(3)) }}{-1}{dtu-blue}{Database}
		\drawHexagon[draw=none]{{0 + 1 * (2 - sqrt(3)) }}{-1}{dtu-blue}{}
		\drawHexagon[draw=none]{{8 - 3 * (2 - sqrt(3)) }}{-1}{dtu-blue}{}

		\drawHexagon[draw=none]{{2 - 0 * (2 - sqrt(3)) }}{-4}{dtu-blue}{}
		\drawHexagon[draw=none]{{6 - 2 * (2 - sqrt(3)) }}{-4}{dtu-blue}{}

		\drawHexagon[draw=none]{{10 - 4 * (2 - sqrt(3)) }}{-4}{dtu-blue}{}
		\drawHexagon[draw=none]{{-2 + 2 * (2 - sqrt(3)) }}{-4}{dtu-blue}{}

		\drawHexagon[draw=none]{{12 - 5 * (2 - sqrt(3)) }}{-1}{dtu-blue}{}
		\drawHexagon[draw=none]{{-4 + 3 * (2 - sqrt(3)) }}{-1}{dtu-blue}{}
	\fi

	\ifmetaheuristicslayer
		\drawHexagon{ 2                      }{ 2}{dtu-blue}{Strategic}
		\drawHexagon{{6 - 2 * (2 - sqrt(3)) }}{ 2}{dtu-green}{Tactical}
		\drawHexagon{{4 - 1 * (2 - sqrt(3)) }}{-1}{dtu-red}{Supervisor}
		\drawHexagon{{0 + 1 * (2 - sqrt(3)) }}{-1}{dtu-red}{Supervisor}
		\drawHexagon{{8 - 3 * (2 - sqrt(3)) }}{-1}{dtu-red}{Supervisor}

		\drawHexagon{{2 - 0 * (2 - sqrt(3)) }}{-4}{dtu-corporate-red}{Technician}
		\drawHexagon{{6 - 2 * (2 - sqrt(3)) }}{-4}{dtu-corporate-red}{Technician}

		\drawHexagon{{10 - 4 * (2 - sqrt(3)) }}{-4}{dtu-corporate-red}{Technician}
		\drawHexagon{{-2 + 2 * (2 - sqrt(3)) }}{-4}{dtu-corporate-red}{Technician}

		\drawHexagon{{12 - 5 * (2 - sqrt(3)) }}{-1}{dtu-corporate-red}{Technician}
		\drawHexagon{{-4 + 3 * (2 - sqrt(3)) }}{-1}{dtu-corporate-red}{Technician}
	\fi

	
	\end{tikzpicture}
}

\ifstandalone
	

\definecolor{red}{HTML}{8A3F3A}
\definecolor{yellow}{HTML}{E0BB3C}
\definecolor{blue}{HTML}{4569E0}
\definecolor{green}{HTML}{17E561}
\definecolor{other}{HTML}{6A939E}

	\drawModelSetupHexagon
\fi

\end{document}

\begin{figure}[H]
    \centering
    \drawModelSetupHexagon[simplified=true]
    \caption{Each model represents a distinct stakeholder with its own UI (yellow) and their solution spaces are tied together with
        lock-free atomic pointer swaps (purple)}
\end{figure}

\section{Product Architecture}

\begin{figure}[H]
    \includegraphics[width=\textwidth]{../../../figures/architecture/ordinator-architecture.png}
    \caption{Software architecture of the Ordinator application. Databases are shown in
    \textbf{grey}, with either SAP or Mongodb based on the data requirements, SchedulingEnvironment
    holds all the needed data for the application to run and saves all the user interactions with
    the system, these interaction are held in the MongoDB database, the Orchestrator is
    shown in \textbf{blue} and is responsible for managing the whole system while it is running,
    the Shared Solution is
    shown in \textbf{yellow} and holds fast mutable state that is needed for the algorithms to optimize
    across the different stakeholders, each optimization algorithms themselves is shown in \textbf{light red},
    where each models and optimizes the scheduling process for that specific stakeholder, finally
    the presentation views for each of the stakeholders which are shown in \textbf{green} }
\end{figure}

\section{Project going forward}

\subsection{Collaboration}

\textbf{Main issues:}
\begin{itemize}
    \item I will need time in Esbjerg to further develop the project
    \begin{itemize}
        \item I have family in Esbjerg that I might be able to stay with
    \end{itemize}
    \item I lack the project management skill to run the whole project by myself
    \item There is one year left of the project, so the collaboration will have to depend on project scope
    \item The project will require developer time from \textbf{Loic}
    \item The project will require developer time from GMC and the maintenance department
\end{itemize}

\textbf{Main questions:}
\begin{itemize}
    \item What do you think that it would take to convince Total to further the project? 
    \item The project could save the company a lot of money. Which steps should be
          taken to gain the best kind of support?
\end{itemize}

\subsection{Milestones}
\begin{enumerate}
    \item Make the first iteration of Ordinator that can schedule for each kind of stakeholder
    \item Providing each stakeholder with high quality excel exports 
    \item Host the program with both a \textbf{Scheduler} and \textbf{Supervisor} frontend
    \item Create frontends together with Total
\end{enumerate}

\subsection{Company: Scipo}
Would a company setup be an option if the project gains financial support. This may be a
long short. I have a good and competent friend and I think that the two of us together
would be able to deliver you a product of high quality. This kind of collaboration would
go along the lines:
\begin{enumerate}
    \item Develop Ordinator so mature that Total gains confidence in its likely success
    \item Drafting a budget, specifying how many hours there is needed to deliver certain milestones.
    \item Shared ownership of the product or open-source the code.
\end{enumerate}

\subsection{Example Budget}
This is simply an example budget for how a development model could look like between \textbf{Total} and \textbf{Scipo} (an example company created for solving the scheduling problem for Total)

\begin{table}[H]
\centering
\begin{tabular}{lllllll}
    \toprule
    \textbf{Scipo}                                                          & Role                            & \makecell[lt]{Total\\Hours} & \makecell[lt]{Cost per\\Hour}        & Skills                                                                  & Period                             &  \\
    \midrule
    Christian Jespersen                                                     & \makecell[lt]{Core\\Developer}  & 320                         & 250 DKK                              & \makecell[lt]{Optimization\\Algorithms}                                 & \makecell[lt]{2025-09 to\\2025-12} &  \\
    Sebastian Dall                                                          & \makecell[lt]{Core\\Developer}  & 320                         & 250 DKK                              & \makecell[lt]{API,\\Frontend,\\Project}                                 & \makecell[lt]{2025-09 to\\2025-12} &  \\
    \toprule
    \textbf{Total}                                                          & Role                            & \makecell[lt]{Total\\Hours} & \makecell[lt]{Cost per\\Hour}        & Skills                                                                  & Period                             &  \\
    \midrule
    \makecell[tl]{TOTAL\_DEVELOPER\\Baptiste Dubillaud}                     & Integration                     & 50-70                       & 500 DKK                              & \makecell[lt]{Azure,\\IT infrastructure}                                & \makecell[lt]{2025-09 to\\2025-12} &  \\
    \makecell[tl]{TOTAL\_MAINTENANCE\_METHOD\\Brian Friis Niels}            & Domain Expert                   & 40                          & 500 DKK                              & \makecell[lt]{Understanding of\\Business Flows}                            & \makecell[lt]{2025-09 to\\2025-12} &  \\
    \makecell[tl]{GMC\_SCHEDULER\\Valentin Ispas}                           & Domain Expert                   & 30                          & 500 DKK                              & \makecell[lt]{Key Stakeholder}                                          & \makecell[lt]{2025-09 to\\2025-12} &  \\
    \makecell[tl]{GMC\_SUPERVISOR\\<UNKNOWN>}                               & Domain Expert                   & 20                          & 500 DKK                              & \makecell[lt]{Key Stakeholder}                                          & \makecell[lt]{2025-09 to\\2025-12} &  \\
    \makecell[tl]{GMC\_TECHNICIAN\\<UNKNOWN>}                               & Domain Expert                   & 20                          & 500 DKK                              & \makecell[lt]{Key Stakeholder}                                          & \makecell[lt]{2025-09 to\\2025-12} &  \\
    \toprule
    \textbf{Material}                                                       & Role                            & \makecell[lt]{Total\\Hours} & \makecell[lt]{Cost per\\Hour}        & Skills                                                                  & Period                             &  \\
    \midrule
    Server                                                                  & -                               & ~2000                       & ~1-5 DKK                             & -                                                                       & \makecell[lt]{2025-09 to\\2025-12} &  \\
    \bottomrule
\end{tabular}

\end{table}

\[
\begin{aligned}
    \text{Total Cost} &= (320 \times 250) + (320 \times 250) \\
    &\quad + (70 \times 500) + (40 \times 500) + (30 \times 500) \\
    &\quad + (20 \times 500) + (20 \times 500) + (2,000 \times 5) \\
    &= 80,000 + 80,000 + 35,000 + 20,000 + 15,000 + 10,000 + 10,000 + 10,000  \\
    &= 160,000 \quad (\text {Direct cost for Scipo}) \\
    &\quad + 100,000 \quad (\text {Indirect cost for Total}) \\
    &= 260,000 \quad \text{Total cost DKK}
\end{aligned}
\]

Total cost per 2 month period \textbf{260,000 DKK}


\newpage
\section{Appendix}

The mathematical model formulations are based on interviews and
the handbook \citep{palmerMaintenancePlanningScheduling2019}.

\section{The Strategic Model}


The Strategic Model have multiple different purposes.
\begin{itemize}
	\item Schedule Work Order out across the weekly periods
	\item Prioritize all the different released work orders
	\item Respect the available weekly hours available for each trait
\end{itemize}

The Strategic model is responsible for grouping work orders into weekly or biweekly periods depending on which kind of maintenance setup that one is running.
This kind of model closely resembles a variant of the multi-compartment multi-knapsack problem. 

\begin{alignat}{2}
	\text{Min} &\sum_{w \in W} \sum_{p \in P} v_{wp}(t) \cdot x_{wp}(t)                                                                                             \\ 
	+ & \sum_{p \in P} \sum_{\tau = 1}^T d \cdot pen_{p\tau}(t)                                                                                                     \\
	+ & \sum_{p \in P} \sum_{w1 \in W} \sum_{w2 \in W} clu_{w1, w2} \cdot x_{w1p} \cdot x_{w2p}                              \label{eqn:objective_function_strategic} \\
    & \text{subject to:} \notag                                                                                                                                       \\
	& \sum_{w \in W} c_{w\tau} \cdot x_{wp}(t) \leq \ cap_{p\tau}(t) + pen_{p\tau}(t) \notag                                                                          \\ 
	& \forall p \in P, \forall \tau \in T                                                                                    \label{eqn:capacity_constraint}          \\
	& \sum_{w \in W} x_{wp}(t) = 1                                              , \quad \forall p \in P                      \label{eqn:single_workorder_constraint}  \\
	& x_{wp}(t) = 0                                                             , \quad \forall (w, p) \in E(t)              \label{eqn:exclusion_constraint}         \\
	& x_{wp}(t) = 1                                                             , \quad \forall (w, p) \in I(t)              \label{eqn:inclusion_constraint}         \\
	& x_{wp}(t) \in \{0, 1\}                                                    , \quad \forall w \in W, \forall p \in P     \label{eqn:x_integrality_constraint}     \\ 
	& pen_{p\tau}(t) \in \mathbb{R}^{+}                                         , \quad \forall p \in P, \forall \tau \in T  \label{eqn:p_non_negativity_constraint}
    & t \in  [0, \infty] 
\end{alignat}

\begin{figure}[H]
    \section{Scheduler: Strategic}
    \strategicModel[clustering=true, beta=false, normal=false, multiskill=true]
\end{figure}


\section{The Tactical Model}
\begin{itemize}
	\item Respect precedence constraints
	\item Respect daily resource requirements for each trait
	\item Penalize exceeded daily capacity
\end{itemize}

After the strategic model has optimized its schedule the tactical agent will continue scheduling the output at a more detailed level. This means that now the tactical agent will schedule 
out on each of the days of the work orders scheduled by the strategic agent. 

The tactical model is responsible for providing an initial suggestion for a weekly schedule, below we see the model for the tactical agent.
\begin{alignat}{2}
\text{Min}     & \sum_{o \in O} \sum_{d \in D} v_{do}(t) \cdot y_{do}(t)                                                      \\  
	         + & \sum_{c \in C} \sum_{d \in D} pen \cdot p_{cd}(t)                                               \\  
			                                                                                                  \\
               &\text{subject to:}                                                          \notag                                                                   \\
	           & \sum_{o \in O} w_{co} \cdot y_{do}(t)  \leq R_{dc} + p_{dc}(t)                                   \\ 
			   & \quad \quad \forall  d \in D, \forall c \in C                              \notag                                    \\ 
	           & \sum_{d \in D} d \cdot y_{do1}(t) + \delta_o  = \sum_{d \in D} d \cdot y_{do2}(t)                    \\ 
			   & \quad \quad \forall (o1, o2) \in \text{finish-start}(t)                        \notag                                   \\ 
	           & \sum_{d \in D} d \cdot y_{do1}(t) = \sum_{d \in D} d \cdot y_{do2}(t)                                \\ 
			   & \quad \quad \forall (o1, o2) \in \text{start-start}(t)                          \notag                             \\ 
			   & y_{do}(t) \leq number_o(t) * operating\_time_o                                                     \\ 
			   & \quad \quad ,\forall d \in D, \forall o \in O                              \notag                                    \\
			   & y_{do}(t) \in \{0, 1\} \quad ,\forall d \in D, o \in O                                          \\
			   & p_{cd}(t) \in \mathbb{R} \quad ,\forall c \in C, d \in D                                        \\
			   & \delta_o \in \{ duration\_lower_o(t),                                                           \\ 
			   & \quad \quad duration\_upper_o \}(t) \quad, \forall o \in O \notag                                      \\
			   & t \in  [0, \infty] 
\end{alignat}


\begin{figure}[H]
    \section{Scheduler: Tactical}
    \tacticalModel[]
\end{figure}

\newif\ifincludenormal\

\pgfkeys{
	/supervisormodel/.is family, /supervisormodel,
	default/.style = {
		normal=true,
	},
	normal/.is if=includenormal,
}
\newcommand{\supervisorModel}[1][]{
	\pgfkeys{/supervisormodel, default, #1}
	\begin{alignat}{2}
		& \text{\rule{\linewidth}{0.8pt}} \notag \label{}                                                                                                                                                                                                                                                                                                                                                                     \\ 
		& \textbf{Meta variables:} \notag\\
		& \ElementSupervisor \in \SetSupervisor \\
		& \VarStrategicWorkOrderAssignment{}{} \\
		& \VarIncludeActivity{} \\
		& \tau \in [0, \infty] \\
		& \text{\rule{\linewidth}{0.4pt}} \notag\\
		& \textbf{Maximize:} \notag\\
		& \sum_{\ElementActivity \in \SetActivity{\VarStrategicWorkOrderAssignment{}{}}{}} \sum_{\ElementTechnician \in \SetTechnician} \ParSupervisorValue \cdot \VarSupervisorAssignment{\ElementActivity}{\ElementTechnician} \\ 
		& \text{\rule{\linewidth}{0.4pt}} \notag\\
		& \textbf{Subject to:} \notag\\ 
		& \sum_{\ElementActivity \in \SetActivity{\VarStrategicWorkOrderAssignment{}{}}{\ElementOperation}} \VarActivityWork{\ElementActivity} = \ParOperationWork{\ElementOperation}    \quad \forall \ElementOperation \in \SetOperation{}{\VarStrategicWorkOrderAssignment{}{}}\\
		& \sum_{\ElementTechnician \in \SetTechnician} \sum_{\ElementActivity \in \SetActivity{\VarStrategicWorkOrderAssignment{}{}}{\ElementOperation}}\VarSupervisorAssignment{\ElementActivity}{\ElementTechnician} = \VarSupervisorOperationWhole \cdot \ParNumberOfPeople  \quad \forall \ElementOperation \in \SetOperation{}{\VarStrategicWorkOrderAssignment{}{}}  \\
		& \sum_{\ElementOperation \in \SetOperation{\ElementWorkOrder}{\VarStrategicWorkOrderAssignment{}{}}} \VarSupervisorOperationWhole = |\SetOperation{\ElementWorkOrder}{\VarStrategicWorkOrderAssignment{}{}}| \cdot \VarSupervisorWorkOrderWhole  \quad \forall \ElementWorkOrder \in \SetWorkOrder{,\VarStrategicWorkOrderAssignment{}{}} \\
		& \sum_{\ElementActivity \in \SetActivity{\VarStrategicWorkOrderAssignment{}{}}{\ElementOperation}} \VarSupervisorAssignment{\ElementActivity}{\ElementTechnician} \leq 1  \quad \forall \ElementOperation \in \SetOperation{}{\VarStrategicWorkOrderAssignment{}{}} \quad \forall \ElementTechnician \in \SetTechnician \\  
		& \VarSupervisorAssignment{\ElementActivity}{\ElementTechnician} \leq \ParFeasible  \quad \forall \ElementActivity \in \SetActivity{\VarTacticalWork}{\ElementOperation} \quad \forall \ElementOperation \in \SetOperation{}{\VarStrategicWorkOrderAssignment{}{}} \quad \forall \ElementTechnician \in \SetTechnician \\
		& \VarSupervisorAssignment{\ElementActivity}{\ElementTechnician} \in \{0, 1\}  \quad \forall \ElementOperation \in \SetOperation{}{\VarStrategicWorkOrderAssignment{}{}} \quad \forall \ElementTechnician \in \SetTechnician \\ 
		& \VarSupervisorOperationWhole \in \{0, 1\}  \quad \forall \ElementOperation \in \SetOperation{}{\VarStrategicWorkOrderAssignment{}{}} \\ 
		& \VarSupervisorWorkOrderWhole \in \{0, 1\}  \quad \forall \ElementWorkOrder \in \SetWorkOrder{,\VarStrategicWorkOrderAssignment{}{}} \\ 
		& \VarActivityWork{\ElementActivity} \in [\ParLowerActivityWork, \ParOperationWork{\ElementActivity}]  \quad \forall \ElementActivity \in \SetActivity{\VarStrategicWorkOrderAssignment{}{}}{}\\ 
		& \text{\rule{\linewidth}{0.8pt}} \notag \label{}                                                                                                                                                                                                                                                                                                                                                                      
	\end{alignat}
}

\begin{figure}[H]
    \section{Supervisor}
    \supervisorModel[]
\end{figure}

\section{The Operational Model}

Here the o is a single operation and o2 is another operation. It is crucial to understand here that the main
decision variable, $x$ defines an ordering of the operations that a single operational agent will do the 
operations in. 

The $\VarStartOfSegment{a}{k}$ is the start time of job $i$ in segment $k$ and $\VarFinishOfSegment{a}{k}$ is the finish time of job $i$ in segment $k$.
$\VarProcessingTime{a}{k}$ is the processing time of each segment. 
\begin{alignat}{2}
	& Max \sum_{a \in \SetActivity{\VarSupervisorAssignment}} \sum_{k \in \SetWorkSegment} \VarProcessingTime                                                         \\
	& \text{Subject to:} \notag                                                                                                                                       \\
    & \sum_{k \in \SetWorkSegment} \VarProcessingTime \cdot \VarActiveSegment{a}{k} = \ParOperationWork \cdot \VarIncludeActivity \quad \forall a \in \SetActivity{\VarSupervisorAssignment} \\
	& \VarStartOfSegment{a2}{1} \geq \VarFinishOfSegment{a1}{last(a1)} + \ParPreparation \notag                                                                       \\ 
	& \quad \forall a1 \in \SetActivity{\VarSupervisorAssignment}, a2 \in \SetActivity{\VarSupervisorAssignment}                                                     \\
	& \VarStartOfSegment{a}{k} \geq \VarFinishOfSegment{a}{k-1} - \ParConstraintLimit \cdot (2 - \VarActiveSegment{a}{k} + \VarActiveSegment{a}{k-1})                \notag\\
	& \quad \forall a \in \SetActivity{\VarSupervisorAssignment}\forall k \in \SetWorkSegment \\ 
	& \VarProcessingTime = \VarFinishOfSegment{a}{k} - \VarStartOfSegment{a}{k}                                                                                       \\
	& \quad \forall a \in \SetActivity{\VarSupervisorAssignment}, k \in \SetWorkSegment \notag                                                                        \\
	& \VarStartOfSegment{a}{k} \geq \ParEvent + \ParEventDuration - \ParConstraintLimit \cdot (1 - \VarSegmentInRelation)                                             \notag\\ 
	& \quad \forall a \in \SetActivity{\VarSupervisorAssignment}, k \in \SetWorkSegment, i \in \SetTimeInstance, e \in \SetEvent                                \\
	& \VarFinishOfSegment{a}{k} \leq \ParEvent + \ParConstraintLimit \cdot \VarSegmentInRelation                                                                      \notag\\ 
	& \quad \forall a \in \SetActivity{\VarSupervisorAssignment}, k \in \SetWorkSegment, i \in \SetTimeInstance, e \in \SetEvent                                \\
	& \VarStartOfSegment{a}{1} \geq \ParTimeWindowStart \forall a \in \SetActivity{\VarSupervisorAssignment}                                                          \\
	& \VarFinishOfSegment{a}{last(a)} \leq \ParTimeWindowFinish \forall a \in \SetActivity{\VarSupervisorAssignment}                                                  \\
	& \VarActiveSegment{a}{k} \in \{0, 1\} \quad \forall a \in \SetActivity{\VarSupervisorAssignment}, k \in \SetWorkSegment                                          \\
	& \VarStartOfSegment{a}{k} \in [\ParAvailabilityStart, \ParAvailabilityFinish] \notag\\
	& \quad \forall a \in \SetActivity{\VarSupervisorAssignment}, k \in \SetWorkSegment  \\
	& \VarFinishOfSegment{a}{k} \in [\ParAvailabilityStart, \ParAvailabilityFinish] \notag\\
	& \quad \forall a \in \SetActivity{\VarSupervisorAssignment}, k \in \SetWorkSegment \\
	& \VarProcessingTime \in [0, \ParOperationWork] \quad \forall a \in \SetActivity{\VarSupervisorAssignment}, k \in \SetWorkSegment                                               \\
	& \VarSegmentInRelation \in \{0, 1\}                                                                                                                              \\ 
	& \quad \forall a \in \SetActivity{\VarSupervisorAssignment}, k \in \SetWorkSegment, i \in \SetTimeInstance, e \in \SetEvent \notag                               \\
	& \theta_i \in \{0, 1\} \quad \forall a \in \SetActivity{\VarSupervisorAssignment}                                                                                \\
\end{alignat}

\begin{figure}[H]
    \section{Technician}
    \operationalModel[]
\end{figure}

\bibliography{../../../papers/actor-based-large-neighborhood-search/refs}
\bibliographystyle{elsarticle-harv}

\end{document}

