\documentclass{article}


\usepackage{standalone}
\usepackage{pgf}
\usepackage{pgfkeys}
\usepackage{gnuplottex}
\usepackage{amssymb}
\usepackage{graphicx}
\usepackage{amsmath}
\usepackage{algorithm}
\usepackage{algorithmicx}
\usepackage{algpseudocode}
\usepackage{pgfplots}
\usepackage{epstopdf}  
\usepackage[utf8]{inputenc}
\usepackage[T1]{fontenc}
\usepackage{graphicx}
\usepackage{amsmath}
\usepackage{makecell}
\usepackage{amssymb}
\usepackage{multicol}
\usepackage{natbib}
\usepackage{hyperref}
\usepackage{booktabs} % For professional quality tables
\usepackage{array}    % For better column alignment
\usepackage{tikz}
\usepackage{pgfkeys}
\usepackage[a4paper, margin=2cm]{geometry}
\usepackage{pgfplots}
\usepackage{epstopdf}  
\usepackage{graphicx}
\usepackage{float}
\usepackage{tikz}
\usepackage{pgf}
\usepackage{pgfkeys}
\usepackage{gnuplottex}
\usepackage{amsmath}
\usepackage{natbib}

% Sets
\newcommand{\SetWorkOrder}[1]{W(\VarMetaTime, \VarStrategicWorkOrderAssignment{}{})}
\newcommand{\SetPeriod}{P(\VarMetaTime)}
\newcommand{\SetResource}{R(\VarMetaTime)} %RECONSIDER | RELEVANT FOR BOTH STRATEGIC AND TACTICAL
\newcommand{\SetOperation}[2]{O_{#1}(\VarMetaTime, #2 )}

\newcommand{\SetDays}[1]{D_{#1}(\VarMetaTime)}
\newcommand{\SetActivity}[2]{A_{#2}(\VarMetaTime, #1)}
\newcommand{\SetTechnician}{T(\VarMetaTime)}
\newcommand{\SetWorkSegment}{K(\VarSupervisorAssignment{}{})}

\newcommand{\SetTimeInstance}{I(\VarMetaTime)}
\newcommand{\SetEvent}{E(\VarMetaTime)}

% Parameters
\newcommand{\ParStrategicValue}{strategic\_value_{wp}(\VarMetaTime)}
\newcommand{\ParStrategicPenalty}{strategic\_penalty}
\newcommand{\ParClusteringValue}{clustering\_value_{w1, w2}}
\newcommand{\ParStrategicResource}{resource_{pr}(\VarMetaTime)}

\newcommand{\ParStrategicWorkOrderWeight}{work\_order\_work_{wr}}
\newcommand{\ParStrategicInclude}{include(\VarMetaTime)}
\newcommand{\ParStrategicExclude}{exclude(\VarMetaTime)}
\newcommand{\ParTacticalValue}{tactical\_value_{do}(\VarMetaTime)}

\newcommand{\ParTacticalPenalty}{tactical\_penalty}
\newcommand{\ParOperationWork}[1]{work_{#1}(\VarMetaTime)}
\newcommand{\ParTacticalResource}{tactical\_resource_{dr}(\VarMetaTime)}
\newcommand{\ParStartStart}{start\_start_{o1, o2}}

\newcommand{\ParFinishStart}{finish\_start_{o1, o2}}
\newcommand{\ParNumberOfPeople}{number_{o}(\VarMetaTime)}
\newcommand{\ParOperatingTime}{operating\_time_{o}}
\newcommand{\ParDurationLower}{duration\_lower_{o}(\VarMetaTime)}

\newcommand{\ParDurationUpper}{duration\_upper_{o}(\VarMetaTime)}
\newcommand{\ParSupervisorValue}{supervisor\_value_{at}(\VarMetaTime)} % We should determine if the supervisor should assign for each operation or for each activity. My guts say that it should be for each activity
\newcommand{\ParFeasible}{feasible_{at}(\VarIncludeActivity{})}
\newcommand{\ParOperationsForWorkOrder}{work\_order\_to\_operations_{w}}

\newcommand{\ParOperationsInWorkOrder}{operations\_in\_work\_order_{w}}
\newcommand{\ParActivitiesForOperation}{activities\_for\_operation_{o}}
\newcommand{\ParLowerActivityWork}{lower\_activity\_work_{a}(\VarMetaTime)}
\newcommand{\ParActivityWork}[1]{activity\_work_{a}(\VarMetaTime, \VarActivityWork{#1})}
\newcommand{\ParPreparation}{preparation_{a1, a2}}

\newcommand{\ParEvent}{event_{ie}}
\newcommand{\ParEventDuration}{duration_{ie}}
\newcommand{\ParConstraintLimit}{constraint\_limit}
\newcommand{\ParTimeWindowStart}{time\_window\_start_{a}(\VarTacticalWork{}{})}

\newcommand{\ParTimeWindowFinish}{time\_window\_finish_{a}(\VarTacticalWork{}{})}
\newcommand{\ParAvailabilityStart}{availability\_start(\VarMetaTime)}
\newcommand{\ParAvailabilityFinish}{availability\_finish(\VarMetaTime)}

% Variables
\newcommand{\VarStrategicWorkOrderAssignment}[2]{\alpha_{#1#2}(\VarMetaTime)}
\newcommand{\VarStrategicExcess}{\epsilon_{pr}(\VarMetaTime)}
\newcommand{\VarTacticalWork}[2]{\beta_{#1#2}(\VarMetaTime)}
\newcommand{\VarTacticalExcess}{\mu_{rd}(\VarMetaTime)} : excess capacity for each day

\newcommand{\VarTacticalWorkBinary}[2]{\sigma_{#1#2}(\VarMetaTime)}
\newcommand{\VarTacticalWorkBinaryConsecutive}{\eta_{do}(\VarMetaTime)}
\newcommand{\VarTacticalOperationDifference}{\Delta_{o}(\VarMetaTime)}
\newcommand{\VarSupervisorAssignment}[2]{\gamma_{#1#2}(\VarMetaTime)}
\newcommand{\VarSupervisorAssignmentWhole}{\phi_{o}(\VarMetaTime)}

\newcommand{\VarActivityWork}[1]{\rho_{#1}(\VarMetaTime)}

\newcommand{\VarProcessingTime}{\delta_{ak}(\VarMetaTime)} 
\newcommand{\VarActiveSegment}[2]{\pi_{#1#2}(\VarMetaTime)}
\newcommand{\VarStartOfSegment}[2]{\lambda_{#1#2}(\VarMetaTime)}
\newcommand{\VarFinishOfSegment}[2]{\Lambda_{#1#2}(\VarMetaTime)}

\newcommand{\VarSegmentInRelation}{\omega_{akie}(\VarMetaTime)}
\newcommand{\VarIncludeActivity}[1]{\theta_{#1}(\VarMetaTime)}

% Meta variables
\newcommand{\VarMetaTime}{\tau}



\definecolor{red}{HTML}{8A3F3A}
\definecolor{yellow}{HTML}{E0BB3C}
\definecolor{blue}{HTML}{4569E0}
\definecolor{green}{HTML}{17E561}
\definecolor{other}{HTML}{6A939E}


\begin{document}

\section{Project Plan}
\subsection{Introduction}
Industrial maintenence scheduling seen as a mathematical optimization problem presents a unique challenge to develop
state of the art modelling and solution approaches. The scope of modern industrial maintenance operations
can be extremely vast, consisting of 1000s of work orders,
100s of technicians on assets that have up to 5 year asset maintenance plans and designed to be operational
for decades~\cite{palmerMaintenancePlanningScheduling2019}. This combined with high levels of uncertainty,
real-time changes, and optimizing across multiple levels of stakeholder, means that maintenance scheduling
touches on many of the known current limitation in the operation research/metaheuristic literature.

Further it remains challenging for human planners, schedulers, and maintenance
supervisors to plan and schedule maintenance manually. This creates a demand for
software tools that can effectively utilize available manpower and prioritize
the most critical maintenance operations whereever in the scheduling process
they occur. The cost of planning maintenance in an inefficient manner can be
extensive as processes in large-scale equipment are often linked together. If a
single piece of equipment fails it can cause whole subsystems to shut down. To
migigate this, it is important to prioritize maintenance operations correctly.
This is especially true in circumstances, where it is infeasible to perform all
maintenance operations, due to either financial or physical constraints.

\subsection{Purpose}
Given the complexity of such operations and the maturity of the practical part of the field, the maintenance
scheduling has throughout many decades designed an effective though often inefficient way of planning and scheduling
these activities. This Ph.D. project seeks to model the practical knowledge and rules of thumb that have been gathered
over decades and model and optimize these using industry proven techniques coming from other mathematical optimization
problems~\cite{talbiMetaheuristicsDesignImplementation2009}, \cite{gendreauHandbookMetaheuristics2019}. Further as this
process presents unique challenges due to many

\subsection{Research Questions}
\begin{itemize}
    \item How to model and optimize across multiple stakeholder each optimizing a part of the total maintenance scheduling process?
    \item How can simulations guide decision-making in maintenance planning? 
    \item How to implement metaheuristics that can adapt to changing requirements in near real-time? 
    \item How to create a heuristic that is effective for both corrective and preventive maintenance? 
    \item How can the maintenance planning problem be solved with imperfect data? 
    \item What is the effect of the maintenance process on the ability to create a robust maintenance plan? 
\end{itemize}

\nocite{ben-dayaHandbookMaintenanceManagement2009}
\nocite{pinedoSchedulingTheoryAlgorithms2022}
\bibliography{refs.bib}

\bibliographystyle{elsarticle-harv}

\end{document}

