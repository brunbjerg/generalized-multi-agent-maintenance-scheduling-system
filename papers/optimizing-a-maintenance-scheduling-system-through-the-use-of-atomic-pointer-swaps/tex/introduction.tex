\section{Introduction}





\subsection{}
The future research direction is to demonstrate that
the actor-based approach described here can be used to model and optimize 
multi-actor/multi-level scheduling processes. 
Figures~\ref{fig:ordinator-hexagon:persistence},
~\ref{fig:ordinator-hexagon:atomicpointerswap},
~\ref{fig:ordinator-hexagon:metaheuristics},
~\ref{fig:ordinator-hexagon:orchestrator},
~\ref{fig:ordinator-hexagon:userinterfaces}
show a larger scale setup of the actor-based approach which is being developed with Total Energies.
Here figure~\ref{fig:ordinator-hexagon:metaheuristics}
shows the metaheuristics of scheduling system architecture where each of the actors run an AbLNS
and that each metaheuristic will share its solutions with the other
metaheuristics through atomic pointer swapping shown in figure~\ref{fig:ordinator-hexagon:atomicpointerswap}.
Communicate with the end-user
through userinterfaces and message passing as shown in figure~\ref{fig:ordinator-hexagon:userinterfaces}, 
integrate with a persistent storage through mutex
locks as shown in figure~\ref{fig:ordinator-hexagon:persistence}, 
and the lifecycle of each of the metaheuristics will be controlled by
the orchestrator also through message passing as shown in figure~\ref{fig:ordinator-hexagon:orchestrator}. 

% \begin{figure}[H]
% 	\centering
% 	\usetikzlibrary {positioning}

\definecolor{red}{HTML}{8A3F3A}
\definecolor{yellow}{HTML}{E0BB3C}
\definecolor{blue}{HTML}{4569E0}
\definecolor{green}{HTML}{17E561}
\definecolor{other}{HTML}{6A939E}

% DTU Colors
\definecolor{dtu-corporate-red}{HTML}{990000}
\definecolor{dtu-white}{HTML}{ffffff}
\definecolor{dtu-black}{HTML}{000000}
\definecolor{dtu-blue}{HTML}{2F3EEA}
\definecolor{dtu-bright-green}{HTML}{1FD082}
\definecolor{dtu-navy-blue}{HTML}{030F4F}
\definecolor{dtu-yellow}{HTML}{F6D04D}
\definecolor{dtu-orange}{HTML}{FC7634}
\definecolor{dtu-pink}{HTML}{F7BBB1}
\definecolor{dtu-grey}{HTML}{DADADA}
\definecolor{dtu-red}{HTML}{E83F48}
\definecolor{dtu-green}{HTML}{008835}
\definecolor{dtu-purple}{HTML}{79238E}


\newcommand{\ModelColor}{dtu-red}
\newcommand{\UserInterfaceColor}{dtu-yellow}
\newcommand{\PersistenceColor}{dtu-blue}
\newcommand{\PointerSwapColor}{dtu-green}
\newcommand{\OrchestratorColor}{dtu-bright-green}

\newcommand{\basisinput}{4cm}  % Default value if not set by /graph/basis

\pgfkeys{
	/graph/.is family, /graph,
	default/.style = {
		show_shared_pointer = false,
		show_orchestrator = false,
		show_persistence = false,
		show_user_interface = false,
		basisinput/.estore in = \basisinput,
	},
	show_shared_pointer/.estore in = \ShowSharedSolutionCommunication,
	show_orchestrator/.estore in = \ShowOrchestratorCommunication,
	show_persistence/.estore in = \ShowPersistenceCommunication,
	show_user_interface/.estore in = \ShowUserInterfaceCommunication,
	basisinput/.estore in = \basisinput,
}

\newlength{\basis}
\tikzset{
  basis/.code={\setlength{\basis}{\basisinput}}, % TikZ assignment code
  basis/.default=3cm,                   % Provide a default (\b@sis is undefined/unassigned)
  basis,                                % Set initial Value (\b@sis is defined/assigned)
}

\newcommand{\drawOrdinatorArchitecture}[1]{
	\pgfkeys{/graph, default, #1}
	\setlength{\basis}{\basisinput}
	\begin{tikzpicture}[scale=0.75, line width=0.05\basis]

		\ifthenelse{\equal{\ShowOrchestratorCommunication}{true}}{
			\draw[color=other,-, ultra thick] (Strategic) -- (Orchestrator);
			\draw[color=other,-, ultra thick] (Tactical) -- (Orchestrator);
			\draw[color=other,-, ultra thick] (Supervisor) -- (Orchestrator);
			\draw[color=other,-, ultra thick] (Operational_1) -- (Orchestrator);
			\draw[color=other,-, ultra thick] (Operational_2) -- (Orchestrator);
			\draw[color=other,-, ultra thick] (Operational_3) -- (Orchestrator);
		}{}
		% \draw[help lines] (0\basis, 0\basis) grid (10\basis, 8\basis);
		\draw (5\basis,4\basis) node[minimum height=5\basis,minimum width=7.0\basis,rounded corners=0.1\basis] {};

	    \draw[draw=black] (4.125\basis,4.0\basis) node[opacity=0.5, minimum height=3.5\basis,minimum width=6.25\basis,rounded corners=0.1\basis,fill=\PointerSwapColor] {} ;
	    \draw (2.5\basis,5.5\basis) node[minimum height=1\basis,minimum width=1\basis,fill=\ModelColor,rounded corners=0.1\basis] (Strategic) {Stra};
	    \draw (5.0\basis,4.0\basis) node[minimum height=1\basis,minimum width=1\basis,fill=\ModelColor,rounded corners=0.1\basis] (Supervisor) {Sup};
		\draw (7.5\basis,5.5\basis) node[minimum height=1\basis,minimum width=1\basis,fill=\ModelColor,rounded corners=0.1\basis] (Tactical) {Tac};

		\draw (2.5\basis,2.5\basis) node[minimum height=1\basis,minimum width=1\basis,fill=\ModelColor,rounded corners=0.1\basis] (Operational_1) {$O_{1}$};
		\draw (5.0\basis,2.5\basis) node[minimum height=1\basis,minimum width=1\basis,fill=\ModelColor,rounded corners=0.1\basis] (Operational_2) {$O_{2}$};
		\draw (7.5\basis,2.5\basis) node[minimum height=1\basis,minimum width=1\basis,fill=\ModelColor,rounded corners=0.1\basis,rounded corners=0.1\basis] (Operational_3) {$O_{3}$};
	
		\draw (Strategic) edge (Tactical);
		\draw (Strategic) edge (Tactical);
		\draw (5\basis,5.5\basis) edge (Supervisor);
		\draw (Supervisor) -- (2.5\basis,4.0\basis) -- (Operational_1);
		\draw (Supervisor) edge (Operational_2);
		\draw (Supervisor) -- (7.5\basis,4.0\basis) -- (Operational_3);
		\draw (5.0\basis,0.5\basis)   node[minimum height=1\basis,minimum width=5.0\basis,                fill=\PersistenceColor,rounded corners=0.1\basis] {persistence};
		\draw (5.0\basis,7.5\basis)   node[minimum height=1\basis,minimum width=5.0\basis,                fill=\OrchestratorColor,rounded corners=0.1\basis] (Orchestrator) {Orchestrator};
		\draw (0.5\basis,4.0\basis)   node[rotate=90, minimum height=1.0\basis, minimum width=3.5\basis,  fill=\PointerSwapColor,rounded corners=0.1\basis] {decision variables};
		\draw (9.5\basis,5.75\basis)  node[rotate=90, minimum height=1.0\basis, minimum width=1.0\basis,  fill=\UserInterfaceColor,rounded corners=0.1\basis] {UI};
		\draw (9.5\basis,4.0\basis)   node[rotate=90, minimum height=1.0\basis, minimum width=1.0\basis,  fill=\UserInterfaceColor,rounded corners=0.1\basis] {UI};
		\draw (9.5\basis,2.25\basis)  node[rotate=90, minimum height=1.0\basis, minimum width=1.0\basis,  fill=\UserInterfaceColor,rounded corners=0.1\basis] {UI};

		% Legend
		\begin{scope}[shift={(11.0\basis,5.7\basis)}]
			\node at (-0.25\basis,1\basis) [right] {};
			\draw[color=\OrchestratorColor,fill,rounded corners=0.1\basis] (0\basis,0.0\basis)   rectangle (0.5\basis, 0.5\basis);
			\node[anchor=west] at (0.5\basis, 0.25\basis) { Managing metaheuristic lifetimes };
			\draw[color=\PointerSwapColor,fill,rounded corners=0.1\basis] (0\basis,-1.0\basis)   rectangle(0.5\basis, -0.5\basis); 
			\node[anchor=west] at (0.5\basis, -0.75\basis) { Solution sharing (Atomic pointer swaps) };
			\draw[color=\ModelColor,fill,rounded corners=0.1\basis] (0\basis,-2.0\basis)         rectangle(0.5\basis, -1.5\basis); 
			\node[anchor=west] at (0.5\basis, -1.75\basis) { Metaheurics (Mathematical Models) };
			\draw[color=\PersistenceColor,fill,rounded corners=0.1\basis] (0\basis,-3.0\basis)   rectangle(0.5\basis, -2.5\basis); 
			\node[anchor=west] at (0.5\basis, -2.75\basis) { Data storage (Memory locks)};
			\draw[color=\UserInterfaceColor,fill,rounded corners=0.1\basis] (0\basis,-4.0\basis) rectangle(0.5\basis, -3.5\basis); 
			\node[anchor=west] at (0.5\basis, -3.75\basis) { Message passing (Memory channels) };

		\end{scope}
		\ifthenelse{\equal{\ShowSharedSolutionCommunication}{true}}{
			\draw[->, thick] (Strategic) -- (Orchestrator);
		}{}
		\ifthenelse{\equal{\ShowUserInterfaceCommunication}{true}}{
			\draw[->, thick] (Strategic) -- (Orchestrator);
		}{}
		\ifthenelse{\equal{\ShowPersistenceCommunication}{true}}{
			\draw[->, thick] (Strategic) -- (Orchestrator);
		}{}
		

	\end{tikzpicture}
}


% 	\drawOrdinatorArchitecture{basisinput=1cm}
% 	\label{fig:ordinator-architecture}
% \end{figure}
\begin{figure}[H]
	\centering
	\usetikzlibrary {positioning}
\newcommand{\drawHexagon}[6][draw=black]{
	\draw[#1, fill=#4] (#2,#3) ++(30:#6) -- ++(90:#6) -- ++(150:#6) -- ++(210:#6) -- ++(270:#6) -- ++(330:#6) -- cycle;
	\node[align=center] at (#2,#3+2) {#5};
}

\newif\ifpersistencelayer
\newif\ifatomicpointerswaplayer
\newif\ifmetaheuristicslayer
\newif\ifuserinterfacelayer
\newif\iforchestratorlayer
\newif\ifsimplifiedlayer

\pgfkeys{
	/hexagon/.is family, /hexagon,
	default/.style = {
		persistence=false,
		atomicpointerswap=false,
		metaheuristics=false,
		orchestrator=false,
		userinterface=false,
		simplified=false,
	},
	persistence/.is if=persistencelayer,
	atomicpointerswap/.is if=atomicpointerswaplayer,
	metaheuristics/.is if=metaheuristicslayer,
	orchestrator/.is if=orchestratorlayer,
	userinterface/.is if=userinterfacelayer,
	simplified/.is if=simplifiedlayer,
}
\newcommand{\drawModelSetupHexagon}[1][]{
	\pgfkeys{/hexagon, default, #1}

	\begin{tikzpicture}[font=\footnotesize, scale=0.5, line width=1.05]
	

	\ifpersistencelayer
		\drawHexagon[draw=none]{ 2                      }{ 2}{dtu-blue}{}{2}
		\drawHexagon[draw=none]{{6 - 2 * (2 - sqrt(3)) }}{ 2}{dtu-blue}{}{2}
		\drawHexagon[draw=none]{{4 - 1 * (2 - sqrt(3)) }}{-1}{dtu-blue}{Persistence}{2}
		\drawHexagon[draw=none]{{0 + 1 * (2 - sqrt(3)) }}{-1}{dtu-blue}{}{2}
		\drawHexagon[draw=none]{{8 - 3 * (2 - sqrt(3)) }}{-1}{dtu-blue}{}{2}

		\drawHexagon[draw=none]{{2 - 0 * (2 - sqrt(3)) }}{-4}{dtu-blue}{}{2}
		\drawHexagon[draw=none]{{6 - 2 * (2 - sqrt(3)) }}{-4}{dtu-blue}{}{2}

		\drawHexagon[draw=none]{{10 - 4 * (2 - sqrt(3)) }}{-4}{dtu-blue}{}{2}
		\drawHexagon[draw=none]{{-2 + 2 * (2 - sqrt(3)) }}{-4}{dtu-blue}{}{2}

		\drawHexagon[draw=none]{{12 - 5 * (2 - sqrt(3)) }}{-1}{dtu-blue}{}{2}
		\drawHexagon[draw=none]{{-4 + 3 * (2 - sqrt(3)) }}{-1}{dtu-blue}{}{2}
		% Legend for each layer
		\drawHexagon{{14.0  }}{+3.0}{dtu-blue}{}{0.75}
		\node[align=right, anchor=west] at ({15.0}, +3.75) {Persistence};
		\drawHexagon{{14.0  }}{+1.5}{dtu-white}{}{0.75}
		\node[align=right, anchor=west] at ({15.0}, +2.25) {Atomic Pointer};
		\drawHexagon{{14.0  }}{+0.0}{dtu-white}{}{0.75}
		\node[align=right, anchor=west] at ({15.0}, +0.75) {Metaheuristics};
		\drawHexagon{{14.0  }}{-1.5}{dtu-white}{}{0.75}
		\node[align=right, anchor=west] at ({15.0}, -0.75) {Orchestration};
		\drawHexagon{{14.0  }}{-3.0}{dtu-white}{}{0.75}
		\node[align=right, anchor=west] at ({15.0}, -2.25) {User interfaces};
	\fi


	\ifatomicpointerswaplayer
		\drawHexagon[]{ 2                      }{ 2}{dtu-green}{Shared\\solution\\pointer}{2}
		\drawHexagon[]{{6 - 2 * (2 - sqrt(3)) }}{ 2}{dtu-green}{Shared\\solution\\pointer}{2}
		\drawHexagon[]{{4 - 1 * (2 - sqrt(3)) }}{-1}{dtu-green}{Shared\\solution\\pointer}{2}
		\drawHexagon[]{{0 + 1 * (2 - sqrt(3)) }}{-1}{dtu-green}{Shared\\solution\\pointer}{2}
		\drawHexagon[]{{8 - 3 * (2 - sqrt(3)) }}{-1}{dtu-green}{Shared\\solution\\pointer}{2}

		\drawHexagon[]{{2 - 0 * (2 - sqrt(3)) }}{-4}{dtu-green}{Shared\\solution\\pointer}{2}
		\drawHexagon[]{{6 - 2 * (2 - sqrt(3)) }}{-4}{dtu-green}{Shared\\solution\\pointer}{2}

		\drawHexagon[]{{10 - 4 * (2 - sqrt(3)) }}{-4}{dtu-green}{Shared\\solution\\pointer}{2}
		\drawHexagon[]{{-2 + 2 * (2 - sqrt(3)) }}{-4}{dtu-green}{Shared\\solution\\pointer}{2}

		\drawHexagon[]{{12 - 5 * (2 - sqrt(3)) }}{-1}{dtu-green}{Shared\\solution\\pointer}{2}
		\drawHexagon[]{{-4 + 3 * (2 - sqrt(3)) }}{-1}{dtu-green}{Shared\\solution\\pointer}{2}
		% Legend for each layer
		\drawHexagon{{14.0  }}{+3.0}{dtu-white}{}{0.75}
		\node[align=right, anchor=west] at ({15.0}, +3.75) {Persistence};
		\drawHexagon{{14.0  }}{+1.5}{dtu-green}{}{0.75}
		\node[align=right, anchor=west] at ({15.0}, +2.25) {Atomic Pointer};
		\drawHexagon{{14.0  }}{+0.0}{dtu-white}{}{0.75}
		\node[align=right, anchor=west] at ({15.0}, +0.75) {Metaheuristics};
		\drawHexagon{{14.0  }}{-1.5}{dtu-white}{}{0.75}
		\node[align=right, anchor=west] at ({15.0}, -0.75) {Orchestration};
		\drawHexagon{{14.0  }}{-3.0}{dtu-white}{}{0.75}
		\node[align=right, anchor=west] at ({15.0}, -2.25) {User interfaces};
	\fi

	\ifsimplifiedlayer

		\node[align=right, anchor=west] at ({-5.5}, +3.75) {};
		\drawHexagon{{+2 + 0 * (2 - sqrt(3)) }}{ 2}{dtu-green}{Scheduler}{2}
		\drawHexagon{{+4 - 1 * (2 - sqrt(3)) }}{-1}{dtu-red}{Supervisor}{2}
		\drawHexagon{{+0 + 1 * (2 - sqrt(3)) }}{-1}{dtu-red}{Supervisor}{2}
		\drawHexagon{{+2 - 0 * (2 - sqrt(3)) }}{-4}{dtu-corporate-red}{Technician}{2}
		\drawHexagon{{+6 - 2 * (2 - sqrt(3)) }}{-4}{dtu-corporate-red}{Technician}{2}
		\drawHexagon{{-2 + 2 * (2 - sqrt(3)) }}{-4}{dtu-corporate-red}{Technician}{2}
		\drawHexagon{{+8 - 3 * (2 - sqrt(3)) }}{-1}{dtu-corporate-red}{Technician}{2}
		\drawHexagon{{-4 + 3 * (2 - sqrt(3)) }}{-1}{dtu-corporate-red}{Technician}{2}

		% Scheduler
		\draw[thin, fill=dtu-yellow] (2, 5) circle (0.35);
		\draw[thin, fill=dtu-purple] (2, 3) circle (0.35);
		% Supervisor 1
		\draw[thin, fill=dtu-yellow] ({+4 - 1 * (2 - sqrt(3)) }, 02) circle (0.35);
		\draw[thin, fill=dtu-purple] ({+4 - 1 * (2 - sqrt(3)) }, -0) circle (0.35);
		% Supervisor 2
		\draw[thin, fill=dtu-yellow] ({+0 + 1 * (2 - sqrt(3)) }, 02) circle (0.35);
		\draw[thin, fill=dtu-purple] ({+0 + 1 * (2 - sqrt(3)) }, -0) circle (0.35);
		% Technician 1
		\draw[thin, fill=dtu-yellow] ({+2 - 0 * (2 - sqrt(3)) }, -1) circle (0.35);
		\draw[thin, fill=dtu-purple] ({+2 - 0 * (2 - sqrt(3)) }, -3) circle (0.35);
		% Technician 2
		\draw[thin, fill=dtu-yellow] ({+6 - 2 * (2 - sqrt(3)) }, -1) circle (0.35);
		\draw[thin, fill=dtu-purple] ({+6 - 2 * (2 - sqrt(3)) }, -3) circle (0.35);
		% Technician 3
		\draw[thin, fill=dtu-yellow] ({-2 + 2 * (2 - sqrt(3)) }, -1) circle (0.35);
		\draw[thin, fill=dtu-purple] ({-2 + 2 * (2 - sqrt(3)) }, -3) circle (0.35);
		% Technician 4
		\draw[thin, fill=dtu-yellow] ({+8 - 3 * (2 - sqrt(3)) }, 02) circle (0.35);
		\draw[thin, fill=dtu-purple] ({+8 - 3 * (2 - sqrt(3)) }, -0) circle (0.35);
		% Technician 5
		\draw[thin, fill=dtu-yellow] ({-4 + 3 * (2 - sqrt(3)) }, 02) circle (0.35);
		\draw[thin, fill=dtu-purple] ({-4 + 3 * (2 - sqrt(3)) }, -0) circle (0.35);

		% Legend for each layer
		\node[align=right, anchor=west] at ({12.0}, +3.75) {Atomic Pointer};
		\draw[fill=dtu-purple] (11.0,  +3.75) circle (0.5);

		\node[align=right, anchor=west] at ({12.0}, +2.25) {Scheduler Metaheuristic};
		\drawHexagon{{11.0  }}{+1.75}{dtu-green}{}{0.5}
		\node[align=right, anchor=west] at ({12.0}, +0.75) {Supervisor Metaheuristic};
		\drawHexagon{{11.0  }}{+0.25}{dtu-red}{}{0.5}
		\node[align=right, anchor=west] at ({12.0}, -0.75) {Technician Metaheuristic};
		\drawHexagon{{11.0  }}{-1.25}{dtu-corporate-red}{}{0.5}
		\node[align=right, anchor=west] at ({12.0}, -2.25) {User interfaces (Message Passing)};
		\draw[fill=dtu-yellow] (11.0, -2.25) circle (0.5);
	\fi

	\ifmetaheuristicslayer
		\drawHexagon{ 2                      }{ 2}{dtu-blue}{Strategic}{2}
		\drawHexagon{{6 - 2 * (2 - sqrt(3)) }}{ 2}{dtu-green}{Tactical}{2}
		\drawHexagon{{4 - 1 * (2 - sqrt(3)) }}{-1}{dtu-red}{Supervisor}{2}
		\drawHexagon{{0 + 1 * (2 - sqrt(3)) }}{-1}{dtu-red}{Supervisor}{2}
		\drawHexagon{{8 - 3 * (2 - sqrt(3)) }}{-1}{dtu-red}{Supervisor}{2}

		\drawHexagon{{2 - 0 * (2 - sqrt(3)) }}{-4}{dtu-corporate-red}{Technician}{2}
		\drawHexagon{{6 - 2 * (2 - sqrt(3)) }}{-4}{dtu-corporate-red}{Technician}{2}

		\drawHexagon{{10 - 4 * (2 - sqrt(3)) }}{-4}{dtu-corporate-red}{Technician}{2}
		\drawHexagon{{-2 + 2 * (2 - sqrt(3)) }}{-4}{dtu-corporate-red}{Technician}{2}

		\drawHexagon{{12 - 5 * (2 - sqrt(3)) }}{-1}{dtu-corporate-red}{Technician}{2}
		\drawHexagon{{-4 + 3 * (2 - sqrt(3)) }}{-1}{dtu-corporate-red}{Technician}{2}

		% Legend for each layer
		\drawHexagon{{14.0  }}{+3.0}{dtu-white}{}{0.75}
		\node[align=right, anchor=west] at ({15.0}, +3.75) {Persistence};
		\drawHexagon{{14.0  }}{+1.5}{dtu-white}{}{0.75}
		\node[align=right, anchor=west] at ({15.0}, +2.25) {Atomic Pointer};
		\drawHexagon{{14.0  }}{+0.0}{dtu-corporate-red}{}{0.75}
		\node[align=right, anchor=west] at ({15.0}, +0.75) {Metaheuristics};
		\drawHexagon{{14.0  }}{-1.5}{dtu-white}{}{0.75}
		\node[align=right, anchor=west] at ({15.0}, -0.75) {Orchestration};
		\drawHexagon{{14.0  }}{-3.0}{dtu-white}{}{0.75}
		\node[align=right, anchor=west] at ({15.0}, -2.25) {User interfaces};
	\fi

	\iforchestratorlayer
		\drawHexagon{ 2                      }{ 2}{dtu-orange}{}{2}
		\drawHexagon{{6 - 2 * (2 - sqrt(3)) }}{ 2}{dtu-orange}{}{2}
		\drawHexagon{{4 - 1 * (2 - sqrt(3)) }}{-1}{dtu-orange}{Orche-\\strator}{2}
		\drawHexagon{{0 + 1 * (2 - sqrt(3)) }}{-1}{dtu-orange}{}{2}
		\drawHexagon{{8 - 3 * (2 - sqrt(3)) }}{-1}{dtu-orange}{}{2}

		\drawHexagon{{2 - 0 * (2 - sqrt(3)) }}{-4}{dtu-orange}{}{2}
		\drawHexagon{{6 - 2 * (2 - sqrt(3)) }}{-4}{dtu-orange}{}{2}

		\drawHexagon{{10 - 4 * (2 - sqrt(3)) }}{-4}{dtu-orange}{}{2}
		\drawHexagon{{-2 + 2 * (2 - sqrt(3)) }}{-4}{dtu-orange}{}{2}

		\drawHexagon{{12 - 5 * (2 - sqrt(3)) }}{-1}{dtu-orange}{}{2}
		\drawHexagon{{-4 + 3 * (2 - sqrt(3)) }}{-1}{dtu-orange}{}{2}
		% Legend for each layer
		\drawHexagon{{14.0  }}{+3.0}{dtu-white}{}{0.75}
		\node[align=right, anchor=west] at ({15.0}, +3.75) {Persistence};
		\drawHexagon{{14.0  }}{+1.5}{dtu-white}{}{0.75}
		\node[align=right, anchor=west] at ({15.0}, +2.25) {Atomic Pointer};
		\drawHexagon{{14.0  }}{+0.0}{dtu-white}{}{0.75}
		\node[align=right, anchor=west] at ({15.0}, +0.75) {Metaheuristics};
		\drawHexagon{{14.0  }}{-1.5}{dtu-orange}{}{0.75}
		\node[align=right, anchor=west] at ({15.0}, -0.75) {Orchestration};
		\drawHexagon{{14.0  }}{-3.0}{dtu-white}{}{0.75}
		\node[align=right, anchor=west] at ({15.0}, -2.25) {User interfaces};
	\fi

	
	\ifuserinterfacelayer
		\drawHexagon{ 2                      }{ 2}{dtu-yellow}{UI}{2}
		\drawHexagon{{6 - 2 * (2 - sqrt(3)) }}{ 2}{dtu-yellow}{UI}{2}
		\drawHexagon{{4 - 1 * (2 - sqrt(3)) }}{-1}{dtu-yellow}{UI}{2}
		\drawHexagon{{0 + 1 * (2 - sqrt(3)) }}{-1}{dtu-yellow}{UI}{2}
		\drawHexagon{{8 - 3 * (2 - sqrt(3)) }}{-1}{dtu-yellow}{UI}{2}

		\drawHexagon{{2 - 0 * (2 - sqrt(3)) }}{-4}{dtu-yellow}{UI}{2}
		\drawHexagon{{6 - 2 * (2 - sqrt(3)) }}{-4}{dtu-yellow}{UI}{2}

		\drawHexagon{{10 - 4 * (2 - sqrt(3)) }}{-4}{dtu-yellow}{UI}{2}
		\drawHexagon{{-2 + 2 * (2 - sqrt(3)) }}{-4}{dtu-yellow}{UI}{2}

		\drawHexagon{{12 - 5 * (2 - sqrt(3)) }}{-1}{dtu-yellow}{UI}{2}
		\drawHexagon{{-4 + 3 * (2 - sqrt(3)) }}{-1}{dtu-yellow}{UI}{2}
		% Legend for each layer
		\drawHexagon{{14.0  }}{+3.0}{dtu-white}{}{0.75}
		\node[align=right, anchor=west] at ({15.0}, +3.75) {Persistence};
		\drawHexagon{{14.0  }}{+1.5}{dtu-white}{}{0.75}
		\node[align=right, anchor=west] at ({15.0}, +2.25) {Atomic Pointer};
		\drawHexagon{{14.0  }}{+0.0}{dtu-white}{}{0.75}
		\node[align=right, anchor=west] at ({15.0}, +0.75) {Metaheuristics};
		\drawHexagon{{14.0  }}{-1.5}{dtu-white}{}{0.75}
		\node[align=right, anchor=west] at ({15.0}, -0.75) {Orchestration};
		\drawHexagon{{14.0  }}{-3.0}{dtu-yellow}{}{0.75}
		\node[align=right, anchor=west] at ({15.0}, -2.25) {User interfaces};
	\fi
	
	\end{tikzpicture}
}

	\resizebox{0.7\textwidth}{!}{
		\drawModelSetupHexagon[persistence=true]
	}
	\caption{
		Overview of the scheduling process when modelled as actors. When LNS is encapsulated 
		is an actor it becomes possible to optimize parts of a large process individually instead of 
		optimizing the scheduling problem globally from a single model implementation.
	}\label{fig:ordinator-hexagon:persistence}
\end{figure}
\begin{figure}[H]
	\centering
	\usetikzlibrary {positioning}
\newcommand{\drawHexagon}[6][draw=black]{
	\draw[#1, fill=#4] (#2,#3) ++(30:#6) -- ++(90:#6) -- ++(150:#6) -- ++(210:#6) -- ++(270:#6) -- ++(330:#6) -- cycle;
	\node[align=center] at (#2,#3+2) {#5};
}

\newif\ifpersistencelayer
\newif\ifatomicpointerswaplayer
\newif\ifmetaheuristicslayer
\newif\ifuserinterfacelayer
\newif\iforchestratorlayer
\newif\ifsimplifiedlayer

\pgfkeys{
	/hexagon/.is family, /hexagon,
	default/.style = {
		persistence=false,
		atomicpointerswap=false,
		metaheuristics=false,
		orchestrator=false,
		userinterface=false,
		simplified=false,
	},
	persistence/.is if=persistencelayer,
	atomicpointerswap/.is if=atomicpointerswaplayer,
	metaheuristics/.is if=metaheuristicslayer,
	orchestrator/.is if=orchestratorlayer,
	userinterface/.is if=userinterfacelayer,
	simplified/.is if=simplifiedlayer,
}
\newcommand{\drawModelSetupHexagon}[1][]{
	\pgfkeys{/hexagon, default, #1}

	\begin{tikzpicture}[font=\footnotesize, scale=0.5, line width=1.05]
	

	\ifpersistencelayer
		\drawHexagon[draw=none]{ 2                      }{ 2}{dtu-blue}{}{2}
		\drawHexagon[draw=none]{{6 - 2 * (2 - sqrt(3)) }}{ 2}{dtu-blue}{}{2}
		\drawHexagon[draw=none]{{4 - 1 * (2 - sqrt(3)) }}{-1}{dtu-blue}{Persistence}{2}
		\drawHexagon[draw=none]{{0 + 1 * (2 - sqrt(3)) }}{-1}{dtu-blue}{}{2}
		\drawHexagon[draw=none]{{8 - 3 * (2 - sqrt(3)) }}{-1}{dtu-blue}{}{2}

		\drawHexagon[draw=none]{{2 - 0 * (2 - sqrt(3)) }}{-4}{dtu-blue}{}{2}
		\drawHexagon[draw=none]{{6 - 2 * (2 - sqrt(3)) }}{-4}{dtu-blue}{}{2}

		\drawHexagon[draw=none]{{10 - 4 * (2 - sqrt(3)) }}{-4}{dtu-blue}{}{2}
		\drawHexagon[draw=none]{{-2 + 2 * (2 - sqrt(3)) }}{-4}{dtu-blue}{}{2}

		\drawHexagon[draw=none]{{12 - 5 * (2 - sqrt(3)) }}{-1}{dtu-blue}{}{2}
		\drawHexagon[draw=none]{{-4 + 3 * (2 - sqrt(3)) }}{-1}{dtu-blue}{}{2}
		% Legend for each layer
		\drawHexagon{{14.0  }}{+3.0}{dtu-blue}{}{0.75}
		\node[align=right, anchor=west] at ({15.0}, +3.75) {Persistence};
		\drawHexagon{{14.0  }}{+1.5}{dtu-white}{}{0.75}
		\node[align=right, anchor=west] at ({15.0}, +2.25) {Atomic Pointer};
		\drawHexagon{{14.0  }}{+0.0}{dtu-white}{}{0.75}
		\node[align=right, anchor=west] at ({15.0}, +0.75) {Metaheuristics};
		\drawHexagon{{14.0  }}{-1.5}{dtu-white}{}{0.75}
		\node[align=right, anchor=west] at ({15.0}, -0.75) {Orchestration};
		\drawHexagon{{14.0  }}{-3.0}{dtu-white}{}{0.75}
		\node[align=right, anchor=west] at ({15.0}, -2.25) {User interfaces};
	\fi


	\ifatomicpointerswaplayer
		\drawHexagon[]{ 2                      }{ 2}{dtu-green}{Shared\\solution\\pointer}{2}
		\drawHexagon[]{{6 - 2 * (2 - sqrt(3)) }}{ 2}{dtu-green}{Shared\\solution\\pointer}{2}
		\drawHexagon[]{{4 - 1 * (2 - sqrt(3)) }}{-1}{dtu-green}{Shared\\solution\\pointer}{2}
		\drawHexagon[]{{0 + 1 * (2 - sqrt(3)) }}{-1}{dtu-green}{Shared\\solution\\pointer}{2}
		\drawHexagon[]{{8 - 3 * (2 - sqrt(3)) }}{-1}{dtu-green}{Shared\\solution\\pointer}{2}

		\drawHexagon[]{{2 - 0 * (2 - sqrt(3)) }}{-4}{dtu-green}{Shared\\solution\\pointer}{2}
		\drawHexagon[]{{6 - 2 * (2 - sqrt(3)) }}{-4}{dtu-green}{Shared\\solution\\pointer}{2}

		\drawHexagon[]{{10 - 4 * (2 - sqrt(3)) }}{-4}{dtu-green}{Shared\\solution\\pointer}{2}
		\drawHexagon[]{{-2 + 2 * (2 - sqrt(3)) }}{-4}{dtu-green}{Shared\\solution\\pointer}{2}

		\drawHexagon[]{{12 - 5 * (2 - sqrt(3)) }}{-1}{dtu-green}{Shared\\solution\\pointer}{2}
		\drawHexagon[]{{-4 + 3 * (2 - sqrt(3)) }}{-1}{dtu-green}{Shared\\solution\\pointer}{2}
		% Legend for each layer
		\drawHexagon{{14.0  }}{+3.0}{dtu-white}{}{0.75}
		\node[align=right, anchor=west] at ({15.0}, +3.75) {Persistence};
		\drawHexagon{{14.0  }}{+1.5}{dtu-green}{}{0.75}
		\node[align=right, anchor=west] at ({15.0}, +2.25) {Atomic Pointer};
		\drawHexagon{{14.0  }}{+0.0}{dtu-white}{}{0.75}
		\node[align=right, anchor=west] at ({15.0}, +0.75) {Metaheuristics};
		\drawHexagon{{14.0  }}{-1.5}{dtu-white}{}{0.75}
		\node[align=right, anchor=west] at ({15.0}, -0.75) {Orchestration};
		\drawHexagon{{14.0  }}{-3.0}{dtu-white}{}{0.75}
		\node[align=right, anchor=west] at ({15.0}, -2.25) {User interfaces};
	\fi

	\ifsimplifiedlayer

		\node[align=right, anchor=west] at ({-5.5}, +3.75) {};
		\drawHexagon{{+2 + 0 * (2 - sqrt(3)) }}{ 2}{dtu-green}{Scheduler}{2}
		\drawHexagon{{+4 - 1 * (2 - sqrt(3)) }}{-1}{dtu-red}{Supervisor}{2}
		\drawHexagon{{+0 + 1 * (2 - sqrt(3)) }}{-1}{dtu-red}{Supervisor}{2}
		\drawHexagon{{+2 - 0 * (2 - sqrt(3)) }}{-4}{dtu-corporate-red}{Technician}{2}
		\drawHexagon{{+6 - 2 * (2 - sqrt(3)) }}{-4}{dtu-corporate-red}{Technician}{2}
		\drawHexagon{{-2 + 2 * (2 - sqrt(3)) }}{-4}{dtu-corporate-red}{Technician}{2}
		\drawHexagon{{+8 - 3 * (2 - sqrt(3)) }}{-1}{dtu-corporate-red}{Technician}{2}
		\drawHexagon{{-4 + 3 * (2 - sqrt(3)) }}{-1}{dtu-corporate-red}{Technician}{2}

		% Scheduler
		\draw[thin, fill=dtu-yellow] (2, 5) circle (0.35);
		\draw[thin, fill=dtu-purple] (2, 3) circle (0.35);
		% Supervisor 1
		\draw[thin, fill=dtu-yellow] ({+4 - 1 * (2 - sqrt(3)) }, 02) circle (0.35);
		\draw[thin, fill=dtu-purple] ({+4 - 1 * (2 - sqrt(3)) }, -0) circle (0.35);
		% Supervisor 2
		\draw[thin, fill=dtu-yellow] ({+0 + 1 * (2 - sqrt(3)) }, 02) circle (0.35);
		\draw[thin, fill=dtu-purple] ({+0 + 1 * (2 - sqrt(3)) }, -0) circle (0.35);
		% Technician 1
		\draw[thin, fill=dtu-yellow] ({+2 - 0 * (2 - sqrt(3)) }, -1) circle (0.35);
		\draw[thin, fill=dtu-purple] ({+2 - 0 * (2 - sqrt(3)) }, -3) circle (0.35);
		% Technician 2
		\draw[thin, fill=dtu-yellow] ({+6 - 2 * (2 - sqrt(3)) }, -1) circle (0.35);
		\draw[thin, fill=dtu-purple] ({+6 - 2 * (2 - sqrt(3)) }, -3) circle (0.35);
		% Technician 3
		\draw[thin, fill=dtu-yellow] ({-2 + 2 * (2 - sqrt(3)) }, -1) circle (0.35);
		\draw[thin, fill=dtu-purple] ({-2 + 2 * (2 - sqrt(3)) }, -3) circle (0.35);
		% Technician 4
		\draw[thin, fill=dtu-yellow] ({+8 - 3 * (2 - sqrt(3)) }, 02) circle (0.35);
		\draw[thin, fill=dtu-purple] ({+8 - 3 * (2 - sqrt(3)) }, -0) circle (0.35);
		% Technician 5
		\draw[thin, fill=dtu-yellow] ({-4 + 3 * (2 - sqrt(3)) }, 02) circle (0.35);
		\draw[thin, fill=dtu-purple] ({-4 + 3 * (2 - sqrt(3)) }, -0) circle (0.35);

		% Legend for each layer
		\node[align=right, anchor=west] at ({12.0}, +3.75) {Atomic Pointer};
		\draw[fill=dtu-purple] (11.0,  +3.75) circle (0.5);

		\node[align=right, anchor=west] at ({12.0}, +2.25) {Scheduler Metaheuristic};
		\drawHexagon{{11.0  }}{+1.75}{dtu-green}{}{0.5}
		\node[align=right, anchor=west] at ({12.0}, +0.75) {Supervisor Metaheuristic};
		\drawHexagon{{11.0  }}{+0.25}{dtu-red}{}{0.5}
		\node[align=right, anchor=west] at ({12.0}, -0.75) {Technician Metaheuristic};
		\drawHexagon{{11.0  }}{-1.25}{dtu-corporate-red}{}{0.5}
		\node[align=right, anchor=west] at ({12.0}, -2.25) {User interfaces (Message Passing)};
		\draw[fill=dtu-yellow] (11.0, -2.25) circle (0.5);
	\fi

	\ifmetaheuristicslayer
		\drawHexagon{ 2                      }{ 2}{dtu-blue}{Strategic}{2}
		\drawHexagon{{6 - 2 * (2 - sqrt(3)) }}{ 2}{dtu-green}{Tactical}{2}
		\drawHexagon{{4 - 1 * (2 - sqrt(3)) }}{-1}{dtu-red}{Supervisor}{2}
		\drawHexagon{{0 + 1 * (2 - sqrt(3)) }}{-1}{dtu-red}{Supervisor}{2}
		\drawHexagon{{8 - 3 * (2 - sqrt(3)) }}{-1}{dtu-red}{Supervisor}{2}

		\drawHexagon{{2 - 0 * (2 - sqrt(3)) }}{-4}{dtu-corporate-red}{Technician}{2}
		\drawHexagon{{6 - 2 * (2 - sqrt(3)) }}{-4}{dtu-corporate-red}{Technician}{2}

		\drawHexagon{{10 - 4 * (2 - sqrt(3)) }}{-4}{dtu-corporate-red}{Technician}{2}
		\drawHexagon{{-2 + 2 * (2 - sqrt(3)) }}{-4}{dtu-corporate-red}{Technician}{2}

		\drawHexagon{{12 - 5 * (2 - sqrt(3)) }}{-1}{dtu-corporate-red}{Technician}{2}
		\drawHexagon{{-4 + 3 * (2 - sqrt(3)) }}{-1}{dtu-corporate-red}{Technician}{2}

		% Legend for each layer
		\drawHexagon{{14.0  }}{+3.0}{dtu-white}{}{0.75}
		\node[align=right, anchor=west] at ({15.0}, +3.75) {Persistence};
		\drawHexagon{{14.0  }}{+1.5}{dtu-white}{}{0.75}
		\node[align=right, anchor=west] at ({15.0}, +2.25) {Atomic Pointer};
		\drawHexagon{{14.0  }}{+0.0}{dtu-corporate-red}{}{0.75}
		\node[align=right, anchor=west] at ({15.0}, +0.75) {Metaheuristics};
		\drawHexagon{{14.0  }}{-1.5}{dtu-white}{}{0.75}
		\node[align=right, anchor=west] at ({15.0}, -0.75) {Orchestration};
		\drawHexagon{{14.0  }}{-3.0}{dtu-white}{}{0.75}
		\node[align=right, anchor=west] at ({15.0}, -2.25) {User interfaces};
	\fi

	\iforchestratorlayer
		\drawHexagon{ 2                      }{ 2}{dtu-orange}{}{2}
		\drawHexagon{{6 - 2 * (2 - sqrt(3)) }}{ 2}{dtu-orange}{}{2}
		\drawHexagon{{4 - 1 * (2 - sqrt(3)) }}{-1}{dtu-orange}{Orche-\\strator}{2}
		\drawHexagon{{0 + 1 * (2 - sqrt(3)) }}{-1}{dtu-orange}{}{2}
		\drawHexagon{{8 - 3 * (2 - sqrt(3)) }}{-1}{dtu-orange}{}{2}

		\drawHexagon{{2 - 0 * (2 - sqrt(3)) }}{-4}{dtu-orange}{}{2}
		\drawHexagon{{6 - 2 * (2 - sqrt(3)) }}{-4}{dtu-orange}{}{2}

		\drawHexagon{{10 - 4 * (2 - sqrt(3)) }}{-4}{dtu-orange}{}{2}
		\drawHexagon{{-2 + 2 * (2 - sqrt(3)) }}{-4}{dtu-orange}{}{2}

		\drawHexagon{{12 - 5 * (2 - sqrt(3)) }}{-1}{dtu-orange}{}{2}
		\drawHexagon{{-4 + 3 * (2 - sqrt(3)) }}{-1}{dtu-orange}{}{2}
		% Legend for each layer
		\drawHexagon{{14.0  }}{+3.0}{dtu-white}{}{0.75}
		\node[align=right, anchor=west] at ({15.0}, +3.75) {Persistence};
		\drawHexagon{{14.0  }}{+1.5}{dtu-white}{}{0.75}
		\node[align=right, anchor=west] at ({15.0}, +2.25) {Atomic Pointer};
		\drawHexagon{{14.0  }}{+0.0}{dtu-white}{}{0.75}
		\node[align=right, anchor=west] at ({15.0}, +0.75) {Metaheuristics};
		\drawHexagon{{14.0  }}{-1.5}{dtu-orange}{}{0.75}
		\node[align=right, anchor=west] at ({15.0}, -0.75) {Orchestration};
		\drawHexagon{{14.0  }}{-3.0}{dtu-white}{}{0.75}
		\node[align=right, anchor=west] at ({15.0}, -2.25) {User interfaces};
	\fi

	
	\ifuserinterfacelayer
		\drawHexagon{ 2                      }{ 2}{dtu-yellow}{UI}{2}
		\drawHexagon{{6 - 2 * (2 - sqrt(3)) }}{ 2}{dtu-yellow}{UI}{2}
		\drawHexagon{{4 - 1 * (2 - sqrt(3)) }}{-1}{dtu-yellow}{UI}{2}
		\drawHexagon{{0 + 1 * (2 - sqrt(3)) }}{-1}{dtu-yellow}{UI}{2}
		\drawHexagon{{8 - 3 * (2 - sqrt(3)) }}{-1}{dtu-yellow}{UI}{2}

		\drawHexagon{{2 - 0 * (2 - sqrt(3)) }}{-4}{dtu-yellow}{UI}{2}
		\drawHexagon{{6 - 2 * (2 - sqrt(3)) }}{-4}{dtu-yellow}{UI}{2}

		\drawHexagon{{10 - 4 * (2 - sqrt(3)) }}{-4}{dtu-yellow}{UI}{2}
		\drawHexagon{{-2 + 2 * (2 - sqrt(3)) }}{-4}{dtu-yellow}{UI}{2}

		\drawHexagon{{12 - 5 * (2 - sqrt(3)) }}{-1}{dtu-yellow}{UI}{2}
		\drawHexagon{{-4 + 3 * (2 - sqrt(3)) }}{-1}{dtu-yellow}{UI}{2}
		% Legend for each layer
		\drawHexagon{{14.0  }}{+3.0}{dtu-white}{}{0.75}
		\node[align=right, anchor=west] at ({15.0}, +3.75) {Persistence};
		\drawHexagon{{14.0  }}{+1.5}{dtu-white}{}{0.75}
		\node[align=right, anchor=west] at ({15.0}, +2.25) {Atomic Pointer};
		\drawHexagon{{14.0  }}{+0.0}{dtu-white}{}{0.75}
		\node[align=right, anchor=west] at ({15.0}, +0.75) {Metaheuristics};
		\drawHexagon{{14.0  }}{-1.5}{dtu-white}{}{0.75}
		\node[align=right, anchor=west] at ({15.0}, -0.75) {Orchestration};
		\drawHexagon{{14.0  }}{-3.0}{dtu-yellow}{}{0.75}
		\node[align=right, anchor=west] at ({15.0}, -2.25) {User interfaces};
	\fi
	
	\end{tikzpicture}
}

	% \resizebox{0.7\textwidth}{!}{
		\drawModelSetupHexagon[atomicpointerswap=true]
	% }
	\caption{
		Overview of the scheduling process when modelled as actors. When LNS is encapsulated 
		is an actor it becomes possible to optimize parts of a large process individually instead of 
		optimizing the scheduling problem globally from a single model implementation.
	}
	\label{fig:ordinator-hexagon:atomicpointerswap}
\end{figure}

\begin{figure}[H]
	\centering
	\usetikzlibrary {positioning}
\newcommand{\drawHexagon}[6][draw=black]{
	\draw[#1, fill=#4] (#2,#3) ++(30:#6) -- ++(90:#6) -- ++(150:#6) -- ++(210:#6) -- ++(270:#6) -- ++(330:#6) -- cycle;
	\node[align=center] at (#2,#3+2) {#5};
}

\newif\ifpersistencelayer
\newif\ifatomicpointerswaplayer
\newif\ifmetaheuristicslayer
\newif\ifuserinterfacelayer
\newif\iforchestratorlayer
\newif\ifsimplifiedlayer

\pgfkeys{
	/hexagon/.is family, /hexagon,
	default/.style = {
		persistence=false,
		atomicpointerswap=false,
		metaheuristics=false,
		orchestrator=false,
		userinterface=false,
		simplified=false,
	},
	persistence/.is if=persistencelayer,
	atomicpointerswap/.is if=atomicpointerswaplayer,
	metaheuristics/.is if=metaheuristicslayer,
	orchestrator/.is if=orchestratorlayer,
	userinterface/.is if=userinterfacelayer,
	simplified/.is if=simplifiedlayer,
}
\newcommand{\drawModelSetupHexagon}[1][]{
	\pgfkeys{/hexagon, default, #1}

	\begin{tikzpicture}[font=\footnotesize, scale=0.5, line width=1.05]
	

	\ifpersistencelayer
		\drawHexagon[draw=none]{ 2                      }{ 2}{dtu-blue}{}{2}
		\drawHexagon[draw=none]{{6 - 2 * (2 - sqrt(3)) }}{ 2}{dtu-blue}{}{2}
		\drawHexagon[draw=none]{{4 - 1 * (2 - sqrt(3)) }}{-1}{dtu-blue}{Persistence}{2}
		\drawHexagon[draw=none]{{0 + 1 * (2 - sqrt(3)) }}{-1}{dtu-blue}{}{2}
		\drawHexagon[draw=none]{{8 - 3 * (2 - sqrt(3)) }}{-1}{dtu-blue}{}{2}

		\drawHexagon[draw=none]{{2 - 0 * (2 - sqrt(3)) }}{-4}{dtu-blue}{}{2}
		\drawHexagon[draw=none]{{6 - 2 * (2 - sqrt(3)) }}{-4}{dtu-blue}{}{2}

		\drawHexagon[draw=none]{{10 - 4 * (2 - sqrt(3)) }}{-4}{dtu-blue}{}{2}
		\drawHexagon[draw=none]{{-2 + 2 * (2 - sqrt(3)) }}{-4}{dtu-blue}{}{2}

		\drawHexagon[draw=none]{{12 - 5 * (2 - sqrt(3)) }}{-1}{dtu-blue}{}{2}
		\drawHexagon[draw=none]{{-4 + 3 * (2 - sqrt(3)) }}{-1}{dtu-blue}{}{2}
		% Legend for each layer
		\drawHexagon{{14.0  }}{+3.0}{dtu-blue}{}{0.75}
		\node[align=right, anchor=west] at ({15.0}, +3.75) {Persistence};
		\drawHexagon{{14.0  }}{+1.5}{dtu-white}{}{0.75}
		\node[align=right, anchor=west] at ({15.0}, +2.25) {Atomic Pointer};
		\drawHexagon{{14.0  }}{+0.0}{dtu-white}{}{0.75}
		\node[align=right, anchor=west] at ({15.0}, +0.75) {Metaheuristics};
		\drawHexagon{{14.0  }}{-1.5}{dtu-white}{}{0.75}
		\node[align=right, anchor=west] at ({15.0}, -0.75) {Orchestration};
		\drawHexagon{{14.0  }}{-3.0}{dtu-white}{}{0.75}
		\node[align=right, anchor=west] at ({15.0}, -2.25) {User interfaces};
	\fi


	\ifatomicpointerswaplayer
		\drawHexagon[]{ 2                      }{ 2}{dtu-green}{Shared\\solution\\pointer}{2}
		\drawHexagon[]{{6 - 2 * (2 - sqrt(3)) }}{ 2}{dtu-green}{Shared\\solution\\pointer}{2}
		\drawHexagon[]{{4 - 1 * (2 - sqrt(3)) }}{-1}{dtu-green}{Shared\\solution\\pointer}{2}
		\drawHexagon[]{{0 + 1 * (2 - sqrt(3)) }}{-1}{dtu-green}{Shared\\solution\\pointer}{2}
		\drawHexagon[]{{8 - 3 * (2 - sqrt(3)) }}{-1}{dtu-green}{Shared\\solution\\pointer}{2}

		\drawHexagon[]{{2 - 0 * (2 - sqrt(3)) }}{-4}{dtu-green}{Shared\\solution\\pointer}{2}
		\drawHexagon[]{{6 - 2 * (2 - sqrt(3)) }}{-4}{dtu-green}{Shared\\solution\\pointer}{2}

		\drawHexagon[]{{10 - 4 * (2 - sqrt(3)) }}{-4}{dtu-green}{Shared\\solution\\pointer}{2}
		\drawHexagon[]{{-2 + 2 * (2 - sqrt(3)) }}{-4}{dtu-green}{Shared\\solution\\pointer}{2}

		\drawHexagon[]{{12 - 5 * (2 - sqrt(3)) }}{-1}{dtu-green}{Shared\\solution\\pointer}{2}
		\drawHexagon[]{{-4 + 3 * (2 - sqrt(3)) }}{-1}{dtu-green}{Shared\\solution\\pointer}{2}
		% Legend for each layer
		\drawHexagon{{14.0  }}{+3.0}{dtu-white}{}{0.75}
		\node[align=right, anchor=west] at ({15.0}, +3.75) {Persistence};
		\drawHexagon{{14.0  }}{+1.5}{dtu-green}{}{0.75}
		\node[align=right, anchor=west] at ({15.0}, +2.25) {Atomic Pointer};
		\drawHexagon{{14.0  }}{+0.0}{dtu-white}{}{0.75}
		\node[align=right, anchor=west] at ({15.0}, +0.75) {Metaheuristics};
		\drawHexagon{{14.0  }}{-1.5}{dtu-white}{}{0.75}
		\node[align=right, anchor=west] at ({15.0}, -0.75) {Orchestration};
		\drawHexagon{{14.0  }}{-3.0}{dtu-white}{}{0.75}
		\node[align=right, anchor=west] at ({15.0}, -2.25) {User interfaces};
	\fi

	\ifsimplifiedlayer

		\node[align=right, anchor=west] at ({-5.5}, +3.75) {};
		\drawHexagon{{+2 + 0 * (2 - sqrt(3)) }}{ 2}{dtu-green}{Scheduler}{2}
		\drawHexagon{{+4 - 1 * (2 - sqrt(3)) }}{-1}{dtu-red}{Supervisor}{2}
		\drawHexagon{{+0 + 1 * (2 - sqrt(3)) }}{-1}{dtu-red}{Supervisor}{2}
		\drawHexagon{{+2 - 0 * (2 - sqrt(3)) }}{-4}{dtu-corporate-red}{Technician}{2}
		\drawHexagon{{+6 - 2 * (2 - sqrt(3)) }}{-4}{dtu-corporate-red}{Technician}{2}
		\drawHexagon{{-2 + 2 * (2 - sqrt(3)) }}{-4}{dtu-corporate-red}{Technician}{2}
		\drawHexagon{{+8 - 3 * (2 - sqrt(3)) }}{-1}{dtu-corporate-red}{Technician}{2}
		\drawHexagon{{-4 + 3 * (2 - sqrt(3)) }}{-1}{dtu-corporate-red}{Technician}{2}

		% Scheduler
		\draw[thin, fill=dtu-yellow] (2, 5) circle (0.35);
		\draw[thin, fill=dtu-purple] (2, 3) circle (0.35);
		% Supervisor 1
		\draw[thin, fill=dtu-yellow] ({+4 - 1 * (2 - sqrt(3)) }, 02) circle (0.35);
		\draw[thin, fill=dtu-purple] ({+4 - 1 * (2 - sqrt(3)) }, -0) circle (0.35);
		% Supervisor 2
		\draw[thin, fill=dtu-yellow] ({+0 + 1 * (2 - sqrt(3)) }, 02) circle (0.35);
		\draw[thin, fill=dtu-purple] ({+0 + 1 * (2 - sqrt(3)) }, -0) circle (0.35);
		% Technician 1
		\draw[thin, fill=dtu-yellow] ({+2 - 0 * (2 - sqrt(3)) }, -1) circle (0.35);
		\draw[thin, fill=dtu-purple] ({+2 - 0 * (2 - sqrt(3)) }, -3) circle (0.35);
		% Technician 2
		\draw[thin, fill=dtu-yellow] ({+6 - 2 * (2 - sqrt(3)) }, -1) circle (0.35);
		\draw[thin, fill=dtu-purple] ({+6 - 2 * (2 - sqrt(3)) }, -3) circle (0.35);
		% Technician 3
		\draw[thin, fill=dtu-yellow] ({-2 + 2 * (2 - sqrt(3)) }, -1) circle (0.35);
		\draw[thin, fill=dtu-purple] ({-2 + 2 * (2 - sqrt(3)) }, -3) circle (0.35);
		% Technician 4
		\draw[thin, fill=dtu-yellow] ({+8 - 3 * (2 - sqrt(3)) }, 02) circle (0.35);
		\draw[thin, fill=dtu-purple] ({+8 - 3 * (2 - sqrt(3)) }, -0) circle (0.35);
		% Technician 5
		\draw[thin, fill=dtu-yellow] ({-4 + 3 * (2 - sqrt(3)) }, 02) circle (0.35);
		\draw[thin, fill=dtu-purple] ({-4 + 3 * (2 - sqrt(3)) }, -0) circle (0.35);

		% Legend for each layer
		\node[align=right, anchor=west] at ({12.0}, +3.75) {Atomic Pointer};
		\draw[fill=dtu-purple] (11.0,  +3.75) circle (0.5);

		\node[align=right, anchor=west] at ({12.0}, +2.25) {Scheduler Metaheuristic};
		\drawHexagon{{11.0  }}{+1.75}{dtu-green}{}{0.5}
		\node[align=right, anchor=west] at ({12.0}, +0.75) {Supervisor Metaheuristic};
		\drawHexagon{{11.0  }}{+0.25}{dtu-red}{}{0.5}
		\node[align=right, anchor=west] at ({12.0}, -0.75) {Technician Metaheuristic};
		\drawHexagon{{11.0  }}{-1.25}{dtu-corporate-red}{}{0.5}
		\node[align=right, anchor=west] at ({12.0}, -2.25) {User interfaces (Message Passing)};
		\draw[fill=dtu-yellow] (11.0, -2.25) circle (0.5);
	\fi

	\ifmetaheuristicslayer
		\drawHexagon{ 2                      }{ 2}{dtu-blue}{Strategic}{2}
		\drawHexagon{{6 - 2 * (2 - sqrt(3)) }}{ 2}{dtu-green}{Tactical}{2}
		\drawHexagon{{4 - 1 * (2 - sqrt(3)) }}{-1}{dtu-red}{Supervisor}{2}
		\drawHexagon{{0 + 1 * (2 - sqrt(3)) }}{-1}{dtu-red}{Supervisor}{2}
		\drawHexagon{{8 - 3 * (2 - sqrt(3)) }}{-1}{dtu-red}{Supervisor}{2}

		\drawHexagon{{2 - 0 * (2 - sqrt(3)) }}{-4}{dtu-corporate-red}{Technician}{2}
		\drawHexagon{{6 - 2 * (2 - sqrt(3)) }}{-4}{dtu-corporate-red}{Technician}{2}

		\drawHexagon{{10 - 4 * (2 - sqrt(3)) }}{-4}{dtu-corporate-red}{Technician}{2}
		\drawHexagon{{-2 + 2 * (2 - sqrt(3)) }}{-4}{dtu-corporate-red}{Technician}{2}

		\drawHexagon{{12 - 5 * (2 - sqrt(3)) }}{-1}{dtu-corporate-red}{Technician}{2}
		\drawHexagon{{-4 + 3 * (2 - sqrt(3)) }}{-1}{dtu-corporate-red}{Technician}{2}

		% Legend for each layer
		\drawHexagon{{14.0  }}{+3.0}{dtu-white}{}{0.75}
		\node[align=right, anchor=west] at ({15.0}, +3.75) {Persistence};
		\drawHexagon{{14.0  }}{+1.5}{dtu-white}{}{0.75}
		\node[align=right, anchor=west] at ({15.0}, +2.25) {Atomic Pointer};
		\drawHexagon{{14.0  }}{+0.0}{dtu-corporate-red}{}{0.75}
		\node[align=right, anchor=west] at ({15.0}, +0.75) {Metaheuristics};
		\drawHexagon{{14.0  }}{-1.5}{dtu-white}{}{0.75}
		\node[align=right, anchor=west] at ({15.0}, -0.75) {Orchestration};
		\drawHexagon{{14.0  }}{-3.0}{dtu-white}{}{0.75}
		\node[align=right, anchor=west] at ({15.0}, -2.25) {User interfaces};
	\fi

	\iforchestratorlayer
		\drawHexagon{ 2                      }{ 2}{dtu-orange}{}{2}
		\drawHexagon{{6 - 2 * (2 - sqrt(3)) }}{ 2}{dtu-orange}{}{2}
		\drawHexagon{{4 - 1 * (2 - sqrt(3)) }}{-1}{dtu-orange}{Orche-\\strator}{2}
		\drawHexagon{{0 + 1 * (2 - sqrt(3)) }}{-1}{dtu-orange}{}{2}
		\drawHexagon{{8 - 3 * (2 - sqrt(3)) }}{-1}{dtu-orange}{}{2}

		\drawHexagon{{2 - 0 * (2 - sqrt(3)) }}{-4}{dtu-orange}{}{2}
		\drawHexagon{{6 - 2 * (2 - sqrt(3)) }}{-4}{dtu-orange}{}{2}

		\drawHexagon{{10 - 4 * (2 - sqrt(3)) }}{-4}{dtu-orange}{}{2}
		\drawHexagon{{-2 + 2 * (2 - sqrt(3)) }}{-4}{dtu-orange}{}{2}

		\drawHexagon{{12 - 5 * (2 - sqrt(3)) }}{-1}{dtu-orange}{}{2}
		\drawHexagon{{-4 + 3 * (2 - sqrt(3)) }}{-1}{dtu-orange}{}{2}
		% Legend for each layer
		\drawHexagon{{14.0  }}{+3.0}{dtu-white}{}{0.75}
		\node[align=right, anchor=west] at ({15.0}, +3.75) {Persistence};
		\drawHexagon{{14.0  }}{+1.5}{dtu-white}{}{0.75}
		\node[align=right, anchor=west] at ({15.0}, +2.25) {Atomic Pointer};
		\drawHexagon{{14.0  }}{+0.0}{dtu-white}{}{0.75}
		\node[align=right, anchor=west] at ({15.0}, +0.75) {Metaheuristics};
		\drawHexagon{{14.0  }}{-1.5}{dtu-orange}{}{0.75}
		\node[align=right, anchor=west] at ({15.0}, -0.75) {Orchestration};
		\drawHexagon{{14.0  }}{-3.0}{dtu-white}{}{0.75}
		\node[align=right, anchor=west] at ({15.0}, -2.25) {User interfaces};
	\fi

	
	\ifuserinterfacelayer
		\drawHexagon{ 2                      }{ 2}{dtu-yellow}{UI}{2}
		\drawHexagon{{6 - 2 * (2 - sqrt(3)) }}{ 2}{dtu-yellow}{UI}{2}
		\drawHexagon{{4 - 1 * (2 - sqrt(3)) }}{-1}{dtu-yellow}{UI}{2}
		\drawHexagon{{0 + 1 * (2 - sqrt(3)) }}{-1}{dtu-yellow}{UI}{2}
		\drawHexagon{{8 - 3 * (2 - sqrt(3)) }}{-1}{dtu-yellow}{UI}{2}

		\drawHexagon{{2 - 0 * (2 - sqrt(3)) }}{-4}{dtu-yellow}{UI}{2}
		\drawHexagon{{6 - 2 * (2 - sqrt(3)) }}{-4}{dtu-yellow}{UI}{2}

		\drawHexagon{{10 - 4 * (2 - sqrt(3)) }}{-4}{dtu-yellow}{UI}{2}
		\drawHexagon{{-2 + 2 * (2 - sqrt(3)) }}{-4}{dtu-yellow}{UI}{2}

		\drawHexagon{{12 - 5 * (2 - sqrt(3)) }}{-1}{dtu-yellow}{UI}{2}
		\drawHexagon{{-4 + 3 * (2 - sqrt(3)) }}{-1}{dtu-yellow}{UI}{2}
		% Legend for each layer
		\drawHexagon{{14.0  }}{+3.0}{dtu-white}{}{0.75}
		\node[align=right, anchor=west] at ({15.0}, +3.75) {Persistence};
		\drawHexagon{{14.0  }}{+1.5}{dtu-white}{}{0.75}
		\node[align=right, anchor=west] at ({15.0}, +2.25) {Atomic Pointer};
		\drawHexagon{{14.0  }}{+0.0}{dtu-white}{}{0.75}
		\node[align=right, anchor=west] at ({15.0}, +0.75) {Metaheuristics};
		\drawHexagon{{14.0  }}{-1.5}{dtu-white}{}{0.75}
		\node[align=right, anchor=west] at ({15.0}, -0.75) {Orchestration};
		\drawHexagon{{14.0  }}{-3.0}{dtu-yellow}{}{0.75}
		\node[align=right, anchor=west] at ({15.0}, -2.25) {User interfaces};
	\fi
	
	\end{tikzpicture}
}

	\resizebox{0.7\textwidth}{!}{
		\drawModelSetupHexagon[metaheuristics=true]
	}
	\caption{
		Overview of the scheduling process when modelled as actors. When LNS is encapsulated 
		is an actor it becomes possible to optimize parts of a large process individually instead of 
		optimizing the scheduling problem globally from a single model implementation.
	}
	\label{fig:ordinator-hexagon:metaheuristics}
\end{figure}

\begin{figure}[H]
	\centering
	\usetikzlibrary {positioning}
\newcommand{\drawHexagon}[6][draw=black]{
	\draw[#1, fill=#4] (#2,#3) ++(30:#6) -- ++(90:#6) -- ++(150:#6) -- ++(210:#6) -- ++(270:#6) -- ++(330:#6) -- cycle;
	\node[align=center] at (#2,#3+2) {#5};
}

\newif\ifpersistencelayer
\newif\ifatomicpointerswaplayer
\newif\ifmetaheuristicslayer
\newif\ifuserinterfacelayer
\newif\iforchestratorlayer
\newif\ifsimplifiedlayer

\pgfkeys{
	/hexagon/.is family, /hexagon,
	default/.style = {
		persistence=false,
		atomicpointerswap=false,
		metaheuristics=false,
		orchestrator=false,
		userinterface=false,
		simplified=false,
	},
	persistence/.is if=persistencelayer,
	atomicpointerswap/.is if=atomicpointerswaplayer,
	metaheuristics/.is if=metaheuristicslayer,
	orchestrator/.is if=orchestratorlayer,
	userinterface/.is if=userinterfacelayer,
	simplified/.is if=simplifiedlayer,
}
\newcommand{\drawModelSetupHexagon}[1][]{
	\pgfkeys{/hexagon, default, #1}

	\begin{tikzpicture}[font=\footnotesize, scale=0.5, line width=1.05]
	

	\ifpersistencelayer
		\drawHexagon[draw=none]{ 2                      }{ 2}{dtu-blue}{}{2}
		\drawHexagon[draw=none]{{6 - 2 * (2 - sqrt(3)) }}{ 2}{dtu-blue}{}{2}
		\drawHexagon[draw=none]{{4 - 1 * (2 - sqrt(3)) }}{-1}{dtu-blue}{Persistence}{2}
		\drawHexagon[draw=none]{{0 + 1 * (2 - sqrt(3)) }}{-1}{dtu-blue}{}{2}
		\drawHexagon[draw=none]{{8 - 3 * (2 - sqrt(3)) }}{-1}{dtu-blue}{}{2}

		\drawHexagon[draw=none]{{2 - 0 * (2 - sqrt(3)) }}{-4}{dtu-blue}{}{2}
		\drawHexagon[draw=none]{{6 - 2 * (2 - sqrt(3)) }}{-4}{dtu-blue}{}{2}

		\drawHexagon[draw=none]{{10 - 4 * (2 - sqrt(3)) }}{-4}{dtu-blue}{}{2}
		\drawHexagon[draw=none]{{-2 + 2 * (2 - sqrt(3)) }}{-4}{dtu-blue}{}{2}

		\drawHexagon[draw=none]{{12 - 5 * (2 - sqrt(3)) }}{-1}{dtu-blue}{}{2}
		\drawHexagon[draw=none]{{-4 + 3 * (2 - sqrt(3)) }}{-1}{dtu-blue}{}{2}
		% Legend for each layer
		\drawHexagon{{14.0  }}{+3.0}{dtu-blue}{}{0.75}
		\node[align=right, anchor=west] at ({15.0}, +3.75) {Persistence};
		\drawHexagon{{14.0  }}{+1.5}{dtu-white}{}{0.75}
		\node[align=right, anchor=west] at ({15.0}, +2.25) {Atomic Pointer};
		\drawHexagon{{14.0  }}{+0.0}{dtu-white}{}{0.75}
		\node[align=right, anchor=west] at ({15.0}, +0.75) {Metaheuristics};
		\drawHexagon{{14.0  }}{-1.5}{dtu-white}{}{0.75}
		\node[align=right, anchor=west] at ({15.0}, -0.75) {Orchestration};
		\drawHexagon{{14.0  }}{-3.0}{dtu-white}{}{0.75}
		\node[align=right, anchor=west] at ({15.0}, -2.25) {User interfaces};
	\fi


	\ifatomicpointerswaplayer
		\drawHexagon[]{ 2                      }{ 2}{dtu-green}{Shared\\solution\\pointer}{2}
		\drawHexagon[]{{6 - 2 * (2 - sqrt(3)) }}{ 2}{dtu-green}{Shared\\solution\\pointer}{2}
		\drawHexagon[]{{4 - 1 * (2 - sqrt(3)) }}{-1}{dtu-green}{Shared\\solution\\pointer}{2}
		\drawHexagon[]{{0 + 1 * (2 - sqrt(3)) }}{-1}{dtu-green}{Shared\\solution\\pointer}{2}
		\drawHexagon[]{{8 - 3 * (2 - sqrt(3)) }}{-1}{dtu-green}{Shared\\solution\\pointer}{2}

		\drawHexagon[]{{2 - 0 * (2 - sqrt(3)) }}{-4}{dtu-green}{Shared\\solution\\pointer}{2}
		\drawHexagon[]{{6 - 2 * (2 - sqrt(3)) }}{-4}{dtu-green}{Shared\\solution\\pointer}{2}

		\drawHexagon[]{{10 - 4 * (2 - sqrt(3)) }}{-4}{dtu-green}{Shared\\solution\\pointer}{2}
		\drawHexagon[]{{-2 + 2 * (2 - sqrt(3)) }}{-4}{dtu-green}{Shared\\solution\\pointer}{2}

		\drawHexagon[]{{12 - 5 * (2 - sqrt(3)) }}{-1}{dtu-green}{Shared\\solution\\pointer}{2}
		\drawHexagon[]{{-4 + 3 * (2 - sqrt(3)) }}{-1}{dtu-green}{Shared\\solution\\pointer}{2}
		% Legend for each layer
		\drawHexagon{{14.0  }}{+3.0}{dtu-white}{}{0.75}
		\node[align=right, anchor=west] at ({15.0}, +3.75) {Persistence};
		\drawHexagon{{14.0  }}{+1.5}{dtu-green}{}{0.75}
		\node[align=right, anchor=west] at ({15.0}, +2.25) {Atomic Pointer};
		\drawHexagon{{14.0  }}{+0.0}{dtu-white}{}{0.75}
		\node[align=right, anchor=west] at ({15.0}, +0.75) {Metaheuristics};
		\drawHexagon{{14.0  }}{-1.5}{dtu-white}{}{0.75}
		\node[align=right, anchor=west] at ({15.0}, -0.75) {Orchestration};
		\drawHexagon{{14.0  }}{-3.0}{dtu-white}{}{0.75}
		\node[align=right, anchor=west] at ({15.0}, -2.25) {User interfaces};
	\fi

	\ifsimplifiedlayer

		\node[align=right, anchor=west] at ({-5.5}, +3.75) {};
		\drawHexagon{{+2 + 0 * (2 - sqrt(3)) }}{ 2}{dtu-green}{Scheduler}{2}
		\drawHexagon{{+4 - 1 * (2 - sqrt(3)) }}{-1}{dtu-red}{Supervisor}{2}
		\drawHexagon{{+0 + 1 * (2 - sqrt(3)) }}{-1}{dtu-red}{Supervisor}{2}
		\drawHexagon{{+2 - 0 * (2 - sqrt(3)) }}{-4}{dtu-corporate-red}{Technician}{2}
		\drawHexagon{{+6 - 2 * (2 - sqrt(3)) }}{-4}{dtu-corporate-red}{Technician}{2}
		\drawHexagon{{-2 + 2 * (2 - sqrt(3)) }}{-4}{dtu-corporate-red}{Technician}{2}
		\drawHexagon{{+8 - 3 * (2 - sqrt(3)) }}{-1}{dtu-corporate-red}{Technician}{2}
		\drawHexagon{{-4 + 3 * (2 - sqrt(3)) }}{-1}{dtu-corporate-red}{Technician}{2}

		% Scheduler
		\draw[thin, fill=dtu-yellow] (2, 5) circle (0.35);
		\draw[thin, fill=dtu-purple] (2, 3) circle (0.35);
		% Supervisor 1
		\draw[thin, fill=dtu-yellow] ({+4 - 1 * (2 - sqrt(3)) }, 02) circle (0.35);
		\draw[thin, fill=dtu-purple] ({+4 - 1 * (2 - sqrt(3)) }, -0) circle (0.35);
		% Supervisor 2
		\draw[thin, fill=dtu-yellow] ({+0 + 1 * (2 - sqrt(3)) }, 02) circle (0.35);
		\draw[thin, fill=dtu-purple] ({+0 + 1 * (2 - sqrt(3)) }, -0) circle (0.35);
		% Technician 1
		\draw[thin, fill=dtu-yellow] ({+2 - 0 * (2 - sqrt(3)) }, -1) circle (0.35);
		\draw[thin, fill=dtu-purple] ({+2 - 0 * (2 - sqrt(3)) }, -3) circle (0.35);
		% Technician 2
		\draw[thin, fill=dtu-yellow] ({+6 - 2 * (2 - sqrt(3)) }, -1) circle (0.35);
		\draw[thin, fill=dtu-purple] ({+6 - 2 * (2 - sqrt(3)) }, -3) circle (0.35);
		% Technician 3
		\draw[thin, fill=dtu-yellow] ({-2 + 2 * (2 - sqrt(3)) }, -1) circle (0.35);
		\draw[thin, fill=dtu-purple] ({-2 + 2 * (2 - sqrt(3)) }, -3) circle (0.35);
		% Technician 4
		\draw[thin, fill=dtu-yellow] ({+8 - 3 * (2 - sqrt(3)) }, 02) circle (0.35);
		\draw[thin, fill=dtu-purple] ({+8 - 3 * (2 - sqrt(3)) }, -0) circle (0.35);
		% Technician 5
		\draw[thin, fill=dtu-yellow] ({-4 + 3 * (2 - sqrt(3)) }, 02) circle (0.35);
		\draw[thin, fill=dtu-purple] ({-4 + 3 * (2 - sqrt(3)) }, -0) circle (0.35);

		% Legend for each layer
		\node[align=right, anchor=west] at ({12.0}, +3.75) {Atomic Pointer};
		\draw[fill=dtu-purple] (11.0,  +3.75) circle (0.5);

		\node[align=right, anchor=west] at ({12.0}, +2.25) {Scheduler Metaheuristic};
		\drawHexagon{{11.0  }}{+1.75}{dtu-green}{}{0.5}
		\node[align=right, anchor=west] at ({12.0}, +0.75) {Supervisor Metaheuristic};
		\drawHexagon{{11.0  }}{+0.25}{dtu-red}{}{0.5}
		\node[align=right, anchor=west] at ({12.0}, -0.75) {Technician Metaheuristic};
		\drawHexagon{{11.0  }}{-1.25}{dtu-corporate-red}{}{0.5}
		\node[align=right, anchor=west] at ({12.0}, -2.25) {User interfaces (Message Passing)};
		\draw[fill=dtu-yellow] (11.0, -2.25) circle (0.5);
	\fi

	\ifmetaheuristicslayer
		\drawHexagon{ 2                      }{ 2}{dtu-blue}{Strategic}{2}
		\drawHexagon{{6 - 2 * (2 - sqrt(3)) }}{ 2}{dtu-green}{Tactical}{2}
		\drawHexagon{{4 - 1 * (2 - sqrt(3)) }}{-1}{dtu-red}{Supervisor}{2}
		\drawHexagon{{0 + 1 * (2 - sqrt(3)) }}{-1}{dtu-red}{Supervisor}{2}
		\drawHexagon{{8 - 3 * (2 - sqrt(3)) }}{-1}{dtu-red}{Supervisor}{2}

		\drawHexagon{{2 - 0 * (2 - sqrt(3)) }}{-4}{dtu-corporate-red}{Technician}{2}
		\drawHexagon{{6 - 2 * (2 - sqrt(3)) }}{-4}{dtu-corporate-red}{Technician}{2}

		\drawHexagon{{10 - 4 * (2 - sqrt(3)) }}{-4}{dtu-corporate-red}{Technician}{2}
		\drawHexagon{{-2 + 2 * (2 - sqrt(3)) }}{-4}{dtu-corporate-red}{Technician}{2}

		\drawHexagon{{12 - 5 * (2 - sqrt(3)) }}{-1}{dtu-corporate-red}{Technician}{2}
		\drawHexagon{{-4 + 3 * (2 - sqrt(3)) }}{-1}{dtu-corporate-red}{Technician}{2}

		% Legend for each layer
		\drawHexagon{{14.0  }}{+3.0}{dtu-white}{}{0.75}
		\node[align=right, anchor=west] at ({15.0}, +3.75) {Persistence};
		\drawHexagon{{14.0  }}{+1.5}{dtu-white}{}{0.75}
		\node[align=right, anchor=west] at ({15.0}, +2.25) {Atomic Pointer};
		\drawHexagon{{14.0  }}{+0.0}{dtu-corporate-red}{}{0.75}
		\node[align=right, anchor=west] at ({15.0}, +0.75) {Metaheuristics};
		\drawHexagon{{14.0  }}{-1.5}{dtu-white}{}{0.75}
		\node[align=right, anchor=west] at ({15.0}, -0.75) {Orchestration};
		\drawHexagon{{14.0  }}{-3.0}{dtu-white}{}{0.75}
		\node[align=right, anchor=west] at ({15.0}, -2.25) {User interfaces};
	\fi

	\iforchestratorlayer
		\drawHexagon{ 2                      }{ 2}{dtu-orange}{}{2}
		\drawHexagon{{6 - 2 * (2 - sqrt(3)) }}{ 2}{dtu-orange}{}{2}
		\drawHexagon{{4 - 1 * (2 - sqrt(3)) }}{-1}{dtu-orange}{Orche-\\strator}{2}
		\drawHexagon{{0 + 1 * (2 - sqrt(3)) }}{-1}{dtu-orange}{}{2}
		\drawHexagon{{8 - 3 * (2 - sqrt(3)) }}{-1}{dtu-orange}{}{2}

		\drawHexagon{{2 - 0 * (2 - sqrt(3)) }}{-4}{dtu-orange}{}{2}
		\drawHexagon{{6 - 2 * (2 - sqrt(3)) }}{-4}{dtu-orange}{}{2}

		\drawHexagon{{10 - 4 * (2 - sqrt(3)) }}{-4}{dtu-orange}{}{2}
		\drawHexagon{{-2 + 2 * (2 - sqrt(3)) }}{-4}{dtu-orange}{}{2}

		\drawHexagon{{12 - 5 * (2 - sqrt(3)) }}{-1}{dtu-orange}{}{2}
		\drawHexagon{{-4 + 3 * (2 - sqrt(3)) }}{-1}{dtu-orange}{}{2}
		% Legend for each layer
		\drawHexagon{{14.0  }}{+3.0}{dtu-white}{}{0.75}
		\node[align=right, anchor=west] at ({15.0}, +3.75) {Persistence};
		\drawHexagon{{14.0  }}{+1.5}{dtu-white}{}{0.75}
		\node[align=right, anchor=west] at ({15.0}, +2.25) {Atomic Pointer};
		\drawHexagon{{14.0  }}{+0.0}{dtu-white}{}{0.75}
		\node[align=right, anchor=west] at ({15.0}, +0.75) {Metaheuristics};
		\drawHexagon{{14.0  }}{-1.5}{dtu-orange}{}{0.75}
		\node[align=right, anchor=west] at ({15.0}, -0.75) {Orchestration};
		\drawHexagon{{14.0  }}{-3.0}{dtu-white}{}{0.75}
		\node[align=right, anchor=west] at ({15.0}, -2.25) {User interfaces};
	\fi

	
	\ifuserinterfacelayer
		\drawHexagon{ 2                      }{ 2}{dtu-yellow}{UI}{2}
		\drawHexagon{{6 - 2 * (2 - sqrt(3)) }}{ 2}{dtu-yellow}{UI}{2}
		\drawHexagon{{4 - 1 * (2 - sqrt(3)) }}{-1}{dtu-yellow}{UI}{2}
		\drawHexagon{{0 + 1 * (2 - sqrt(3)) }}{-1}{dtu-yellow}{UI}{2}
		\drawHexagon{{8 - 3 * (2 - sqrt(3)) }}{-1}{dtu-yellow}{UI}{2}

		\drawHexagon{{2 - 0 * (2 - sqrt(3)) }}{-4}{dtu-yellow}{UI}{2}
		\drawHexagon{{6 - 2 * (2 - sqrt(3)) }}{-4}{dtu-yellow}{UI}{2}

		\drawHexagon{{10 - 4 * (2 - sqrt(3)) }}{-4}{dtu-yellow}{UI}{2}
		\drawHexagon{{-2 + 2 * (2 - sqrt(3)) }}{-4}{dtu-yellow}{UI}{2}

		\drawHexagon{{12 - 5 * (2 - sqrt(3)) }}{-1}{dtu-yellow}{UI}{2}
		\drawHexagon{{-4 + 3 * (2 - sqrt(3)) }}{-1}{dtu-yellow}{UI}{2}
		% Legend for each layer
		\drawHexagon{{14.0  }}{+3.0}{dtu-white}{}{0.75}
		\node[align=right, anchor=west] at ({15.0}, +3.75) {Persistence};
		\drawHexagon{{14.0  }}{+1.5}{dtu-white}{}{0.75}
		\node[align=right, anchor=west] at ({15.0}, +2.25) {Atomic Pointer};
		\drawHexagon{{14.0  }}{+0.0}{dtu-white}{}{0.75}
		\node[align=right, anchor=west] at ({15.0}, +0.75) {Metaheuristics};
		\drawHexagon{{14.0  }}{-1.5}{dtu-white}{}{0.75}
		\node[align=right, anchor=west] at ({15.0}, -0.75) {Orchestration};
		\drawHexagon{{14.0  }}{-3.0}{dtu-yellow}{}{0.75}
		\node[align=right, anchor=west] at ({15.0}, -2.25) {User interfaces};
	\fi
	
	\end{tikzpicture}
}

	\resizebox{0.7\textwidth}{!}{
		\drawModelSetupHexagon[orchestrator=true]
	}
	\caption{
		Overview of the scheduling process when modelled as actors. When LNS is encapsulated 
		is an actor it becomes possible to optimize parts of a large process individually instead of 
		optimizing the scheduling problem globally from a single model implementation.
	}
	\label{fig:ordinator-hexagon:orchestrator}
\end{figure}

\begin{figure}[H]
	\centering
	\usetikzlibrary {positioning}
\newcommand{\drawHexagon}[6][draw=black]{
	\draw[#1, fill=#4] (#2,#3) ++(30:#6) -- ++(90:#6) -- ++(150:#6) -- ++(210:#6) -- ++(270:#6) -- ++(330:#6) -- cycle;
	\node[align=center] at (#2,#3+2) {#5};
}

\newif\ifpersistencelayer
\newif\ifatomicpointerswaplayer
\newif\ifmetaheuristicslayer
\newif\ifuserinterfacelayer
\newif\iforchestratorlayer
\newif\ifsimplifiedlayer

\pgfkeys{
	/hexagon/.is family, /hexagon,
	default/.style = {
		persistence=false,
		atomicpointerswap=false,
		metaheuristics=false,
		orchestrator=false,
		userinterface=false,
		simplified=false,
	},
	persistence/.is if=persistencelayer,
	atomicpointerswap/.is if=atomicpointerswaplayer,
	metaheuristics/.is if=metaheuristicslayer,
	orchestrator/.is if=orchestratorlayer,
	userinterface/.is if=userinterfacelayer,
	simplified/.is if=simplifiedlayer,
}
\newcommand{\drawModelSetupHexagon}[1][]{
	\pgfkeys{/hexagon, default, #1}

	\begin{tikzpicture}[font=\footnotesize, scale=0.5, line width=1.05]
	

	\ifpersistencelayer
		\drawHexagon[draw=none]{ 2                      }{ 2}{dtu-blue}{}{2}
		\drawHexagon[draw=none]{{6 - 2 * (2 - sqrt(3)) }}{ 2}{dtu-blue}{}{2}
		\drawHexagon[draw=none]{{4 - 1 * (2 - sqrt(3)) }}{-1}{dtu-blue}{Persistence}{2}
		\drawHexagon[draw=none]{{0 + 1 * (2 - sqrt(3)) }}{-1}{dtu-blue}{}{2}
		\drawHexagon[draw=none]{{8 - 3 * (2 - sqrt(3)) }}{-1}{dtu-blue}{}{2}

		\drawHexagon[draw=none]{{2 - 0 * (2 - sqrt(3)) }}{-4}{dtu-blue}{}{2}
		\drawHexagon[draw=none]{{6 - 2 * (2 - sqrt(3)) }}{-4}{dtu-blue}{}{2}

		\drawHexagon[draw=none]{{10 - 4 * (2 - sqrt(3)) }}{-4}{dtu-blue}{}{2}
		\drawHexagon[draw=none]{{-2 + 2 * (2 - sqrt(3)) }}{-4}{dtu-blue}{}{2}

		\drawHexagon[draw=none]{{12 - 5 * (2 - sqrt(3)) }}{-1}{dtu-blue}{}{2}
		\drawHexagon[draw=none]{{-4 + 3 * (2 - sqrt(3)) }}{-1}{dtu-blue}{}{2}
		% Legend for each layer
		\drawHexagon{{14.0  }}{+3.0}{dtu-blue}{}{0.75}
		\node[align=right, anchor=west] at ({15.0}, +3.75) {Persistence};
		\drawHexagon{{14.0  }}{+1.5}{dtu-white}{}{0.75}
		\node[align=right, anchor=west] at ({15.0}, +2.25) {Atomic Pointer};
		\drawHexagon{{14.0  }}{+0.0}{dtu-white}{}{0.75}
		\node[align=right, anchor=west] at ({15.0}, +0.75) {Metaheuristics};
		\drawHexagon{{14.0  }}{-1.5}{dtu-white}{}{0.75}
		\node[align=right, anchor=west] at ({15.0}, -0.75) {Orchestration};
		\drawHexagon{{14.0  }}{-3.0}{dtu-white}{}{0.75}
		\node[align=right, anchor=west] at ({15.0}, -2.25) {User interfaces};
	\fi


	\ifatomicpointerswaplayer
		\drawHexagon[]{ 2                      }{ 2}{dtu-green}{Shared\\solution\\pointer}{2}
		\drawHexagon[]{{6 - 2 * (2 - sqrt(3)) }}{ 2}{dtu-green}{Shared\\solution\\pointer}{2}
		\drawHexagon[]{{4 - 1 * (2 - sqrt(3)) }}{-1}{dtu-green}{Shared\\solution\\pointer}{2}
		\drawHexagon[]{{0 + 1 * (2 - sqrt(3)) }}{-1}{dtu-green}{Shared\\solution\\pointer}{2}
		\drawHexagon[]{{8 - 3 * (2 - sqrt(3)) }}{-1}{dtu-green}{Shared\\solution\\pointer}{2}

		\drawHexagon[]{{2 - 0 * (2 - sqrt(3)) }}{-4}{dtu-green}{Shared\\solution\\pointer}{2}
		\drawHexagon[]{{6 - 2 * (2 - sqrt(3)) }}{-4}{dtu-green}{Shared\\solution\\pointer}{2}

		\drawHexagon[]{{10 - 4 * (2 - sqrt(3)) }}{-4}{dtu-green}{Shared\\solution\\pointer}{2}
		\drawHexagon[]{{-2 + 2 * (2 - sqrt(3)) }}{-4}{dtu-green}{Shared\\solution\\pointer}{2}

		\drawHexagon[]{{12 - 5 * (2 - sqrt(3)) }}{-1}{dtu-green}{Shared\\solution\\pointer}{2}
		\drawHexagon[]{{-4 + 3 * (2 - sqrt(3)) }}{-1}{dtu-green}{Shared\\solution\\pointer}{2}
		% Legend for each layer
		\drawHexagon{{14.0  }}{+3.0}{dtu-white}{}{0.75}
		\node[align=right, anchor=west] at ({15.0}, +3.75) {Persistence};
		\drawHexagon{{14.0  }}{+1.5}{dtu-green}{}{0.75}
		\node[align=right, anchor=west] at ({15.0}, +2.25) {Atomic Pointer};
		\drawHexagon{{14.0  }}{+0.0}{dtu-white}{}{0.75}
		\node[align=right, anchor=west] at ({15.0}, +0.75) {Metaheuristics};
		\drawHexagon{{14.0  }}{-1.5}{dtu-white}{}{0.75}
		\node[align=right, anchor=west] at ({15.0}, -0.75) {Orchestration};
		\drawHexagon{{14.0  }}{-3.0}{dtu-white}{}{0.75}
		\node[align=right, anchor=west] at ({15.0}, -2.25) {User interfaces};
	\fi

	\ifsimplifiedlayer

		\node[align=right, anchor=west] at ({-5.5}, +3.75) {};
		\drawHexagon{{+2 + 0 * (2 - sqrt(3)) }}{ 2}{dtu-green}{Scheduler}{2}
		\drawHexagon{{+4 - 1 * (2 - sqrt(3)) }}{-1}{dtu-red}{Supervisor}{2}
		\drawHexagon{{+0 + 1 * (2 - sqrt(3)) }}{-1}{dtu-red}{Supervisor}{2}
		\drawHexagon{{+2 - 0 * (2 - sqrt(3)) }}{-4}{dtu-corporate-red}{Technician}{2}
		\drawHexagon{{+6 - 2 * (2 - sqrt(3)) }}{-4}{dtu-corporate-red}{Technician}{2}
		\drawHexagon{{-2 + 2 * (2 - sqrt(3)) }}{-4}{dtu-corporate-red}{Technician}{2}
		\drawHexagon{{+8 - 3 * (2 - sqrt(3)) }}{-1}{dtu-corporate-red}{Technician}{2}
		\drawHexagon{{-4 + 3 * (2 - sqrt(3)) }}{-1}{dtu-corporate-red}{Technician}{2}

		% Scheduler
		\draw[thin, fill=dtu-yellow] (2, 5) circle (0.35);
		\draw[thin, fill=dtu-purple] (2, 3) circle (0.35);
		% Supervisor 1
		\draw[thin, fill=dtu-yellow] ({+4 - 1 * (2 - sqrt(3)) }, 02) circle (0.35);
		\draw[thin, fill=dtu-purple] ({+4 - 1 * (2 - sqrt(3)) }, -0) circle (0.35);
		% Supervisor 2
		\draw[thin, fill=dtu-yellow] ({+0 + 1 * (2 - sqrt(3)) }, 02) circle (0.35);
		\draw[thin, fill=dtu-purple] ({+0 + 1 * (2 - sqrt(3)) }, -0) circle (0.35);
		% Technician 1
		\draw[thin, fill=dtu-yellow] ({+2 - 0 * (2 - sqrt(3)) }, -1) circle (0.35);
		\draw[thin, fill=dtu-purple] ({+2 - 0 * (2 - sqrt(3)) }, -3) circle (0.35);
		% Technician 2
		\draw[thin, fill=dtu-yellow] ({+6 - 2 * (2 - sqrt(3)) }, -1) circle (0.35);
		\draw[thin, fill=dtu-purple] ({+6 - 2 * (2 - sqrt(3)) }, -3) circle (0.35);
		% Technician 3
		\draw[thin, fill=dtu-yellow] ({-2 + 2 * (2 - sqrt(3)) }, -1) circle (0.35);
		\draw[thin, fill=dtu-purple] ({-2 + 2 * (2 - sqrt(3)) }, -3) circle (0.35);
		% Technician 4
		\draw[thin, fill=dtu-yellow] ({+8 - 3 * (2 - sqrt(3)) }, 02) circle (0.35);
		\draw[thin, fill=dtu-purple] ({+8 - 3 * (2 - sqrt(3)) }, -0) circle (0.35);
		% Technician 5
		\draw[thin, fill=dtu-yellow] ({-4 + 3 * (2 - sqrt(3)) }, 02) circle (0.35);
		\draw[thin, fill=dtu-purple] ({-4 + 3 * (2 - sqrt(3)) }, -0) circle (0.35);

		% Legend for each layer
		\node[align=right, anchor=west] at ({12.0}, +3.75) {Atomic Pointer};
		\draw[fill=dtu-purple] (11.0,  +3.75) circle (0.5);

		\node[align=right, anchor=west] at ({12.0}, +2.25) {Scheduler Metaheuristic};
		\drawHexagon{{11.0  }}{+1.75}{dtu-green}{}{0.5}
		\node[align=right, anchor=west] at ({12.0}, +0.75) {Supervisor Metaheuristic};
		\drawHexagon{{11.0  }}{+0.25}{dtu-red}{}{0.5}
		\node[align=right, anchor=west] at ({12.0}, -0.75) {Technician Metaheuristic};
		\drawHexagon{{11.0  }}{-1.25}{dtu-corporate-red}{}{0.5}
		\node[align=right, anchor=west] at ({12.0}, -2.25) {User interfaces (Message Passing)};
		\draw[fill=dtu-yellow] (11.0, -2.25) circle (0.5);
	\fi

	\ifmetaheuristicslayer
		\drawHexagon{ 2                      }{ 2}{dtu-blue}{Strategic}{2}
		\drawHexagon{{6 - 2 * (2 - sqrt(3)) }}{ 2}{dtu-green}{Tactical}{2}
		\drawHexagon{{4 - 1 * (2 - sqrt(3)) }}{-1}{dtu-red}{Supervisor}{2}
		\drawHexagon{{0 + 1 * (2 - sqrt(3)) }}{-1}{dtu-red}{Supervisor}{2}
		\drawHexagon{{8 - 3 * (2 - sqrt(3)) }}{-1}{dtu-red}{Supervisor}{2}

		\drawHexagon{{2 - 0 * (2 - sqrt(3)) }}{-4}{dtu-corporate-red}{Technician}{2}
		\drawHexagon{{6 - 2 * (2 - sqrt(3)) }}{-4}{dtu-corporate-red}{Technician}{2}

		\drawHexagon{{10 - 4 * (2 - sqrt(3)) }}{-4}{dtu-corporate-red}{Technician}{2}
		\drawHexagon{{-2 + 2 * (2 - sqrt(3)) }}{-4}{dtu-corporate-red}{Technician}{2}

		\drawHexagon{{12 - 5 * (2 - sqrt(3)) }}{-1}{dtu-corporate-red}{Technician}{2}
		\drawHexagon{{-4 + 3 * (2 - sqrt(3)) }}{-1}{dtu-corporate-red}{Technician}{2}

		% Legend for each layer
		\drawHexagon{{14.0  }}{+3.0}{dtu-white}{}{0.75}
		\node[align=right, anchor=west] at ({15.0}, +3.75) {Persistence};
		\drawHexagon{{14.0  }}{+1.5}{dtu-white}{}{0.75}
		\node[align=right, anchor=west] at ({15.0}, +2.25) {Atomic Pointer};
		\drawHexagon{{14.0  }}{+0.0}{dtu-corporate-red}{}{0.75}
		\node[align=right, anchor=west] at ({15.0}, +0.75) {Metaheuristics};
		\drawHexagon{{14.0  }}{-1.5}{dtu-white}{}{0.75}
		\node[align=right, anchor=west] at ({15.0}, -0.75) {Orchestration};
		\drawHexagon{{14.0  }}{-3.0}{dtu-white}{}{0.75}
		\node[align=right, anchor=west] at ({15.0}, -2.25) {User interfaces};
	\fi

	\iforchestratorlayer
		\drawHexagon{ 2                      }{ 2}{dtu-orange}{}{2}
		\drawHexagon{{6 - 2 * (2 - sqrt(3)) }}{ 2}{dtu-orange}{}{2}
		\drawHexagon{{4 - 1 * (2 - sqrt(3)) }}{-1}{dtu-orange}{Orche-\\strator}{2}
		\drawHexagon{{0 + 1 * (2 - sqrt(3)) }}{-1}{dtu-orange}{}{2}
		\drawHexagon{{8 - 3 * (2 - sqrt(3)) }}{-1}{dtu-orange}{}{2}

		\drawHexagon{{2 - 0 * (2 - sqrt(3)) }}{-4}{dtu-orange}{}{2}
		\drawHexagon{{6 - 2 * (2 - sqrt(3)) }}{-4}{dtu-orange}{}{2}

		\drawHexagon{{10 - 4 * (2 - sqrt(3)) }}{-4}{dtu-orange}{}{2}
		\drawHexagon{{-2 + 2 * (2 - sqrt(3)) }}{-4}{dtu-orange}{}{2}

		\drawHexagon{{12 - 5 * (2 - sqrt(3)) }}{-1}{dtu-orange}{}{2}
		\drawHexagon{{-4 + 3 * (2 - sqrt(3)) }}{-1}{dtu-orange}{}{2}
		% Legend for each layer
		\drawHexagon{{14.0  }}{+3.0}{dtu-white}{}{0.75}
		\node[align=right, anchor=west] at ({15.0}, +3.75) {Persistence};
		\drawHexagon{{14.0  }}{+1.5}{dtu-white}{}{0.75}
		\node[align=right, anchor=west] at ({15.0}, +2.25) {Atomic Pointer};
		\drawHexagon{{14.0  }}{+0.0}{dtu-white}{}{0.75}
		\node[align=right, anchor=west] at ({15.0}, +0.75) {Metaheuristics};
		\drawHexagon{{14.0  }}{-1.5}{dtu-orange}{}{0.75}
		\node[align=right, anchor=west] at ({15.0}, -0.75) {Orchestration};
		\drawHexagon{{14.0  }}{-3.0}{dtu-white}{}{0.75}
		\node[align=right, anchor=west] at ({15.0}, -2.25) {User interfaces};
	\fi

	
	\ifuserinterfacelayer
		\drawHexagon{ 2                      }{ 2}{dtu-yellow}{UI}{2}
		\drawHexagon{{6 - 2 * (2 - sqrt(3)) }}{ 2}{dtu-yellow}{UI}{2}
		\drawHexagon{{4 - 1 * (2 - sqrt(3)) }}{-1}{dtu-yellow}{UI}{2}
		\drawHexagon{{0 + 1 * (2 - sqrt(3)) }}{-1}{dtu-yellow}{UI}{2}
		\drawHexagon{{8 - 3 * (2 - sqrt(3)) }}{-1}{dtu-yellow}{UI}{2}

		\drawHexagon{{2 - 0 * (2 - sqrt(3)) }}{-4}{dtu-yellow}{UI}{2}
		\drawHexagon{{6 - 2 * (2 - sqrt(3)) }}{-4}{dtu-yellow}{UI}{2}

		\drawHexagon{{10 - 4 * (2 - sqrt(3)) }}{-4}{dtu-yellow}{UI}{2}
		\drawHexagon{{-2 + 2 * (2 - sqrt(3)) }}{-4}{dtu-yellow}{UI}{2}

		\drawHexagon{{12 - 5 * (2 - sqrt(3)) }}{-1}{dtu-yellow}{UI}{2}
		\drawHexagon{{-4 + 3 * (2 - sqrt(3)) }}{-1}{dtu-yellow}{UI}{2}
		% Legend for each layer
		\drawHexagon{{14.0  }}{+3.0}{dtu-white}{}{0.75}
		\node[align=right, anchor=west] at ({15.0}, +3.75) {Persistence};
		\drawHexagon{{14.0  }}{+1.5}{dtu-white}{}{0.75}
		\node[align=right, anchor=west] at ({15.0}, +2.25) {Atomic Pointer};
		\drawHexagon{{14.0  }}{+0.0}{dtu-white}{}{0.75}
		\node[align=right, anchor=west] at ({15.0}, +0.75) {Metaheuristics};
		\drawHexagon{{14.0  }}{-1.5}{dtu-white}{}{0.75}
		\node[align=right, anchor=west] at ({15.0}, -0.75) {Orchestration};
		\drawHexagon{{14.0  }}{-3.0}{dtu-yellow}{}{0.75}
		\node[align=right, anchor=west] at ({15.0}, -2.25) {User interfaces};
	\fi
	
	\end{tikzpicture}
}

	\resizebox{0.7\textwidth}{!}{
		\drawModelSetupHexagon[userinterface=true]
	}
	\caption{
		Overview of the scheduling process when modelled as actors. When LNS is encapsulated 
		is an actor it becomes possible to optimize parts of a large process individually instead of 
		optimizing the scheduling problem globally from a single model implementation.
	}
	\label{fig:ordinator-hexagon:userinterfaces}
\end{figure}

