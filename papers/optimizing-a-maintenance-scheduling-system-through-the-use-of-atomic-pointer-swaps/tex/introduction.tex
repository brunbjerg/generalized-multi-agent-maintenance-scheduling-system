\section{Introduction}\label{sec:introduction}
This paper will be structured into four sections: Section~\ref{sec:introduction} will explain the
problem setting that necessitates the solution method; describe the relevant software architecture
literature to fix the problem; and provide mathematical models for each component of the system for 
a detailed understanding of the problem that the optimization system aims to solve. Section~\ref{sec:solution_method}
will describe the chosen approach to solve the system, highlighting why low-level memory operations
are necessary to effective scale metaheuristics. Section~\ref{sec:results} will provide a high level
understanding of how the system is behaving. Finally Section~\ref{sec:discussion} will discuss the 
implication of the provided solution approach and which other optimization problems could benefit 
from the presented optimization approach.

\subsection{Problem Setting}
Many operational optimzation problems are difficult to solve in practice as the underlying business 
process is made up of many different actors. This makes single model approaches difficult as it is 
difficult to assign responsibility for the output of the solution provided by the optimization
approach \citep{INTERACTIVEOPTIMIZATION}. Furthermore this decomposition of business process into 
specific actors is not only done to make the underlying optimization problem easier to solve, but 
also to handle dynamic changes in the optimization process and make each process compliant with 
respect to safety, regulation, etc. 

Maintenance scheduling has all these components and it is one of the keys reasons that the presented
solution method was developed. 

\begin{figure}
	\usetikzlibrary{positioning}


\definecolor{red}{HTML}{8A3F3A}
\definecolor{yellow}{HTML}{E0BB3C}
\definecolor{blue}{HTML}{4569E0}
\definecolor{green}{HTML}{17E561}
\definecolor{other}{HTML}{6A939E}


\newlength{\basisa}
\setlength{\basisa}{1cm}
\centering
\begin{tikzpicture}[line width=0.0\basisa]
    \draw (4.0\basisa,4.0\basisa) 
		node[minimum height=2\basisa,fill=dtu-grey,minimum width=3\basisa,rounded corners=0.1\basisa] 
			(Planner) {Planner};
    \draw (12.0\basisa,4.0\basisa) 
		node[minimum height=2\basisa,fill=dtu-grey,minimum width=3\basisa,rounded corners=0.1\basisa] 
			(Supervisor) {Supervisor};

			
    \draw (8.0\basisa,4.0\basisa) 
		node[minimum height=2\basisa,fill=dtu-green,minimum width=3\basisa,rounded corners=0.1\basisa] 
			(Scheduler) {Scheduler};
	
    \draw (8.0\basisa,1.5\basisa) 
		node[minimum height=1\basisa,fill=dtu-blue,minimum width=3\basisa,rounded corners=0.1\basisa] 
			(Database) {Database};

    \draw (8.0\basisa,6.5\basisa) 
		node[minimum height=1\basisa,fill=dtu-yellow,minimum width=3\basisa,rounded corners=0.1\basisa] 
			(UserInterface) {User Interface};

	\draw[<->, thick, line width=0.1\basisa] (Planner) -- (Scheduler);
	\draw[<->, thick, line width=0.1\basisa] (Scheduler) -- (Supervisor);
	\draw[<->, thick, line width=0.1\basisa] (Scheduler) -- (Database);
	\draw[<->, thick, line width=0.1\basisa] (Scheduler) -- (UserInterface);
\end{tikzpicture}

	\caption{Simple overview of the scheduling process with its primary types of
		actors. The planner, the scheduler, the supervisor(s), and the technicians. 		
		The green color highlights the scheduler as it the actor in the maintenance
		scheduling process that is modelled in this paper.
	}\label{fig:integrated:maintenance-process}
\end{figure}

\subsection{Software Architecture}
Metaheuristics present unique challenges when it comes to designing a high level software architecture
where the metaheuristics can be reliably implemented and scale. One of the reasons for this is that 
metaheuristics have very high requirements when it comes to state mutation. State mutation is the 
primary way in which metaheuristics achieve optimization by changing the solution space rapidly
according to special rules \citep{gendreauHandbookMetaheuristics2019}. 

If you want to scale metaheuristics if becomes necessary to determine a software architecture that 
can accomodate high levels of state changes. Most attempts at this has centered around message
passing \citep{talbiMetaheuristicsDesignImplementation2009} and in more advanced setup might 
also incorporate methods from the multi-agent systems literature to 
speed up the optimization process \citep{MultiagentMetaheuristicOptimization}.

This paper will present a different way of scaling metaheuristics using low-level memory operations
such as atomic operations. \citep{!!! CITE THE LOCK FREE DATA STRUCTURE ARTICLE}.

% \begin{figure}[H]
% 	\centering
% 	\usetikzlibrary {positioning}


\definecolor{red}{HTML}{8A3F3A}
\definecolor{yellow}{HTML}{E0BB3C}
\definecolor{blue}{HTML}{4569E0}
\definecolor{green}{HTML}{17E561}
\definecolor{other}{HTML}{6A939E}

\newcommand{\ModelColor}{red}
\newcommand{\UserInterfaceColor}{yellow}
\newcommand{\PersistenceColor}{blue}
\newcommand{\PointerSwapColor}{green}
\newcommand{\OrchestratorColor}{other}

\pgfkeys{
	/graph/.is family, /graph,
	default/.style = {
		show_shared_pointer = false,
		show_orchestrator = false,
		show_persistence = false,
		show_user_interface = false,
		basis/.estore in = 2cm,
	},
	show_shared_pointer/.estore in = \ShowSharedSolutionCommunication,
	show_orchestrator/.estore in = \ShowOrchestratorCommunication,
	show_persistence/.estore in = \ShowPersistenceCommunication,
	show_user_interface/.estore in = \ShowUserInterfaceCommunication,
	basis/.estore in = \basisinput,
}
\newlength{\basis}
\tikzset{
  basis/.code={\setlength{\basis}{\basisinput}}, % TikZ assignment code
  basis/.default=1cm,                   % Provide a default (\b@sis is undefined/unassigned)
  basis,                                % Set initial Value (\b@sis is defined/assigned)
 }

\newcommand{\drawGraph}[1]{
	\pgfkeys{/graph, default, #1}
	
	\begin{tikzpicture}[scale=0.75][
		% Define styles and settings
		node distance=2cm,
		block/.style={rectangle, draw, fill=blue!20, text centered, minimum height=3em},
		arrow/.style={-Stealth, thick}
		]


		\ifthenelse{\equal{\ShowOrchestratorCommunication}{true}}{
			\draw[color=other,-, ultra thick] (Strategic) -- (Orchestrator);
			\draw[color=other,-, ultra thick] (Tactical) -- (Orchestrator);
			\draw[color=other,-, ultra thick] (Supervisor) -- (Orchestrator);
			\draw[color=other,-, ultra thick] (Operational_1) -- (Orchestrator);
			\draw[color=other,-, ultra thick] (Operational_2) -- (Orchestrator);
			\draw[color=other,-, ultra thick] (Operational_3) -- (Orchestrator);
		}{}
		% \draw[help lines] (0\basis, 0\basis) grid (10\basis, 8\basis);
		\draw (5\basis,4\basis) node[minimum height=5cm,minimum width=7.0cm,rounded corners=5pt] {};

	    \draw (2.5\basis,5.5\basis) node[minimum height=1cm,minimum width=1cm,fill=\ModelColor,rounded corners=5pt] (Strategic) {Stra};
	    \draw (5.0\basis,4.0\basis) node[minimum height=1cm,minimum width=1\basis,fill=\ModelColor,rounded corners=5pt] (Supervisor) {Sup};
		\draw (7.5\basis,5.5\basis) node[minimum height=1cm,minimum width=1cm,fill=\ModelColor,rounded corners=5pt] (Tactical) {Tac};

		\draw (2.5\basis,2.5\basis) node[minimum height=1cm,minimum width=1cm,fill=\ModelColor,rounded corners=5pt] (Operational_1) {$O_{1}$};
		\draw (5.0\basis,2.5\basis) node[minimum height=1cm,minimum width=1cm,fill=\ModelColor,rounded corners=5pt] (Operational_2) {$O_{2}$};
		\draw (7.5\basis,2.5\basis) node[minimum height=1cm,minimum width=1cm,fill=\ModelColor,rounded corners=5pt,rounded corners=5pt] (Operational_3) {$O_{3}$};
	
		\draw (Strategic) edge (Tactical);
		\draw (Strategic) edge (Tactical);
		\draw (5\basis,5.5\basis) edge (Supervisor);
		\draw (Supervisor) edge (Operational_1);
		\draw (Supervisor) edge (Operational_2);
		\draw (Supervisor) edge (Operational_3);
		\draw (5.0\basis,0.5\basis)   node[minimum height=1cm,minimum width=5.0cm,            fill=\PersistenceColor,rounded corners=5pt] {SchedulingEnvironment};
		\draw (5.0\basis,7.5\basis)   node[minimum height=1cm,minimum width=5.0cm,            fill=\OrchestratorColor,rounded corners=5pt] (Orchestrator) {Orchestrator};
		\draw (0.5\basis,4.0\basis)   node[rotate=90, minimum height=1cm, minimum width=3.25cm,fill=\PointerSwapColor,rounded corners=5pt] {SharedSolution};
		\draw (9.5\basis,5.5\basis) node[rotate=90, minimum height=1cm, minimum width=1cm,fill=\UserInterfaceColor,rounded corners=5pt] {UI};
		\draw (9.5\basis,4.0\basis)   node[rotate=90, minimum height=1cm, minimum width=1cm,fill=\UserInterfaceColor,rounded corners=5pt] {UI};
		\draw (9.5\basis,2.5\basis) node[rotate=90, minimum height=1cm, minimum width=1cm,fill=\UserInterfaceColor,rounded corners=5pt] {UI};

		% Legend
		\begin{scope}[shift={(10.6\basis,5.7\basis)}]
			\node at (-0.25\basis,1\basis) [right] {Communication Type};
			\draw[color=\OrchestratorColor,fill] (0\basis,0.0\basis) rectangle (0.5cm, 0.5cm);
			\node[anchor=west] at (0.5\basis, 0.25\basis) {\scriptsize Channels};
			\draw[color=\PointerSwapColor,fill] (0\basis,-1.0\basis) rectangle(0.5cm, -0.5cm); 
			\node[anchor=west] at (0.5\basis, -0.75\basis) {\scriptsize Atomic Pointer Swap};
			\draw[color=\ModelColor,fill] (0\basis,-2.0\basis) rectangle(0.5cm, -1.5cm); 
			\node[anchor=west] at (0.5\basis, -1.75\basis) {\scriptsize Metaheurics};
			\draw[color=\PersistenceColor,fill] (0\basis,-3.0\basis) rectangle(0.5cm, -2.5cm); 
			\node[anchor=west] at (0.5\basis, -2.75\basis) {\scriptsize Mutex lock};
			\draw[color=\UserInterfaceColor,fill] (0\basis,-4.0\basis) rectangle(0.5cm, -3.5cm); 
			\node[anchor=west] at (0.5\basis, -3.75\basis) {\scriptsize Channels};
		\end{scope}
		\ifthenelse{\equal{\ShowSharedSolutionCommunication}{true}}{
			\draw[->, thick] (Strategic) -- (Orchestrator);
		}{}
		\ifthenelse{\equal{\ShowUserInterfaceCommunication}{true}}{
			\draw[->, thick] (Strategic) -- (Orchestrator);
		}{}
		\ifthenelse{\equal{\ShowPersistenceCommunication}{true}}{
			\draw[->, thick] (Strategic) -- (Orchestrator);
		}{}
		

	\end{tikzpicture}
}


% 	\drawOrdinatorArchitecture{basisinput=1cm}
% 	\label{fig:ordinator-architecture}
% \end{figure}
\begin{figure}[H]
	\centering
	\documentclass{standalone}
\usepackage{tikz}
\usetikzlibrary {positioning}

<<<<<<< HEAD
\newcommand{\drawHexagon}[5][draw=black]{
	\draw[#1, fill=#4, ] (#2,#3) ++(30:2) -- ++(90:2) -- ++(150:2) -- ++(210:2) -- ++(270:2) -- ++(330:2) -- cycle;
	\node at (#2,#3+2) {#5};
||||||| 2d4327a
\newcommand{\drawHexagon}[4]{
	\draw[fill=#3] (#1,#2) ++(30:2) -- ++(90:2) -- ++(150:2) -- ++(210:2) -- ++(270:2) -- ++(330:2) -- cycle;
	\node at (#1,#2+2) {#4};
=======
\begin{document}
\newcommand{\drawHexagon}[4]{
	\draw[fill=#3] (#1,#2) ++(30:2) -- ++(90:2) -- ++(150:2) -- ++(210:2) -- ++(270:2) -- ++(330:2) -- cycle;
	\node at (#1,#2+2) {#4};
>>>>>>> 023797133fb426c1bb01f920d8f5635c343d11a6
}

\newif\ifuserinterfacelayer
\newif\ifpersistencelayer
\newif\ifmetaheuristicslayer
\newif\ifatomicpointerswaplayer

\pgfkeys{
	/hexagon/.is family, /hexagon,
	default/.style = {
		persistence=false,
		userinterface=false,
		metaheuristics=true,
	},
	persistence/.is if=persistencelayer,
	userinterface/.is if=userinterfacelayer,
	metaheuristics/.is if=metaheuristicslayer,
}
\newcommand{\drawModelSetupHexagon}[1][]{
	\pgfkeys{/hexagon, default, #1}

	\begin{tikzpicture}[scale=0.6, line width=1.05]
	
	\ifuserinterfacelayer
		\drawHexagon{ 2                      }{ 2}{dtu-yellow}{UI}
		\drawHexagon{{6 - 2 * (2 - sqrt(3)) }}{ 2}{dtu-yellow}{UI}
		\drawHexagon{{4 - 1 * (2 - sqrt(3)) }}{-1}{dtu-yellow}{UI}
		\drawHexagon{{0 + 1 * (2 - sqrt(3)) }}{-1}{dtu-yellow}{UI}
		\drawHexagon{{8 - 3 * (2 - sqrt(3)) }}{-1}{dtu-yellow}{UI}

		\drawHexagon{{2 - 0 * (2 - sqrt(3)) }}{-4}{dtu-yellow}{UI}
		\drawHexagon{{6 - 2 * (2 - sqrt(3)) }}{-4}{dtu-yellow}{UI}

		\drawHexagon{{10 - 4 * (2 - sqrt(3)) }}{-4}{dtu-yellow}{UI}
		\drawHexagon{{-2 + 2 * (2 - sqrt(3)) }}{-4}{dtu-yellow}{UI}

		\drawHexagon{{12 - 5 * (2 - sqrt(3)) }}{-1}{dtu-yellow}{UI}
		\drawHexagon{{-4 + 3 * (2 - sqrt(3)) }}{-1}{dtu-yellow}{UI}
	\fi

	\ifpersistencelayer
		\drawHexagon[draw=none]{ 2                      }{ 2}{dtu-blue}{}
		\drawHexagon[draw=none]{{6 - 2 * (2 - sqrt(3)) }}{ 2}{dtu-blue}{}
		\drawHexagon[draw=none]{{4 - 1 * (2 - sqrt(3)) }}{-1}{dtu-blue}{Database}
		\drawHexagon[draw=none]{{0 + 1 * (2 - sqrt(3)) }}{-1}{dtu-blue}{}
		\drawHexagon[draw=none]{{8 - 3 * (2 - sqrt(3)) }}{-1}{dtu-blue}{}

		\drawHexagon[draw=none]{{2 - 0 * (2 - sqrt(3)) }}{-4}{dtu-blue}{}
		\drawHexagon[draw=none]{{6 - 2 * (2 - sqrt(3)) }}{-4}{dtu-blue}{}

		\drawHexagon[draw=none]{{10 - 4 * (2 - sqrt(3)) }}{-4}{dtu-blue}{}
		\drawHexagon[draw=none]{{-2 + 2 * (2 - sqrt(3)) }}{-4}{dtu-blue}{}

		\drawHexagon[draw=none]{{12 - 5 * (2 - sqrt(3)) }}{-1}{dtu-blue}{}
		\drawHexagon[draw=none]{{-4 + 3 * (2 - sqrt(3)) }}{-1}{dtu-blue}{}
	\fi

	\ifmetaheuristicslayer
		\drawHexagon{ 2                      }{ 2}{dtu-blue}{Strategic}
		\drawHexagon{{6 - 2 * (2 - sqrt(3)) }}{ 2}{dtu-green}{Tactical}
		\drawHexagon{{4 - 1 * (2 - sqrt(3)) }}{-1}{dtu-red}{Supervisor}
		\drawHexagon{{0 + 1 * (2 - sqrt(3)) }}{-1}{dtu-red}{Supervisor}
		\drawHexagon{{8 - 3 * (2 - sqrt(3)) }}{-1}{dtu-red}{Supervisor}

		\drawHexagon{{2 - 0 * (2 - sqrt(3)) }}{-4}{dtu-corporate-red}{Technician}
		\drawHexagon{{6 - 2 * (2 - sqrt(3)) }}{-4}{dtu-corporate-red}{Technician}

		\drawHexagon{{10 - 4 * (2 - sqrt(3)) }}{-4}{dtu-corporate-red}{Technician}
		\drawHexagon{{-2 + 2 * (2 - sqrt(3)) }}{-4}{dtu-corporate-red}{Technician}

		\drawHexagon{{12 - 5 * (2 - sqrt(3)) }}{-1}{dtu-corporate-red}{Technician}
		\drawHexagon{{-4 + 3 * (2 - sqrt(3)) }}{-1}{dtu-corporate-red}{Technician}
	\fi

	
	\end{tikzpicture}
}

\ifstandalone
	

\definecolor{red}{HTML}{8A3F3A}
\definecolor{yellow}{HTML}{E0BB3C}
\definecolor{blue}{HTML}{4569E0}
\definecolor{green}{HTML}{17E561}
\definecolor{other}{HTML}{6A939E}

	\drawModelSetupHexagon
\fi

\end{document}

	\resizebox{0.7\textwidth}{!}{
		\drawModelSetupHexagon[persistence=true]
	}
	\caption{
		Overview of the scheduling process when modelled as actors. When LNS is encapsulated 
		is an actor it becomes possible to optimize parts of a large process individually instead of 
		optimizing the scheduling problem globally from a single model implementation.
	}\label{fig:ordinator-hexagon:persistence}
\end{figure}
\begin{figure}[H]
	\centering
	\documentclass{standalone}
\usepackage{tikz}
\usetikzlibrary {positioning}

<<<<<<< HEAD
\newcommand{\drawHexagon}[5][draw=black]{
	\draw[#1, fill=#4, ] (#2,#3) ++(30:2) -- ++(90:2) -- ++(150:2) -- ++(210:2) -- ++(270:2) -- ++(330:2) -- cycle;
	\node at (#2,#3+2) {#5};
||||||| 2d4327a
\newcommand{\drawHexagon}[4]{
	\draw[fill=#3] (#1,#2) ++(30:2) -- ++(90:2) -- ++(150:2) -- ++(210:2) -- ++(270:2) -- ++(330:2) -- cycle;
	\node at (#1,#2+2) {#4};
=======
\begin{document}
\newcommand{\drawHexagon}[4]{
	\draw[fill=#3] (#1,#2) ++(30:2) -- ++(90:2) -- ++(150:2) -- ++(210:2) -- ++(270:2) -- ++(330:2) -- cycle;
	\node at (#1,#2+2) {#4};
>>>>>>> 023797133fb426c1bb01f920d8f5635c343d11a6
}

\newif\ifuserinterfacelayer
\newif\ifpersistencelayer
\newif\ifmetaheuristicslayer
\newif\ifatomicpointerswaplayer

\pgfkeys{
	/hexagon/.is family, /hexagon,
	default/.style = {
		persistence=false,
		userinterface=false,
		metaheuristics=true,
	},
	persistence/.is if=persistencelayer,
	userinterface/.is if=userinterfacelayer,
	metaheuristics/.is if=metaheuristicslayer,
}
\newcommand{\drawModelSetupHexagon}[1][]{
	\pgfkeys{/hexagon, default, #1}

	\begin{tikzpicture}[scale=0.6, line width=1.05]
	
	\ifuserinterfacelayer
		\drawHexagon{ 2                      }{ 2}{dtu-yellow}{UI}
		\drawHexagon{{6 - 2 * (2 - sqrt(3)) }}{ 2}{dtu-yellow}{UI}
		\drawHexagon{{4 - 1 * (2 - sqrt(3)) }}{-1}{dtu-yellow}{UI}
		\drawHexagon{{0 + 1 * (2 - sqrt(3)) }}{-1}{dtu-yellow}{UI}
		\drawHexagon{{8 - 3 * (2 - sqrt(3)) }}{-1}{dtu-yellow}{UI}

		\drawHexagon{{2 - 0 * (2 - sqrt(3)) }}{-4}{dtu-yellow}{UI}
		\drawHexagon{{6 - 2 * (2 - sqrt(3)) }}{-4}{dtu-yellow}{UI}

		\drawHexagon{{10 - 4 * (2 - sqrt(3)) }}{-4}{dtu-yellow}{UI}
		\drawHexagon{{-2 + 2 * (2 - sqrt(3)) }}{-4}{dtu-yellow}{UI}

		\drawHexagon{{12 - 5 * (2 - sqrt(3)) }}{-1}{dtu-yellow}{UI}
		\drawHexagon{{-4 + 3 * (2 - sqrt(3)) }}{-1}{dtu-yellow}{UI}
	\fi

	\ifpersistencelayer
		\drawHexagon[draw=none]{ 2                      }{ 2}{dtu-blue}{}
		\drawHexagon[draw=none]{{6 - 2 * (2 - sqrt(3)) }}{ 2}{dtu-blue}{}
		\drawHexagon[draw=none]{{4 - 1 * (2 - sqrt(3)) }}{-1}{dtu-blue}{Database}
		\drawHexagon[draw=none]{{0 + 1 * (2 - sqrt(3)) }}{-1}{dtu-blue}{}
		\drawHexagon[draw=none]{{8 - 3 * (2 - sqrt(3)) }}{-1}{dtu-blue}{}

		\drawHexagon[draw=none]{{2 - 0 * (2 - sqrt(3)) }}{-4}{dtu-blue}{}
		\drawHexagon[draw=none]{{6 - 2 * (2 - sqrt(3)) }}{-4}{dtu-blue}{}

		\drawHexagon[draw=none]{{10 - 4 * (2 - sqrt(3)) }}{-4}{dtu-blue}{}
		\drawHexagon[draw=none]{{-2 + 2 * (2 - sqrt(3)) }}{-4}{dtu-blue}{}

		\drawHexagon[draw=none]{{12 - 5 * (2 - sqrt(3)) }}{-1}{dtu-blue}{}
		\drawHexagon[draw=none]{{-4 + 3 * (2 - sqrt(3)) }}{-1}{dtu-blue}{}
	\fi

	\ifmetaheuristicslayer
		\drawHexagon{ 2                      }{ 2}{dtu-blue}{Strategic}
		\drawHexagon{{6 - 2 * (2 - sqrt(3)) }}{ 2}{dtu-green}{Tactical}
		\drawHexagon{{4 - 1 * (2 - sqrt(3)) }}{-1}{dtu-red}{Supervisor}
		\drawHexagon{{0 + 1 * (2 - sqrt(3)) }}{-1}{dtu-red}{Supervisor}
		\drawHexagon{{8 - 3 * (2 - sqrt(3)) }}{-1}{dtu-red}{Supervisor}

		\drawHexagon{{2 - 0 * (2 - sqrt(3)) }}{-4}{dtu-corporate-red}{Technician}
		\drawHexagon{{6 - 2 * (2 - sqrt(3)) }}{-4}{dtu-corporate-red}{Technician}

		\drawHexagon{{10 - 4 * (2 - sqrt(3)) }}{-4}{dtu-corporate-red}{Technician}
		\drawHexagon{{-2 + 2 * (2 - sqrt(3)) }}{-4}{dtu-corporate-red}{Technician}

		\drawHexagon{{12 - 5 * (2 - sqrt(3)) }}{-1}{dtu-corporate-red}{Technician}
		\drawHexagon{{-4 + 3 * (2 - sqrt(3)) }}{-1}{dtu-corporate-red}{Technician}
	\fi

	
	\end{tikzpicture}
}

\ifstandalone
	

\definecolor{red}{HTML}{8A3F3A}
\definecolor{yellow}{HTML}{E0BB3C}
\definecolor{blue}{HTML}{4569E0}
\definecolor{green}{HTML}{17E561}
\definecolor{other}{HTML}{6A939E}

	\drawModelSetupHexagon
\fi

\end{document}

	\resizebox{0.7\textwidth}{!}{
		\drawModelSetupHexagon[atomicpointerswap=true]
	}
	\caption{
		Overview of the scheduling process when modelled as actors. When LNS is encapsulated 
		is an actor it becomes possible to optimize parts of a large process individually instead of 
		optimizing the scheduling problem globally from a single model implementation.
	}
	\label{fig:ordinator-hexagon:atomicpointerswap}
\end{figure}

\begin{figure}[H]
	\centering
	\documentclass{standalone}
\usepackage{tikz}
\usetikzlibrary {positioning}

<<<<<<< HEAD
\newcommand{\drawHexagon}[5][draw=black]{
	\draw[#1, fill=#4, ] (#2,#3) ++(30:2) -- ++(90:2) -- ++(150:2) -- ++(210:2) -- ++(270:2) -- ++(330:2) -- cycle;
	\node at (#2,#3+2) {#5};
||||||| 2d4327a
\newcommand{\drawHexagon}[4]{
	\draw[fill=#3] (#1,#2) ++(30:2) -- ++(90:2) -- ++(150:2) -- ++(210:2) -- ++(270:2) -- ++(330:2) -- cycle;
	\node at (#1,#2+2) {#4};
=======
\begin{document}
\newcommand{\drawHexagon}[4]{
	\draw[fill=#3] (#1,#2) ++(30:2) -- ++(90:2) -- ++(150:2) -- ++(210:2) -- ++(270:2) -- ++(330:2) -- cycle;
	\node at (#1,#2+2) {#4};
>>>>>>> 023797133fb426c1bb01f920d8f5635c343d11a6
}

\newif\ifuserinterfacelayer
\newif\ifpersistencelayer
\newif\ifmetaheuristicslayer
\newif\ifatomicpointerswaplayer

\pgfkeys{
	/hexagon/.is family, /hexagon,
	default/.style = {
		persistence=false,
		userinterface=false,
		metaheuristics=true,
	},
	persistence/.is if=persistencelayer,
	userinterface/.is if=userinterfacelayer,
	metaheuristics/.is if=metaheuristicslayer,
}
\newcommand{\drawModelSetupHexagon}[1][]{
	\pgfkeys{/hexagon, default, #1}

	\begin{tikzpicture}[scale=0.6, line width=1.05]
	
	\ifuserinterfacelayer
		\drawHexagon{ 2                      }{ 2}{dtu-yellow}{UI}
		\drawHexagon{{6 - 2 * (2 - sqrt(3)) }}{ 2}{dtu-yellow}{UI}
		\drawHexagon{{4 - 1 * (2 - sqrt(3)) }}{-1}{dtu-yellow}{UI}
		\drawHexagon{{0 + 1 * (2 - sqrt(3)) }}{-1}{dtu-yellow}{UI}
		\drawHexagon{{8 - 3 * (2 - sqrt(3)) }}{-1}{dtu-yellow}{UI}

		\drawHexagon{{2 - 0 * (2 - sqrt(3)) }}{-4}{dtu-yellow}{UI}
		\drawHexagon{{6 - 2 * (2 - sqrt(3)) }}{-4}{dtu-yellow}{UI}

		\drawHexagon{{10 - 4 * (2 - sqrt(3)) }}{-4}{dtu-yellow}{UI}
		\drawHexagon{{-2 + 2 * (2 - sqrt(3)) }}{-4}{dtu-yellow}{UI}

		\drawHexagon{{12 - 5 * (2 - sqrt(3)) }}{-1}{dtu-yellow}{UI}
		\drawHexagon{{-4 + 3 * (2 - sqrt(3)) }}{-1}{dtu-yellow}{UI}
	\fi

	\ifpersistencelayer
		\drawHexagon[draw=none]{ 2                      }{ 2}{dtu-blue}{}
		\drawHexagon[draw=none]{{6 - 2 * (2 - sqrt(3)) }}{ 2}{dtu-blue}{}
		\drawHexagon[draw=none]{{4 - 1 * (2 - sqrt(3)) }}{-1}{dtu-blue}{Database}
		\drawHexagon[draw=none]{{0 + 1 * (2 - sqrt(3)) }}{-1}{dtu-blue}{}
		\drawHexagon[draw=none]{{8 - 3 * (2 - sqrt(3)) }}{-1}{dtu-blue}{}

		\drawHexagon[draw=none]{{2 - 0 * (2 - sqrt(3)) }}{-4}{dtu-blue}{}
		\drawHexagon[draw=none]{{6 - 2 * (2 - sqrt(3)) }}{-4}{dtu-blue}{}

		\drawHexagon[draw=none]{{10 - 4 * (2 - sqrt(3)) }}{-4}{dtu-blue}{}
		\drawHexagon[draw=none]{{-2 + 2 * (2 - sqrt(3)) }}{-4}{dtu-blue}{}

		\drawHexagon[draw=none]{{12 - 5 * (2 - sqrt(3)) }}{-1}{dtu-blue}{}
		\drawHexagon[draw=none]{{-4 + 3 * (2 - sqrt(3)) }}{-1}{dtu-blue}{}
	\fi

	\ifmetaheuristicslayer
		\drawHexagon{ 2                      }{ 2}{dtu-blue}{Strategic}
		\drawHexagon{{6 - 2 * (2 - sqrt(3)) }}{ 2}{dtu-green}{Tactical}
		\drawHexagon{{4 - 1 * (2 - sqrt(3)) }}{-1}{dtu-red}{Supervisor}
		\drawHexagon{{0 + 1 * (2 - sqrt(3)) }}{-1}{dtu-red}{Supervisor}
		\drawHexagon{{8 - 3 * (2 - sqrt(3)) }}{-1}{dtu-red}{Supervisor}

		\drawHexagon{{2 - 0 * (2 - sqrt(3)) }}{-4}{dtu-corporate-red}{Technician}
		\drawHexagon{{6 - 2 * (2 - sqrt(3)) }}{-4}{dtu-corporate-red}{Technician}

		\drawHexagon{{10 - 4 * (2 - sqrt(3)) }}{-4}{dtu-corporate-red}{Technician}
		\drawHexagon{{-2 + 2 * (2 - sqrt(3)) }}{-4}{dtu-corporate-red}{Technician}

		\drawHexagon{{12 - 5 * (2 - sqrt(3)) }}{-1}{dtu-corporate-red}{Technician}
		\drawHexagon{{-4 + 3 * (2 - sqrt(3)) }}{-1}{dtu-corporate-red}{Technician}
	\fi

	
	\end{tikzpicture}
}

\ifstandalone
	

\definecolor{red}{HTML}{8A3F3A}
\definecolor{yellow}{HTML}{E0BB3C}
\definecolor{blue}{HTML}{4569E0}
\definecolor{green}{HTML}{17E561}
\definecolor{other}{HTML}{6A939E}

	\drawModelSetupHexagon
\fi

\end{document}

	\resizebox{0.7\textwidth}{!}{
		\drawModelSetupHexagon[metaheuristics=true]
	}
	\caption{
		Overview of the scheduling process when modelled as actors. When LNS is encapsulated 
		is an actor it becomes possible to optimize parts of a large process individually instead of 
		optimizing the scheduling problem globally from a single model implementation.
	}
	\label{fig:ordinator-hexagon:metaheuristics}
\end{figure}

\begin{figure}[H]
	\centering
	\documentclass{standalone}
\usepackage{tikz}
\usetikzlibrary {positioning}

<<<<<<< HEAD
\newcommand{\drawHexagon}[5][draw=black]{
	\draw[#1, fill=#4, ] (#2,#3) ++(30:2) -- ++(90:2) -- ++(150:2) -- ++(210:2) -- ++(270:2) -- ++(330:2) -- cycle;
	\node at (#2,#3+2) {#5};
||||||| 2d4327a
\newcommand{\drawHexagon}[4]{
	\draw[fill=#3] (#1,#2) ++(30:2) -- ++(90:2) -- ++(150:2) -- ++(210:2) -- ++(270:2) -- ++(330:2) -- cycle;
	\node at (#1,#2+2) {#4};
=======
\begin{document}
\newcommand{\drawHexagon}[4]{
	\draw[fill=#3] (#1,#2) ++(30:2) -- ++(90:2) -- ++(150:2) -- ++(210:2) -- ++(270:2) -- ++(330:2) -- cycle;
	\node at (#1,#2+2) {#4};
>>>>>>> 023797133fb426c1bb01f920d8f5635c343d11a6
}

\newif\ifuserinterfacelayer
\newif\ifpersistencelayer
\newif\ifmetaheuristicslayer
\newif\ifatomicpointerswaplayer

\pgfkeys{
	/hexagon/.is family, /hexagon,
	default/.style = {
		persistence=false,
		userinterface=false,
		metaheuristics=true,
	},
	persistence/.is if=persistencelayer,
	userinterface/.is if=userinterfacelayer,
	metaheuristics/.is if=metaheuristicslayer,
}
\newcommand{\drawModelSetupHexagon}[1][]{
	\pgfkeys{/hexagon, default, #1}

	\begin{tikzpicture}[scale=0.6, line width=1.05]
	
	\ifuserinterfacelayer
		\drawHexagon{ 2                      }{ 2}{dtu-yellow}{UI}
		\drawHexagon{{6 - 2 * (2 - sqrt(3)) }}{ 2}{dtu-yellow}{UI}
		\drawHexagon{{4 - 1 * (2 - sqrt(3)) }}{-1}{dtu-yellow}{UI}
		\drawHexagon{{0 + 1 * (2 - sqrt(3)) }}{-1}{dtu-yellow}{UI}
		\drawHexagon{{8 - 3 * (2 - sqrt(3)) }}{-1}{dtu-yellow}{UI}

		\drawHexagon{{2 - 0 * (2 - sqrt(3)) }}{-4}{dtu-yellow}{UI}
		\drawHexagon{{6 - 2 * (2 - sqrt(3)) }}{-4}{dtu-yellow}{UI}

		\drawHexagon{{10 - 4 * (2 - sqrt(3)) }}{-4}{dtu-yellow}{UI}
		\drawHexagon{{-2 + 2 * (2 - sqrt(3)) }}{-4}{dtu-yellow}{UI}

		\drawHexagon{{12 - 5 * (2 - sqrt(3)) }}{-1}{dtu-yellow}{UI}
		\drawHexagon{{-4 + 3 * (2 - sqrt(3)) }}{-1}{dtu-yellow}{UI}
	\fi

	\ifpersistencelayer
		\drawHexagon[draw=none]{ 2                      }{ 2}{dtu-blue}{}
		\drawHexagon[draw=none]{{6 - 2 * (2 - sqrt(3)) }}{ 2}{dtu-blue}{}
		\drawHexagon[draw=none]{{4 - 1 * (2 - sqrt(3)) }}{-1}{dtu-blue}{Database}
		\drawHexagon[draw=none]{{0 + 1 * (2 - sqrt(3)) }}{-1}{dtu-blue}{}
		\drawHexagon[draw=none]{{8 - 3 * (2 - sqrt(3)) }}{-1}{dtu-blue}{}

		\drawHexagon[draw=none]{{2 - 0 * (2 - sqrt(3)) }}{-4}{dtu-blue}{}
		\drawHexagon[draw=none]{{6 - 2 * (2 - sqrt(3)) }}{-4}{dtu-blue}{}

		\drawHexagon[draw=none]{{10 - 4 * (2 - sqrt(3)) }}{-4}{dtu-blue}{}
		\drawHexagon[draw=none]{{-2 + 2 * (2 - sqrt(3)) }}{-4}{dtu-blue}{}

		\drawHexagon[draw=none]{{12 - 5 * (2 - sqrt(3)) }}{-1}{dtu-blue}{}
		\drawHexagon[draw=none]{{-4 + 3 * (2 - sqrt(3)) }}{-1}{dtu-blue}{}
	\fi

	\ifmetaheuristicslayer
		\drawHexagon{ 2                      }{ 2}{dtu-blue}{Strategic}
		\drawHexagon{{6 - 2 * (2 - sqrt(3)) }}{ 2}{dtu-green}{Tactical}
		\drawHexagon{{4 - 1 * (2 - sqrt(3)) }}{-1}{dtu-red}{Supervisor}
		\drawHexagon{{0 + 1 * (2 - sqrt(3)) }}{-1}{dtu-red}{Supervisor}
		\drawHexagon{{8 - 3 * (2 - sqrt(3)) }}{-1}{dtu-red}{Supervisor}

		\drawHexagon{{2 - 0 * (2 - sqrt(3)) }}{-4}{dtu-corporate-red}{Technician}
		\drawHexagon{{6 - 2 * (2 - sqrt(3)) }}{-4}{dtu-corporate-red}{Technician}

		\drawHexagon{{10 - 4 * (2 - sqrt(3)) }}{-4}{dtu-corporate-red}{Technician}
		\drawHexagon{{-2 + 2 * (2 - sqrt(3)) }}{-4}{dtu-corporate-red}{Technician}

		\drawHexagon{{12 - 5 * (2 - sqrt(3)) }}{-1}{dtu-corporate-red}{Technician}
		\drawHexagon{{-4 + 3 * (2 - sqrt(3)) }}{-1}{dtu-corporate-red}{Technician}
	\fi

	
	\end{tikzpicture}
}

\ifstandalone
	

\definecolor{red}{HTML}{8A3F3A}
\definecolor{yellow}{HTML}{E0BB3C}
\definecolor{blue}{HTML}{4569E0}
\definecolor{green}{HTML}{17E561}
\definecolor{other}{HTML}{6A939E}

	\drawModelSetupHexagon
\fi

\end{document}

	\resizebox{0.7\textwidth}{!}{
		\drawModelSetupHexagon[orchestrator=true]
	}
	\caption{
		Overview of the scheduling process when modelled as actors. When LNS is encapsulated 
		is an actor it becomes possible to optimize parts of a large process individually instead of 
		optimizing the scheduling problem globally from a single model implementation.
	}
	\label{fig:ordinator-hexagon:orchestrator}
\end{figure}

\begin{figure}[H]
	\centering
	\documentclass{standalone}
\usepackage{tikz}
\usetikzlibrary {positioning}

<<<<<<< HEAD
\newcommand{\drawHexagon}[5][draw=black]{
	\draw[#1, fill=#4, ] (#2,#3) ++(30:2) -- ++(90:2) -- ++(150:2) -- ++(210:2) -- ++(270:2) -- ++(330:2) -- cycle;
	\node at (#2,#3+2) {#5};
||||||| 2d4327a
\newcommand{\drawHexagon}[4]{
	\draw[fill=#3] (#1,#2) ++(30:2) -- ++(90:2) -- ++(150:2) -- ++(210:2) -- ++(270:2) -- ++(330:2) -- cycle;
	\node at (#1,#2+2) {#4};
=======
\begin{document}
\newcommand{\drawHexagon}[4]{
	\draw[fill=#3] (#1,#2) ++(30:2) -- ++(90:2) -- ++(150:2) -- ++(210:2) -- ++(270:2) -- ++(330:2) -- cycle;
	\node at (#1,#2+2) {#4};
>>>>>>> 023797133fb426c1bb01f920d8f5635c343d11a6
}

\newif\ifuserinterfacelayer
\newif\ifpersistencelayer
\newif\ifmetaheuristicslayer
\newif\ifatomicpointerswaplayer

\pgfkeys{
	/hexagon/.is family, /hexagon,
	default/.style = {
		persistence=false,
		userinterface=false,
		metaheuristics=true,
	},
	persistence/.is if=persistencelayer,
	userinterface/.is if=userinterfacelayer,
	metaheuristics/.is if=metaheuristicslayer,
}
\newcommand{\drawModelSetupHexagon}[1][]{
	\pgfkeys{/hexagon, default, #1}

	\begin{tikzpicture}[scale=0.6, line width=1.05]
	
	\ifuserinterfacelayer
		\drawHexagon{ 2                      }{ 2}{dtu-yellow}{UI}
		\drawHexagon{{6 - 2 * (2 - sqrt(3)) }}{ 2}{dtu-yellow}{UI}
		\drawHexagon{{4 - 1 * (2 - sqrt(3)) }}{-1}{dtu-yellow}{UI}
		\drawHexagon{{0 + 1 * (2 - sqrt(3)) }}{-1}{dtu-yellow}{UI}
		\drawHexagon{{8 - 3 * (2 - sqrt(3)) }}{-1}{dtu-yellow}{UI}

		\drawHexagon{{2 - 0 * (2 - sqrt(3)) }}{-4}{dtu-yellow}{UI}
		\drawHexagon{{6 - 2 * (2 - sqrt(3)) }}{-4}{dtu-yellow}{UI}

		\drawHexagon{{10 - 4 * (2 - sqrt(3)) }}{-4}{dtu-yellow}{UI}
		\drawHexagon{{-2 + 2 * (2 - sqrt(3)) }}{-4}{dtu-yellow}{UI}

		\drawHexagon{{12 - 5 * (2 - sqrt(3)) }}{-1}{dtu-yellow}{UI}
		\drawHexagon{{-4 + 3 * (2 - sqrt(3)) }}{-1}{dtu-yellow}{UI}
	\fi

	\ifpersistencelayer
		\drawHexagon[draw=none]{ 2                      }{ 2}{dtu-blue}{}
		\drawHexagon[draw=none]{{6 - 2 * (2 - sqrt(3)) }}{ 2}{dtu-blue}{}
		\drawHexagon[draw=none]{{4 - 1 * (2 - sqrt(3)) }}{-1}{dtu-blue}{Database}
		\drawHexagon[draw=none]{{0 + 1 * (2 - sqrt(3)) }}{-1}{dtu-blue}{}
		\drawHexagon[draw=none]{{8 - 3 * (2 - sqrt(3)) }}{-1}{dtu-blue}{}

		\drawHexagon[draw=none]{{2 - 0 * (2 - sqrt(3)) }}{-4}{dtu-blue}{}
		\drawHexagon[draw=none]{{6 - 2 * (2 - sqrt(3)) }}{-4}{dtu-blue}{}

		\drawHexagon[draw=none]{{10 - 4 * (2 - sqrt(3)) }}{-4}{dtu-blue}{}
		\drawHexagon[draw=none]{{-2 + 2 * (2 - sqrt(3)) }}{-4}{dtu-blue}{}

		\drawHexagon[draw=none]{{12 - 5 * (2 - sqrt(3)) }}{-1}{dtu-blue}{}
		\drawHexagon[draw=none]{{-4 + 3 * (2 - sqrt(3)) }}{-1}{dtu-blue}{}
	\fi

	\ifmetaheuristicslayer
		\drawHexagon{ 2                      }{ 2}{dtu-blue}{Strategic}
		\drawHexagon{{6 - 2 * (2 - sqrt(3)) }}{ 2}{dtu-green}{Tactical}
		\drawHexagon{{4 - 1 * (2 - sqrt(3)) }}{-1}{dtu-red}{Supervisor}
		\drawHexagon{{0 + 1 * (2 - sqrt(3)) }}{-1}{dtu-red}{Supervisor}
		\drawHexagon{{8 - 3 * (2 - sqrt(3)) }}{-1}{dtu-red}{Supervisor}

		\drawHexagon{{2 - 0 * (2 - sqrt(3)) }}{-4}{dtu-corporate-red}{Technician}
		\drawHexagon{{6 - 2 * (2 - sqrt(3)) }}{-4}{dtu-corporate-red}{Technician}

		\drawHexagon{{10 - 4 * (2 - sqrt(3)) }}{-4}{dtu-corporate-red}{Technician}
		\drawHexagon{{-2 + 2 * (2 - sqrt(3)) }}{-4}{dtu-corporate-red}{Technician}

		\drawHexagon{{12 - 5 * (2 - sqrt(3)) }}{-1}{dtu-corporate-red}{Technician}
		\drawHexagon{{-4 + 3 * (2 - sqrt(3)) }}{-1}{dtu-corporate-red}{Technician}
	\fi

	
	\end{tikzpicture}
}

\ifstandalone
	

\definecolor{red}{HTML}{8A3F3A}
\definecolor{yellow}{HTML}{E0BB3C}
\definecolor{blue}{HTML}{4569E0}
\definecolor{green}{HTML}{17E561}
\definecolor{other}{HTML}{6A939E}

	\drawModelSetupHexagon
\fi

\end{document}

	\resizebox{0.7\textwidth}{!}{
		\drawModelSetupHexagon[userinterface=true]
	}
	\caption{
		Overview of the scheduling process when modelled as actors. When LNS is encapsulated 
		is an actor it becomes possible to optimize parts of a large process individually instead of 
		optimizing the scheduling problem globally from a single model implementation.
	}
	\label{fig:ordinator-hexagon:userinterfaces}
\end{figure}

\subsection{Mathematical Models}
Each actor of the maintenance scheduling process will here be modelled by a mathematical model. There
are multiple driving ideas behind these approaches:
\begin{itemize}
	\item A model should encapsulate the decision that a single decision-maker is responsible for and
		nothing more
	\item Coordination between different actors is done by sharing relevant information 
		by turning the solution of one actor/model into parameters of the other model.
\end{itemize}

\subsubsection{Strategic Model: Scheduler}
\section{The Strategic Model}


The Strategic Model have multiple different purposes.
\begin{itemize}
	\item Schedule Work Order out across the weekly periods
	\item Prioritize all the different released work orders
	\item Respect the available weekly hours available for each trait
\end{itemize}

The Strategic model is responsible for grouping work orders into weekly or biweekly periods depending on which kind of maintenance setup that one is running.
This kind of model closely resembles a variant of the multi-compartment multi-knapsack problem. 

\begin{alignat}{2}
	\text{Min} &\sum_{w \in W} \sum_{p \in P} v_{wp}(t) \cdot x_{wp}(t)                                                                                             \\ 
	+ & \sum_{p \in P} \sum_{\tau = 1}^T d \cdot pen_{p\tau}(t)                                                                                                     \\
	+ & \sum_{p \in P} \sum_{w1 \in W} \sum_{w2 \in W} clu_{w1, w2} \cdot x_{w1p} \cdot x_{w2p}                              \label{eqn:objective_function_strategic} \\
    & \text{subject to:} \notag                                                                                                                                       \\
	& \sum_{w \in W} c_{w\tau} \cdot x_{wp}(t) \leq \ cap_{p\tau}(t) + pen_{p\tau}(t) \notag                                                                          \\ 
	& \forall p \in P, \forall \tau \in T                                                                                    \label{eqn:capacity_constraint}          \\
	& \sum_{w \in W} x_{wp}(t) = 1                                              , \quad \forall p \in P                      \label{eqn:single_workorder_constraint}  \\
	& x_{wp}(t) = 0                                                             , \quad \forall (w, p) \in E(t)              \label{eqn:exclusion_constraint}         \\
	& x_{wp}(t) = 1                                                             , \quad \forall (w, p) \in I(t)              \label{eqn:inclusion_constraint}         \\
	& x_{wp}(t) \in \{0, 1\}                                                    , \quad \forall w \in W, \forall p \in P     \label{eqn:x_integrality_constraint}     \\ 
	& pen_{p\tau}(t) \in \mathbb{R}^{+}                                         , \quad \forall p \in P, \forall \tau \in T  \label{eqn:p_non_negativity_constraint}
    & t \in  [0, \infty] 
\end{alignat}

\strategicmodel

\subsubsection{Tactical Model: Scheduler}
\section{The Tactical Model}
\begin{itemize}
	\item Respect precedence constraints
	\item Respect daily resource requirements for each trait
	\item Penalize exceeded daily capacity
\end{itemize}

After the strategic model has optimized its schedule the tactical agent will continue scheduling the output at a more detailed level. This means that now the tactical agent will schedule 
out on each of the days of the work orders scheduled by the strategic agent. 

The tactical model is responsible for providing an initial suggestion for a weekly schedule, below we see the model for the tactical agent.
\begin{alignat}{2}
\text{Min}     & \sum_{o \in O} \sum_{d \in D} v_{do}(t) \cdot y_{do}(t)                                                      \\  
	         + & \sum_{c \in C} \sum_{d \in D} pen \cdot p_{cd}(t)                                               \\  
			                                                                                                  \\
               &\text{subject to:}                                                          \notag                                                                   \\
	           & \sum_{o \in O} w_{co} \cdot y_{do}(t)  \leq R_{dc} + p_{dc}(t)                                   \\ 
			   & \quad \quad \forall  d \in D, \forall c \in C                              \notag                                    \\ 
	           & \sum_{d \in D} d \cdot y_{do1}(t) + \delta_o  = \sum_{d \in D} d \cdot y_{do2}(t)                    \\ 
			   & \quad \quad \forall (o1, o2) \in \text{finish-start}(t)                        \notag                                   \\ 
	           & \sum_{d \in D} d \cdot y_{do1}(t) = \sum_{d \in D} d \cdot y_{do2}(t)                                \\ 
			   & \quad \quad \forall (o1, o2) \in \text{start-start}(t)                          \notag                             \\ 
			   & y_{do}(t) \leq number_o(t) * operating\_time_o                                                     \\ 
			   & \quad \quad ,\forall d \in D, \forall o \in O                              \notag                                    \\
			   & y_{do}(t) \in \{0, 1\} \quad ,\forall d \in D, o \in O                                          \\
			   & p_{cd}(t) \in \mathbb{R} \quad ,\forall c \in C, d \in D                                        \\
			   & \delta_o \in \{ duration\_lower_o(t),                                                           \\ 
			   & \quad \quad duration\_upper_o \}(t) \quad, \forall o \in O \notag                                      \\
			   & t \in  [0, \infty] 
\end{alignat}


\subsubsection{Supervisor Model: Supervisor}
\newif\ifincludenormal\

\pgfkeys{
	/supervisormodel/.is family, /supervisormodel,
	default/.style = {
		normal=true,
	},
	normal/.is if=includenormal,
}
\newcommand{\supervisorModel}[1][]{
	\pgfkeys{/supervisormodel, default, #1}
	\begin{alignat}{2}
		& \text{\rule{\linewidth}{0.8pt}} \notag \label{}                                                                                                                                                                                                                                                                                                                                                                     \\ 
		& \textbf{Meta variables:} \notag\\
		& \ElementSupervisor \in \SetSupervisor \\
		& \VarStrategicWorkOrderAssignment{}{} \\
		& \VarIncludeActivity{} \\
		& \tau \in [0, \infty] \\
		& \text{\rule{\linewidth}{0.4pt}} \notag\\
		& \textbf{Maximize:} \notag\\
		& \sum_{\ElementActivity \in \SetActivity{\VarStrategicWorkOrderAssignment{}{}}{}} \sum_{\ElementTechnician \in \SetTechnician} \ParSupervisorValue \cdot \VarSupervisorAssignment{\ElementActivity}{\ElementTechnician} \\ 
		& \text{\rule{\linewidth}{0.4pt}} \notag\\
		& \textbf{Subject to:} \notag\\ 
		& \sum_{\ElementActivity \in \SetActivity{\VarStrategicWorkOrderAssignment{}{}}{\ElementOperation}} \VarActivityWork{\ElementActivity} = \ParOperationWork{\ElementOperation}    \quad \forall \ElementOperation \in \SetOperation{}{\VarStrategicWorkOrderAssignment{}{}}\\
		& \sum_{\ElementTechnician \in \SetTechnician} \sum_{\ElementActivity \in \SetActivity{\VarStrategicWorkOrderAssignment{}{}}{\ElementOperation}}\VarSupervisorAssignment{\ElementActivity}{\ElementTechnician} = \VarSupervisorOperationWhole \cdot \ParNumberOfPeople  \quad \forall \ElementOperation \in \SetOperation{}{\VarStrategicWorkOrderAssignment{}{}}  \\
		& \sum_{\ElementOperation \in \SetOperation{\ElementWorkOrder}{\VarStrategicWorkOrderAssignment{}{}}} \VarSupervisorOperationWhole = |\SetOperation{\ElementWorkOrder}{\VarStrategicWorkOrderAssignment{}{}}| \cdot \VarSupervisorWorkOrderWhole  \quad \forall \ElementWorkOrder \in \SetWorkOrder{,\VarStrategicWorkOrderAssignment{}{}} \\
		& \sum_{\ElementActivity \in \SetActivity{\VarStrategicWorkOrderAssignment{}{}}{\ElementOperation}} \VarSupervisorAssignment{\ElementActivity}{\ElementTechnician} \leq 1  \quad \forall \ElementOperation \in \SetOperation{}{\VarStrategicWorkOrderAssignment{}{}} \quad \forall \ElementTechnician \in \SetTechnician \\  
		& \VarSupervisorAssignment{\ElementActivity}{\ElementTechnician} \leq \ParFeasible  \quad \forall \ElementActivity \in \SetActivity{\VarTacticalWork}{\ElementOperation} \quad \forall \ElementOperation \in \SetOperation{}{\VarStrategicWorkOrderAssignment{}{}} \quad \forall \ElementTechnician \in \SetTechnician \\
		& \VarSupervisorAssignment{\ElementActivity}{\ElementTechnician} \in \{0, 1\}  \quad \forall \ElementOperation \in \SetOperation{}{\VarStrategicWorkOrderAssignment{}{}} \quad \forall \ElementTechnician \in \SetTechnician \\ 
		& \VarSupervisorOperationWhole \in \{0, 1\}  \quad \forall \ElementOperation \in \SetOperation{}{\VarStrategicWorkOrderAssignment{}{}} \\ 
		& \VarSupervisorWorkOrderWhole \in \{0, 1\}  \quad \forall \ElementWorkOrder \in \SetWorkOrder{,\VarStrategicWorkOrderAssignment{}{}} \\ 
		& \VarActivityWork{\ElementActivity} \in [\ParLowerActivityWork, \ParOperationWork{\ElementActivity}]  \quad \forall \ElementActivity \in \SetActivity{\VarStrategicWorkOrderAssignment{}{}}{}\\ 
		& \text{\rule{\linewidth}{0.8pt}} \notag \label{}                                                                                                                                                                                                                                                                                                                                                                      
	\end{alignat}
}

\subsubsection{Operational Model: Technicial}
\section{The Operational Model}

Here the o is a single operation and o2 is another operation. It is crucial to understand here that the main
decision variable, $x$ defines an ordering of the operations that a single operational agent will do the 
operations in. 

The $\VarStartOfSegment{a}{k}$ is the start time of job $i$ in segment $k$ and $\VarFinishOfSegment{a}{k}$ is the finish time of job $i$ in segment $k$.
$\VarProcessingTime{a}{k}$ is the processing time of each segment. 
\begin{alignat}{2}
	& Max \sum_{a \in \SetActivity{\VarSupervisorAssignment}} \sum_{k \in \SetWorkSegment} \VarProcessingTime                                                         \\
	& \text{Subject to:} \notag                                                                                                                                       \\
    & \sum_{k \in \SetWorkSegment} \VarProcessingTime \cdot \VarActiveSegment{a}{k} = \ParOperationWork \cdot \VarIncludeActivity \quad \forall a \in \SetActivity{\VarSupervisorAssignment} \\
	& \VarStartOfSegment{a2}{1} \geq \VarFinishOfSegment{a1}{last(a1)} + \ParPreparation \notag                                                                       \\ 
	& \quad \forall a1 \in \SetActivity{\VarSupervisorAssignment}, a2 \in \SetActivity{\VarSupervisorAssignment}                                                     \\
	& \VarStartOfSegment{a}{k} \geq \VarFinishOfSegment{a}{k-1} - \ParConstraintLimit \cdot (2 - \VarActiveSegment{a}{k} + \VarActiveSegment{a}{k-1})                \notag\\
	& \quad \forall a \in \SetActivity{\VarSupervisorAssignment}\forall k \in \SetWorkSegment \\ 
	& \VarProcessingTime = \VarFinishOfSegment{a}{k} - \VarStartOfSegment{a}{k}                                                                                       \\
	& \quad \forall a \in \SetActivity{\VarSupervisorAssignment}, k \in \SetWorkSegment \notag                                                                        \\
	& \VarStartOfSegment{a}{k} \geq \ParEvent + \ParEventDuration - \ParConstraintLimit \cdot (1 - \VarSegmentInRelation)                                             \notag\\ 
	& \quad \forall a \in \SetActivity{\VarSupervisorAssignment}, k \in \SetWorkSegment, i \in \SetTimeInstance, e \in \SetEvent                                \\
	& \VarFinishOfSegment{a}{k} \leq \ParEvent + \ParConstraintLimit \cdot \VarSegmentInRelation                                                                      \notag\\ 
	& \quad \forall a \in \SetActivity{\VarSupervisorAssignment}, k \in \SetWorkSegment, i \in \SetTimeInstance, e \in \SetEvent                                \\
	& \VarStartOfSegment{a}{1} \geq \ParTimeWindowStart \forall a \in \SetActivity{\VarSupervisorAssignment}                                                          \\
	& \VarFinishOfSegment{a}{last(a)} \leq \ParTimeWindowFinish \forall a \in \SetActivity{\VarSupervisorAssignment}                                                  \\
	& \VarActiveSegment{a}{k} \in \{0, 1\} \quad \forall a \in \SetActivity{\VarSupervisorAssignment}, k \in \SetWorkSegment                                          \\
	& \VarStartOfSegment{a}{k} \in [\ParAvailabilityStart, \ParAvailabilityFinish] \notag\\
	& \quad \forall a \in \SetActivity{\VarSupervisorAssignment}, k \in \SetWorkSegment  \\
	& \VarFinishOfSegment{a}{k} \in [\ParAvailabilityStart, \ParAvailabilityFinish] \notag\\
	& \quad \forall a \in \SetActivity{\VarSupervisorAssignment}, k \in \SetWorkSegment \\
	& \VarProcessingTime \in [0, \ParOperationWork] \quad \forall a \in \SetActivity{\VarSupervisorAssignment}, k \in \SetWorkSegment                                               \\
	& \VarSegmentInRelation \in \{0, 1\}                                                                                                                              \\ 
	& \quad \forall a \in \SetActivity{\VarSupervisorAssignment}, k \in \SetWorkSegment, i \in \SetTimeInstance, e \in \SetEvent \notag                               \\
	& \theta_i \in \{0, 1\} \quad \forall a \in \SetActivity{\VarSupervisorAssignment}                                                                                \\
\end{alignat}



\subsection{Contribution}
Provide a software architectual pattern to allow metaheuristics to be integrated together in real-time.
