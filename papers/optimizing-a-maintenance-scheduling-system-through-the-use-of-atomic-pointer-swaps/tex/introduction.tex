\section{Introduction}\label{sec:introduction}
This paper will be structured into four sections: Section~\ref{sec:introduction} will explain the
problem setting that necessitates the solution method; describe the relevant software architecture
literature to fix the problem; and provide mathematical models for each component of the system for 
a detailed understanding of the problem that the optimization system aims to solve. Section~\ref{sec:solution_method}
will describe the chosen approach to solve the system, highlighting why low-level memory operations
are necessary to effective scale metaheuristics. Section~\ref{sec:results} will provide a high level
understanding of how the system is behaving. Finally Section~\ref{sec:discussion} will discuss the 
implication of the provided solution approach and which other optimization problems could benefit 
from the presented optimization approach.

\subsection{Problem Setting}
Many operational optimzation problems are difficult to solve in practice as the underlying business 
process is made up of many different actors. This makes single model approaches difficult as it is 
difficult to assign responsibility for the output of the solution provided by the optimization
approach \citep{INTERACTIVEOPTIMIZATION}. Furthermore this decomposition of business process into 
specific actors is not only done to make the underlying optimization problem easier to solve, but 
also to handle dynamic changes in the optimization process and make each process compliant with 
respect to safety, regulation, etc. 

Maintenance scheduling has all these components and it is one of the keys reasons that the presented
solution method was developed. 

\begin{figure}
	\usetikzlibrary{positioning}
\definecolor{red}{HTML}{8A3F3A}
\definecolor{yellow}{HTML}{E0BB3C}
\definecolor{blue}{HTML}{4569E0}
\definecolor{green}{HTML}{17E561}
\definecolor{other}{HTML}{6A939E}

% DTU Colors
\definecolor{dtu-corporate-red}{HTML}{990000}
\definecolor{dtu-white}{HTML}{ffffff}
\definecolor{dtu-black}{HTML}{000000}
\definecolor{dtu-blue}{HTML}{2F3EEA}
\definecolor{dtu-bright-green}{HTML}{1FD082}
\definecolor{dtu-navy-blue}{HTML}{030F4F}
\definecolor{dtu-yellow}{HTML}{F6D04D}
\definecolor{dtu-orange}{HTML}{FC7634}
\definecolor{dtu-pink}{HTML}{F7BBB1}
\definecolor{dtu-grey}{HTML}{DADADA}
\definecolor{dtu-red}{HTML}{E83F48}
\definecolor{dtu-green}{HTML}{008835}
\definecolor{dtu-purple}{HTML}{79238E}


\newlength{\basisa}
\setlength{\basisa}{1cm}
\centering
\begin{tikzpicture}[line width=0.0\basisa]
    \draw (4.0\basisa,4.0\basisa) 
		node[minimum height=2\basisa,fill=dtu-grey,minimum width=3\basisa,rounded corners=0.1\basisa] 
			(Planner) {Planner};
    \draw (12.0\basisa,4.0\basisa) 
		node[minimum height=2\basisa,fill=dtu-grey,minimum width=3\basisa,rounded corners=0.1\basisa] 
			(Supervisor) {Supervisor};

			
    \draw (8.0\basisa,4.0\basisa) 
		node[minimum height=2\basisa,fill=dtu-green,minimum width=3\basisa,rounded corners=0.1\basisa] 
			(Scheduler) {Scheduler};
	
    \draw (8.0\basisa,1.5\basisa) 
		node[minimum height=1\basisa,fill=dtu-blue,minimum width=3\basisa,rounded corners=0.1\basisa] 
			(Database) {Database};

    \draw (8.0\basisa,6.5\basisa) 
		node[minimum height=1\basisa,fill=dtu-yellow,minimum width=3\basisa,rounded corners=0.1\basisa] 
			(UserInterface) {User Interface};

	\draw[<->, thick, line width=0.1\basisa] (Planner) -- (Scheduler);
	\draw[<->, thick, line width=0.1\basisa] (Scheduler) -- (Supervisor);
	\draw[<->, thick, line width=0.1\basisa] (Scheduler) -- (Database);
	\draw[<->, thick, line width=0.1\basisa] (Scheduler) -- (UserInterface);
\end{tikzpicture}

	\caption{Simple overview of the scheduling process with its primary types of
		actors. The planner, the scheduler, the supervisor(s), and the technicians. 		
		The green color highlights the scheduler as it the actor in the maintenance
		scheduling process that is modelled in this paper.
	}\label{fig:integrated:maintenance-process}
\end{figure}

\subsection{Software Architecture}
Metaheuristics present unique challenges when it comes to designing a high level software architecture
where the metaheuristics can be reliably implemented and scale. One of the reasons for this is that 
metaheuristics have very high requirements when it comes to state mutation. State mutation is the 
primary way in which metaheuristics achieve optimization by changing the solution space rapidly
according to special rules \citep{gendreauHandbookMetaheuristics2019}. 

If you want to scale metaheuristics if becomes necessary to determine a software architecture that 
can accomodate high levels of state changes. Most attempts at this has centered around message
passing \citep{talbiMetaheuristicsDesignImplementation2009} and in more advanced setup might 
also incorporate methods from the multi-agent systems literature to 
speed up the optimization process \citep{MultiagentMetaheuristicOptimization}.

This paper will present a different way of scaling metaheuristics using low-level memory operations
such as atomic operations. \citep{!!! CITE THE LOCK FREE DATA STRUCTURE ARTICLE}.

% \begin{figure}[H]
% 	\centering
% 	\usetikzlibrary {positioning}

\definecolor{red}{HTML}{8A3F3A}
\definecolor{yellow}{HTML}{E0BB3C}
\definecolor{blue}{HTML}{4569E0}
\definecolor{green}{HTML}{17E561}
\definecolor{other}{HTML}{6A939E}

% DTU Colors
\definecolor{dtu-corporate-red}{HTML}{990000}
\definecolor{dtu-white}{HTML}{ffffff}
\definecolor{dtu-black}{HTML}{000000}
\definecolor{dtu-blue}{HTML}{2F3EEA}
\definecolor{dtu-bright-green}{HTML}{1FD082}
\definecolor{dtu-navy-blue}{HTML}{030F4F}
\definecolor{dtu-yellow}{HTML}{F6D04D}
\definecolor{dtu-orange}{HTML}{FC7634}
\definecolor{dtu-pink}{HTML}{F7BBB1}
\definecolor{dtu-grey}{HTML}{DADADA}
\definecolor{dtu-red}{HTML}{E83F48}
\definecolor{dtu-green}{HTML}{008835}
\definecolor{dtu-purple}{HTML}{79238E}


\newcommand{\ModelColor}{dtu-red}
\newcommand{\UserInterfaceColor}{dtu-yellow}
\newcommand{\PersistenceColor}{dtu-blue}
\newcommand{\PointerSwapColor}{dtu-green}
\newcommand{\OrchestratorColor}{dtu-bright-green}

\newcommand{\basisinput}{4cm}  % Default value if not set by /graph/basis

\pgfkeys{
	/graph/.is family, /graph,
	default/.style = {
		show_shared_pointer = false,
		show_orchestrator = false,
		show_persistence = false,
		show_user_interface = false,
		basisinput/.estore in = \basisinput,
	},
	show_shared_pointer/.estore in = \ShowSharedSolutionCommunication,
	show_orchestrator/.estore in = \ShowOrchestratorCommunication,
	show_persistence/.estore in = \ShowPersistenceCommunication,
	show_user_interface/.estore in = \ShowUserInterfaceCommunication,
	basisinput/.estore in = \basisinput,
}

\newlength{\basis}
\tikzset{
  basis/.code={\setlength{\basis}{\basisinput}}, % TikZ assignment code
  basis/.default=3cm,                   % Provide a default (\b@sis is undefined/unassigned)
  basis,                                % Set initial Value (\b@sis is defined/assigned)
}

\newcommand{\drawOrdinatorArchitecture}[1]{
	\pgfkeys{/graph, default, #1}
	\setlength{\basis}{\basisinput}
	\begin{tikzpicture}[scale=0.75, line width=0.05\basis]

		\ifthenelse{\equal{\ShowOrchestratorCommunication}{true}}{
			\draw[color=other,-, ultra thick] (Strategic) -- (Orchestrator);
			\draw[color=other,-, ultra thick] (Tactical) -- (Orchestrator);
			\draw[color=other,-, ultra thick] (Supervisor) -- (Orchestrator);
			\draw[color=other,-, ultra thick] (Operational_1) -- (Orchestrator);
			\draw[color=other,-, ultra thick] (Operational_2) -- (Orchestrator);
			\draw[color=other,-, ultra thick] (Operational_3) -- (Orchestrator);
		}{}
		% \draw[help lines] (0\basis, 0\basis) grid (10\basis, 8\basis);
		\draw (5\basis,4\basis) node[minimum height=5\basis,minimum width=7.0\basis,rounded corners=0.1\basis] {};

	    \draw[draw=black] (4.125\basis,4.0\basis) node[opacity=0.5, minimum height=3.5\basis,minimum width=6.25\basis,rounded corners=0.1\basis,fill=\PointerSwapColor] {} ;
	    \draw (2.5\basis,5.5\basis) node[minimum height=1\basis,minimum width=1\basis,fill=\ModelColor,rounded corners=0.1\basis] (Strategic) {Stra};
	    \draw (5.0\basis,4.0\basis) node[minimum height=1\basis,minimum width=1\basis,fill=\ModelColor,rounded corners=0.1\basis] (Supervisor) {Sup};
		\draw (7.5\basis,5.5\basis) node[minimum height=1\basis,minimum width=1\basis,fill=\ModelColor,rounded corners=0.1\basis] (Tactical) {Tac};

		\draw (2.5\basis,2.5\basis) node[minimum height=1\basis,minimum width=1\basis,fill=\ModelColor,rounded corners=0.1\basis] (Operational_1) {$O_{1}$};
		\draw (5.0\basis,2.5\basis) node[minimum height=1\basis,minimum width=1\basis,fill=\ModelColor,rounded corners=0.1\basis] (Operational_2) {$O_{2}$};
		\draw (7.5\basis,2.5\basis) node[minimum height=1\basis,minimum width=1\basis,fill=\ModelColor,rounded corners=0.1\basis,rounded corners=0.1\basis] (Operational_3) {$O_{3}$};
	
		\draw (Strategic) edge (Tactical);
		\draw (Strategic) edge (Tactical);
		\draw (5\basis,5.5\basis) edge (Supervisor);
		\draw (Supervisor) -- (2.5\basis,4.0\basis) -- (Operational_1);
		\draw (Supervisor) edge (Operational_2);
		\draw (Supervisor) -- (7.5\basis,4.0\basis) -- (Operational_3);
		\draw (5.0\basis,0.5\basis)   node[minimum height=1\basis,minimum width=5.0\basis,                fill=\PersistenceColor,rounded corners=0.1\basis] {persistence};
		\draw (5.0\basis,7.5\basis)   node[minimum height=1\basis,minimum width=5.0\basis,                fill=\OrchestratorColor,rounded corners=0.1\basis] (Orchestrator) {Orchestrator};
		\draw (0.5\basis,4.0\basis)   node[rotate=90, minimum height=1.0\basis, minimum width=3.5\basis,  fill=\PointerSwapColor,rounded corners=0.1\basis] {decision variables};
		\draw (9.5\basis,5.75\basis)  node[rotate=90, minimum height=1.0\basis, minimum width=1.0\basis,  fill=\UserInterfaceColor,rounded corners=0.1\basis] {UI};
		\draw (9.5\basis,4.0\basis)   node[rotate=90, minimum height=1.0\basis, minimum width=1.0\basis,  fill=\UserInterfaceColor,rounded corners=0.1\basis] {UI};
		\draw (9.5\basis,2.25\basis)  node[rotate=90, minimum height=1.0\basis, minimum width=1.0\basis,  fill=\UserInterfaceColor,rounded corners=0.1\basis] {UI};

		% Legend
		\begin{scope}[shift={(11.0\basis,5.7\basis)}]
			\node at (-0.25\basis,1\basis) [right] {};
			\draw[color=\OrchestratorColor,fill,rounded corners=0.1\basis] (0\basis,0.0\basis)   rectangle (0.5\basis, 0.5\basis);
			\node[anchor=west] at (0.5\basis, 0.25\basis) { Managing metaheuristic lifetimes };
			\draw[color=\PointerSwapColor,fill,rounded corners=0.1\basis] (0\basis,-1.0\basis)   rectangle(0.5\basis, -0.5\basis); 
			\node[anchor=west] at (0.5\basis, -0.75\basis) { Solution sharing (Atomic pointer swaps) };
			\draw[color=\ModelColor,fill,rounded corners=0.1\basis] (0\basis,-2.0\basis)         rectangle(0.5\basis, -1.5\basis); 
			\node[anchor=west] at (0.5\basis, -1.75\basis) { Metaheurics (Mathematical Models) };
			\draw[color=\PersistenceColor,fill,rounded corners=0.1\basis] (0\basis,-3.0\basis)   rectangle(0.5\basis, -2.5\basis); 
			\node[anchor=west] at (0.5\basis, -2.75\basis) { Data storage (Memory locks)};
			\draw[color=\UserInterfaceColor,fill,rounded corners=0.1\basis] (0\basis,-4.0\basis) rectangle(0.5\basis, -3.5\basis); 
			\node[anchor=west] at (0.5\basis, -3.75\basis) { Message passing (Memory channels) };

		\end{scope}
		\ifthenelse{\equal{\ShowSharedSolutionCommunication}{true}}{
			\draw[->, thick] (Strategic) -- (Orchestrator);
		}{}
		\ifthenelse{\equal{\ShowUserInterfaceCommunication}{true}}{
			\draw[->, thick] (Strategic) -- (Orchestrator);
		}{}
		\ifthenelse{\equal{\ShowPersistenceCommunication}{true}}{
			\draw[->, thick] (Strategic) -- (Orchestrator);
		}{}
		

	\end{tikzpicture}
}


% 	\drawOrdinatorArchitecture{basisinput=1cm}
% 	\label{fig:ordinator-architecture}
% \end{figure}
\begin{figure}[H]
	\centering
	\usetikzlibrary {positioning}
\newcommand{\drawHexagon}[6][draw=black]{
	\draw[#1, fill=#4] (#2,#3) ++(30:#6) -- ++(90:#6) -- ++(150:#6) -- ++(210:#6) -- ++(270:#6) -- ++(330:#6) -- cycle;
	\node[align=center] at (#2,#3+2) {#5};
}

\newif\ifpersistencelayer
\newif\ifatomicpointerswaplayer
\newif\ifmetaheuristicslayer
\newif\ifuserinterfacelayer
\newif\iforchestratorlayer
\newif\ifsimplifiedlayer

\pgfkeys{
	/hexagon/.is family, /hexagon,
	default/.style = {
		persistence=false,
		atomicpointerswap=false,
		metaheuristics=false,
		orchestrator=false,
		userinterface=false,
		simplified=false,
	},
	persistence/.is if=persistencelayer,
	atomicpointerswap/.is if=atomicpointerswaplayer,
	metaheuristics/.is if=metaheuristicslayer,
	orchestrator/.is if=orchestratorlayer,
	userinterface/.is if=userinterfacelayer,
	simplified/.is if=simplifiedlayer,
}
\newcommand{\drawModelSetupHexagon}[1][]{
	\pgfkeys{/hexagon, default, #1}

	\begin{tikzpicture}[font=\footnotesize, scale=0.5, line width=1.05]
	

	\ifpersistencelayer
		\drawHexagon[draw=none]{ 2                      }{ 2}{dtu-blue}{}{2}
		\drawHexagon[draw=none]{{6 - 2 * (2 - sqrt(3)) }}{ 2}{dtu-blue}{}{2}
		\drawHexagon[draw=none]{{4 - 1 * (2 - sqrt(3)) }}{-1}{dtu-blue}{Persistence}{2}
		\drawHexagon[draw=none]{{0 + 1 * (2 - sqrt(3)) }}{-1}{dtu-blue}{}{2}
		\drawHexagon[draw=none]{{8 - 3 * (2 - sqrt(3)) }}{-1}{dtu-blue}{}{2}

		\drawHexagon[draw=none]{{2 - 0 * (2 - sqrt(3)) }}{-4}{dtu-blue}{}{2}
		\drawHexagon[draw=none]{{6 - 2 * (2 - sqrt(3)) }}{-4}{dtu-blue}{}{2}

		\drawHexagon[draw=none]{{10 - 4 * (2 - sqrt(3)) }}{-4}{dtu-blue}{}{2}
		\drawHexagon[draw=none]{{-2 + 2 * (2 - sqrt(3)) }}{-4}{dtu-blue}{}{2}

		\drawHexagon[draw=none]{{12 - 5 * (2 - sqrt(3)) }}{-1}{dtu-blue}{}{2}
		\drawHexagon[draw=none]{{-4 + 3 * (2 - sqrt(3)) }}{-1}{dtu-blue}{}{2}
		% Legend for each layer
		\drawHexagon{{14.0  }}{+3.0}{dtu-blue}{}{0.75}
		\node[align=right, anchor=west] at ({15.0}, +3.75) {Persistence};
		\drawHexagon{{14.0  }}{+1.5}{dtu-white}{}{0.75}
		\node[align=right, anchor=west] at ({15.0}, +2.25) {Atomic Pointer};
		\drawHexagon{{14.0  }}{+0.0}{dtu-white}{}{0.75}
		\node[align=right, anchor=west] at ({15.0}, +0.75) {Metaheuristics};
		\drawHexagon{{14.0  }}{-1.5}{dtu-white}{}{0.75}
		\node[align=right, anchor=west] at ({15.0}, -0.75) {Orchestration};
		\drawHexagon{{14.0  }}{-3.0}{dtu-white}{}{0.75}
		\node[align=right, anchor=west] at ({15.0}, -2.25) {User interfaces};
	\fi


	\ifatomicpointerswaplayer
		\drawHexagon[]{ 2                      }{ 2}{dtu-green}{Shared\\solution\\pointer}{2}
		\drawHexagon[]{{6 - 2 * (2 - sqrt(3)) }}{ 2}{dtu-green}{Shared\\solution\\pointer}{2}
		\drawHexagon[]{{4 - 1 * (2 - sqrt(3)) }}{-1}{dtu-green}{Shared\\solution\\pointer}{2}
		\drawHexagon[]{{0 + 1 * (2 - sqrt(3)) }}{-1}{dtu-green}{Shared\\solution\\pointer}{2}
		\drawHexagon[]{{8 - 3 * (2 - sqrt(3)) }}{-1}{dtu-green}{Shared\\solution\\pointer}{2}

		\drawHexagon[]{{2 - 0 * (2 - sqrt(3)) }}{-4}{dtu-green}{Shared\\solution\\pointer}{2}
		\drawHexagon[]{{6 - 2 * (2 - sqrt(3)) }}{-4}{dtu-green}{Shared\\solution\\pointer}{2}

		\drawHexagon[]{{10 - 4 * (2 - sqrt(3)) }}{-4}{dtu-green}{Shared\\solution\\pointer}{2}
		\drawHexagon[]{{-2 + 2 * (2 - sqrt(3)) }}{-4}{dtu-green}{Shared\\solution\\pointer}{2}

		\drawHexagon[]{{12 - 5 * (2 - sqrt(3)) }}{-1}{dtu-green}{Shared\\solution\\pointer}{2}
		\drawHexagon[]{{-4 + 3 * (2 - sqrt(3)) }}{-1}{dtu-green}{Shared\\solution\\pointer}{2}
		% Legend for each layer
		\drawHexagon{{14.0  }}{+3.0}{dtu-white}{}{0.75}
		\node[align=right, anchor=west] at ({15.0}, +3.75) {Persistence};
		\drawHexagon{{14.0  }}{+1.5}{dtu-green}{}{0.75}
		\node[align=right, anchor=west] at ({15.0}, +2.25) {Atomic Pointer};
		\drawHexagon{{14.0  }}{+0.0}{dtu-white}{}{0.75}
		\node[align=right, anchor=west] at ({15.0}, +0.75) {Metaheuristics};
		\drawHexagon{{14.0  }}{-1.5}{dtu-white}{}{0.75}
		\node[align=right, anchor=west] at ({15.0}, -0.75) {Orchestration};
		\drawHexagon{{14.0  }}{-3.0}{dtu-white}{}{0.75}
		\node[align=right, anchor=west] at ({15.0}, -2.25) {User interfaces};
	\fi

	\ifsimplifiedlayer

		\node[align=right, anchor=west] at ({-5.5}, +3.75) {};
		\drawHexagon{{+2 + 0 * (2 - sqrt(3)) }}{ 2}{dtu-green}{Scheduler}{2}
		\drawHexagon{{+4 - 1 * (2 - sqrt(3)) }}{-1}{dtu-red}{Supervisor}{2}
		\drawHexagon{{+0 + 1 * (2 - sqrt(3)) }}{-1}{dtu-red}{Supervisor}{2}
		\drawHexagon{{+2 - 0 * (2 - sqrt(3)) }}{-4}{dtu-corporate-red}{Technician}{2}
		\drawHexagon{{+6 - 2 * (2 - sqrt(3)) }}{-4}{dtu-corporate-red}{Technician}{2}
		\drawHexagon{{-2 + 2 * (2 - sqrt(3)) }}{-4}{dtu-corporate-red}{Technician}{2}
		\drawHexagon{{+8 - 3 * (2 - sqrt(3)) }}{-1}{dtu-corporate-red}{Technician}{2}
		\drawHexagon{{-4 + 3 * (2 - sqrt(3)) }}{-1}{dtu-corporate-red}{Technician}{2}

		% Scheduler
		\draw[thin, fill=dtu-yellow] (2, 5) circle (0.35);
		\draw[thin, fill=dtu-purple] (2, 3) circle (0.35);
		% Supervisor 1
		\draw[thin, fill=dtu-yellow] ({+4 - 1 * (2 - sqrt(3)) }, 02) circle (0.35);
		\draw[thin, fill=dtu-purple] ({+4 - 1 * (2 - sqrt(3)) }, -0) circle (0.35);
		% Supervisor 2
		\draw[thin, fill=dtu-yellow] ({+0 + 1 * (2 - sqrt(3)) }, 02) circle (0.35);
		\draw[thin, fill=dtu-purple] ({+0 + 1 * (2 - sqrt(3)) }, -0) circle (0.35);
		% Technician 1
		\draw[thin, fill=dtu-yellow] ({+2 - 0 * (2 - sqrt(3)) }, -1) circle (0.35);
		\draw[thin, fill=dtu-purple] ({+2 - 0 * (2 - sqrt(3)) }, -3) circle (0.35);
		% Technician 2
		\draw[thin, fill=dtu-yellow] ({+6 - 2 * (2 - sqrt(3)) }, -1) circle (0.35);
		\draw[thin, fill=dtu-purple] ({+6 - 2 * (2 - sqrt(3)) }, -3) circle (0.35);
		% Technician 3
		\draw[thin, fill=dtu-yellow] ({-2 + 2 * (2 - sqrt(3)) }, -1) circle (0.35);
		\draw[thin, fill=dtu-purple] ({-2 + 2 * (2 - sqrt(3)) }, -3) circle (0.35);
		% Technician 4
		\draw[thin, fill=dtu-yellow] ({+8 - 3 * (2 - sqrt(3)) }, 02) circle (0.35);
		\draw[thin, fill=dtu-purple] ({+8 - 3 * (2 - sqrt(3)) }, -0) circle (0.35);
		% Technician 5
		\draw[thin, fill=dtu-yellow] ({-4 + 3 * (2 - sqrt(3)) }, 02) circle (0.35);
		\draw[thin, fill=dtu-purple] ({-4 + 3 * (2 - sqrt(3)) }, -0) circle (0.35);

		% Legend for each layer
		\node[align=right, anchor=west] at ({12.0}, +3.75) {Atomic Pointer};
		\draw[fill=dtu-purple] (11.0,  +3.75) circle (0.5);

		\node[align=right, anchor=west] at ({12.0}, +2.25) {Scheduler Metaheuristic};
		\drawHexagon{{11.0  }}{+1.75}{dtu-green}{}{0.5}
		\node[align=right, anchor=west] at ({12.0}, +0.75) {Supervisor Metaheuristic};
		\drawHexagon{{11.0  }}{+0.25}{dtu-red}{}{0.5}
		\node[align=right, anchor=west] at ({12.0}, -0.75) {Technician Metaheuristic};
		\drawHexagon{{11.0  }}{-1.25}{dtu-corporate-red}{}{0.5}
		\node[align=right, anchor=west] at ({12.0}, -2.25) {User interfaces (Message Passing)};
		\draw[fill=dtu-yellow] (11.0, -2.25) circle (0.5);
	\fi

	\ifmetaheuristicslayer
		\drawHexagon{ 2                      }{ 2}{dtu-blue}{Strategic}{2}
		\drawHexagon{{6 - 2 * (2 - sqrt(3)) }}{ 2}{dtu-green}{Tactical}{2}
		\drawHexagon{{4 - 1 * (2 - sqrt(3)) }}{-1}{dtu-red}{Supervisor}{2}
		\drawHexagon{{0 + 1 * (2 - sqrt(3)) }}{-1}{dtu-red}{Supervisor}{2}
		\drawHexagon{{8 - 3 * (2 - sqrt(3)) }}{-1}{dtu-red}{Supervisor}{2}

		\drawHexagon{{2 - 0 * (2 - sqrt(3)) }}{-4}{dtu-corporate-red}{Technician}{2}
		\drawHexagon{{6 - 2 * (2 - sqrt(3)) }}{-4}{dtu-corporate-red}{Technician}{2}

		\drawHexagon{{10 - 4 * (2 - sqrt(3)) }}{-4}{dtu-corporate-red}{Technician}{2}
		\drawHexagon{{-2 + 2 * (2 - sqrt(3)) }}{-4}{dtu-corporate-red}{Technician}{2}

		\drawHexagon{{12 - 5 * (2 - sqrt(3)) }}{-1}{dtu-corporate-red}{Technician}{2}
		\drawHexagon{{-4 + 3 * (2 - sqrt(3)) }}{-1}{dtu-corporate-red}{Technician}{2}

		% Legend for each layer
		\drawHexagon{{14.0  }}{+3.0}{dtu-white}{}{0.75}
		\node[align=right, anchor=west] at ({15.0}, +3.75) {Persistence};
		\drawHexagon{{14.0  }}{+1.5}{dtu-white}{}{0.75}
		\node[align=right, anchor=west] at ({15.0}, +2.25) {Atomic Pointer};
		\drawHexagon{{14.0  }}{+0.0}{dtu-corporate-red}{}{0.75}
		\node[align=right, anchor=west] at ({15.0}, +0.75) {Metaheuristics};
		\drawHexagon{{14.0  }}{-1.5}{dtu-white}{}{0.75}
		\node[align=right, anchor=west] at ({15.0}, -0.75) {Orchestration};
		\drawHexagon{{14.0  }}{-3.0}{dtu-white}{}{0.75}
		\node[align=right, anchor=west] at ({15.0}, -2.25) {User interfaces};
	\fi

	\iforchestratorlayer
		\drawHexagon{ 2                      }{ 2}{dtu-orange}{}{2}
		\drawHexagon{{6 - 2 * (2 - sqrt(3)) }}{ 2}{dtu-orange}{}{2}
		\drawHexagon{{4 - 1 * (2 - sqrt(3)) }}{-1}{dtu-orange}{Orche-\\strator}{2}
		\drawHexagon{{0 + 1 * (2 - sqrt(3)) }}{-1}{dtu-orange}{}{2}
		\drawHexagon{{8 - 3 * (2 - sqrt(3)) }}{-1}{dtu-orange}{}{2}

		\drawHexagon{{2 - 0 * (2 - sqrt(3)) }}{-4}{dtu-orange}{}{2}
		\drawHexagon{{6 - 2 * (2 - sqrt(3)) }}{-4}{dtu-orange}{}{2}

		\drawHexagon{{10 - 4 * (2 - sqrt(3)) }}{-4}{dtu-orange}{}{2}
		\drawHexagon{{-2 + 2 * (2 - sqrt(3)) }}{-4}{dtu-orange}{}{2}

		\drawHexagon{{12 - 5 * (2 - sqrt(3)) }}{-1}{dtu-orange}{}{2}
		\drawHexagon{{-4 + 3 * (2 - sqrt(3)) }}{-1}{dtu-orange}{}{2}
		% Legend for each layer
		\drawHexagon{{14.0  }}{+3.0}{dtu-white}{}{0.75}
		\node[align=right, anchor=west] at ({15.0}, +3.75) {Persistence};
		\drawHexagon{{14.0  }}{+1.5}{dtu-white}{}{0.75}
		\node[align=right, anchor=west] at ({15.0}, +2.25) {Atomic Pointer};
		\drawHexagon{{14.0  }}{+0.0}{dtu-white}{}{0.75}
		\node[align=right, anchor=west] at ({15.0}, +0.75) {Metaheuristics};
		\drawHexagon{{14.0  }}{-1.5}{dtu-orange}{}{0.75}
		\node[align=right, anchor=west] at ({15.0}, -0.75) {Orchestration};
		\drawHexagon{{14.0  }}{-3.0}{dtu-white}{}{0.75}
		\node[align=right, anchor=west] at ({15.0}, -2.25) {User interfaces};
	\fi

	
	\ifuserinterfacelayer
		\drawHexagon{ 2                      }{ 2}{dtu-yellow}{UI}{2}
		\drawHexagon{{6 - 2 * (2 - sqrt(3)) }}{ 2}{dtu-yellow}{UI}{2}
		\drawHexagon{{4 - 1 * (2 - sqrt(3)) }}{-1}{dtu-yellow}{UI}{2}
		\drawHexagon{{0 + 1 * (2 - sqrt(3)) }}{-1}{dtu-yellow}{UI}{2}
		\drawHexagon{{8 - 3 * (2 - sqrt(3)) }}{-1}{dtu-yellow}{UI}{2}

		\drawHexagon{{2 - 0 * (2 - sqrt(3)) }}{-4}{dtu-yellow}{UI}{2}
		\drawHexagon{{6 - 2 * (2 - sqrt(3)) }}{-4}{dtu-yellow}{UI}{2}

		\drawHexagon{{10 - 4 * (2 - sqrt(3)) }}{-4}{dtu-yellow}{UI}{2}
		\drawHexagon{{-2 + 2 * (2 - sqrt(3)) }}{-4}{dtu-yellow}{UI}{2}

		\drawHexagon{{12 - 5 * (2 - sqrt(3)) }}{-1}{dtu-yellow}{UI}{2}
		\drawHexagon{{-4 + 3 * (2 - sqrt(3)) }}{-1}{dtu-yellow}{UI}{2}
		% Legend for each layer
		\drawHexagon{{14.0  }}{+3.0}{dtu-white}{}{0.75}
		\node[align=right, anchor=west] at ({15.0}, +3.75) {Persistence};
		\drawHexagon{{14.0  }}{+1.5}{dtu-white}{}{0.75}
		\node[align=right, anchor=west] at ({15.0}, +2.25) {Atomic Pointer};
		\drawHexagon{{14.0  }}{+0.0}{dtu-white}{}{0.75}
		\node[align=right, anchor=west] at ({15.0}, +0.75) {Metaheuristics};
		\drawHexagon{{14.0  }}{-1.5}{dtu-white}{}{0.75}
		\node[align=right, anchor=west] at ({15.0}, -0.75) {Orchestration};
		\drawHexagon{{14.0  }}{-3.0}{dtu-yellow}{}{0.75}
		\node[align=right, anchor=west] at ({15.0}, -2.25) {User interfaces};
	\fi
	
	\end{tikzpicture}
}

	\resizebox{0.7\textwidth}{!}{
		\drawModelSetupHexagon[persistence=true]
	}
	\caption{
		Overview of the scheduling process when modelled as actors. When LNS is encapsulated 
		is an actor it becomes possible to optimize parts of a large process individually instead of 
		optimizing the scheduling problem globally from a single model implementation.
	}\label{fig:ordinator-hexagon:persistence}
\end{figure}
\begin{figure}[H]
	\centering
	\usetikzlibrary {positioning}
\newcommand{\drawHexagon}[6][draw=black]{
	\draw[#1, fill=#4] (#2,#3) ++(30:#6) -- ++(90:#6) -- ++(150:#6) -- ++(210:#6) -- ++(270:#6) -- ++(330:#6) -- cycle;
	\node[align=center] at (#2,#3+2) {#5};
}

\newif\ifpersistencelayer
\newif\ifatomicpointerswaplayer
\newif\ifmetaheuristicslayer
\newif\ifuserinterfacelayer
\newif\iforchestratorlayer
\newif\ifsimplifiedlayer

\pgfkeys{
	/hexagon/.is family, /hexagon,
	default/.style = {
		persistence=false,
		atomicpointerswap=false,
		metaheuristics=false,
		orchestrator=false,
		userinterface=false,
		simplified=false,
	},
	persistence/.is if=persistencelayer,
	atomicpointerswap/.is if=atomicpointerswaplayer,
	metaheuristics/.is if=metaheuristicslayer,
	orchestrator/.is if=orchestratorlayer,
	userinterface/.is if=userinterfacelayer,
	simplified/.is if=simplifiedlayer,
}
\newcommand{\drawModelSetupHexagon}[1][]{
	\pgfkeys{/hexagon, default, #1}

	\begin{tikzpicture}[font=\footnotesize, scale=0.5, line width=1.05]
	

	\ifpersistencelayer
		\drawHexagon[draw=none]{ 2                      }{ 2}{dtu-blue}{}{2}
		\drawHexagon[draw=none]{{6 - 2 * (2 - sqrt(3)) }}{ 2}{dtu-blue}{}{2}
		\drawHexagon[draw=none]{{4 - 1 * (2 - sqrt(3)) }}{-1}{dtu-blue}{Persistence}{2}
		\drawHexagon[draw=none]{{0 + 1 * (2 - sqrt(3)) }}{-1}{dtu-blue}{}{2}
		\drawHexagon[draw=none]{{8 - 3 * (2 - sqrt(3)) }}{-1}{dtu-blue}{}{2}

		\drawHexagon[draw=none]{{2 - 0 * (2 - sqrt(3)) }}{-4}{dtu-blue}{}{2}
		\drawHexagon[draw=none]{{6 - 2 * (2 - sqrt(3)) }}{-4}{dtu-blue}{}{2}

		\drawHexagon[draw=none]{{10 - 4 * (2 - sqrt(3)) }}{-4}{dtu-blue}{}{2}
		\drawHexagon[draw=none]{{-2 + 2 * (2 - sqrt(3)) }}{-4}{dtu-blue}{}{2}

		\drawHexagon[draw=none]{{12 - 5 * (2 - sqrt(3)) }}{-1}{dtu-blue}{}{2}
		\drawHexagon[draw=none]{{-4 + 3 * (2 - sqrt(3)) }}{-1}{dtu-blue}{}{2}
		% Legend for each layer
		\drawHexagon{{14.0  }}{+3.0}{dtu-blue}{}{0.75}
		\node[align=right, anchor=west] at ({15.0}, +3.75) {Persistence};
		\drawHexagon{{14.0  }}{+1.5}{dtu-white}{}{0.75}
		\node[align=right, anchor=west] at ({15.0}, +2.25) {Atomic Pointer};
		\drawHexagon{{14.0  }}{+0.0}{dtu-white}{}{0.75}
		\node[align=right, anchor=west] at ({15.0}, +0.75) {Metaheuristics};
		\drawHexagon{{14.0  }}{-1.5}{dtu-white}{}{0.75}
		\node[align=right, anchor=west] at ({15.0}, -0.75) {Orchestration};
		\drawHexagon{{14.0  }}{-3.0}{dtu-white}{}{0.75}
		\node[align=right, anchor=west] at ({15.0}, -2.25) {User interfaces};
	\fi


	\ifatomicpointerswaplayer
		\drawHexagon[]{ 2                      }{ 2}{dtu-green}{Shared\\solution\\pointer}{2}
		\drawHexagon[]{{6 - 2 * (2 - sqrt(3)) }}{ 2}{dtu-green}{Shared\\solution\\pointer}{2}
		\drawHexagon[]{{4 - 1 * (2 - sqrt(3)) }}{-1}{dtu-green}{Shared\\solution\\pointer}{2}
		\drawHexagon[]{{0 + 1 * (2 - sqrt(3)) }}{-1}{dtu-green}{Shared\\solution\\pointer}{2}
		\drawHexagon[]{{8 - 3 * (2 - sqrt(3)) }}{-1}{dtu-green}{Shared\\solution\\pointer}{2}

		\drawHexagon[]{{2 - 0 * (2 - sqrt(3)) }}{-4}{dtu-green}{Shared\\solution\\pointer}{2}
		\drawHexagon[]{{6 - 2 * (2 - sqrt(3)) }}{-4}{dtu-green}{Shared\\solution\\pointer}{2}

		\drawHexagon[]{{10 - 4 * (2 - sqrt(3)) }}{-4}{dtu-green}{Shared\\solution\\pointer}{2}
		\drawHexagon[]{{-2 + 2 * (2 - sqrt(3)) }}{-4}{dtu-green}{Shared\\solution\\pointer}{2}

		\drawHexagon[]{{12 - 5 * (2 - sqrt(3)) }}{-1}{dtu-green}{Shared\\solution\\pointer}{2}
		\drawHexagon[]{{-4 + 3 * (2 - sqrt(3)) }}{-1}{dtu-green}{Shared\\solution\\pointer}{2}
		% Legend for each layer
		\drawHexagon{{14.0  }}{+3.0}{dtu-white}{}{0.75}
		\node[align=right, anchor=west] at ({15.0}, +3.75) {Persistence};
		\drawHexagon{{14.0  }}{+1.5}{dtu-green}{}{0.75}
		\node[align=right, anchor=west] at ({15.0}, +2.25) {Atomic Pointer};
		\drawHexagon{{14.0  }}{+0.0}{dtu-white}{}{0.75}
		\node[align=right, anchor=west] at ({15.0}, +0.75) {Metaheuristics};
		\drawHexagon{{14.0  }}{-1.5}{dtu-white}{}{0.75}
		\node[align=right, anchor=west] at ({15.0}, -0.75) {Orchestration};
		\drawHexagon{{14.0  }}{-3.0}{dtu-white}{}{0.75}
		\node[align=right, anchor=west] at ({15.0}, -2.25) {User interfaces};
	\fi

	\ifsimplifiedlayer

		\node[align=right, anchor=west] at ({-5.5}, +3.75) {};
		\drawHexagon{{+2 + 0 * (2 - sqrt(3)) }}{ 2}{dtu-green}{Scheduler}{2}
		\drawHexagon{{+4 - 1 * (2 - sqrt(3)) }}{-1}{dtu-red}{Supervisor}{2}
		\drawHexagon{{+0 + 1 * (2 - sqrt(3)) }}{-1}{dtu-red}{Supervisor}{2}
		\drawHexagon{{+2 - 0 * (2 - sqrt(3)) }}{-4}{dtu-corporate-red}{Technician}{2}
		\drawHexagon{{+6 - 2 * (2 - sqrt(3)) }}{-4}{dtu-corporate-red}{Technician}{2}
		\drawHexagon{{-2 + 2 * (2 - sqrt(3)) }}{-4}{dtu-corporate-red}{Technician}{2}
		\drawHexagon{{+8 - 3 * (2 - sqrt(3)) }}{-1}{dtu-corporate-red}{Technician}{2}
		\drawHexagon{{-4 + 3 * (2 - sqrt(3)) }}{-1}{dtu-corporate-red}{Technician}{2}

		% Scheduler
		\draw[thin, fill=dtu-yellow] (2, 5) circle (0.35);
		\draw[thin, fill=dtu-purple] (2, 3) circle (0.35);
		% Supervisor 1
		\draw[thin, fill=dtu-yellow] ({+4 - 1 * (2 - sqrt(3)) }, 02) circle (0.35);
		\draw[thin, fill=dtu-purple] ({+4 - 1 * (2 - sqrt(3)) }, -0) circle (0.35);
		% Supervisor 2
		\draw[thin, fill=dtu-yellow] ({+0 + 1 * (2 - sqrt(3)) }, 02) circle (0.35);
		\draw[thin, fill=dtu-purple] ({+0 + 1 * (2 - sqrt(3)) }, -0) circle (0.35);
		% Technician 1
		\draw[thin, fill=dtu-yellow] ({+2 - 0 * (2 - sqrt(3)) }, -1) circle (0.35);
		\draw[thin, fill=dtu-purple] ({+2 - 0 * (2 - sqrt(3)) }, -3) circle (0.35);
		% Technician 2
		\draw[thin, fill=dtu-yellow] ({+6 - 2 * (2 - sqrt(3)) }, -1) circle (0.35);
		\draw[thin, fill=dtu-purple] ({+6 - 2 * (2 - sqrt(3)) }, -3) circle (0.35);
		% Technician 3
		\draw[thin, fill=dtu-yellow] ({-2 + 2 * (2 - sqrt(3)) }, -1) circle (0.35);
		\draw[thin, fill=dtu-purple] ({-2 + 2 * (2 - sqrt(3)) }, -3) circle (0.35);
		% Technician 4
		\draw[thin, fill=dtu-yellow] ({+8 - 3 * (2 - sqrt(3)) }, 02) circle (0.35);
		\draw[thin, fill=dtu-purple] ({+8 - 3 * (2 - sqrt(3)) }, -0) circle (0.35);
		% Technician 5
		\draw[thin, fill=dtu-yellow] ({-4 + 3 * (2 - sqrt(3)) }, 02) circle (0.35);
		\draw[thin, fill=dtu-purple] ({-4 + 3 * (2 - sqrt(3)) }, -0) circle (0.35);

		% Legend for each layer
		\node[align=right, anchor=west] at ({12.0}, +3.75) {Atomic Pointer};
		\draw[fill=dtu-purple] (11.0,  +3.75) circle (0.5);

		\node[align=right, anchor=west] at ({12.0}, +2.25) {Scheduler Metaheuristic};
		\drawHexagon{{11.0  }}{+1.75}{dtu-green}{}{0.5}
		\node[align=right, anchor=west] at ({12.0}, +0.75) {Supervisor Metaheuristic};
		\drawHexagon{{11.0  }}{+0.25}{dtu-red}{}{0.5}
		\node[align=right, anchor=west] at ({12.0}, -0.75) {Technician Metaheuristic};
		\drawHexagon{{11.0  }}{-1.25}{dtu-corporate-red}{}{0.5}
		\node[align=right, anchor=west] at ({12.0}, -2.25) {User interfaces (Message Passing)};
		\draw[fill=dtu-yellow] (11.0, -2.25) circle (0.5);
	\fi

	\ifmetaheuristicslayer
		\drawHexagon{ 2                      }{ 2}{dtu-blue}{Strategic}{2}
		\drawHexagon{{6 - 2 * (2 - sqrt(3)) }}{ 2}{dtu-green}{Tactical}{2}
		\drawHexagon{{4 - 1 * (2 - sqrt(3)) }}{-1}{dtu-red}{Supervisor}{2}
		\drawHexagon{{0 + 1 * (2 - sqrt(3)) }}{-1}{dtu-red}{Supervisor}{2}
		\drawHexagon{{8 - 3 * (2 - sqrt(3)) }}{-1}{dtu-red}{Supervisor}{2}

		\drawHexagon{{2 - 0 * (2 - sqrt(3)) }}{-4}{dtu-corporate-red}{Technician}{2}
		\drawHexagon{{6 - 2 * (2 - sqrt(3)) }}{-4}{dtu-corporate-red}{Technician}{2}

		\drawHexagon{{10 - 4 * (2 - sqrt(3)) }}{-4}{dtu-corporate-red}{Technician}{2}
		\drawHexagon{{-2 + 2 * (2 - sqrt(3)) }}{-4}{dtu-corporate-red}{Technician}{2}

		\drawHexagon{{12 - 5 * (2 - sqrt(3)) }}{-1}{dtu-corporate-red}{Technician}{2}
		\drawHexagon{{-4 + 3 * (2 - sqrt(3)) }}{-1}{dtu-corporate-red}{Technician}{2}

		% Legend for each layer
		\drawHexagon{{14.0  }}{+3.0}{dtu-white}{}{0.75}
		\node[align=right, anchor=west] at ({15.0}, +3.75) {Persistence};
		\drawHexagon{{14.0  }}{+1.5}{dtu-white}{}{0.75}
		\node[align=right, anchor=west] at ({15.0}, +2.25) {Atomic Pointer};
		\drawHexagon{{14.0  }}{+0.0}{dtu-corporate-red}{}{0.75}
		\node[align=right, anchor=west] at ({15.0}, +0.75) {Metaheuristics};
		\drawHexagon{{14.0  }}{-1.5}{dtu-white}{}{0.75}
		\node[align=right, anchor=west] at ({15.0}, -0.75) {Orchestration};
		\drawHexagon{{14.0  }}{-3.0}{dtu-white}{}{0.75}
		\node[align=right, anchor=west] at ({15.0}, -2.25) {User interfaces};
	\fi

	\iforchestratorlayer
		\drawHexagon{ 2                      }{ 2}{dtu-orange}{}{2}
		\drawHexagon{{6 - 2 * (2 - sqrt(3)) }}{ 2}{dtu-orange}{}{2}
		\drawHexagon{{4 - 1 * (2 - sqrt(3)) }}{-1}{dtu-orange}{Orche-\\strator}{2}
		\drawHexagon{{0 + 1 * (2 - sqrt(3)) }}{-1}{dtu-orange}{}{2}
		\drawHexagon{{8 - 3 * (2 - sqrt(3)) }}{-1}{dtu-orange}{}{2}

		\drawHexagon{{2 - 0 * (2 - sqrt(3)) }}{-4}{dtu-orange}{}{2}
		\drawHexagon{{6 - 2 * (2 - sqrt(3)) }}{-4}{dtu-orange}{}{2}

		\drawHexagon{{10 - 4 * (2 - sqrt(3)) }}{-4}{dtu-orange}{}{2}
		\drawHexagon{{-2 + 2 * (2 - sqrt(3)) }}{-4}{dtu-orange}{}{2}

		\drawHexagon{{12 - 5 * (2 - sqrt(3)) }}{-1}{dtu-orange}{}{2}
		\drawHexagon{{-4 + 3 * (2 - sqrt(3)) }}{-1}{dtu-orange}{}{2}
		% Legend for each layer
		\drawHexagon{{14.0  }}{+3.0}{dtu-white}{}{0.75}
		\node[align=right, anchor=west] at ({15.0}, +3.75) {Persistence};
		\drawHexagon{{14.0  }}{+1.5}{dtu-white}{}{0.75}
		\node[align=right, anchor=west] at ({15.0}, +2.25) {Atomic Pointer};
		\drawHexagon{{14.0  }}{+0.0}{dtu-white}{}{0.75}
		\node[align=right, anchor=west] at ({15.0}, +0.75) {Metaheuristics};
		\drawHexagon{{14.0  }}{-1.5}{dtu-orange}{}{0.75}
		\node[align=right, anchor=west] at ({15.0}, -0.75) {Orchestration};
		\drawHexagon{{14.0  }}{-3.0}{dtu-white}{}{0.75}
		\node[align=right, anchor=west] at ({15.0}, -2.25) {User interfaces};
	\fi

	
	\ifuserinterfacelayer
		\drawHexagon{ 2                      }{ 2}{dtu-yellow}{UI}{2}
		\drawHexagon{{6 - 2 * (2 - sqrt(3)) }}{ 2}{dtu-yellow}{UI}{2}
		\drawHexagon{{4 - 1 * (2 - sqrt(3)) }}{-1}{dtu-yellow}{UI}{2}
		\drawHexagon{{0 + 1 * (2 - sqrt(3)) }}{-1}{dtu-yellow}{UI}{2}
		\drawHexagon{{8 - 3 * (2 - sqrt(3)) }}{-1}{dtu-yellow}{UI}{2}

		\drawHexagon{{2 - 0 * (2 - sqrt(3)) }}{-4}{dtu-yellow}{UI}{2}
		\drawHexagon{{6 - 2 * (2 - sqrt(3)) }}{-4}{dtu-yellow}{UI}{2}

		\drawHexagon{{10 - 4 * (2 - sqrt(3)) }}{-4}{dtu-yellow}{UI}{2}
		\drawHexagon{{-2 + 2 * (2 - sqrt(3)) }}{-4}{dtu-yellow}{UI}{2}

		\drawHexagon{{12 - 5 * (2 - sqrt(3)) }}{-1}{dtu-yellow}{UI}{2}
		\drawHexagon{{-4 + 3 * (2 - sqrt(3)) }}{-1}{dtu-yellow}{UI}{2}
		% Legend for each layer
		\drawHexagon{{14.0  }}{+3.0}{dtu-white}{}{0.75}
		\node[align=right, anchor=west] at ({15.0}, +3.75) {Persistence};
		\drawHexagon{{14.0  }}{+1.5}{dtu-white}{}{0.75}
		\node[align=right, anchor=west] at ({15.0}, +2.25) {Atomic Pointer};
		\drawHexagon{{14.0  }}{+0.0}{dtu-white}{}{0.75}
		\node[align=right, anchor=west] at ({15.0}, +0.75) {Metaheuristics};
		\drawHexagon{{14.0  }}{-1.5}{dtu-white}{}{0.75}
		\node[align=right, anchor=west] at ({15.0}, -0.75) {Orchestration};
		\drawHexagon{{14.0  }}{-3.0}{dtu-yellow}{}{0.75}
		\node[align=right, anchor=west] at ({15.0}, -2.25) {User interfaces};
	\fi
	
	\end{tikzpicture}
}

	\resizebox{0.7\textwidth}{!}{
		\drawModelSetupHexagon[atomicpointerswap=true]
	}
	\caption{
		Overview of the scheduling process when modelled as actors. When LNS is encapsulated 
		is an actor it becomes possible to optimize parts of a large process individually instead of 
		optimizing the scheduling problem globally from a single model implementation.
	}
	\label{fig:ordinator-hexagon:atomicpointerswap}
\end{figure}

\begin{figure}[H]
	\centering
	\usetikzlibrary {positioning}
\newcommand{\drawHexagon}[6][draw=black]{
	\draw[#1, fill=#4] (#2,#3) ++(30:#6) -- ++(90:#6) -- ++(150:#6) -- ++(210:#6) -- ++(270:#6) -- ++(330:#6) -- cycle;
	\node[align=center] at (#2,#3+2) {#5};
}

\newif\ifpersistencelayer
\newif\ifatomicpointerswaplayer
\newif\ifmetaheuristicslayer
\newif\ifuserinterfacelayer
\newif\iforchestratorlayer
\newif\ifsimplifiedlayer

\pgfkeys{
	/hexagon/.is family, /hexagon,
	default/.style = {
		persistence=false,
		atomicpointerswap=false,
		metaheuristics=false,
		orchestrator=false,
		userinterface=false,
		simplified=false,
	},
	persistence/.is if=persistencelayer,
	atomicpointerswap/.is if=atomicpointerswaplayer,
	metaheuristics/.is if=metaheuristicslayer,
	orchestrator/.is if=orchestratorlayer,
	userinterface/.is if=userinterfacelayer,
	simplified/.is if=simplifiedlayer,
}
\newcommand{\drawModelSetupHexagon}[1][]{
	\pgfkeys{/hexagon, default, #1}

	\begin{tikzpicture}[font=\footnotesize, scale=0.5, line width=1.05]
	

	\ifpersistencelayer
		\drawHexagon[draw=none]{ 2                      }{ 2}{dtu-blue}{}{2}
		\drawHexagon[draw=none]{{6 - 2 * (2 - sqrt(3)) }}{ 2}{dtu-blue}{}{2}
		\drawHexagon[draw=none]{{4 - 1 * (2 - sqrt(3)) }}{-1}{dtu-blue}{Persistence}{2}
		\drawHexagon[draw=none]{{0 + 1 * (2 - sqrt(3)) }}{-1}{dtu-blue}{}{2}
		\drawHexagon[draw=none]{{8 - 3 * (2 - sqrt(3)) }}{-1}{dtu-blue}{}{2}

		\drawHexagon[draw=none]{{2 - 0 * (2 - sqrt(3)) }}{-4}{dtu-blue}{}{2}
		\drawHexagon[draw=none]{{6 - 2 * (2 - sqrt(3)) }}{-4}{dtu-blue}{}{2}

		\drawHexagon[draw=none]{{10 - 4 * (2 - sqrt(3)) }}{-4}{dtu-blue}{}{2}
		\drawHexagon[draw=none]{{-2 + 2 * (2 - sqrt(3)) }}{-4}{dtu-blue}{}{2}

		\drawHexagon[draw=none]{{12 - 5 * (2 - sqrt(3)) }}{-1}{dtu-blue}{}{2}
		\drawHexagon[draw=none]{{-4 + 3 * (2 - sqrt(3)) }}{-1}{dtu-blue}{}{2}
		% Legend for each layer
		\drawHexagon{{14.0  }}{+3.0}{dtu-blue}{}{0.75}
		\node[align=right, anchor=west] at ({15.0}, +3.75) {Persistence};
		\drawHexagon{{14.0  }}{+1.5}{dtu-white}{}{0.75}
		\node[align=right, anchor=west] at ({15.0}, +2.25) {Atomic Pointer};
		\drawHexagon{{14.0  }}{+0.0}{dtu-white}{}{0.75}
		\node[align=right, anchor=west] at ({15.0}, +0.75) {Metaheuristics};
		\drawHexagon{{14.0  }}{-1.5}{dtu-white}{}{0.75}
		\node[align=right, anchor=west] at ({15.0}, -0.75) {Orchestration};
		\drawHexagon{{14.0  }}{-3.0}{dtu-white}{}{0.75}
		\node[align=right, anchor=west] at ({15.0}, -2.25) {User interfaces};
	\fi


	\ifatomicpointerswaplayer
		\drawHexagon[]{ 2                      }{ 2}{dtu-green}{Shared\\solution\\pointer}{2}
		\drawHexagon[]{{6 - 2 * (2 - sqrt(3)) }}{ 2}{dtu-green}{Shared\\solution\\pointer}{2}
		\drawHexagon[]{{4 - 1 * (2 - sqrt(3)) }}{-1}{dtu-green}{Shared\\solution\\pointer}{2}
		\drawHexagon[]{{0 + 1 * (2 - sqrt(3)) }}{-1}{dtu-green}{Shared\\solution\\pointer}{2}
		\drawHexagon[]{{8 - 3 * (2 - sqrt(3)) }}{-1}{dtu-green}{Shared\\solution\\pointer}{2}

		\drawHexagon[]{{2 - 0 * (2 - sqrt(3)) }}{-4}{dtu-green}{Shared\\solution\\pointer}{2}
		\drawHexagon[]{{6 - 2 * (2 - sqrt(3)) }}{-4}{dtu-green}{Shared\\solution\\pointer}{2}

		\drawHexagon[]{{10 - 4 * (2 - sqrt(3)) }}{-4}{dtu-green}{Shared\\solution\\pointer}{2}
		\drawHexagon[]{{-2 + 2 * (2 - sqrt(3)) }}{-4}{dtu-green}{Shared\\solution\\pointer}{2}

		\drawHexagon[]{{12 - 5 * (2 - sqrt(3)) }}{-1}{dtu-green}{Shared\\solution\\pointer}{2}
		\drawHexagon[]{{-4 + 3 * (2 - sqrt(3)) }}{-1}{dtu-green}{Shared\\solution\\pointer}{2}
		% Legend for each layer
		\drawHexagon{{14.0  }}{+3.0}{dtu-white}{}{0.75}
		\node[align=right, anchor=west] at ({15.0}, +3.75) {Persistence};
		\drawHexagon{{14.0  }}{+1.5}{dtu-green}{}{0.75}
		\node[align=right, anchor=west] at ({15.0}, +2.25) {Atomic Pointer};
		\drawHexagon{{14.0  }}{+0.0}{dtu-white}{}{0.75}
		\node[align=right, anchor=west] at ({15.0}, +0.75) {Metaheuristics};
		\drawHexagon{{14.0  }}{-1.5}{dtu-white}{}{0.75}
		\node[align=right, anchor=west] at ({15.0}, -0.75) {Orchestration};
		\drawHexagon{{14.0  }}{-3.0}{dtu-white}{}{0.75}
		\node[align=right, anchor=west] at ({15.0}, -2.25) {User interfaces};
	\fi

	\ifsimplifiedlayer

		\node[align=right, anchor=west] at ({-5.5}, +3.75) {};
		\drawHexagon{{+2 + 0 * (2 - sqrt(3)) }}{ 2}{dtu-green}{Scheduler}{2}
		\drawHexagon{{+4 - 1 * (2 - sqrt(3)) }}{-1}{dtu-red}{Supervisor}{2}
		\drawHexagon{{+0 + 1 * (2 - sqrt(3)) }}{-1}{dtu-red}{Supervisor}{2}
		\drawHexagon{{+2 - 0 * (2 - sqrt(3)) }}{-4}{dtu-corporate-red}{Technician}{2}
		\drawHexagon{{+6 - 2 * (2 - sqrt(3)) }}{-4}{dtu-corporate-red}{Technician}{2}
		\drawHexagon{{-2 + 2 * (2 - sqrt(3)) }}{-4}{dtu-corporate-red}{Technician}{2}
		\drawHexagon{{+8 - 3 * (2 - sqrt(3)) }}{-1}{dtu-corporate-red}{Technician}{2}
		\drawHexagon{{-4 + 3 * (2 - sqrt(3)) }}{-1}{dtu-corporate-red}{Technician}{2}

		% Scheduler
		\draw[thin, fill=dtu-yellow] (2, 5) circle (0.35);
		\draw[thin, fill=dtu-purple] (2, 3) circle (0.35);
		% Supervisor 1
		\draw[thin, fill=dtu-yellow] ({+4 - 1 * (2 - sqrt(3)) }, 02) circle (0.35);
		\draw[thin, fill=dtu-purple] ({+4 - 1 * (2 - sqrt(3)) }, -0) circle (0.35);
		% Supervisor 2
		\draw[thin, fill=dtu-yellow] ({+0 + 1 * (2 - sqrt(3)) }, 02) circle (0.35);
		\draw[thin, fill=dtu-purple] ({+0 + 1 * (2 - sqrt(3)) }, -0) circle (0.35);
		% Technician 1
		\draw[thin, fill=dtu-yellow] ({+2 - 0 * (2 - sqrt(3)) }, -1) circle (0.35);
		\draw[thin, fill=dtu-purple] ({+2 - 0 * (2 - sqrt(3)) }, -3) circle (0.35);
		% Technician 2
		\draw[thin, fill=dtu-yellow] ({+6 - 2 * (2 - sqrt(3)) }, -1) circle (0.35);
		\draw[thin, fill=dtu-purple] ({+6 - 2 * (2 - sqrt(3)) }, -3) circle (0.35);
		% Technician 3
		\draw[thin, fill=dtu-yellow] ({-2 + 2 * (2 - sqrt(3)) }, -1) circle (0.35);
		\draw[thin, fill=dtu-purple] ({-2 + 2 * (2 - sqrt(3)) }, -3) circle (0.35);
		% Technician 4
		\draw[thin, fill=dtu-yellow] ({+8 - 3 * (2 - sqrt(3)) }, 02) circle (0.35);
		\draw[thin, fill=dtu-purple] ({+8 - 3 * (2 - sqrt(3)) }, -0) circle (0.35);
		% Technician 5
		\draw[thin, fill=dtu-yellow] ({-4 + 3 * (2 - sqrt(3)) }, 02) circle (0.35);
		\draw[thin, fill=dtu-purple] ({-4 + 3 * (2 - sqrt(3)) }, -0) circle (0.35);

		% Legend for each layer
		\node[align=right, anchor=west] at ({12.0}, +3.75) {Atomic Pointer};
		\draw[fill=dtu-purple] (11.0,  +3.75) circle (0.5);

		\node[align=right, anchor=west] at ({12.0}, +2.25) {Scheduler Metaheuristic};
		\drawHexagon{{11.0  }}{+1.75}{dtu-green}{}{0.5}
		\node[align=right, anchor=west] at ({12.0}, +0.75) {Supervisor Metaheuristic};
		\drawHexagon{{11.0  }}{+0.25}{dtu-red}{}{0.5}
		\node[align=right, anchor=west] at ({12.0}, -0.75) {Technician Metaheuristic};
		\drawHexagon{{11.0  }}{-1.25}{dtu-corporate-red}{}{0.5}
		\node[align=right, anchor=west] at ({12.0}, -2.25) {User interfaces (Message Passing)};
		\draw[fill=dtu-yellow] (11.0, -2.25) circle (0.5);
	\fi

	\ifmetaheuristicslayer
		\drawHexagon{ 2                      }{ 2}{dtu-blue}{Strategic}{2}
		\drawHexagon{{6 - 2 * (2 - sqrt(3)) }}{ 2}{dtu-green}{Tactical}{2}
		\drawHexagon{{4 - 1 * (2 - sqrt(3)) }}{-1}{dtu-red}{Supervisor}{2}
		\drawHexagon{{0 + 1 * (2 - sqrt(3)) }}{-1}{dtu-red}{Supervisor}{2}
		\drawHexagon{{8 - 3 * (2 - sqrt(3)) }}{-1}{dtu-red}{Supervisor}{2}

		\drawHexagon{{2 - 0 * (2 - sqrt(3)) }}{-4}{dtu-corporate-red}{Technician}{2}
		\drawHexagon{{6 - 2 * (2 - sqrt(3)) }}{-4}{dtu-corporate-red}{Technician}{2}

		\drawHexagon{{10 - 4 * (2 - sqrt(3)) }}{-4}{dtu-corporate-red}{Technician}{2}
		\drawHexagon{{-2 + 2 * (2 - sqrt(3)) }}{-4}{dtu-corporate-red}{Technician}{2}

		\drawHexagon{{12 - 5 * (2 - sqrt(3)) }}{-1}{dtu-corporate-red}{Technician}{2}
		\drawHexagon{{-4 + 3 * (2 - sqrt(3)) }}{-1}{dtu-corporate-red}{Technician}{2}

		% Legend for each layer
		\drawHexagon{{14.0  }}{+3.0}{dtu-white}{}{0.75}
		\node[align=right, anchor=west] at ({15.0}, +3.75) {Persistence};
		\drawHexagon{{14.0  }}{+1.5}{dtu-white}{}{0.75}
		\node[align=right, anchor=west] at ({15.0}, +2.25) {Atomic Pointer};
		\drawHexagon{{14.0  }}{+0.0}{dtu-corporate-red}{}{0.75}
		\node[align=right, anchor=west] at ({15.0}, +0.75) {Metaheuristics};
		\drawHexagon{{14.0  }}{-1.5}{dtu-white}{}{0.75}
		\node[align=right, anchor=west] at ({15.0}, -0.75) {Orchestration};
		\drawHexagon{{14.0  }}{-3.0}{dtu-white}{}{0.75}
		\node[align=right, anchor=west] at ({15.0}, -2.25) {User interfaces};
	\fi

	\iforchestratorlayer
		\drawHexagon{ 2                      }{ 2}{dtu-orange}{}{2}
		\drawHexagon{{6 - 2 * (2 - sqrt(3)) }}{ 2}{dtu-orange}{}{2}
		\drawHexagon{{4 - 1 * (2 - sqrt(3)) }}{-1}{dtu-orange}{Orche-\\strator}{2}
		\drawHexagon{{0 + 1 * (2 - sqrt(3)) }}{-1}{dtu-orange}{}{2}
		\drawHexagon{{8 - 3 * (2 - sqrt(3)) }}{-1}{dtu-orange}{}{2}

		\drawHexagon{{2 - 0 * (2 - sqrt(3)) }}{-4}{dtu-orange}{}{2}
		\drawHexagon{{6 - 2 * (2 - sqrt(3)) }}{-4}{dtu-orange}{}{2}

		\drawHexagon{{10 - 4 * (2 - sqrt(3)) }}{-4}{dtu-orange}{}{2}
		\drawHexagon{{-2 + 2 * (2 - sqrt(3)) }}{-4}{dtu-orange}{}{2}

		\drawHexagon{{12 - 5 * (2 - sqrt(3)) }}{-1}{dtu-orange}{}{2}
		\drawHexagon{{-4 + 3 * (2 - sqrt(3)) }}{-1}{dtu-orange}{}{2}
		% Legend for each layer
		\drawHexagon{{14.0  }}{+3.0}{dtu-white}{}{0.75}
		\node[align=right, anchor=west] at ({15.0}, +3.75) {Persistence};
		\drawHexagon{{14.0  }}{+1.5}{dtu-white}{}{0.75}
		\node[align=right, anchor=west] at ({15.0}, +2.25) {Atomic Pointer};
		\drawHexagon{{14.0  }}{+0.0}{dtu-white}{}{0.75}
		\node[align=right, anchor=west] at ({15.0}, +0.75) {Metaheuristics};
		\drawHexagon{{14.0  }}{-1.5}{dtu-orange}{}{0.75}
		\node[align=right, anchor=west] at ({15.0}, -0.75) {Orchestration};
		\drawHexagon{{14.0  }}{-3.0}{dtu-white}{}{0.75}
		\node[align=right, anchor=west] at ({15.0}, -2.25) {User interfaces};
	\fi

	
	\ifuserinterfacelayer
		\drawHexagon{ 2                      }{ 2}{dtu-yellow}{UI}{2}
		\drawHexagon{{6 - 2 * (2 - sqrt(3)) }}{ 2}{dtu-yellow}{UI}{2}
		\drawHexagon{{4 - 1 * (2 - sqrt(3)) }}{-1}{dtu-yellow}{UI}{2}
		\drawHexagon{{0 + 1 * (2 - sqrt(3)) }}{-1}{dtu-yellow}{UI}{2}
		\drawHexagon{{8 - 3 * (2 - sqrt(3)) }}{-1}{dtu-yellow}{UI}{2}

		\drawHexagon{{2 - 0 * (2 - sqrt(3)) }}{-4}{dtu-yellow}{UI}{2}
		\drawHexagon{{6 - 2 * (2 - sqrt(3)) }}{-4}{dtu-yellow}{UI}{2}

		\drawHexagon{{10 - 4 * (2 - sqrt(3)) }}{-4}{dtu-yellow}{UI}{2}
		\drawHexagon{{-2 + 2 * (2 - sqrt(3)) }}{-4}{dtu-yellow}{UI}{2}

		\drawHexagon{{12 - 5 * (2 - sqrt(3)) }}{-1}{dtu-yellow}{UI}{2}
		\drawHexagon{{-4 + 3 * (2 - sqrt(3)) }}{-1}{dtu-yellow}{UI}{2}
		% Legend for each layer
		\drawHexagon{{14.0  }}{+3.0}{dtu-white}{}{0.75}
		\node[align=right, anchor=west] at ({15.0}, +3.75) {Persistence};
		\drawHexagon{{14.0  }}{+1.5}{dtu-white}{}{0.75}
		\node[align=right, anchor=west] at ({15.0}, +2.25) {Atomic Pointer};
		\drawHexagon{{14.0  }}{+0.0}{dtu-white}{}{0.75}
		\node[align=right, anchor=west] at ({15.0}, +0.75) {Metaheuristics};
		\drawHexagon{{14.0  }}{-1.5}{dtu-white}{}{0.75}
		\node[align=right, anchor=west] at ({15.0}, -0.75) {Orchestration};
		\drawHexagon{{14.0  }}{-3.0}{dtu-yellow}{}{0.75}
		\node[align=right, anchor=west] at ({15.0}, -2.25) {User interfaces};
	\fi
	
	\end{tikzpicture}
}

	\resizebox{0.7\textwidth}{!}{
		\drawModelSetupHexagon[metaheuristics=true]
	}
	\caption{
		Overview of the scheduling process when modelled as actors. When LNS is encapsulated 
		is an actor it becomes possible to optimize parts of a large process individually instead of 
		optimizing the scheduling problem globally from a single model implementation.
	}
	\label{fig:ordinator-hexagon:metaheuristics}
\end{figure}

\begin{figure}[H]
	\centering
	\usetikzlibrary {positioning}
\newcommand{\drawHexagon}[6][draw=black]{
	\draw[#1, fill=#4] (#2,#3) ++(30:#6) -- ++(90:#6) -- ++(150:#6) -- ++(210:#6) -- ++(270:#6) -- ++(330:#6) -- cycle;
	\node[align=center] at (#2,#3+2) {#5};
}

\newif\ifpersistencelayer
\newif\ifatomicpointerswaplayer
\newif\ifmetaheuristicslayer
\newif\ifuserinterfacelayer
\newif\iforchestratorlayer
\newif\ifsimplifiedlayer

\pgfkeys{
	/hexagon/.is family, /hexagon,
	default/.style = {
		persistence=false,
		atomicpointerswap=false,
		metaheuristics=false,
		orchestrator=false,
		userinterface=false,
		simplified=false,
	},
	persistence/.is if=persistencelayer,
	atomicpointerswap/.is if=atomicpointerswaplayer,
	metaheuristics/.is if=metaheuristicslayer,
	orchestrator/.is if=orchestratorlayer,
	userinterface/.is if=userinterfacelayer,
	simplified/.is if=simplifiedlayer,
}
\newcommand{\drawModelSetupHexagon}[1][]{
	\pgfkeys{/hexagon, default, #1}

	\begin{tikzpicture}[font=\footnotesize, scale=0.5, line width=1.05]
	

	\ifpersistencelayer
		\drawHexagon[draw=none]{ 2                      }{ 2}{dtu-blue}{}{2}
		\drawHexagon[draw=none]{{6 - 2 * (2 - sqrt(3)) }}{ 2}{dtu-blue}{}{2}
		\drawHexagon[draw=none]{{4 - 1 * (2 - sqrt(3)) }}{-1}{dtu-blue}{Persistence}{2}
		\drawHexagon[draw=none]{{0 + 1 * (2 - sqrt(3)) }}{-1}{dtu-blue}{}{2}
		\drawHexagon[draw=none]{{8 - 3 * (2 - sqrt(3)) }}{-1}{dtu-blue}{}{2}

		\drawHexagon[draw=none]{{2 - 0 * (2 - sqrt(3)) }}{-4}{dtu-blue}{}{2}
		\drawHexagon[draw=none]{{6 - 2 * (2 - sqrt(3)) }}{-4}{dtu-blue}{}{2}

		\drawHexagon[draw=none]{{10 - 4 * (2 - sqrt(3)) }}{-4}{dtu-blue}{}{2}
		\drawHexagon[draw=none]{{-2 + 2 * (2 - sqrt(3)) }}{-4}{dtu-blue}{}{2}

		\drawHexagon[draw=none]{{12 - 5 * (2 - sqrt(3)) }}{-1}{dtu-blue}{}{2}
		\drawHexagon[draw=none]{{-4 + 3 * (2 - sqrt(3)) }}{-1}{dtu-blue}{}{2}
		% Legend for each layer
		\drawHexagon{{14.0  }}{+3.0}{dtu-blue}{}{0.75}
		\node[align=right, anchor=west] at ({15.0}, +3.75) {Persistence};
		\drawHexagon{{14.0  }}{+1.5}{dtu-white}{}{0.75}
		\node[align=right, anchor=west] at ({15.0}, +2.25) {Atomic Pointer};
		\drawHexagon{{14.0  }}{+0.0}{dtu-white}{}{0.75}
		\node[align=right, anchor=west] at ({15.0}, +0.75) {Metaheuristics};
		\drawHexagon{{14.0  }}{-1.5}{dtu-white}{}{0.75}
		\node[align=right, anchor=west] at ({15.0}, -0.75) {Orchestration};
		\drawHexagon{{14.0  }}{-3.0}{dtu-white}{}{0.75}
		\node[align=right, anchor=west] at ({15.0}, -2.25) {User interfaces};
	\fi


	\ifatomicpointerswaplayer
		\drawHexagon[]{ 2                      }{ 2}{dtu-green}{Shared\\solution\\pointer}{2}
		\drawHexagon[]{{6 - 2 * (2 - sqrt(3)) }}{ 2}{dtu-green}{Shared\\solution\\pointer}{2}
		\drawHexagon[]{{4 - 1 * (2 - sqrt(3)) }}{-1}{dtu-green}{Shared\\solution\\pointer}{2}
		\drawHexagon[]{{0 + 1 * (2 - sqrt(3)) }}{-1}{dtu-green}{Shared\\solution\\pointer}{2}
		\drawHexagon[]{{8 - 3 * (2 - sqrt(3)) }}{-1}{dtu-green}{Shared\\solution\\pointer}{2}

		\drawHexagon[]{{2 - 0 * (2 - sqrt(3)) }}{-4}{dtu-green}{Shared\\solution\\pointer}{2}
		\drawHexagon[]{{6 - 2 * (2 - sqrt(3)) }}{-4}{dtu-green}{Shared\\solution\\pointer}{2}

		\drawHexagon[]{{10 - 4 * (2 - sqrt(3)) }}{-4}{dtu-green}{Shared\\solution\\pointer}{2}
		\drawHexagon[]{{-2 + 2 * (2 - sqrt(3)) }}{-4}{dtu-green}{Shared\\solution\\pointer}{2}

		\drawHexagon[]{{12 - 5 * (2 - sqrt(3)) }}{-1}{dtu-green}{Shared\\solution\\pointer}{2}
		\drawHexagon[]{{-4 + 3 * (2 - sqrt(3)) }}{-1}{dtu-green}{Shared\\solution\\pointer}{2}
		% Legend for each layer
		\drawHexagon{{14.0  }}{+3.0}{dtu-white}{}{0.75}
		\node[align=right, anchor=west] at ({15.0}, +3.75) {Persistence};
		\drawHexagon{{14.0  }}{+1.5}{dtu-green}{}{0.75}
		\node[align=right, anchor=west] at ({15.0}, +2.25) {Atomic Pointer};
		\drawHexagon{{14.0  }}{+0.0}{dtu-white}{}{0.75}
		\node[align=right, anchor=west] at ({15.0}, +0.75) {Metaheuristics};
		\drawHexagon{{14.0  }}{-1.5}{dtu-white}{}{0.75}
		\node[align=right, anchor=west] at ({15.0}, -0.75) {Orchestration};
		\drawHexagon{{14.0  }}{-3.0}{dtu-white}{}{0.75}
		\node[align=right, anchor=west] at ({15.0}, -2.25) {User interfaces};
	\fi

	\ifsimplifiedlayer

		\node[align=right, anchor=west] at ({-5.5}, +3.75) {};
		\drawHexagon{{+2 + 0 * (2 - sqrt(3)) }}{ 2}{dtu-green}{Scheduler}{2}
		\drawHexagon{{+4 - 1 * (2 - sqrt(3)) }}{-1}{dtu-red}{Supervisor}{2}
		\drawHexagon{{+0 + 1 * (2 - sqrt(3)) }}{-1}{dtu-red}{Supervisor}{2}
		\drawHexagon{{+2 - 0 * (2 - sqrt(3)) }}{-4}{dtu-corporate-red}{Technician}{2}
		\drawHexagon{{+6 - 2 * (2 - sqrt(3)) }}{-4}{dtu-corporate-red}{Technician}{2}
		\drawHexagon{{-2 + 2 * (2 - sqrt(3)) }}{-4}{dtu-corporate-red}{Technician}{2}
		\drawHexagon{{+8 - 3 * (2 - sqrt(3)) }}{-1}{dtu-corporate-red}{Technician}{2}
		\drawHexagon{{-4 + 3 * (2 - sqrt(3)) }}{-1}{dtu-corporate-red}{Technician}{2}

		% Scheduler
		\draw[thin, fill=dtu-yellow] (2, 5) circle (0.35);
		\draw[thin, fill=dtu-purple] (2, 3) circle (0.35);
		% Supervisor 1
		\draw[thin, fill=dtu-yellow] ({+4 - 1 * (2 - sqrt(3)) }, 02) circle (0.35);
		\draw[thin, fill=dtu-purple] ({+4 - 1 * (2 - sqrt(3)) }, -0) circle (0.35);
		% Supervisor 2
		\draw[thin, fill=dtu-yellow] ({+0 + 1 * (2 - sqrt(3)) }, 02) circle (0.35);
		\draw[thin, fill=dtu-purple] ({+0 + 1 * (2 - sqrt(3)) }, -0) circle (0.35);
		% Technician 1
		\draw[thin, fill=dtu-yellow] ({+2 - 0 * (2 - sqrt(3)) }, -1) circle (0.35);
		\draw[thin, fill=dtu-purple] ({+2 - 0 * (2 - sqrt(3)) }, -3) circle (0.35);
		% Technician 2
		\draw[thin, fill=dtu-yellow] ({+6 - 2 * (2 - sqrt(3)) }, -1) circle (0.35);
		\draw[thin, fill=dtu-purple] ({+6 - 2 * (2 - sqrt(3)) }, -3) circle (0.35);
		% Technician 3
		\draw[thin, fill=dtu-yellow] ({-2 + 2 * (2 - sqrt(3)) }, -1) circle (0.35);
		\draw[thin, fill=dtu-purple] ({-2 + 2 * (2 - sqrt(3)) }, -3) circle (0.35);
		% Technician 4
		\draw[thin, fill=dtu-yellow] ({+8 - 3 * (2 - sqrt(3)) }, 02) circle (0.35);
		\draw[thin, fill=dtu-purple] ({+8 - 3 * (2 - sqrt(3)) }, -0) circle (0.35);
		% Technician 5
		\draw[thin, fill=dtu-yellow] ({-4 + 3 * (2 - sqrt(3)) }, 02) circle (0.35);
		\draw[thin, fill=dtu-purple] ({-4 + 3 * (2 - sqrt(3)) }, -0) circle (0.35);

		% Legend for each layer
		\node[align=right, anchor=west] at ({12.0}, +3.75) {Atomic Pointer};
		\draw[fill=dtu-purple] (11.0,  +3.75) circle (0.5);

		\node[align=right, anchor=west] at ({12.0}, +2.25) {Scheduler Metaheuristic};
		\drawHexagon{{11.0  }}{+1.75}{dtu-green}{}{0.5}
		\node[align=right, anchor=west] at ({12.0}, +0.75) {Supervisor Metaheuristic};
		\drawHexagon{{11.0  }}{+0.25}{dtu-red}{}{0.5}
		\node[align=right, anchor=west] at ({12.0}, -0.75) {Technician Metaheuristic};
		\drawHexagon{{11.0  }}{-1.25}{dtu-corporate-red}{}{0.5}
		\node[align=right, anchor=west] at ({12.0}, -2.25) {User interfaces (Message Passing)};
		\draw[fill=dtu-yellow] (11.0, -2.25) circle (0.5);
	\fi

	\ifmetaheuristicslayer
		\drawHexagon{ 2                      }{ 2}{dtu-blue}{Strategic}{2}
		\drawHexagon{{6 - 2 * (2 - sqrt(3)) }}{ 2}{dtu-green}{Tactical}{2}
		\drawHexagon{{4 - 1 * (2 - sqrt(3)) }}{-1}{dtu-red}{Supervisor}{2}
		\drawHexagon{{0 + 1 * (2 - sqrt(3)) }}{-1}{dtu-red}{Supervisor}{2}
		\drawHexagon{{8 - 3 * (2 - sqrt(3)) }}{-1}{dtu-red}{Supervisor}{2}

		\drawHexagon{{2 - 0 * (2 - sqrt(3)) }}{-4}{dtu-corporate-red}{Technician}{2}
		\drawHexagon{{6 - 2 * (2 - sqrt(3)) }}{-4}{dtu-corporate-red}{Technician}{2}

		\drawHexagon{{10 - 4 * (2 - sqrt(3)) }}{-4}{dtu-corporate-red}{Technician}{2}
		\drawHexagon{{-2 + 2 * (2 - sqrt(3)) }}{-4}{dtu-corporate-red}{Technician}{2}

		\drawHexagon{{12 - 5 * (2 - sqrt(3)) }}{-1}{dtu-corporate-red}{Technician}{2}
		\drawHexagon{{-4 + 3 * (2 - sqrt(3)) }}{-1}{dtu-corporate-red}{Technician}{2}

		% Legend for each layer
		\drawHexagon{{14.0  }}{+3.0}{dtu-white}{}{0.75}
		\node[align=right, anchor=west] at ({15.0}, +3.75) {Persistence};
		\drawHexagon{{14.0  }}{+1.5}{dtu-white}{}{0.75}
		\node[align=right, anchor=west] at ({15.0}, +2.25) {Atomic Pointer};
		\drawHexagon{{14.0  }}{+0.0}{dtu-corporate-red}{}{0.75}
		\node[align=right, anchor=west] at ({15.0}, +0.75) {Metaheuristics};
		\drawHexagon{{14.0  }}{-1.5}{dtu-white}{}{0.75}
		\node[align=right, anchor=west] at ({15.0}, -0.75) {Orchestration};
		\drawHexagon{{14.0  }}{-3.0}{dtu-white}{}{0.75}
		\node[align=right, anchor=west] at ({15.0}, -2.25) {User interfaces};
	\fi

	\iforchestratorlayer
		\drawHexagon{ 2                      }{ 2}{dtu-orange}{}{2}
		\drawHexagon{{6 - 2 * (2 - sqrt(3)) }}{ 2}{dtu-orange}{}{2}
		\drawHexagon{{4 - 1 * (2 - sqrt(3)) }}{-1}{dtu-orange}{Orche-\\strator}{2}
		\drawHexagon{{0 + 1 * (2 - sqrt(3)) }}{-1}{dtu-orange}{}{2}
		\drawHexagon{{8 - 3 * (2 - sqrt(3)) }}{-1}{dtu-orange}{}{2}

		\drawHexagon{{2 - 0 * (2 - sqrt(3)) }}{-4}{dtu-orange}{}{2}
		\drawHexagon{{6 - 2 * (2 - sqrt(3)) }}{-4}{dtu-orange}{}{2}

		\drawHexagon{{10 - 4 * (2 - sqrt(3)) }}{-4}{dtu-orange}{}{2}
		\drawHexagon{{-2 + 2 * (2 - sqrt(3)) }}{-4}{dtu-orange}{}{2}

		\drawHexagon{{12 - 5 * (2 - sqrt(3)) }}{-1}{dtu-orange}{}{2}
		\drawHexagon{{-4 + 3 * (2 - sqrt(3)) }}{-1}{dtu-orange}{}{2}
		% Legend for each layer
		\drawHexagon{{14.0  }}{+3.0}{dtu-white}{}{0.75}
		\node[align=right, anchor=west] at ({15.0}, +3.75) {Persistence};
		\drawHexagon{{14.0  }}{+1.5}{dtu-white}{}{0.75}
		\node[align=right, anchor=west] at ({15.0}, +2.25) {Atomic Pointer};
		\drawHexagon{{14.0  }}{+0.0}{dtu-white}{}{0.75}
		\node[align=right, anchor=west] at ({15.0}, +0.75) {Metaheuristics};
		\drawHexagon{{14.0  }}{-1.5}{dtu-orange}{}{0.75}
		\node[align=right, anchor=west] at ({15.0}, -0.75) {Orchestration};
		\drawHexagon{{14.0  }}{-3.0}{dtu-white}{}{0.75}
		\node[align=right, anchor=west] at ({15.0}, -2.25) {User interfaces};
	\fi

	
	\ifuserinterfacelayer
		\drawHexagon{ 2                      }{ 2}{dtu-yellow}{UI}{2}
		\drawHexagon{{6 - 2 * (2 - sqrt(3)) }}{ 2}{dtu-yellow}{UI}{2}
		\drawHexagon{{4 - 1 * (2 - sqrt(3)) }}{-1}{dtu-yellow}{UI}{2}
		\drawHexagon{{0 + 1 * (2 - sqrt(3)) }}{-1}{dtu-yellow}{UI}{2}
		\drawHexagon{{8 - 3 * (2 - sqrt(3)) }}{-1}{dtu-yellow}{UI}{2}

		\drawHexagon{{2 - 0 * (2 - sqrt(3)) }}{-4}{dtu-yellow}{UI}{2}
		\drawHexagon{{6 - 2 * (2 - sqrt(3)) }}{-4}{dtu-yellow}{UI}{2}

		\drawHexagon{{10 - 4 * (2 - sqrt(3)) }}{-4}{dtu-yellow}{UI}{2}
		\drawHexagon{{-2 + 2 * (2 - sqrt(3)) }}{-4}{dtu-yellow}{UI}{2}

		\drawHexagon{{12 - 5 * (2 - sqrt(3)) }}{-1}{dtu-yellow}{UI}{2}
		\drawHexagon{{-4 + 3 * (2 - sqrt(3)) }}{-1}{dtu-yellow}{UI}{2}
		% Legend for each layer
		\drawHexagon{{14.0  }}{+3.0}{dtu-white}{}{0.75}
		\node[align=right, anchor=west] at ({15.0}, +3.75) {Persistence};
		\drawHexagon{{14.0  }}{+1.5}{dtu-white}{}{0.75}
		\node[align=right, anchor=west] at ({15.0}, +2.25) {Atomic Pointer};
		\drawHexagon{{14.0  }}{+0.0}{dtu-white}{}{0.75}
		\node[align=right, anchor=west] at ({15.0}, +0.75) {Metaheuristics};
		\drawHexagon{{14.0  }}{-1.5}{dtu-white}{}{0.75}
		\node[align=right, anchor=west] at ({15.0}, -0.75) {Orchestration};
		\drawHexagon{{14.0  }}{-3.0}{dtu-yellow}{}{0.75}
		\node[align=right, anchor=west] at ({15.0}, -2.25) {User interfaces};
	\fi
	
	\end{tikzpicture}
}

	\resizebox{0.7\textwidth}{!}{
		\drawModelSetupHexagon[orchestrator=true]
	}
	\caption{
		Overview of the scheduling process when modelled as actors. When LNS is encapsulated 
		is an actor it becomes possible to optimize parts of a large process individually instead of 
		optimizing the scheduling problem globally from a single model implementation.
	}
	\label{fig:ordinator-hexagon:orchestrator}
\end{figure}

\begin{figure}[H]
	\centering
	\usetikzlibrary {positioning}
\newcommand{\drawHexagon}[6][draw=black]{
	\draw[#1, fill=#4] (#2,#3) ++(30:#6) -- ++(90:#6) -- ++(150:#6) -- ++(210:#6) -- ++(270:#6) -- ++(330:#6) -- cycle;
	\node[align=center] at (#2,#3+2) {#5};
}

\newif\ifpersistencelayer
\newif\ifatomicpointerswaplayer
\newif\ifmetaheuristicslayer
\newif\ifuserinterfacelayer
\newif\iforchestratorlayer
\newif\ifsimplifiedlayer

\pgfkeys{
	/hexagon/.is family, /hexagon,
	default/.style = {
		persistence=false,
		atomicpointerswap=false,
		metaheuristics=false,
		orchestrator=false,
		userinterface=false,
		simplified=false,
	},
	persistence/.is if=persistencelayer,
	atomicpointerswap/.is if=atomicpointerswaplayer,
	metaheuristics/.is if=metaheuristicslayer,
	orchestrator/.is if=orchestratorlayer,
	userinterface/.is if=userinterfacelayer,
	simplified/.is if=simplifiedlayer,
}
\newcommand{\drawModelSetupHexagon}[1][]{
	\pgfkeys{/hexagon, default, #1}

	\begin{tikzpicture}[font=\footnotesize, scale=0.5, line width=1.05]
	

	\ifpersistencelayer
		\drawHexagon[draw=none]{ 2                      }{ 2}{dtu-blue}{}{2}
		\drawHexagon[draw=none]{{6 - 2 * (2 - sqrt(3)) }}{ 2}{dtu-blue}{}{2}
		\drawHexagon[draw=none]{{4 - 1 * (2 - sqrt(3)) }}{-1}{dtu-blue}{Persistence}{2}
		\drawHexagon[draw=none]{{0 + 1 * (2 - sqrt(3)) }}{-1}{dtu-blue}{}{2}
		\drawHexagon[draw=none]{{8 - 3 * (2 - sqrt(3)) }}{-1}{dtu-blue}{}{2}

		\drawHexagon[draw=none]{{2 - 0 * (2 - sqrt(3)) }}{-4}{dtu-blue}{}{2}
		\drawHexagon[draw=none]{{6 - 2 * (2 - sqrt(3)) }}{-4}{dtu-blue}{}{2}

		\drawHexagon[draw=none]{{10 - 4 * (2 - sqrt(3)) }}{-4}{dtu-blue}{}{2}
		\drawHexagon[draw=none]{{-2 + 2 * (2 - sqrt(3)) }}{-4}{dtu-blue}{}{2}

		\drawHexagon[draw=none]{{12 - 5 * (2 - sqrt(3)) }}{-1}{dtu-blue}{}{2}
		\drawHexagon[draw=none]{{-4 + 3 * (2 - sqrt(3)) }}{-1}{dtu-blue}{}{2}
		% Legend for each layer
		\drawHexagon{{14.0  }}{+3.0}{dtu-blue}{}{0.75}
		\node[align=right, anchor=west] at ({15.0}, +3.75) {Persistence};
		\drawHexagon{{14.0  }}{+1.5}{dtu-white}{}{0.75}
		\node[align=right, anchor=west] at ({15.0}, +2.25) {Atomic Pointer};
		\drawHexagon{{14.0  }}{+0.0}{dtu-white}{}{0.75}
		\node[align=right, anchor=west] at ({15.0}, +0.75) {Metaheuristics};
		\drawHexagon{{14.0  }}{-1.5}{dtu-white}{}{0.75}
		\node[align=right, anchor=west] at ({15.0}, -0.75) {Orchestration};
		\drawHexagon{{14.0  }}{-3.0}{dtu-white}{}{0.75}
		\node[align=right, anchor=west] at ({15.0}, -2.25) {User interfaces};
	\fi


	\ifatomicpointerswaplayer
		\drawHexagon[]{ 2                      }{ 2}{dtu-green}{Shared\\solution\\pointer}{2}
		\drawHexagon[]{{6 - 2 * (2 - sqrt(3)) }}{ 2}{dtu-green}{Shared\\solution\\pointer}{2}
		\drawHexagon[]{{4 - 1 * (2 - sqrt(3)) }}{-1}{dtu-green}{Shared\\solution\\pointer}{2}
		\drawHexagon[]{{0 + 1 * (2 - sqrt(3)) }}{-1}{dtu-green}{Shared\\solution\\pointer}{2}
		\drawHexagon[]{{8 - 3 * (2 - sqrt(3)) }}{-1}{dtu-green}{Shared\\solution\\pointer}{2}

		\drawHexagon[]{{2 - 0 * (2 - sqrt(3)) }}{-4}{dtu-green}{Shared\\solution\\pointer}{2}
		\drawHexagon[]{{6 - 2 * (2 - sqrt(3)) }}{-4}{dtu-green}{Shared\\solution\\pointer}{2}

		\drawHexagon[]{{10 - 4 * (2 - sqrt(3)) }}{-4}{dtu-green}{Shared\\solution\\pointer}{2}
		\drawHexagon[]{{-2 + 2 * (2 - sqrt(3)) }}{-4}{dtu-green}{Shared\\solution\\pointer}{2}

		\drawHexagon[]{{12 - 5 * (2 - sqrt(3)) }}{-1}{dtu-green}{Shared\\solution\\pointer}{2}
		\drawHexagon[]{{-4 + 3 * (2 - sqrt(3)) }}{-1}{dtu-green}{Shared\\solution\\pointer}{2}
		% Legend for each layer
		\drawHexagon{{14.0  }}{+3.0}{dtu-white}{}{0.75}
		\node[align=right, anchor=west] at ({15.0}, +3.75) {Persistence};
		\drawHexagon{{14.0  }}{+1.5}{dtu-green}{}{0.75}
		\node[align=right, anchor=west] at ({15.0}, +2.25) {Atomic Pointer};
		\drawHexagon{{14.0  }}{+0.0}{dtu-white}{}{0.75}
		\node[align=right, anchor=west] at ({15.0}, +0.75) {Metaheuristics};
		\drawHexagon{{14.0  }}{-1.5}{dtu-white}{}{0.75}
		\node[align=right, anchor=west] at ({15.0}, -0.75) {Orchestration};
		\drawHexagon{{14.0  }}{-3.0}{dtu-white}{}{0.75}
		\node[align=right, anchor=west] at ({15.0}, -2.25) {User interfaces};
	\fi

	\ifsimplifiedlayer

		\node[align=right, anchor=west] at ({-5.5}, +3.75) {};
		\drawHexagon{{+2 + 0 * (2 - sqrt(3)) }}{ 2}{dtu-green}{Scheduler}{2}
		\drawHexagon{{+4 - 1 * (2 - sqrt(3)) }}{-1}{dtu-red}{Supervisor}{2}
		\drawHexagon{{+0 + 1 * (2 - sqrt(3)) }}{-1}{dtu-red}{Supervisor}{2}
		\drawHexagon{{+2 - 0 * (2 - sqrt(3)) }}{-4}{dtu-corporate-red}{Technician}{2}
		\drawHexagon{{+6 - 2 * (2 - sqrt(3)) }}{-4}{dtu-corporate-red}{Technician}{2}
		\drawHexagon{{-2 + 2 * (2 - sqrt(3)) }}{-4}{dtu-corporate-red}{Technician}{2}
		\drawHexagon{{+8 - 3 * (2 - sqrt(3)) }}{-1}{dtu-corporate-red}{Technician}{2}
		\drawHexagon{{-4 + 3 * (2 - sqrt(3)) }}{-1}{dtu-corporate-red}{Technician}{2}

		% Scheduler
		\draw[thin, fill=dtu-yellow] (2, 5) circle (0.35);
		\draw[thin, fill=dtu-purple] (2, 3) circle (0.35);
		% Supervisor 1
		\draw[thin, fill=dtu-yellow] ({+4 - 1 * (2 - sqrt(3)) }, 02) circle (0.35);
		\draw[thin, fill=dtu-purple] ({+4 - 1 * (2 - sqrt(3)) }, -0) circle (0.35);
		% Supervisor 2
		\draw[thin, fill=dtu-yellow] ({+0 + 1 * (2 - sqrt(3)) }, 02) circle (0.35);
		\draw[thin, fill=dtu-purple] ({+0 + 1 * (2 - sqrt(3)) }, -0) circle (0.35);
		% Technician 1
		\draw[thin, fill=dtu-yellow] ({+2 - 0 * (2 - sqrt(3)) }, -1) circle (0.35);
		\draw[thin, fill=dtu-purple] ({+2 - 0 * (2 - sqrt(3)) }, -3) circle (0.35);
		% Technician 2
		\draw[thin, fill=dtu-yellow] ({+6 - 2 * (2 - sqrt(3)) }, -1) circle (0.35);
		\draw[thin, fill=dtu-purple] ({+6 - 2 * (2 - sqrt(3)) }, -3) circle (0.35);
		% Technician 3
		\draw[thin, fill=dtu-yellow] ({-2 + 2 * (2 - sqrt(3)) }, -1) circle (0.35);
		\draw[thin, fill=dtu-purple] ({-2 + 2 * (2 - sqrt(3)) }, -3) circle (0.35);
		% Technician 4
		\draw[thin, fill=dtu-yellow] ({+8 - 3 * (2 - sqrt(3)) }, 02) circle (0.35);
		\draw[thin, fill=dtu-purple] ({+8 - 3 * (2 - sqrt(3)) }, -0) circle (0.35);
		% Technician 5
		\draw[thin, fill=dtu-yellow] ({-4 + 3 * (2 - sqrt(3)) }, 02) circle (0.35);
		\draw[thin, fill=dtu-purple] ({-4 + 3 * (2 - sqrt(3)) }, -0) circle (0.35);

		% Legend for each layer
		\node[align=right, anchor=west] at ({12.0}, +3.75) {Atomic Pointer};
		\draw[fill=dtu-purple] (11.0,  +3.75) circle (0.5);

		\node[align=right, anchor=west] at ({12.0}, +2.25) {Scheduler Metaheuristic};
		\drawHexagon{{11.0  }}{+1.75}{dtu-green}{}{0.5}
		\node[align=right, anchor=west] at ({12.0}, +0.75) {Supervisor Metaheuristic};
		\drawHexagon{{11.0  }}{+0.25}{dtu-red}{}{0.5}
		\node[align=right, anchor=west] at ({12.0}, -0.75) {Technician Metaheuristic};
		\drawHexagon{{11.0  }}{-1.25}{dtu-corporate-red}{}{0.5}
		\node[align=right, anchor=west] at ({12.0}, -2.25) {User interfaces (Message Passing)};
		\draw[fill=dtu-yellow] (11.0, -2.25) circle (0.5);
	\fi

	\ifmetaheuristicslayer
		\drawHexagon{ 2                      }{ 2}{dtu-blue}{Strategic}{2}
		\drawHexagon{{6 - 2 * (2 - sqrt(3)) }}{ 2}{dtu-green}{Tactical}{2}
		\drawHexagon{{4 - 1 * (2 - sqrt(3)) }}{-1}{dtu-red}{Supervisor}{2}
		\drawHexagon{{0 + 1 * (2 - sqrt(3)) }}{-1}{dtu-red}{Supervisor}{2}
		\drawHexagon{{8 - 3 * (2 - sqrt(3)) }}{-1}{dtu-red}{Supervisor}{2}

		\drawHexagon{{2 - 0 * (2 - sqrt(3)) }}{-4}{dtu-corporate-red}{Technician}{2}
		\drawHexagon{{6 - 2 * (2 - sqrt(3)) }}{-4}{dtu-corporate-red}{Technician}{2}

		\drawHexagon{{10 - 4 * (2 - sqrt(3)) }}{-4}{dtu-corporate-red}{Technician}{2}
		\drawHexagon{{-2 + 2 * (2 - sqrt(3)) }}{-4}{dtu-corporate-red}{Technician}{2}

		\drawHexagon{{12 - 5 * (2 - sqrt(3)) }}{-1}{dtu-corporate-red}{Technician}{2}
		\drawHexagon{{-4 + 3 * (2 - sqrt(3)) }}{-1}{dtu-corporate-red}{Technician}{2}

		% Legend for each layer
		\drawHexagon{{14.0  }}{+3.0}{dtu-white}{}{0.75}
		\node[align=right, anchor=west] at ({15.0}, +3.75) {Persistence};
		\drawHexagon{{14.0  }}{+1.5}{dtu-white}{}{0.75}
		\node[align=right, anchor=west] at ({15.0}, +2.25) {Atomic Pointer};
		\drawHexagon{{14.0  }}{+0.0}{dtu-corporate-red}{}{0.75}
		\node[align=right, anchor=west] at ({15.0}, +0.75) {Metaheuristics};
		\drawHexagon{{14.0  }}{-1.5}{dtu-white}{}{0.75}
		\node[align=right, anchor=west] at ({15.0}, -0.75) {Orchestration};
		\drawHexagon{{14.0  }}{-3.0}{dtu-white}{}{0.75}
		\node[align=right, anchor=west] at ({15.0}, -2.25) {User interfaces};
	\fi

	\iforchestratorlayer
		\drawHexagon{ 2                      }{ 2}{dtu-orange}{}{2}
		\drawHexagon{{6 - 2 * (2 - sqrt(3)) }}{ 2}{dtu-orange}{}{2}
		\drawHexagon{{4 - 1 * (2 - sqrt(3)) }}{-1}{dtu-orange}{Orche-\\strator}{2}
		\drawHexagon{{0 + 1 * (2 - sqrt(3)) }}{-1}{dtu-orange}{}{2}
		\drawHexagon{{8 - 3 * (2 - sqrt(3)) }}{-1}{dtu-orange}{}{2}

		\drawHexagon{{2 - 0 * (2 - sqrt(3)) }}{-4}{dtu-orange}{}{2}
		\drawHexagon{{6 - 2 * (2 - sqrt(3)) }}{-4}{dtu-orange}{}{2}

		\drawHexagon{{10 - 4 * (2 - sqrt(3)) }}{-4}{dtu-orange}{}{2}
		\drawHexagon{{-2 + 2 * (2 - sqrt(3)) }}{-4}{dtu-orange}{}{2}

		\drawHexagon{{12 - 5 * (2 - sqrt(3)) }}{-1}{dtu-orange}{}{2}
		\drawHexagon{{-4 + 3 * (2 - sqrt(3)) }}{-1}{dtu-orange}{}{2}
		% Legend for each layer
		\drawHexagon{{14.0  }}{+3.0}{dtu-white}{}{0.75}
		\node[align=right, anchor=west] at ({15.0}, +3.75) {Persistence};
		\drawHexagon{{14.0  }}{+1.5}{dtu-white}{}{0.75}
		\node[align=right, anchor=west] at ({15.0}, +2.25) {Atomic Pointer};
		\drawHexagon{{14.0  }}{+0.0}{dtu-white}{}{0.75}
		\node[align=right, anchor=west] at ({15.0}, +0.75) {Metaheuristics};
		\drawHexagon{{14.0  }}{-1.5}{dtu-orange}{}{0.75}
		\node[align=right, anchor=west] at ({15.0}, -0.75) {Orchestration};
		\drawHexagon{{14.0  }}{-3.0}{dtu-white}{}{0.75}
		\node[align=right, anchor=west] at ({15.0}, -2.25) {User interfaces};
	\fi

	
	\ifuserinterfacelayer
		\drawHexagon{ 2                      }{ 2}{dtu-yellow}{UI}{2}
		\drawHexagon{{6 - 2 * (2 - sqrt(3)) }}{ 2}{dtu-yellow}{UI}{2}
		\drawHexagon{{4 - 1 * (2 - sqrt(3)) }}{-1}{dtu-yellow}{UI}{2}
		\drawHexagon{{0 + 1 * (2 - sqrt(3)) }}{-1}{dtu-yellow}{UI}{2}
		\drawHexagon{{8 - 3 * (2 - sqrt(3)) }}{-1}{dtu-yellow}{UI}{2}

		\drawHexagon{{2 - 0 * (2 - sqrt(3)) }}{-4}{dtu-yellow}{UI}{2}
		\drawHexagon{{6 - 2 * (2 - sqrt(3)) }}{-4}{dtu-yellow}{UI}{2}

		\drawHexagon{{10 - 4 * (2 - sqrt(3)) }}{-4}{dtu-yellow}{UI}{2}
		\drawHexagon{{-2 + 2 * (2 - sqrt(3)) }}{-4}{dtu-yellow}{UI}{2}

		\drawHexagon{{12 - 5 * (2 - sqrt(3)) }}{-1}{dtu-yellow}{UI}{2}
		\drawHexagon{{-4 + 3 * (2 - sqrt(3)) }}{-1}{dtu-yellow}{UI}{2}
		% Legend for each layer
		\drawHexagon{{14.0  }}{+3.0}{dtu-white}{}{0.75}
		\node[align=right, anchor=west] at ({15.0}, +3.75) {Persistence};
		\drawHexagon{{14.0  }}{+1.5}{dtu-white}{}{0.75}
		\node[align=right, anchor=west] at ({15.0}, +2.25) {Atomic Pointer};
		\drawHexagon{{14.0  }}{+0.0}{dtu-white}{}{0.75}
		\node[align=right, anchor=west] at ({15.0}, +0.75) {Metaheuristics};
		\drawHexagon{{14.0  }}{-1.5}{dtu-white}{}{0.75}
		\node[align=right, anchor=west] at ({15.0}, -0.75) {Orchestration};
		\drawHexagon{{14.0  }}{-3.0}{dtu-yellow}{}{0.75}
		\node[align=right, anchor=west] at ({15.0}, -2.25) {User interfaces};
	\fi
	
	\end{tikzpicture}
}

	\resizebox{0.7\textwidth}{!}{
		\drawModelSetupHexagon[userinterface=true]
	}
	\caption{
		Overview of the scheduling process when modelled as actors. When LNS is encapsulated 
		is an actor it becomes possible to optimize parts of a large process individually instead of 
		optimizing the scheduling problem globally from a single model implementation.
	}
	\label{fig:ordinator-hexagon:userinterfaces}
\end{figure}

\subsection{Mathematical Models}
Each actor of the maintenance scheduling process will here be modelled by a mathematical model. There
are multiple driving ideas behind these approaches:
\begin{itemize}
	\item A model should encapsulate the decision that a single decision-maker is responsible for and
		nothing more
	\item Coordination between different actors is done by sharing relevant information 
		by turning the solution of one actor/model into parameters of the other model.
\end{itemize}

\newpage
\begin{figure}[H]
\subsubsection{Strategic Model: Scheduler}

\newpage
\begin{alignat}{2}
	& \text{\rule{\linewidth}{0.4pt}} \notag\\
	& \textbf{Meta variables:} \notag\\
	& \ElementScheduler \in \SetScheduler \\
	& \VarTacticalWork{}{} \\ 
	& \tau \in [0, \infty] \\
	& \text{\rule{\linewidth}{0.4pt}} \notag\\
	& \textbf{Minimize:} \notag                                                                                                                                                        \\
	& \sum_{\ElementWorkOrder \in \SetWorkOrder{}} \sum_{\ElementPeriod \in \SetPeriod} \ParStrategicValue \cdot \VarStrategicWorkOrderAssignment{\ElementWorkOrder}{\ElementPeriod}  \notag\\ 
	& + \sum_{\ElementPeriod \in \SetPeriod} \sum_{\ElementResource \in \SetResource} \ParStrategicPenalty \cdot \VarStrategicExcess     \notag                                              \\
	& + \sum_{\ElementPeriod \in \SetPeriod} \sum_{\ElementWorkOrder1 \in \SetWorkOrder{}} \sum_{\ElementWorkOrder2 \in \SetWorkOrder{}} 	 \quad \ParClusteringValue \cdot \VarStrategicWorkOrderAssignment{\ElementWorkOrder1}{\ElementPeriod} \cdot \VarStrategicWorkOrderAssignment{\ElementWorkOrder2}{\ElementPeriod}  \\
	& \text{\rule{\linewidth}{0.4pt}} \notag\\
	& \textbf{Subject to:} \notag                                                                                                                                                      \\
	& \sum_{\ElementWorkOrder \in \SetWorkOrder{}} \ParStrategicWorkOrderWeight \cdot \VarStrategicWorkOrderAssignment{\ElementWorkOrder}{\ElementPeriod} \leq \ \ParStrategicResource + \VarStrategicExcess                                                                           \quad \forall \ElementPeriod \in \SetPeriod \quad \forall \ElementResource \in \SetResource                                                                                      \\
	& \sum_{\ElementWorkOrder \in \SetWorkOrder{}} \VarStrategicWorkOrderAssignment{\ElementWorkOrder}{\ElementPeriod} = 1              \quad \forall \ElementPeriod \in \SetPeriod                                                                                                                                      \\
	& \VarStrategicWorkOrderAssignment{\ElementWorkOrder}{\ElementPeriod} = 0                                                            \quad \forall (\ElementWorkOrder, \ElementPeriod) \in \ParStrategicExclude                                                                                                       \\
	& \VarStrategicWorkOrderAssignment{\ElementWorkOrder}{\ElementPeriod} = 1                                                            \quad \forall (\ElementWorkOrder, \ElementPeriod) \in \ParStrategicInclude                                                                                                       \\
	& \VarStrategicWorkOrderAssignment{\ElementWorkOrder}{\ElementPeriod} \in \{0, 1\}                                                   \quad \forall \ElementWorkOrder \in \SetWorkOrder{} \quad \forall \ElementPeriod \in \SetPeriod                                                                                 \\ 
	& \VarStrategicExcess \in \mathbb{R}^{+}                                                                                             \quad \forall \ElementPeriod \in \SetPeriod \quad \forall \ElementResource \in \SetResource                                                                                  \\ 
	& \text{\rule{\linewidth}{0.4pt}} \notag
\end{alignat}
\newpage


\strategicModel[clustering=true, beta=false, normal=false, multiskill=true]
\end{figure}

\begin{figure}[H]
\subsubsection{Tactical Model: Scheduler}
\section{The Tactical Model}
\begin{itemize}
	\item Respect precedence constraints
	\item Respect daily resource requirements for each trait
	\item Penalize exceeded daily capacity
\end{itemize}

After the strategic model has optimized its schedule the tactical agent will continue scheduling the output at a more detailed level. This means that now the tactical agent will schedule 
out on each of the days of the work orders scheduled by the strategic agent. 

The tactical model is responsible for providing an initial suggestion for a weekly schedule, below we see the model for the tactical agent.
\begin{alignat}{2}
	& \textbf{Meta variables:} \notag\\
	& \ElementScheduler = \SetScheduler \notag\\
	& \tau \in [0, \infty] \notag\\
	& \VarStrategicWorkOrderAssignment{}{} \notag\\
	& \textbf{Minimize:} \notag\\
	& \sum_{\ElementOperation \in \SetOperation{}{\VarStrategicWorkOrderAssignment{}{}}} \sum_{\ElementDays \in \SetDays{}} \ParTacticalValue!!!!!!!!!!!!! \cdot \VarTacticalWork{\ElementDays}{\ElementOperation}\notag\\  
	& + \sum_{r \in \SetResource} \sum_{\ElementDays \in \SetDays{}} \ParTacticalPenalty \cdot \VarTacticalExcess                                               \\  
	& \textbf{Subject to:}                                                          \notag                                                                   \\
	& \sum_{\ElementOperation \in \SetOperation{}{\VarStrategicWorkOrderAssignment{}{}}} \ParOperationWork{\ElementOperation} \cdot \VarTacticalWork{\ElementDays}{\ElementOperation}\notag\\
	& \quad \leq \ParTacticalResource + \VarTacticalExcess\notag\\ 
	& \quad \forall \ElementDays \in \SetDays{} \quad \forall r \in \SetResource\\ 
	& \sum_{\ElementDays = \ParEarliestStart}^{\ParLatestFinish} \VarTacticalWorkBinary{\ElementDays}{\ElementOperation} = \ParDuration \notag\\
	& \quad \forall \ElementOperation \in \SetOperation{}{\VarStrategicWorkOrderAssignment{}{}} \\
	& \sum_{\ElementDays^* \in  \SetDays{\ParDuration}} \VarTacticalWorkBinary{\ElementDays^*}{\ElementOperation} \notag\\
	& \quad = \ParDuration \cdot \VarTacticalWorkBinaryConsecutive \notag\\ 
	& \quad \forall \ElementOperation \in \SetOperation{}{\VarStrategicWorkOrderAssignment{}{}} \quad \forall \ElementDays \in \SetDays{} \\
	& \sum_{\ElementOperation \in \SetOperation{}{\VarStrategicWorkOrderAssignment{}{}}} \VarTacticalWorkBinaryConsecutive = 1, \notag\\
	& \quad \forall \ElementDays \in \SetDays{} \notag\\
	& \sum_{\ElementDays \in \SetDays{}} \ElementDays \cdot \VarTacticalWorkBinary{\ElementDays}{\ElementOperation1} + \VarTacticalOperationDifference  = \sum_{\ElementDays \in \SetDays{}} \ElementDays \cdot \VarTacticalWorkBinary{\ElementDays}{\ElementOperation2}                   \notag  \\ 
	& \quad \forall (o1, \ElementOperation2) \in \ParFinishStart                                                           \\ 
	& \sum_{\ElementDays \in \SetDays{}} \ElementDays \cdot \VarTacticalWorkBinary{\ElementDays}{\ElementOperation1} = \sum_{\ElementDays \in \SetDays{}} \ElementDays \cdot \VarTacticalWorkBinary{\ElementDays}{\ElementOperation2}  \notag                               \\ 
	& \quad \forall (o1, \ElementOperation2) \in \ParStartStart                                                       \\ 
	& \VarTacticalWork{\ElementDays}{\ElementOperation} \leq \ParNumberOfPeople \cdot \ParOperatingTime \notag                                                     \\ 
	& \quad \forall \ElementDays \in \SetDays{} \quad \forall \ElementOperation \in \SetOperation{}{\VarStrategicWorkOrderAssignment{}{}}                                                                  \\
	& \VarTacticalWork{\ElementDays}{\ElementOperation} \in \mathbb{R} \quad \notag\\
	& \quad \forall \ElementDays \in \SetDays{} \quad \forall \ElementOperation \in \SetOperation{}{\VarStrategicWorkOrderAssignment{}{}}                                         \\
	& \VarTacticalExcess \in \mathbb{R} \quad\notag\\
	& \quad \forall r \in \SetResource \quad \forall \ElementDays \in \SetDays{}                                        \\
	& \VarTacticalWorkBinary{\ElementDays}{\ElementOperation} \in \{0, 1\}\quad \notag\\
	& \quad \forall \ElementDays \in \SetDays{} \quad \forall \ElementOperation \in \SetOperation{}{\VarStrategicWorkOrderAssignment{}{}} \\
	& \VarTacticalWorkBinaryConsecutive \in \{0, 1\}\quad \notag\\
	& \quad \forall \ElementDays \in \SetDays{} \quad \forall \ElementOperation \in \SetOperation{}{\VarStrategicWorkOrderAssignment{}{}} \\
	& \VarTacticalOperationDifference \in \{0, 1\} \\
	& \quad \forall \ElementOperation \in \SetOperation{}{\VarStrategicWorkOrderAssignment{}{}}                                      \\
	& \VarMetaTime \in  [0, \infty] 
\end{alignat}


\tacticalModel[]
\end{figure}
\begin{figure}[H]
\subsubsection{Supervisor Model: Supervisor}
\section{The Supervisor Model}
The maintenance supervisor is considered the most central person in a maintenance scheduling system. 
All the work of the planner and scheduler should be considered a service for the supervisor.

The supervisor has multiple different responsibilities among them are: 

\begin{itemize}
	\item Assigning work orders
	\item Creating a daily schedule
	\item Keeping the schedule up-to-date
\end{itemize}


\begin{alignat}{2}
	& \textbf{Meta variables:} \notag\\
	& \tau \in [0, \infty] \notag\\
	& \ElementSupervisor \in \SetSupervisor \notag\\
	& \VarStrategicWorkOrderAssignment{}{} \notag\\
	& \VarIncludeActivity{} \notag\\
	& \textbf{Maximize:} \notag\\
	& \sum_{\ElementActivity \in \SetActivity{\VarStrategicWorkOrderAssignment{}{}}{}} \sum_{\ElementTechnician \in \SetTechnician} \ParSupervisorValue \cdot \VarSupervisorAssignment{\ElementActivity}{\ElementTechnician} \\ 
	& \textbf{Subject to:} \notag\\ 
	& \sum_{\ElementActivity \in \SetActivity{\VarStrategicWorkOrderAssignment{}{}}{\ElementOperation}} \VarActivityWork{\ElementActivity} = \ParOperationWork{\ElementOperation}   \notag\\
	& \quad \forall \ElementOperation \in \SetOperation{}{\VarStrategicWorkOrderAssignment{}{}}\\
	& \sum_{\ElementTechnician \in \SetTechnician} \sum_{\ElementActivity \in \SetActivity{\VarStrategicWorkOrderAssignment{}{}}{\ElementOperation}}\VarSupervisorAssignment{\ElementActivity}{\ElementTechnician} = \VarSupervisorAssignmentWhole \cdot \ParNumberOfPeople \notag\\
	& \quad \forall \ElementOperation \in \SetOperation{}{\VarStrategicWorkOrderAssignment{}{}}  \\
	& \sum_{\ElementOperation \in \SetOperation{\ElementWorkOrder}{\VarStrategicWorkOrderAssignment{}{}}} \VarSupervisorAssignmentWhole = !!!! |\SetOperation{\ElementWorkOrder}{\VarStrategicWorkOrderAssignment{}{}}| \notag\\ 
	& \quad \forall \ElementWorkOrder \in \SetWorkOrder{,\VarStrategicWorkOrderAssignment{}{}} \\
	& \sum_{\ElementActivity \in \SetActivity{\VarStrategicWorkOrderAssignment{}{}}{\ElementOperation}} \VarSupervisorAssignment{\ElementActivity}{\ElementTechnician} \leq 1 \notag\\
	& \quad \forall \ElementOperation \in \SetOperation{}{\VarStrategicWorkOrderAssignment{}{}} \quad \forall \ElementTechnician \in \SetTechnician \\  
	& \VarSupervisorAssignment{\ElementActivity}{\ElementTechnician} \leq \ParFeasible \notag\\
	& \quad \forall \ElementOperation \in \SetOperation{}{\VarStrategicWorkOrderAssignment{}{}} \quad \forall \ElementTechnician \in \SetTechnician \\
	& \VarSupervisorAssignment{\ElementActivity}{\ElementTechnician} \in \{0, 1\} \notag\\
	& \quad \forall \ElementOperation \in \SetOperation{}{\VarStrategicWorkOrderAssignment{}{}} \quad \forall \ElementTechnician \in \SetTechnician \\ 
	& \VarActivityWork{\ElementActivity} \in [\ParLowerActivityWork, \ParOperationWork{\ElementActivity}] \notag\\
	& \quad \forall \ElementActivity \in \SetActivity{\VarStrategicWorkOrderAssignment{}{}}{} \\
    & \VarMetaTime \in  [0, \infty] 
\end{alignat}

In the supervisor model shown in \ref{} the set $O$ and $W$ comes from the tactical algorithm
and value $v$ and the information of whether or not the operation can be assigned to a 
specific operational model comes from the operational model itself and is captured in the.

Can this be done? What should the Supervisor have here? He should have what is necessary to
handle the. 

\supervisorModel[]
\end{figure}


\begin{figure}[H]
\subsubsection{Operational Model: Technicial}
\begin{alignat}{2}
	& \text{\rule{\linewidth}{0.4pt}} \notag\\
	& \textbf{Meta variables:}                                                                                                                                                                         \notag\\
	& \ElementTechnician \in \SetTechnician                                                                                                                                                            \\
	& \VarStrategicWorkOrderAssignment{}{}                                                                                                                                                             \\
	& \VarSupervisorAssignment{}{}                                                                                                                                                                     \\
	& \tau \in [0, \infty]                                                                                                                                                                             \\
	& \text{\rule{\linewidth}{0.4pt}} \notag\\
	& \textbf{Maximize:}                                                                                                                                                                               \notag\\
	& \sum_{\ElementActivity \in \SetActivity{\VarSupervisorAssignment{}{\ElementTechnician}}{}} \sum_{\ElementWorkSegment \in \SetWorkSegment} \VarProcessingTime                                           \\
	& \text{\rule{\linewidth}{0.4pt}} \notag\\
	& \textbf{Subject to:}                                                                                                                                                                             \notag\\
    & \sum_{\ElementWorkSegment \in \SetWorkSegment} \VarProcessingTime \cdot \VarActiveSegment{\ElementActivity}{\ElementWorkSegment} = \ParActivityWork{} \cdot \VarIncludeActivity{\ElementActivity}                                                                                                                                            \quad \forall \ElementActivity \in \SetActivity{\VarSupervisorAssignment{}{\ElementTechnician}}{}                                                                                                      \\
	& \VarStartOfSegment{\ElementActivity2}{1} \geq \VarFinishOfSegment{\ElementActivity1}{last(\ElementActivity1)} + \ParPreparation                                                                    \quad \forall \ElementActivity1 \in \SetActivity{\VarSupervisorAssignment{}{\ElementTechnician}}{} \quad \forall \ElementActivity2 \in \SetActivity{\VarSupervisorAssignment{}{\ElementTechnician}}{}  \\
	& \VarStartOfSegment{\ElementActivity}{\ElementWorkSegment} \geq \VarFinishOfSegment{\ElementActivity}{\ElementWorkSegment-1} - \ParConstraintLimit \cdot (2 - \VarActiveSegment{\ElementActivity}{\ElementWorkSegment} + \VarActiveSegment{\ElementActivity}{\ElementWorkSegment-1})                                                                      \notag\\
	& \quad \forall \ElementActivity \in \SetActivity{\VarSupervisorAssignment{}{\ElementTechnician}}{} \quad\forall \ElementWorkSegment \in \SetWorkSegment                                                 \\ 
	& \VarProcessingTime = \VarFinishOfSegment{\ElementActivity}{\ElementWorkSegment} - \VarStartOfSegment{\ElementActivity}{\ElementWorkSegment}                                                                                                                                              \quad \forall \ElementActivity \in \SetActivity{\VarSupervisorAssignment{}{\ElementTechnician}}{}  \quad\forall \ElementWorkSegment \in \SetWorkSegment                                                \\
	& \VarStartOfSegment{\ElementActivity}{\ElementWorkSegment} \geq \ParEvent + \ParEventDuration - \ParConstraintLimit \cdot (1 - \VarSegmentInRelation) \notag\\ 
	& \quad \forall \ElementActivity \in \SetActivity{\VarSupervisorAssignment{}{\ElementTechnician}}{}  \quad\forall \ElementWorkSegment \in \SetWorkSegment                                           \quad \forall i \in \SetTimeInstance  \quad\forall \ElementEvent \in \SetEvent                                                                                                                         \\
	& \VarFinishOfSegment{\ElementActivity}{\ElementWorkSegment} \leq \ParEvent + \ParConstraintLimit \cdot \VarSegmentInRelation \notag\\
	& \quad \forall \ElementActivity \in \SetActivity{\VarSupervisorAssignment{}{\ElementTechnician}}{}  \quad\forall \ElementWorkSegment \in \SetWorkSegment                                         \quad \forall i \in \SetTimeInstance  \quad\forall \ElementEvent \in \SetEvent                                                                                                                         \\
	& \VarStartOfSegment{\ElementActivity}{1} \geq \ParTimeWindowStart  \quad \forall \ElementActivity \in \SetActivity{\VarSupervisorAssignment{}{\ElementTechnician}}{}                                                                                                      \\
	& \VarFinishOfSegment{\ElementActivity}{last(\ElementActivity)} \leq \ParTimeWindowFinish  \quad \forall \ElementActivity \in \SetActivity{\VarSupervisorAssignment{}{\ElementTechnician}}{}                                                                                                      \\
	& \VarActiveSegment{\ElementActivity}{\ElementWorkSegment} \in \{0, 1\}  \quad \forall \ElementActivity \in \SetActivity{\VarSupervisorAssignment{}{\ElementTechnician}}{} \quad \forall \ElementWorkSegment \in \SetWorkSegment                                                \\
	& \VarStartOfSegment{\ElementActivity}{\ElementWorkSegment} \in [\ParAvailabilityStart, \ParAvailabilityFinish]                                                                                                           \notag\\
	& \quad \forall \ElementActivity \in \SetActivity{\VarSupervisorAssignment{}{\ElementTechnician}}{} \quad \forall \ElementWorkSegment \in \SetWorkSegment                                                \\
	& \VarFinishOfSegment{\ElementActivity}{\ElementWorkSegment} \in [\ParAvailabilityStart, \ParAvailabilityFinish]                                                                                                             \notag\\
	& \quad \forall \ElementActivity \in \SetActivity{\VarSupervisorAssignment{}{\ElementTechnician}}{} \quad \forall \ElementWorkSegment \in \SetWorkSegment                                                \\
	& \VarProcessingTime \in [0, \ParOperationWork{\ElementActivity\_to\_o(\ElementActivity)} ]  \quad \forall \ElementActivity \in \SetActivity{\VarSupervisorAssignment{}{\ElementTechnician}}{} \quad \forall \ElementWorkSegment \in \SetWorkSegment                                                \\
	& \VarSegmentInRelation \in \{0 , 1\}                                                                                                                                                                                                                                                                                     \quad \forall \ElementActivity \in \SetActivity{\VarSupervisorAssignment{}{\ElementTechnician}}{} \quad \forall \ElementWorkSegment \in \SetWorkSegment  \quad \forall i \in \SetTimeInstance \quad \forall \ElementEvent \in \SetEvent                                                                                                                         \\
	& \VarIncludeActivity{\ElementActivity} \in \{0, 1\}                                                                                                                                      \quad \forall \ElementActivity \in \SetActivity{\VarSupervisorAssignment{}{\ElementTechnician}}{}                                                                                                     \\ 
	& \text{\rule{\linewidth}{0.4pt}} \notag
\end{alignat}

\operationalModel[]
\end{figure}


\subsection{Contribution}
Provide a software architectual pattern to allow metaheuristics to be integrated together in real-time.
