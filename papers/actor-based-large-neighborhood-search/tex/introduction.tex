\section{Introduction}
\label{sec:1-introduction}
%% Labels are used to cross-reference an item using \ref command.

Maintenance scheduling is part of a class of operational problems that have proven hard to solve in practive (NP-hard~\citep{garey1979computers}).
Furthermore, for optimization to be useful in dynamic and uncertain environments where maintenance scheduling 
is performed requires tight integration into existing IT infrastructure to allow the  tacit knowledge of decision makers to influence
the planning process easily. Often a number of different decision makers at different company levels take part in the
planning process. In this way the industry usually assigns responsibility for decision-making to an individual
representing only a small part of the complete process. These multiple smaller planning processes are often difficult
to map to a single mathematical model describing the whole system as elaborated by~\citep{barthelemy2002human}. Solving
operation research problems that are operational in nature have additional requirements over more typical static
problems: they have to be responsive to changing parameters; able to be assimilated into the decision-makers workflow;
allow for integration with dynamic data sources such as databases and APIs~\citep{meignan_review_2015}. Operational
aspects of operation research, as opposed to higher level strategic and tactical ones, are characterized by extensive
amounts coordination and negotiation on proposed schedules often in a short amount of time.

The lack of integration into the schedulers and supervisors workflow and lack of responsiveness can lead to a situation 
where solutions are not directly implemented in practice but instead only provides initial suggestions~\citep{meignan_review_2015}.
Theses initial suggestions are then iterated on elsewhere in the scheduling process usually through much more manual means. 
In~\citep{barthelemy2002human} the authors argue that many problems that operation research aim to solve are often composed
of a group of individuals whose decisions are consolidated into an "epistemic subject" for which a mathematical model can be formulated
and solved, with many scheduling problems being good examples. However often multiple actors have different
views on what constitutes an optimal schedule hence resulting in multiple-objectives. Even if multi-objective
optimization~\citep{ehrgott2002multiple} is applied to find a Pareto Front~\citep{Pareto1897} a negotiating process
still is needed between the actors to select the final schedule. 

This paper proposes a solution method that will allow for real-time optimization
based on actor/user interaction and connection to a dynamic data source,
effectively managing the changes to the parameter space as they occur. The proposed solution
method will be tested on the weekly maintenance scheduling problem \citet{palmerMaintenancePlanningScheduling2019} 
which closely resembles a variant of the multi-compartment multi-knapsack problem (MCMKP). It
should be noted that the scientific maintenance scheduling literature deviates
significantly from its practical implementation which is also detailed in
\citep{palmerMaintenancePlanningScheduling2019}. The solution method is
based on the large neighborhood search (LNS) metaheuristic. LNS 
was chosen due to its properties of naturally being able to work with and fix
infeasible solutions and its state of the art performance on various scheduling
problems \citep{gendreauHandbookMetaheuristics2019}. 

To understand the need for actor-based methods some
background knowledge will be required about the maintenance scheduling process.
In figure \ref{fig:integrated:maintenance-process} illustrates the general setup
of a maintenance planning and scheduling system. The system's actors
have the following responsibilities: the planner generates the work orders that
are to be scheduled; the scheduler creates weekly schedules based on a knapsack
formulation; based on the weekly schedule the supervisor assigns work order
activities that the work order is composed of (the assignment problem); the
technicians executes the work in sequential pattern (single machine scheduling).

A final point on the necessity of actor-based approaches to model such a setup
is the idea of ownership of a work order. Throughout the scheduling process a
work order is owned by a specific actor and he alone is allow to modify/execute it. This
means that a single model approach is very difficult to implement in practice
as a work order is modelled differently depending on the actor that currently
owns it. This also highlight another an point in maintenance scheduling: that
the stochastic nature of the maintenance scheduling process can be handled using
a change of model each with different levels of aggregation and different sets
of constraints, opposed to more academic approaches such as fuzzy logic and
stochastic optimization. When the inherrent uncertainties manifest themselves
during planning or execution, work orders are rescheduled by moving between
the different actors, meaning that the stochastic elements of maintenance
scheduling are handled by \textbf{dynamic rescheduling between the actors}.

\begin{figure}[H]
\centering
\includegraphics[width=0.5\textwidth]{figures/Scheduling Process Integrated.png}
\caption{Simple overview of the scheduling process with its primary types of actors. The planner, the scheduler, the supervisor(s), and the technicians. 
The green color highlights the scheduler as it the actor in the maintenance scheduling process that is the foundation for the paper.}
\label{fig:simple-maintenance-process}
\end{figure}

This article has one clear contribution:
\begin{itemize}
	\item To provide a scalable algorithmic component based on the large neighborhood search metaheuristic that will allow business processes to be composed from smaller units of mathematical model implementation, instead of relying of integrated formulations.
\end{itemize}

The paper is divided into four different sections. Section \ref{sec:2-solution-method} explains the weekly maintenance scheduling model in detail and forms the fundation of the paper. Section \ref{sec:3-results} shows that results coming from the implemented system where the implementation will be affected by simulated user-interaction. Section \ref{sec:4-discussion} will discuss the implications of the research and possible future research directions.

\subsection{A weekly maintenance scheduling model}
\label{sub2sec2}
A company needs to create a weekly maintenance plan for the next $\ElementPeriod \in \SetPeriod$
weeks. The weekly schedule is created centrally and consists of scheduling the
$\ElementWorkOrder \in \SetWorkOrder$ work orders, i.e. maintenance tasks, such that all are scheduled into
different weeks. Each work order requires some resourses to be carried out, e.g.
man-power with different qualifications, equipment etc. all of these resourses
are available in limited amounts and are called traits $\ElementResource \in \SetResource$.
To correct for possible manual interventions that can make the problem infeasible 
$\ParStrategicPenalty$. The urgency of the different maintenance operations varies and
is reflected in a penalty for carrying out a maintenance work order in a certain
week $\ParStrategicValue$. Urgent tasks have quickly increasing penalties for the later
weeks week $\ElementPeriod$. Furthermore, two sets exists which will either require the
work order to be carried out in week $\ElementPeriod$, i.e. $(\ElementWorkOrder,\ElementPeriod) \in \ParStrategicInclude$ and a set which
forbids a work order to carried out in week $\ElementPeriod$, i.e. $(\ElementWorkOrder, \ElementPeriod) \in \ParStrategicExclude$. The model
for the problem is the Multi-compartment Multi-knapsack Problem with capacity
penalties MCMKP.

The notation used in the model is based on the notation
from the dynamic metaheuristics literature as found in
\cite{yangMetaheuristicsDynamicCombinatorial2013}, where $\VarMetaTime$ is added as a time
variable on all sets, parameters, and variables that are subject to change.
This allows us to be precise in the timing on the messages that are send to
the Ab-LNS.

%\subsection{The Weekly Schedule: Multi-compartment Multi-knapsack Problem with capacity penalties}
%\label{sub2sec2}
%The actor-based large neighborhood search is implemented on the MCMKP which models that weekly schedule in maintenance. The model is comprised of five different sets. $P$ is the number of weekly periods; $W$ is the number of work orders; $\tau$ is the number of different traits; $E$ is a set that defines which work orders should be excluded from a specific weekly period; $I$ is an inclusion set that defines the allocation of specific work orders which should be included in a specific weekly period. The model has four parameters. $v_{pw}$ is the value of work order $w$ in weekly period $p$; $d$ is the penalty for exceeding a specific trait capacity; $c_{w\tau}$ is the capacity requirement for work order $w$ for trait $t$; $cap_{p\tau}$ is the total amount of capacity available in for weekly period p for for trait t. The model has 2 decision variables. $x_{wp}$, is a binary decision variable equal to one if work order w is in weekly period p and zero otherwise; $pen_{p\tau}$ is non-negative decision variable equal to the amount of excess capacity above the $cap_{p\tau}$ in weekly period p for trait $\tau$. The parameters $v$, $cap$, $Q$, and $P$ are functions of time, $\tau$, in this case as they will be subject to change during the solution process.

\section{The Strategic Model}


The Strategic Model have multiple different purposes.
\begin{itemize}
	\item Schedule Work Order out across the weekly periods
	\item Prioritize all the different released work orders
	\item Respect the available weekly hours available for each trait
\end{itemize}

The Strategic model is responsible for grouping work orders into weekly or biweekly periods depending on which kind of maintenance setup that one is running.
This kind of model closely resembles a variant of the multi-compartment multi-knapsack problem. 

\begin{alignat}{2}
	\text{Min} &\sum_{w \in W} \sum_{p \in P} v_{wp}(t) \cdot x_{wp}(t)                                                                                             \\ 
	+ & \sum_{p \in P} \sum_{\tau = 1}^T d \cdot pen_{p\tau}(t)                                                                                                     \\
	+ & \sum_{p \in P} \sum_{w1 \in W} \sum_{w2 \in W} clu_{w1, w2} \cdot x_{w1p} \cdot x_{w2p}                              \label{eqn:objective_function_strategic} \\
    & \text{subject to:} \notag                                                                                                                                       \\
	& \sum_{w \in W} c_{w\tau} \cdot x_{wp}(t) \leq \ cap_{p\tau}(t) + pen_{p\tau}(t) \notag                                                                          \\ 
	& \forall p \in P, \forall \tau \in T                                                                                    \label{eqn:capacity_constraint}          \\
	& \sum_{w \in W} x_{wp}(t) = 1                                              , \quad \forall p \in P                      \label{eqn:single_workorder_constraint}  \\
	& x_{wp}(t) = 0                                                             , \quad \forall (w, p) \in E(t)              \label{eqn:exclusion_constraint}         \\
	& x_{wp}(t) = 1                                                             , \quad \forall (w, p) \in I(t)              \label{eqn:inclusion_constraint}         \\
	& x_{wp}(t) \in \{0, 1\}                                                    , \quad \forall w \in W, \forall p \in P     \label{eqn:x_integrality_constraint}     \\ 
	& pen_{p\tau}(t) \in \mathbb{R}^{+}                                         , \quad \forall p \in P, \forall \tau \in T  \label{eqn:p_non_negativity_constraint}
    & t \in  [0, \infty] 
\end{alignat}


% TODO start here
The objective function 
\eqref{eqn:objective_function_strategic_value}, 
\eqref{eqn:objective_function_strategic_penalty}, and
\eqref{eqn:objective_function_strategic_clustering}
maximizes the total weighted delay of all work order assignments together
with the penalty $\ParStrategicPenalty$ for exceeding the resource capacity given in constraint
\eqref{eqn:capacity_constraint}. The thrid term  
of the model contains the $\ParClusteringValue$ which turns the model into a 
quadratic problem. This term optimizes the value of putting two work orders
in the same period, if they have share similarity like close proximity, 
same functional location, etc. 

Constraint \eqref{eqn:capacity_constraint}
ensures that all the weights $\ParOperationWork{wr}$ for each activity in an work
order, given that it has been assigned, is lower than the capacity for each
period and for each trait $\tau$. $pen_{p\tau}$ is the amount of exceeded
capacity that is needed for the current assignment of work order to be
feasible. Constraint \eqref{eqn:single_workorder_constraint} makes sure
that each work order is assigned to at least a single period. Constraint
\eqref{eqn:exclusion_constraint} excludes a work order from a certain period
and constraint \eqref{eqn:inclusion_constraint} forces a specific work order
to be in a specific period. Constraint \eqref{eqn:x_integrality_constraint} and
\eqref{eqn:p_non_negativity_constraint} specify the variable domain for $x_{wp}$
and $pen_{p\tau}$ respectively. The effects of changing $E$, $I$, $cap$, and $v$
in real-time will be examined to determine their effects on the weekly schedules
and objective value.

The objective function \eqref{eqn:objective_function_strategic} minimizes the total weight of all work order assignments together with the penalty $d$ for exceeding the capacity given in constraint \eqref{eqn:capacity_constraint}. Constraint \eqref{eqn:capacity_constraint} ensures that all the weights $c_{w\tau}$ for each activity in an work order, given that it has been assigned, is lower than the capacity for each period and for each trait $\tau$. $pen_{p\tau}$ is the amount of exceeded capacity that is needed for the current assignment of work order to be feasible. Constraint \eqref{eqn:single_workorder_constraint} makes sure that each work order is assigned to at least a single period. Constraint \eqref{eqn:exclusion_constraint} excludes a work order from a certain period and constraint \eqref{eqn:inclusion_constraint} forces a specific work order to be in a specific period. Constraint \eqref{eqn:x_integrality_constraint} and \eqref{eqn:p_non_negativity_constraint} specify the variable domain for $x_{wp}$ and $pen_{p\tau}$ respectively. The effects of changing $E$, $I$, $cap$, and $v$ in real-time will be examined to determine their effects on the weekly schedules and objective value.
