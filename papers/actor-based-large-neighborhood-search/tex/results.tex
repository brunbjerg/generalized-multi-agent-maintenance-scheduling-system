\section{Results}
\label{sec:3-results}
To test the AbLNS algorithm in a framework setup a number of tests, where the data are disrupted
during the AbLNS algorithm is tested in this section. The data and company setup is presented in
Section~\ref{sec:data_instance}. Then the effect of forcing work orders into specific weekly 
$\ElementPeriod \in \SetPeriod$ schedules is presented in Section~\ref{sec:response_work_orders} 
and excluding them from periods is shown in Section~\ref{sec:exclusion}.
Afterwards, the effect of reducing the period resource capacities $\ParStrategicResource$ is 
tested in Section~\ref{sec:results:reduced_weekly_capacity} and increasing them is tested in 
Section~\ref{sec:increase_week_cap}. 
Finally the effect of changing the work order values $\ParStrategicValue$, 
is tested in Section~\ref{sec:results:strategic_value_changes}.

\subsection{Data Instance}\label{sec:data_instance}
% TODO: What should be cited here?
This data instance comes from Total Energies~\cite{} , where data is extracted from their SAP ERP system. 
The data instance used in this paper is from a specific offshore platform that for 
the 2 year time horizon contains 3487 outstanding work orders ($\SetWorkOrder{}$), 
have 16 different resource skill sets ($\SetResource$) (e.g. mechanics, electricians, 
turbine specialists, etc.), and finally the scheduling horizon stretches over a 52 bi-weekly 
periods ($\SetPeriod$) meaning approximately 2 years.
 
\begin{table}[H]
\centering
\begin{tabular}{|c|c|c|c|c|}
\hline
           & \begin{tabular}[c]{@{}c@{}}$|\SetWorkOrder{}|$\end{tabular}   & \begin{tabular}[c]{@{}c@{}}$|\SetResource|$\end{tabular}         & \begin{tabular}[c]{@{}c@{}}$|\SetPeriod|$\end{tabular} \\ \hline
Instance 1 & 3487                                                          & 16                                                               & 52                                                            \\ \hline
\end{tabular}

\caption{specific data instances from the case company. Here $\SetWorkOrder$ is the set of work orders, $\SetResource$ is the set of resources, and $\SetPeriod$ is the set of weekly periods.} % \label{fig1}
\end{table}

\subsection{Response to Inclusion of Work Orders}\label{sec:response_work_orders}
The $\ParStrategicInclude$ parameter specifies whether a work order should be scheduled into a specific period.
As the parameter has the time variable $\VarMetaTime$ it means that this parameter can change at any time while
the metaheuristic is running. The $\ParStrategicInclude$ parameter constrains the model in equation~\ref{eqn:constraint:strategic:include}.
Table~\ref{tab:responses:inclusion} shows the responses that the model will be subject to while it is running for different 
points in time $\VarMetaTime$. 

\begin{table}[H]
	\centering
	\begin{tabular}{|c|c|c|c|c|c|}
	\hline
	\begin{tabular}[c]{@{}c@{}}\end{tabular}                                & \begin{tabular}[c]{@{}c@{}}$\VarMetaTime_1 = 60$\end{tabular} & \begin{tabular}[c]{@{}c@{}} $\VarMetaTime_2 = 120$ \end{tabular} & \begin{tabular}[c]{@{}c@{}}$\VarMetaTime_3 = 180$\end{tabular} & \begin{tabular}[c]{@{}c@{}}$\VarMetaTime_4 = 240$\end{tabular} & \begin{tabular}[c]{@{}c@{}}$\VarMetaTime_5 = 300$\end{tabular} \\ \hline
	\begin{tabular}[c]{@{}c@{}}$\Delta |\ParStrategicInclude|$\end{tabular} & 50                                                            & 50                                                               & 50                                                             & 50                                                             & 50                                                       \\ \hline
	\end{tabular}
	\caption{The pertubations of including work orders into the weekly scheduling that the AbLNS is affected by. 
	Pertubations occur at 60 second time intervals affecting 50 randomly chosen work orders included into random periods.}
	\label{tab:responses:inclusion}
\end{table}

Figure~\ref{fig:responses:inclusion} shows the effects of changing the $
\ParStrategicInclude$ parameter in real-time. The model quickly
converges and when the system is pertubed by an input response the objective
value \ref{eqn:objective:strategic} shows a small spike and then quickly converges to a new solution. 

\begin{figure}[H]
	\centering
	\resizebox{10cm}{!}{
		\tikzpicture[gnuplot]
%% generated with GNUPLOT 6.0p1 (Lua 5.2; terminal rev. Jun 2020, script rev. 118)
%% Mon 06 Jan 2025 03:43:49 PM UTC
\path (0.000,0.000) rectangle (16.000,12.000);
\gpcolor{rgb color={0.753,0.753,0.753}}
\gpsetlinetype{gp lt border}
\gpsetdashtype{gp dt solid}
\gpsetlinewidth{2.00}
\draw[gp path] (2.424,1.845)--(13.667,1.845);
\gpcolor{color=gp lt color border}
\gpsetlinewidth{1.00}
\draw[gp path] (2.424,1.845)--(2.604,1.845);
\node[gp node right] at (2.240,1.845) {$8.85\times10^{10}$};
\gpcolor{rgb color={0.753,0.753,0.753}}
\gpsetlinewidth{2.00}
\draw[gp path] (2.424,3.383)--(13.667,3.383);
\gpcolor{color=gp lt color border}
\gpsetlinewidth{1.00}
\draw[gp path] (2.424,3.383)--(2.604,3.383);
\node[gp node right] at (2.240,3.383) {$8.9\times10^{10}$};
\gpcolor{rgb color={0.753,0.753,0.753}}
\gpsetlinewidth{2.00}
\draw[gp path] (2.424,4.922)--(13.667,4.922);
\gpcolor{color=gp lt color border}
\gpsetlinewidth{1.00}
\draw[gp path] (2.424,4.922)--(2.604,4.922);
\node[gp node right] at (2.240,4.922) {$8.95\times10^{10}$};
\gpcolor{rgb color={0.753,0.753,0.753}}
\gpsetlinewidth{2.00}
\draw[gp path] (2.424,6.460)--(13.667,6.460);
\gpcolor{color=gp lt color border}
\gpsetlinewidth{1.00}
\draw[gp path] (2.424,6.460)--(2.604,6.460);
\node[gp node right] at (2.240,6.460) {$9\times10^{10}$};
\gpcolor{rgb color={0.753,0.753,0.753}}
\gpsetlinewidth{2.00}
\draw[gp path] (2.424,7.998)--(13.667,7.998);
\gpcolor{color=gp lt color border}
\gpsetlinewidth{1.00}
\draw[gp path] (2.424,7.998)--(2.604,7.998);
\node[gp node right] at (2.240,7.998) {$9.05\times10^{10}$};
\gpcolor{rgb color={0.753,0.753,0.753}}
\gpsetlinewidth{2.00}
\draw[gp path] (2.424,9.537)--(7.047,9.537);
\draw[gp path] (13.483,9.537)--(13.667,9.537);
\gpcolor{color=gp lt color border}
\gpsetlinewidth{1.00}
\draw[gp path] (2.424,9.537)--(2.604,9.537);
\node[gp node right] at (2.240,9.537) {$9.1\times10^{10}$};
\gpcolor{rgb color={0.753,0.753,0.753}}
\gpsetlinewidth{2.00}
\draw[gp path] (2.424,11.075)--(13.667,11.075);
\gpcolor{color=gp lt color border}
\gpsetlinewidth{1.00}
\draw[gp path] (2.424,11.075)--(2.604,11.075);
\node[gp node right] at (2.240,11.075) {$9.15\times10^{10}$};
\gpcolor{rgb color={0.753,0.753,0.753}}
\gpsetlinewidth{2.00}
\draw[gp path] (2.424,1.845)--(2.424,11.075);
\gpcolor{color=gp lt color border}
\gpsetlinewidth{1.00}
\draw[gp path] (2.424,1.845)--(2.424,2.025);
\draw[gp path] (2.424,11.075)--(2.424,10.895);
\node[gp node left,rotate=270] at (2.424,1.661) {$0$};
\gpcolor{rgb color={0.753,0.753,0.753}}
\gpsetlinewidth{2.00}
\draw[gp path] (4.030,1.845)--(4.030,11.075);
\gpcolor{color=gp lt color border}
\gpsetlinewidth{1.00}
\draw[gp path] (4.030,1.845)--(4.030,2.025);
\draw[gp path] (4.030,11.075)--(4.030,10.895);
\node[gp node left,rotate=270] at (4.030,1.661) {$50$};
\gpcolor{rgb color={0.753,0.753,0.753}}
\gpsetlinewidth{2.00}
\draw[gp path] (5.636,1.845)--(5.636,11.075);
\gpcolor{color=gp lt color border}
\gpsetlinewidth{1.00}
\draw[gp path] (5.636,1.845)--(5.636,2.025);
\draw[gp path] (5.636,11.075)--(5.636,10.895);
\node[gp node left,rotate=270] at (5.636,1.661) {$100$};
\gpcolor{rgb color={0.753,0.753,0.753}}
\gpsetlinewidth{2.00}
\draw[gp path] (7.242,1.845)--(7.242,9.355);
\draw[gp path] (7.242,10.895)--(7.242,11.075);
\gpcolor{color=gp lt color border}
\gpsetlinewidth{1.00}
\draw[gp path] (7.242,1.845)--(7.242,2.025);
\draw[gp path] (7.242,11.075)--(7.242,10.895);
\node[gp node left,rotate=270] at (7.242,1.661) {$150$};
\gpcolor{rgb color={0.753,0.753,0.753}}
\gpsetlinewidth{2.00}
\draw[gp path] (8.849,1.845)--(8.849,9.355);
\draw[gp path] (8.849,10.895)--(8.849,11.075);
\gpcolor{color=gp lt color border}
\gpsetlinewidth{1.00}
\draw[gp path] (8.849,1.845)--(8.849,2.025);
\draw[gp path] (8.849,11.075)--(8.849,10.895);
\node[gp node left,rotate=270] at (8.849,1.661) {$200$};
\gpcolor{rgb color={0.753,0.753,0.753}}
\gpsetlinewidth{2.00}
\draw[gp path] (10.455,1.845)--(10.455,9.355);
\draw[gp path] (10.455,10.895)--(10.455,11.075);
\gpcolor{color=gp lt color border}
\gpsetlinewidth{1.00}
\draw[gp path] (10.455,1.845)--(10.455,2.025);
\draw[gp path] (10.455,11.075)--(10.455,10.895);
\node[gp node left,rotate=270] at (10.455,1.661) {$250$};
\gpcolor{rgb color={0.753,0.753,0.753}}
\gpsetlinewidth{2.00}
\draw[gp path] (12.061,1.845)--(12.061,9.355);
\draw[gp path] (12.061,10.895)--(12.061,11.075);
\gpcolor{color=gp lt color border}
\gpsetlinewidth{1.00}
\draw[gp path] (12.061,1.845)--(12.061,2.025);
\draw[gp path] (12.061,11.075)--(12.061,10.895);
\node[gp node left,rotate=270] at (12.061,1.661) {$300$};
\gpcolor{rgb color={0.753,0.753,0.753}}
\gpsetlinewidth{2.00}
\draw[gp path] (13.667,1.845)--(13.667,11.075);
\gpcolor{color=gp lt color border}
\gpsetlinewidth{1.00}
\draw[gp path] (13.667,1.845)--(13.667,2.025);
\draw[gp path] (13.667,11.075)--(13.667,10.895);
\node[gp node left,rotate=270] at (13.667,1.661) {$350$};
\draw[gp path] (13.667,1.845)--(13.487,1.845);
\node[gp node left] at (13.851,1.845) {$102000$};
\draw[gp path] (13.667,3.164)--(13.487,3.164);
\node[gp node left] at (13.851,3.164) {$103000$};
\draw[gp path] (13.667,4.482)--(13.487,4.482);
\node[gp node left] at (13.851,4.482) {$104000$};
\draw[gp path] (13.667,5.801)--(13.487,5.801);
\node[gp node left] at (13.851,5.801) {$105000$};
\draw[gp path] (13.667,7.119)--(13.487,7.119);
\node[gp node left] at (13.851,7.119) {$106000$};
\draw[gp path] (13.667,8.438)--(13.487,8.438);
\node[gp node left] at (13.851,8.438) {$107000$};
\draw[gp path] (13.667,9.756)--(13.487,9.756);
\node[gp node left] at (13.851,9.756) {$108000$};
\draw[gp path] (13.667,11.075)--(13.487,11.075);
\node[gp node left] at (13.851,11.075) {$109000$};
\draw[gp path] (2.424,11.075)--(2.424,1.845)--(13.667,1.845)--(13.667,11.075)--cycle;
\draw[gp path] (7.047,9.355)--(7.047,10.895)--(13.483,10.895)--(13.483,9.355)--cycle;
\node[gp node right] at (12.199,10.510) {Strategic urgency};
\gpcolor{rgb color={0.000,0.000,0.000}}
\gpsetlinewidth{2.50}
\draw[gp path] (12.383,10.510)--(13.299,10.510);
\draw[gp path] (2.424,2.222)--(2.424,2.236)--(2.424,2.405)--(2.424,2.338)--(2.424,2.361)%
  --(2.424,2.819)--(2.456,3.178)--(2.456,4.057)--(2.456,4.163)--(2.456,4.366)--(2.456,4.711)%
  --(2.456,4.877)--(2.456,4.986)--(2.456,5.164)--(2.456,5.190)--(2.456,5.376)--(2.456,5.421)%
  --(2.456,5.401)--(2.456,5.692)--(2.456,5.583)--(2.456,5.840)--(2.456,5.824)--(2.456,6.014)%
  --(2.456,6.473)--(2.456,6.903)--(2.456,7.279)--(2.456,7.695)--(2.456,7.539)--(2.456,7.569)%
  --(2.456,8.194)--(2.456,8.416)--(2.456,8.452)--(2.456,8.416)--(2.456,8.510)--(2.456,8.617)%
  --(2.456,8.559)--(2.456,8.355)--(2.456,8.410)--(2.456,8.292)--(2.456,8.382)--(2.456,8.591)%
  --(2.456,8.948)--(2.456,9.087)--(2.456,9.102)--(2.456,8.972)--(2.456,8.877)--(2.456,8.819)%
  --(2.456,8.801)--(2.456,8.728)--(2.456,8.793)--(2.456,8.857)--(2.456,8.971)--(2.456,9.176)%
  --(2.456,9.500)--(2.456,9.614)--(2.456,9.511)--(2.456,9.634)--(2.456,9.595)--(2.456,9.666)%
  --(2.456,9.537)--(2.456,9.606)--(2.456,9.600)--(2.456,9.475)--(2.456,9.544)--(2.456,9.480)%
  --(2.456,9.498)--(2.456,9.477)--(2.456,9.396)--(2.456,9.444)--(2.456,9.451)--(2.456,9.527)%
  --(2.488,9.493)--(2.488,9.325)--(2.488,9.370)--(2.488,9.276)--(2.488,9.016)--(2.488,9.091)%
  --(2.488,9.062)--(2.488,9.143)--(2.488,9.157)--(2.488,9.105)--(2.488,9.003)--(2.488,8.978)%
  --(2.488,9.211)--(2.488,9.190)--(2.488,9.145)--(2.488,9.114)--(2.488,9.092)--(2.488,9.204)%
  --(2.488,9.657)--(2.488,9.543)--(2.488,9.528)--(2.488,9.566)--(2.488,9.680)--(2.488,9.636)%
  --(2.488,9.587)--(2.488,9.583)--(2.488,9.483)--(2.488,9.446)--(2.488,9.495)--(2.520,9.375)%
  --(2.520,9.261)--(2.520,9.174)--(2.520,9.403)--(2.520,9.323)--(2.520,9.283)--(2.520,9.212)%
  --(2.520,9.073)--(2.520,9.031)--(2.520,9.026)--(2.520,9.029)--(2.520,8.932)--(2.520,8.860)%
  --(2.520,8.820)--(2.520,8.769)--(2.520,8.789)--(2.520,8.774)--(2.520,8.905)--(2.520,8.854)%
  --(2.520,8.857)--(2.520,8.853)--(2.520,8.964)--(2.520,9.105)--(2.520,9.018)--(2.520,8.863)%
  --(2.520,8.862)--(2.520,8.851)--(2.520,8.848)--(2.520,8.843)--(2.552,8.774)--(2.552,8.723)%
  --(2.552,8.701)--(2.552,8.628)--(2.552,8.584)--(2.552,8.585)--(2.552,8.568)--(2.552,8.434)%
  --(2.552,8.423)--(2.552,8.334)--(2.552,8.204)--(2.552,8.172)--(2.552,8.134)--(2.552,8.121)%
  --(2.552,8.120)--(2.552,8.091)--(2.585,7.969)--(2.585,7.949)--(2.585,7.945)--(2.585,7.933)%
  --(2.585,7.930)--(2.585,7.912)--(2.585,7.884)--(2.585,7.805)--(2.585,7.755)--(2.585,7.745)%
  --(2.585,7.744)--(2.585,7.732)--(2.585,7.727)--(2.585,7.664)--(2.585,7.625)--(2.585,7.620)%
  --(2.585,7.594)--(2.585,7.571)--(2.585,7.559)--(2.617,7.528)--(2.617,7.515)--(2.617,7.492)%
  --(2.617,7.483)--(2.617,7.453)--(2.617,7.346)--(2.617,7.319)--(2.617,7.305)--(2.617,7.295)%
  --(2.617,7.293)--(2.617,7.288)--(2.649,7.273)--(2.649,7.122)--(2.649,7.098)--(2.649,7.091)%
  --(2.681,7.062)--(2.681,7.060)--(2.681,7.005)--(2.681,6.999)--(2.681,6.962)--(2.681,6.928)%
  --(2.681,6.908)--(2.713,6.904)--(2.713,6.894)--(2.713,6.853)--(2.713,6.815)--(2.713,6.787)%
  --(2.713,6.780)--(2.745,6.766)--(2.745,6.733)--(2.745,6.726)--(2.745,6.706)--(2.777,6.658)%
  --(2.777,6.631)--(2.809,6.631)--(2.809,6.613)--(2.809,6.595)--(2.809,6.573)--(2.809,6.534)%
  --(2.809,6.499)--(2.809,6.491)--(2.809,6.468)--(2.809,6.495)--(2.809,6.484)--(2.809,6.475)%
  --(2.809,6.464)--(2.809,6.448)--(2.842,6.417)--(2.842,6.363)--(2.842,6.357)--(2.874,6.357)%
  --(2.874,6.343)--(2.874,6.329)--(2.874,6.358)--(2.874,6.320)--(2.906,6.313)--(2.906,6.291)%
  --(2.938,6.268)--(2.938,6.176)--(2.938,6.172)--(2.938,6.140)--(2.938,6.138)--(2.970,6.110)%
  --(2.970,6.067)--(3.034,6.029)--(3.066,5.985)--(3.066,5.968)--(3.099,5.934)--(3.099,5.888)%
  --(3.163,5.858)--(3.195,5.848)--(3.195,5.826)--(3.195,5.763)--(3.195,5.749)--(3.195,5.746)%
  --(3.195,5.734)--(3.227,5.728)--(3.259,5.717)--(3.259,5.708)--(3.259,5.686)--(3.259,5.682)%
  --(3.291,5.667)--(3.291,5.645)--(3.356,5.635)--(3.388,5.606)--(3.420,5.606)--(3.420,5.603)%
  --(3.484,5.581)--(3.516,5.561)--(3.516,5.532)--(3.516,5.530)--(3.548,5.507)--(3.548,5.479)%
  --(3.645,5.473)--(3.709,5.469)--(3.709,5.468)--(3.741,5.455)--(3.773,5.449)--(3.837,5.445)%
  --(3.902,5.435)--(3.902,4.747)--(3.902,4.789)--(3.902,4.928)--(3.902,5.022)--(3.902,5.178)%
  --(3.902,5.113)--(3.902,5.257)--(3.902,5.440)--(3.902,5.603)--(3.902,5.950)--(3.902,6.056)%
  --(3.902,6.514)--(3.902,6.540)--(3.902,6.579)--(3.902,6.568)--(3.902,6.485)--(3.902,6.472)%
  --(3.934,6.576)--(3.934,6.567)--(3.934,6.518)--(3.934,6.523)--(3.934,6.628)--(3.934,6.646)%
  --(3.934,6.546)--(3.934,6.806)--(3.934,6.846)--(3.934,6.820)--(3.934,6.903)--(3.934,6.950)%
  --(3.934,6.875)--(3.934,6.716)--(3.934,6.707)--(3.934,6.678)--(3.934,6.643)--(3.966,6.680)%
  --(3.966,6.741)--(3.966,6.736)--(3.966,6.742)--(3.966,6.717)--(3.966,6.683)--(3.966,6.795)%
  --(3.966,6.808)--(3.966,6.744)--(3.966,6.776)--(3.998,6.745)--(3.998,6.717)--(3.998,6.716)%
  --(3.998,6.803)--(3.998,6.772)--(3.998,6.839)--(3.998,6.818)--(4.030,6.788)--(4.030,6.774)%
  --(4.030,6.766)--(4.030,6.723)--(4.062,6.718)--(4.062,6.704)--(4.062,6.661)--(4.062,6.686)%
  --(4.062,6.642)--(4.062,6.602)--(4.062,6.529)--(4.094,6.508)--(4.094,6.477)--(4.094,6.467)%
  --(4.159,6.466)--(4.159,6.450)--(4.191,6.435)--(4.191,6.390)--(4.191,6.360)--(4.191,6.329)%
  --(4.223,6.328)--(4.223,6.304)--(4.223,6.291)--(4.223,6.278)--(4.255,6.263)--(4.287,6.257)%
  --(4.287,6.250)--(4.319,6.239)--(4.351,6.198)--(4.351,6.182)--(4.383,6.155)--(4.383,6.131)%
  --(4.416,6.109)--(4.416,6.098)--(4.480,6.093)--(4.480,6.064)--(4.480,6.089)--(4.480,6.081)%
  --(4.512,6.075)--(4.512,6.064)--(4.544,6.044)--(4.576,6.018)--(4.576,6.014)--(4.576,5.984)%
  --(4.608,5.977)--(4.640,5.918)--(4.640,5.907)--(4.640,5.903)--(4.640,5.874)--(4.640,5.852)%
  --(4.640,5.842)--(4.673,5.826)--(4.769,5.789)--(4.769,5.745)--(4.801,5.744)--(4.801,5.713)%
  --(4.865,5.712)--(4.865,5.711)--(4.897,5.710)--(4.897,5.709)--(4.897,5.704)--(4.897,5.681)%
  --(4.930,5.668)--(4.930,5.667)--(4.962,5.656)--(4.962,5.624)--(4.962,5.599)--(4.962,5.533)%
  --(5.058,5.506)--(5.058,5.505)--(5.122,5.496)--(5.219,5.492)--(5.315,5.447)--(5.636,5.447)%
  --(5.668,5.432)--(5.765,5.394)--(5.797,5.365)--(5.829,5.316)--(5.829,5.281)--(5.829,5.210)%
  --(5.829,5.291)--(5.829,5.280)--(5.829,5.265)--(5.829,5.548)--(5.829,5.588)--(5.829,5.699)%
  --(5.829,5.819)--(5.829,5.742)--(5.829,5.906)--(5.829,5.892)--(5.829,5.994)--(5.829,5.943)%
  --(5.829,5.932)--(5.861,6.049)--(5.861,6.010)--(5.861,6.021)--(5.861,6.018)--(5.861,5.986)%
  --(5.861,6.064)--(5.861,6.044)--(5.861,6.309)--(5.861,6.499)--(5.861,6.533)--(5.861,6.550)%
  --(5.861,6.662)--(5.861,6.614)--(5.861,6.592)--(5.861,6.547)--(5.861,6.593)--(5.861,6.571)%
  --(5.861,6.564)--(5.861,6.539)--(5.861,6.521)--(5.893,6.511)--(5.893,6.507)--(5.893,6.477)%
  --(5.893,6.462)--(5.893,6.631)--(5.893,6.528)--(5.893,6.504)--(5.893,6.485)--(5.893,6.484)%
  --(5.893,6.519)--(5.893,6.518)--(5.893,6.509)--(5.893,6.446)--(5.893,6.405)--(5.893,6.404)%
  --(5.893,6.388)--(5.925,6.291)--(5.958,6.270)--(5.958,6.183)--(5.958,6.164)--(5.958,6.270)%
  --(5.990,6.226)--(5.990,6.207)--(5.990,6.161)--(5.990,6.157)--(5.990,6.136)--(6.022,6.077)%
  --(6.022,6.056)--(6.022,6.053)--(6.022,6.017)--(6.022,5.920)--(6.054,5.946)--(6.054,5.925)%
  --(6.054,5.888)--(6.054,5.861)--(6.054,5.931)--(6.054,5.888)--(6.054,5.856)--(6.086,5.846)%
  --(6.086,5.805)--(6.086,5.766)--(6.086,5.743)--(6.086,5.709)--(6.118,5.698)--(6.118,5.690)%
  --(6.118,5.667)--(6.118,5.646)--(6.150,5.645)--(6.182,5.601)--(6.182,5.582)--(6.182,5.567)%
  --(6.214,5.565)--(6.214,5.563)--(6.214,5.552)--(6.214,5.519)--(6.214,5.516)--(6.247,5.515)%
  --(6.247,5.509)--(6.247,5.491)--(6.279,5.485)--(6.311,5.472)--(6.311,5.445)--(6.343,5.423)%
  --(6.343,5.421)--(6.343,5.410)--(6.375,5.397)--(6.375,5.334)--(6.375,5.330)--(6.375,5.326)%
  --(6.375,5.271)--(6.407,5.259)--(6.407,5.257)--(6.407,5.252)--(6.407,5.239)--(6.439,5.213)%
  --(6.439,5.207)--(6.439,5.162)--(6.439,5.183)--(6.471,5.145)--(6.471,5.133)--(6.504,5.118)%
  --(6.504,5.117)--(6.536,5.109)--(6.536,5.106)--(6.664,5.106)--(6.664,5.104)--(6.728,5.102)%
  --(6.793,5.096)--(6.825,5.093)--(6.825,5.090)--(6.889,5.075)--(6.921,5.071)--(6.921,5.053)%
  --(6.985,5.053)--(7.018,5.026)--(7.050,5.019)--(7.050,5.018)--(7.146,5.012)--(7.178,4.989)%
  --(7.210,4.953)--(7.275,4.936)--(7.371,4.924)--(7.371,4.921)--(7.403,4.914)--(7.403,4.911)%
  --(7.403,4.902)--(7.564,4.891)--(7.564,4.865)--(7.596,4.861)--(7.596,4.858)--(7.660,4.836)%
  --(7.660,4.831)--(7.692,4.829)--(7.724,4.826)--(7.756,5.302)--(7.756,5.443)--(7.756,5.694)%
  --(7.756,5.838)--(7.756,5.774)--(7.756,5.689)--(7.756,5.656)--(7.756,5.782)--(7.756,5.779)%
  --(7.756,5.918)--(7.756,6.019)--(7.756,5.982)--(7.789,5.923)--(7.789,5.880)--(7.789,6.043)%
  --(7.789,6.033)--(7.789,6.022)--(7.789,6.178)--(7.789,6.147)--(7.789,6.382)--(7.789,6.468)%
  --(7.789,6.674)--(7.789,6.591)--(7.789,6.615)--(7.789,6.691)--(7.789,6.715)--(7.789,6.700)%
  --(7.789,6.668)--(7.789,6.643)--(7.789,6.867)--(7.789,6.816)--(7.789,6.787)--(7.789,6.792)%
  --(7.789,6.813)--(7.789,6.786)--(7.789,6.836)--(7.789,6.840)--(7.789,6.789)--(7.821,6.785)%
  --(7.821,6.706)--(7.821,6.688)--(7.821,6.667)--(7.821,6.648)--(7.821,6.619)--(7.853,6.588)%
  --(7.853,6.562)--(7.853,6.618)--(7.853,6.563)--(7.853,6.519)--(7.853,6.497)--(7.853,6.467)%
  --(7.853,6.407)--(7.853,6.398)--(7.853,6.377)--(7.853,6.340)--(7.885,6.329)--(7.885,6.320)%
  --(7.885,6.278)--(7.885,6.191)--(7.917,6.181)--(7.917,6.153)--(7.949,6.126)--(7.949,6.093)%
  --(7.949,6.057)--(7.949,6.033)--(7.981,6.033)--(7.981,5.996)--(7.981,5.991)--(8.013,5.982)%
  --(8.013,5.908)--(8.013,5.902)--(8.013,5.886)--(8.013,5.861)--(8.046,5.833)--(8.046,5.813)%
  --(8.046,5.771)--(8.078,5.736)--(8.078,5.682)--(8.078,5.666)--(8.078,5.653)--(8.078,5.617)%
  --(8.078,5.613)--(8.078,5.583)--(8.078,5.568)--(8.110,5.548)--(8.110,5.517)--(8.110,5.494)%
  --(8.110,5.490)--(8.110,5.465)--(8.110,5.454)--(8.142,5.446)--(8.142,5.434)--(8.174,5.434)%
  --(8.174,5.433)--(8.174,5.422)--(8.206,5.401)--(8.206,5.379)--(8.206,5.374)--(8.238,5.373)%
  --(8.238,5.353)--(8.238,5.351)--(8.238,5.338)--(8.238,5.320)--(8.270,5.319)--(8.270,5.296)%
  --(8.270,5.282)--(8.302,5.252)--(8.335,5.251)--(8.335,5.221)--(8.335,5.205)--(8.367,5.201)%
  --(8.399,5.164)--(8.399,5.157)--(8.399,5.156)--(8.463,5.135)--(8.495,5.073)--(8.527,5.068)%
  --(8.559,5.063)--(8.559,5.059)--(8.559,5.041)--(8.559,5.009)--(8.592,5.004)--(8.592,4.997)%
  --(8.624,4.990)--(8.656,4.965)--(8.656,4.964)--(8.656,4.957)--(8.688,4.936)--(8.720,4.935)%
  --(8.784,4.905)--(8.816,4.903)--(8.816,4.891)--(8.849,4.872)--(8.881,4.864)--(8.913,4.847)%
  --(8.945,4.841)--(8.977,4.809)--(9.041,4.808)--(9.041,4.795)--(9.041,4.789)--(9.073,4.782)%
  --(9.073,4.765)--(9.106,4.731)--(9.106,4.720)--(9.138,4.712)--(9.298,4.699)--(9.298,4.672)%
  --(9.363,4.663)--(9.395,4.652)--(9.587,4.645)--(9.620,4.643)--(9.620,4.634)--(9.652,4.623)%
  --(9.684,4.678)--(9.684,4.725)--(9.684,4.774)--(9.684,4.670)--(9.684,4.798)--(9.684,4.746)%
  --(9.684,5.288)--(9.684,5.351)--(9.684,5.615)--(9.684,5.871)--(9.684,5.835)--(9.684,6.026)%
  --(9.684,5.972)--(9.716,6.140)--(9.716,6.107)--(9.716,6.078)--(9.716,6.406)--(9.716,6.455)%
  --(9.716,6.499)--(9.716,6.849)--(9.716,6.844)--(9.716,6.829)--(9.716,6.785)--(9.716,6.736)%
  --(9.716,6.634)--(9.716,6.628)--(9.716,6.640)--(9.716,7.202)--(9.716,7.141)--(9.716,7.106)%
  --(9.716,7.113)--(9.716,7.216)--(9.716,7.209)--(9.716,7.302)--(9.748,7.277)--(9.748,7.545)%
  --(9.748,7.502)--(9.748,7.458)--(9.748,7.455)--(9.748,7.487)--(9.748,7.305)--(9.748,7.253)%
  --(9.748,7.218)--(9.748,7.210)--(9.748,7.129)--(9.748,7.257)--(9.780,7.254)--(9.780,7.250)%
  --(9.780,7.191)--(9.780,7.180)--(9.780,7.141)--(9.780,7.087)--(9.780,7.079)--(9.780,7.102)%
  --(9.780,7.087)--(9.780,7.041)--(9.780,7.027)--(9.780,6.958)--(9.780,6.885)--(9.780,6.856)%
  --(9.780,6.845)--(9.812,6.838)--(9.812,6.812)--(9.812,6.713)--(9.812,6.690)--(9.812,6.688)%
  --(9.812,6.682)--(9.812,6.677)--(9.812,6.603)--(9.812,6.576)--(9.812,6.542)--(9.812,6.536)%
  --(9.844,6.523)--(9.844,6.506)--(9.844,6.502)--(9.844,6.489)--(9.844,6.441)--(9.844,6.385)%
  --(9.844,6.377)--(9.844,6.374)--(9.844,6.349)--(9.844,6.324)--(9.844,6.242)--(9.844,6.227)%
  --(9.844,6.194)--(9.844,6.180)--(9.877,6.146)--(9.877,6.105)--(9.877,6.096)--(9.877,6.080)%
  --(9.909,6.069)--(9.909,6.040)--(9.909,6.013)--(9.909,6.004)--(9.909,6.003)--(9.941,5.938)%
  --(9.941,5.911)--(9.941,5.886)--(9.973,5.882)--(9.973,5.868)--(10.005,5.868)--(10.005,5.861)%
  --(10.005,5.851)--(10.005,5.844)--(10.005,5.831)--(10.037,5.829)--(10.037,5.803)--(10.037,5.773)%
  --(10.037,5.711)--(10.037,5.707)--(10.037,5.695)--(10.037,5.689)--(10.101,5.664)--(10.101,5.650)%
  --(10.133,5.630)--(10.133,5.614)--(10.166,5.591)--(10.166,5.583)--(10.230,5.556)--(10.230,5.503)%
  --(10.230,5.484)--(10.262,5.461)--(10.262,5.459)--(10.262,5.457)--(10.262,5.429)--(10.262,5.411)%
  --(10.326,5.403)--(10.326,5.358)--(10.358,5.357)--(10.390,5.350)--(10.390,5.345)--(10.455,5.335)%
  --(10.487,5.329)--(10.551,5.327)--(10.551,5.312)--(10.551,5.306)--(10.583,5.302)--(10.583,5.287)%
  --(10.615,5.270)--(10.647,5.247)--(10.680,5.242)--(10.712,5.232)--(10.744,5.229)--(10.776,5.228)%
  --(10.872,5.203)--(10.904,5.180)--(10.937,5.175)--(10.969,5.133)--(10.969,5.125)--(11.001,5.119)%
  --(11.001,5.144)--(11.001,5.137)--(11.065,5.134)--(11.065,5.126)--(11.097,5.117)--(11.097,5.105)%
  --(11.161,5.104)--(11.194,5.091)--(11.226,5.085)--(11.226,5.064)--(11.258,5.061)--(11.290,5.057)%
  --(11.290,5.044)--(11.290,5.030)--(11.290,5.019)--(11.290,5.015)--(11.322,5.011)--(11.354,4.990)%
  --(11.386,4.988)--(11.418,4.980)--(11.451,4.969)--(11.515,4.951)--(11.579,4.933)--(11.611,4.863)%
  --(11.611,4.860)--(11.611,4.786)--(11.611,4.977)--(11.611,4.988)--(11.611,5.229)--(11.611,5.188)%
  --(11.611,5.332)--(11.643,5.464)--(11.643,5.608)--(11.643,5.639)--(11.643,6.055)--(11.643,6.021)%
  --(11.643,6.056)--(11.643,5.898)--(11.643,5.852)--(11.643,5.815)--(11.643,6.039)--(11.643,5.945)%
  --(11.643,6.026)--(11.643,5.991)--(11.643,6.071)--(11.643,6.132)--(11.643,6.348)--(11.643,6.271)%
  --(11.643,6.266)--(11.643,6.237)--(11.675,6.528)--(11.675,6.649)--(11.675,6.635)--(11.675,6.617)%
  --(11.675,6.523)--(11.675,6.657)--(11.675,6.652)--(11.675,6.651)--(11.675,6.728)--(11.675,6.631)%
  --(11.675,6.523)--(11.675,6.515)--(11.675,6.474)--(11.708,6.493)--(11.708,6.410)--(11.708,6.831)%
  --(11.708,6.760)--(11.708,6.770)--(11.708,6.756)--(11.708,6.742)--(11.708,6.713)--(11.708,6.699)%
  --(11.740,6.693)--(11.740,6.659)--(11.868,6.652)--(11.900,6.652)--(11.900,6.568)--(11.900,6.522)%
  --(11.932,6.510)--(11.932,6.494)--(11.932,6.435)--(11.964,6.411)--(11.964,6.363)--(11.964,6.340)%
  --(11.964,6.325)--(11.964,6.286)--(11.997,6.236)--(11.997,6.192)--(11.997,6.183)--(12.029,6.140)%
  --(12.061,6.123)--(12.093,6.073)--(12.093,6.017)--(12.125,5.996)--(12.125,5.884)--(12.157,5.851)%
  --(12.157,5.832)--(12.189,5.829)--(12.221,5.786)--(12.221,5.759)--(12.221,5.750)--(12.254,5.749)%
  --(12.286,5.721)--(12.286,5.704)--(12.318,5.697)--(12.318,5.675)--(12.318,5.641)--(12.350,5.633)%
  --(12.350,5.609)--(12.446,5.593)--(12.446,5.579)--(12.446,5.509)--(12.446,5.507)--(12.446,5.485)%
  --(12.446,5.473)--(12.478,5.470)--(12.511,5.456)--(12.511,5.442)--(12.543,5.425)--(12.575,5.416)%
  --(12.607,5.416)--(12.607,5.399)--(12.607,5.388)--(12.671,5.359)--(12.703,5.316)--(12.703,5.303)%
  --(12.703,5.298)--(12.703,5.293)--(12.703,5.281)--(12.703,5.253)--(12.735,5.245)--(12.735,5.238)%
  --(12.735,5.222)--(12.768,5.199)--(12.832,5.194)--(12.832,5.188)--(12.864,5.183)--(12.864,5.181)%
  --(12.928,5.177)--(12.960,5.162)--(13.025,5.160)--(13.025,5.151)--(13.025,5.148)--(13.025,5.141)%
  --(13.057,5.131)--(13.089,5.118)--(13.089,5.081)--(13.089,5.058)--(13.089,5.033)--(13.089,5.032)%
  --(13.089,5.018)--(13.121,5.014)--(13.121,5.004)--(13.153,4.991)--(13.153,4.956)--(13.153,4.942)%
  --(13.217,4.937)--(13.249,4.920)--(13.249,4.915)--(13.282,4.909)--(13.346,4.902)--(13.410,4.897)%
  --(13.474,4.856)--(13.506,4.856)--(13.506,4.814)--(13.506,4.807)--(13.506,4.793)--(13.506,4.775);
\gpcolor{color=gp lt color border}
\node[gp node right] at (12.199,9.740) {Strategic Resource penalty};
\gpcolor{rgb color={0.184,0.243,0.918}}
\draw[gp path] (12.383,9.740)--(13.299,9.740);
\draw[gp path] (2.424,10.425)--(2.424,10.290)--(2.424,10.132)--(2.424,9.933)--(2.424,9.710)%
  --(2.424,9.436)--(2.456,9.309)--(2.456,9.150)--(2.456,8.935)--(2.456,8.692)--(2.456,8.522)%
  --(2.456,8.356)--(2.456,8.232)--(2.456,7.962)--(2.456,7.818)--(2.456,7.693)--(2.456,7.681)%
  --(2.456,7.630)--(2.456,7.424)--(2.456,7.391)--(2.456,7.235)--(2.456,7.144)--(2.456,7.076)%
  --(2.456,7.049)--(2.456,6.798)--(2.456,6.629)--(2.456,6.360)--(2.456,6.341)--(2.456,6.261)%
  --(2.456,6.183)--(2.456,6.103)--(2.456,6.046)--(2.456,5.846)--(2.456,5.776)--(2.456,5.681)%
  --(2.456,5.650)--(2.456,5.586)--(2.456,5.501)--(2.456,5.491)--(2.456,5.410)--(2.456,5.384)%
  --(2.456,5.083)--(2.456,5.068)--(2.456,5.008)--(2.456,4.905)--(2.456,4.900)--(2.456,4.858)%
  --(2.456,4.832)--(2.456,4.766)--(2.456,4.722)--(2.456,4.669)--(2.456,4.652)--(2.456,4.549)%
  --(2.456,4.538)--(2.456,4.495)--(2.456,4.365)--(2.456,4.349)--(2.456,4.336)--(2.456,4.308)%
  --(2.456,4.236)--(2.456,4.143)--(2.456,4.137)--(2.456,4.071)--(2.456,3.942)--(2.456,3.920)%
  --(2.456,3.913)--(2.456,3.894)--(2.488,3.889)--(2.488,3.843)--(2.488,3.828)--(2.488,3.783)%
  --(2.488,3.670)--(2.488,3.659)--(2.488,3.617)--(2.488,3.615)--(2.488,3.608)--(2.488,3.582)%
  --(2.488,3.568)--(2.488,3.566)--(2.488,3.543)--(2.488,3.504)--(2.488,3.481)--(2.488,3.460)%
  --(2.488,3.444)--(2.488,3.437)--(2.488,3.429)--(2.488,3.418)--(2.488,3.394)--(2.520,3.392)%
  --(2.520,3.368)--(2.520,3.347)--(2.520,3.129)--(2.520,3.123)--(2.520,2.942)--(2.520,2.934)%
  --(2.520,2.928)--(2.520,2.921)--(2.520,2.859)--(2.520,2.851)--(2.520,2.834)--(2.552,2.834)%
  --(2.552,2.835)--(2.552,2.830)--(2.552,2.640)--(2.585,2.640)--(2.585,2.628)--(2.617,2.628)%
  --(2.617,2.438)--(2.649,2.438)--(2.681,2.438)--(2.713,2.438)--(2.745,2.438)--(2.777,2.438)%
  --(2.809,2.438)--(2.809,2.437)--(2.842,2.437)--(2.842,2.438)--(2.874,2.438)--(2.874,2.437)%
  --(2.906,2.437)--(2.938,2.437)--(2.970,2.437)--(3.034,2.437)--(3.066,2.437)--(3.099,2.437)%
  --(3.163,2.437)--(3.195,2.437)--(3.227,2.437)--(3.259,2.437)--(3.291,2.437)--(3.356,2.437)%
  --(3.388,2.437)--(3.420,2.437)--(3.484,2.437)--(3.516,2.437)--(3.548,2.437)--(3.645,2.437)%
  --(3.709,2.437)--(3.741,2.437)--(3.773,2.437)--(3.837,2.437)--(3.902,2.437)--(3.902,2.692)%
  --(3.902,2.670)--(3.902,2.640)--(3.902,2.603)--(3.902,2.564)--(3.902,2.545)--(3.902,2.510)%
  --(3.902,2.495)--(3.902,2.430)--(3.902,2.362)--(3.902,2.242)--(3.902,2.219)--(3.902,2.132)%
  --(3.902,2.101)--(3.934,2.093)--(3.934,2.082)--(3.934,2.084)--(3.934,2.076)--(3.934,2.070)%
  --(3.934,2.069)--(3.934,2.048)--(3.934,2.040)--(3.934,2.034)--(3.934,1.968)--(3.966,1.962)%
  --(3.966,1.960)--(3.966,1.954)--(3.966,1.949)--(3.966,1.944)--(3.966,1.935)--(3.998,1.935)%
  --(3.998,1.914)--(3.998,1.904)--(4.030,1.904)--(4.062,1.904)--(4.062,1.899)--(4.094,1.899)%
  --(4.094,1.898)--(4.159,1.898)--(4.191,1.898)--(4.223,1.898)--(4.255,1.898)--(4.287,1.898)%
  --(4.319,1.898)--(4.351,1.899)--(4.383,1.899)--(4.416,1.899)--(4.480,1.899)--(4.480,1.898)%
  --(4.512,1.898)--(4.544,1.898)--(4.576,1.898)--(4.608,1.898)--(4.640,1.898)--(4.673,1.898)%
  --(4.769,1.898)--(4.801,1.898)--(4.865,1.898)--(4.897,1.898)--(4.930,1.898)--(4.962,1.898)%
  --(5.058,1.898)--(5.122,1.898)--(5.219,1.898)--(5.315,1.898)--(5.636,1.898)--(5.668,1.898)%
  --(5.765,1.898)--(5.797,1.898)--(5.829,1.898)--(5.829,2.705)--(5.829,2.697)--(5.829,2.634)%
  --(5.829,2.618)--(5.829,2.558)--(5.829,2.556)--(5.829,2.543)--(5.829,2.536)--(5.829,2.516)%
  --(5.829,2.495)--(5.861,2.487)--(5.861,2.441)--(5.861,2.437)--(5.861,2.383)--(5.861,2.362)%
  --(5.861,2.342)--(5.861,2.318)--(5.861,2.304)--(5.861,2.293)--(5.861,2.292)--(5.861,2.280)%
  --(5.893,2.280)--(5.893,2.272)--(5.893,2.262)--(5.893,2.255)--(5.925,2.255)--(5.958,2.255)%
  --(5.958,2.245)--(5.990,2.245)--(6.022,2.245)--(6.054,2.239)--(6.054,2.226)--(6.086,2.226)%
  --(6.086,2.223)--(6.118,2.223)--(6.150,2.223)--(6.182,2.223)--(6.214,2.223)--(6.247,2.223)%
  --(6.279,2.223)--(6.311,2.223)--(6.343,2.223)--(6.375,2.223)--(6.375,2.225)--(6.407,2.225)%
  --(6.439,2.225)--(6.439,2.223)--(6.471,2.223)--(6.504,2.223)--(6.536,2.223)--(6.664,2.223)%
  --(6.728,2.223)--(6.793,2.223)--(6.825,2.223)--(6.889,2.223)--(6.921,2.223)--(6.985,2.223)%
  --(7.018,2.223)--(7.050,2.223)--(7.146,2.223)--(7.178,2.223)--(7.210,2.223)--(7.275,2.223)%
  --(7.371,2.223)--(7.403,2.223)--(7.564,2.223)--(7.596,2.223)--(7.660,2.223)--(7.692,2.223)%
  --(7.724,2.223)--(7.756,2.870)--(7.756,2.862)--(7.756,2.812)--(7.756,2.779)--(7.756,2.760)%
  --(7.756,2.751)--(7.756,2.742)--(7.756,2.694)--(7.756,2.643)--(7.756,2.632)--(7.789,2.632)%
  --(7.789,2.630)--(7.789,2.594)--(7.789,2.582)--(7.789,2.521)--(7.789,2.458)--(7.789,2.450)%
  --(7.789,2.433)--(7.789,2.430)--(7.789,2.404)--(7.789,2.397)--(7.789,2.376)--(7.789,2.358)%
  --(7.789,2.350)--(7.789,2.349)--(7.789,2.346)--(7.789,2.343)--(7.821,2.343)--(7.853,2.343)%
  --(7.853,2.338)--(7.885,2.338)--(7.917,2.338)--(7.949,2.338)--(7.981,2.338)--(8.013,2.338)%
  --(8.046,2.338)--(8.078,2.338)--(8.110,2.338)--(8.142,2.338)--(8.174,2.338)--(8.206,2.338)%
  --(8.238,2.338)--(8.270,2.338)--(8.302,2.338)--(8.335,2.338)--(8.367,2.338)--(8.399,2.338)%
  --(8.463,2.338)--(8.495,2.338)--(8.527,2.338)--(8.559,2.338)--(8.592,2.338)--(8.624,2.338)%
  --(8.656,2.338)--(8.688,2.338)--(8.720,2.338)--(8.784,2.338)--(8.816,2.338)--(8.849,2.338)%
  --(8.881,2.338)--(8.913,2.338)--(8.945,2.338)--(8.977,2.338)--(9.041,2.338)--(9.073,2.338)%
  --(9.106,2.338)--(9.138,2.338)--(9.298,2.338)--(9.363,2.338)--(9.395,2.338)--(9.587,2.338)%
  --(9.620,2.338)--(9.652,2.338)--(9.684,2.783)--(9.684,2.739)--(9.684,2.724)--(9.684,2.690)%
  --(9.684,2.685)--(9.684,2.517)--(9.684,2.503)--(9.684,2.454)--(9.684,2.371)--(9.684,2.362)%
  --(9.684,2.342)--(9.684,2.337)--(9.716,2.321)--(9.716,2.299)--(9.716,2.238)--(9.716,2.226)%
  --(9.716,2.221)--(9.716,2.181)--(9.716,2.173)--(9.716,2.152)--(9.716,2.146)--(9.716,2.092)%
  --(9.716,2.052)--(9.716,2.030)--(9.716,2.022)--(9.716,2.006)--(9.716,1.998)--(9.748,1.998)%
  --(9.748,1.976)--(9.748,1.973)--(9.748,1.970)--(9.748,1.964)--(9.780,1.964)--(9.780,1.944)%
  --(9.780,1.941)--(9.812,1.941)--(9.844,1.941)--(9.877,1.941)--(9.909,1.941)--(9.941,1.941)%
  --(9.973,1.941)--(10.005,1.941)--(10.037,1.941)--(10.037,1.943)--(10.101,1.943)--(10.133,1.943)%
  --(10.166,1.943)--(10.230,1.943)--(10.262,1.943)--(10.326,1.943)--(10.358,1.943)--(10.390,1.943)%
  --(10.455,1.943)--(10.487,1.943)--(10.551,1.943)--(10.583,1.943)--(10.615,1.943)--(10.647,1.943)%
  --(10.680,1.943)--(10.712,1.943)--(10.744,1.943)--(10.776,1.943)--(10.872,1.943)--(10.904,1.943)%
  --(10.937,1.943)--(10.969,1.943)--(11.001,1.943)--(11.001,1.941)--(11.065,1.941)--(11.097,1.941)%
  --(11.161,1.941)--(11.194,1.941)--(11.226,1.941)--(11.258,1.941)--(11.290,1.941)--(11.322,1.941)%
  --(11.354,1.941)--(11.386,1.941)--(11.418,1.941)--(11.451,1.941)--(11.515,1.941)--(11.579,1.941)%
  --(11.611,1.941)--(11.611,2.721)--(11.611,2.705)--(11.611,2.697)--(11.611,2.492)--(11.611,2.471)%
  --(11.611,2.444)--(11.643,2.436)--(11.643,2.401)--(11.643,2.394)--(11.643,2.206)--(11.643,2.198)%
  --(11.643,2.161)--(11.643,2.154)--(11.643,2.131)--(11.643,2.103)--(11.643,2.092)--(11.643,2.063)%
  --(11.643,2.060)--(11.675,2.040)--(11.675,2.009)--(11.675,1.999)--(11.675,1.993)--(11.708,1.989)%
  --(11.708,1.960)--(11.708,1.949)--(11.708,1.947)--(11.740,1.947)--(11.868,1.947)--(11.900,1.947)%
  --(11.932,1.947)--(11.964,1.947)--(11.997,1.947)--(12.029,1.947)--(12.061,1.947)--(12.093,1.947)%
  --(12.125,1.947)--(12.157,1.947)--(12.189,1.947)--(12.221,1.947)--(12.254,1.947)--(12.286,1.947)%
  --(12.318,1.947)--(12.350,1.947)--(12.446,1.947)--(12.478,1.947)--(12.511,1.947)--(12.543,1.947)%
  --(12.575,1.947)--(12.607,1.947)--(12.607,1.945)--(12.671,1.945)--(12.703,1.945)--(12.735,1.945)%
  --(12.768,1.945)--(12.832,1.945)--(12.864,1.945)--(12.928,1.945)--(12.960,1.945)--(13.025,1.945)%
  --(13.057,1.945)--(13.089,1.945)--(13.121,1.945)--(13.153,1.945)--(13.217,1.945)--(13.249,1.945)%
  --(13.282,1.945)--(13.346,1.945)--(13.410,1.945)--(13.474,1.945)--(13.506,1.945);
\gpcolor{color=gp lt color border}
\gpsetlinewidth{1.00}
\draw[gp path] (2.424,11.075)--(2.424,1.845)--(13.667,1.845)--(13.667,11.075)--cycle;
\node[gp node center,rotate=-270] at (-0.812,6.460) {Objective value};
\node[gp node center,rotate=-270] at (16.397,6.460) {Resource Penalty [Hours]};
\node[gp node center] at (8.045,0.215) {Relative time ($\tau$) [S]};
\node[gp node center] at (8.045,11.537) {Objective Value for Weekly Schedule};
%% coordinates of the plot area
\gpdefrectangularnode{gp plot 1}{\pgfpoint{2.424cm}{1.845cm}}{\pgfpoint{13.667cm}{11.075cm}}
\endtikzpicture
%% gnuplot variables

	}
	\caption{Effects of pertubing the solution by forcing work orders into specific periods.}
	\label{fig:responses:inclusion}
\end{figure}
Figure~\ref{fig:responses:inclusion} 
show 5 pertubations with the first at
time = 60s where the objective value  slightly increases in response to the
inclusion, the objective value increases due to the inclusion either causing
the capacity to be exceeded or the inclusion resulting in a selected $
\ParStrategicValue$ that has a lower value. The remaining 4 pertubations
all show the same  pattern, an increase in the objective value followed by a
subsequent convergence.

\subsection{Response to Exclusion}\label{sec:exclusion}
The response to exclusion is associated with the $\ParStrategicExclude$
parameter and is found in equation~\ref{eqn:strategic:constraint:exclude}.
Here specific work orders ($\ElementWorkOrder \in \SetWorkOrder{}$) are
being excluded from specific periods ($\ElementPeriod \in \SetPeriod$).
The pertubations that the AbLNS will be affected by are shown in
table~\ref{tab:responses:exclusion}
with the setup being very similar to the
one in table~\ref{tab:responses:inclusion}.
The main distinction being that the pertubation affects 500 instead of 50 work orders, the higher number 
of affected work orders is chosen as many exclusions of do not affect the assignment of a work order.

\begin{table}[H]
	\centering
	\begin{tabular}{|c|c|c|c|c|c|}
	\hline
	\begin{tabular}[c]{@{}c@{}}\end{tabular}                                & \begin{tabular}[c]{@{}c@{}}$\VarMetaTime_1 = 60$\end{tabular} & \begin{tabular}[c]{@{}c@{}} $\VarMetaTime_2 = 120$ \end{tabular} & \begin{tabular}[c]{@{}c@{}}$\VarMetaTime_3 = 180$\end{tabular} & \begin{tabular}[c]{@{}c@{}}$\VarMetaTime_4 = 240$\end{tabular} & \begin{tabular}[c]{@{}c@{}}$\VarMetaTime_5 = 300$\end{tabular} \\ \hline
	\begin{tabular}[c]{@{}c@{}}$\Delta |\ParStrategicExclude|$\end{tabular} & 500                                                           & 500                                                              & 500                                                            & 500                                                            & 500                                                       \\ \hline
	\end{tabular}
	\caption{The pertubations of excluding a work order from specific periods in the weekly schedule. 
	Pertubations occur at 60 second time intervals affecting 500 work orders each time.}
	\label{tab:responses:exclusion}
\end{table}

Figure~\ref{fig:responses:exclusion}  show a substantial spike in the objective value  
after a pertubation as given in table~\ref{tab:responses:exclusion}. 

\begin{figure}[H]
	\centering
	\resizebox{10cm}{!}{
		\tikzpicture[gnuplot]
%% generated with GNUPLOT 6.0p1 (Lua 5.2; terminal rev. Jun 2020, script rev. 118)
%% Mon 02 Dec 2024 08:12:10 PM UTC
\path (0.000,0.000) rectangle (16.000,12.000);
\gpcolor{color=gp lt color axes}
\gpsetlinetype{gp lt axes}
\gpsetdashtype{gp dt axes}
\gpsetlinewidth{0.50}
\draw[gp path] (2.608,2.025)--(15.447,2.025);
\gpcolor{color=gp lt color border}
\gpsetlinetype{gp lt border}
\gpsetdashtype{gp dt solid}
\gpsetlinewidth{1.00}
\draw[gp path] (2.608,2.025)--(2.788,2.025);
\draw[gp path] (15.447,2.025)--(15.267,2.025);
\node[gp node right] at (2.424,2.025) {$1.915\times10^{11}$};
\gpcolor{color=gp lt color axes}
\gpsetlinetype{gp lt axes}
\gpsetdashtype{gp dt axes}
\gpsetlinewidth{0.50}
\draw[gp path] (2.608,2.930)--(15.447,2.930);
\gpcolor{color=gp lt color border}
\gpsetlinetype{gp lt border}
\gpsetdashtype{gp dt solid}
\gpsetlinewidth{1.00}
\draw[gp path] (2.608,2.930)--(2.788,2.930);
\draw[gp path] (15.447,2.930)--(15.267,2.930);
\node[gp node right] at (2.424,2.930) {$1.92\times10^{11}$};
\gpcolor{color=gp lt color axes}
\gpsetlinetype{gp lt axes}
\gpsetdashtype{gp dt axes}
\gpsetlinewidth{0.50}
\draw[gp path] (2.608,3.835)--(15.447,3.835);
\gpcolor{color=gp lt color border}
\gpsetlinetype{gp lt border}
\gpsetdashtype{gp dt solid}
\gpsetlinewidth{1.00}
\draw[gp path] (2.608,3.835)--(2.788,3.835);
\draw[gp path] (15.447,3.835)--(15.267,3.835);
\node[gp node right] at (2.424,3.835) {$1.925\times10^{11}$};
\gpcolor{color=gp lt color axes}
\gpsetlinetype{gp lt axes}
\gpsetdashtype{gp dt axes}
\gpsetlinewidth{0.50}
\draw[gp path] (2.608,4.740)--(15.447,4.740);
\gpcolor{color=gp lt color border}
\gpsetlinetype{gp lt border}
\gpsetdashtype{gp dt solid}
\gpsetlinewidth{1.00}
\draw[gp path] (2.608,4.740)--(2.788,4.740);
\draw[gp path] (15.447,4.740)--(15.267,4.740);
\node[gp node right] at (2.424,4.740) {$1.93\times10^{11}$};
\gpcolor{color=gp lt color axes}
\gpsetlinetype{gp lt axes}
\gpsetdashtype{gp dt axes}
\gpsetlinewidth{0.50}
\draw[gp path] (2.608,5.645)--(15.447,5.645);
\gpcolor{color=gp lt color border}
\gpsetlinetype{gp lt border}
\gpsetdashtype{gp dt solid}
\gpsetlinewidth{1.00}
\draw[gp path] (2.608,5.645)--(2.788,5.645);
\draw[gp path] (15.447,5.645)--(15.267,5.645);
\node[gp node right] at (2.424,5.645) {$1.935\times10^{11}$};
\gpcolor{color=gp lt color axes}
\gpsetlinetype{gp lt axes}
\gpsetdashtype{gp dt axes}
\gpsetlinewidth{0.50}
\draw[gp path] (2.608,6.550)--(15.447,6.550);
\gpcolor{color=gp lt color border}
\gpsetlinetype{gp lt border}
\gpsetdashtype{gp dt solid}
\gpsetlinewidth{1.00}
\draw[gp path] (2.608,6.550)--(2.788,6.550);
\draw[gp path] (15.447,6.550)--(15.267,6.550);
\node[gp node right] at (2.424,6.550) {$1.94\times10^{11}$};
\gpcolor{color=gp lt color axes}
\gpsetlinetype{gp lt axes}
\gpsetdashtype{gp dt axes}
\gpsetlinewidth{0.50}
\draw[gp path] (2.608,7.455)--(15.447,7.455);
\gpcolor{color=gp lt color border}
\gpsetlinetype{gp lt border}
\gpsetdashtype{gp dt solid}
\gpsetlinewidth{1.00}
\draw[gp path] (2.608,7.455)--(2.788,7.455);
\draw[gp path] (15.447,7.455)--(15.267,7.455);
\node[gp node right] at (2.424,7.455) {$1.945\times10^{11}$};
\gpcolor{color=gp lt color axes}
\gpsetlinetype{gp lt axes}
\gpsetdashtype{gp dt axes}
\gpsetlinewidth{0.50}
\draw[gp path] (2.608,8.360)--(15.447,8.360);
\gpcolor{color=gp lt color border}
\gpsetlinetype{gp lt border}
\gpsetdashtype{gp dt solid}
\gpsetlinewidth{1.00}
\draw[gp path] (2.608,8.360)--(2.788,8.360);
\draw[gp path] (15.447,8.360)--(15.267,8.360);
\node[gp node right] at (2.424,8.360) {$1.95\times10^{11}$};
\gpcolor{color=gp lt color axes}
\gpsetlinetype{gp lt axes}
\gpsetdashtype{gp dt axes}
\gpsetlinewidth{0.50}
\draw[gp path] (2.608,9.265)--(15.447,9.265);
\gpcolor{color=gp lt color border}
\gpsetlinetype{gp lt border}
\gpsetdashtype{gp dt solid}
\gpsetlinewidth{1.00}
\draw[gp path] (2.608,9.265)--(2.788,9.265);
\draw[gp path] (15.447,9.265)--(15.267,9.265);
\node[gp node right] at (2.424,9.265) {$1.955\times10^{11}$};
\gpcolor{color=gp lt color axes}
\gpsetlinetype{gp lt axes}
\gpsetdashtype{gp dt axes}
\gpsetlinewidth{0.50}
\draw[gp path] (2.608,10.170)--(15.447,10.170);
\gpcolor{color=gp lt color border}
\gpsetlinetype{gp lt border}
\gpsetdashtype{gp dt solid}
\gpsetlinewidth{1.00}
\draw[gp path] (2.608,10.170)--(2.788,10.170);
\draw[gp path] (15.447,10.170)--(15.267,10.170);
\node[gp node right] at (2.424,10.170) {$1.96\times10^{11}$};
\gpcolor{color=gp lt color axes}
\gpsetlinetype{gp lt axes}
\gpsetdashtype{gp dt axes}
\gpsetlinewidth{0.50}
\draw[gp path] (2.608,11.075)--(15.447,11.075);
\gpcolor{color=gp lt color border}
\gpsetlinetype{gp lt border}
\gpsetdashtype{gp dt solid}
\gpsetlinewidth{1.00}
\draw[gp path] (2.608,11.075)--(2.788,11.075);
\draw[gp path] (15.447,11.075)--(15.267,11.075);
\node[gp node right] at (2.424,11.075) {$1.965\times10^{11}$};
\gpcolor{color=gp lt color axes}
\gpsetlinetype{gp lt axes}
\gpsetdashtype{gp dt axes}
\gpsetlinewidth{0.50}
\draw[gp path] (2.608,2.025)--(2.608,11.075);
\gpcolor{color=gp lt color border}
\gpsetlinetype{gp lt border}
\gpsetdashtype{gp dt solid}
\gpsetlinewidth{1.00}
\draw[gp path] (2.608,2.025)--(2.608,1.845);
\draw[gp path] (2.608,11.075)--(2.608,11.255);
\node[gp node left,rotate=270] at (2.608,1.661) {$0$};
\gpcolor{color=gp lt color axes}
\gpsetlinetype{gp lt axes}
\gpsetdashtype{gp dt axes}
\gpsetlinewidth{0.50}
\draw[gp path] (4.748,2.025)--(4.748,11.075);
\gpcolor{color=gp lt color border}
\gpsetlinetype{gp lt border}
\gpsetdashtype{gp dt solid}
\gpsetlinewidth{1.00}
\draw[gp path] (4.748,2.025)--(4.748,1.845);
\draw[gp path] (4.748,11.075)--(4.748,11.255);
\node[gp node left,rotate=270] at (4.748,1.661) {$50$};
\gpcolor{color=gp lt color axes}
\gpsetlinetype{gp lt axes}
\gpsetdashtype{gp dt axes}
\gpsetlinewidth{0.50}
\draw[gp path] (6.888,2.025)--(6.888,11.075);
\gpcolor{color=gp lt color border}
\gpsetlinetype{gp lt border}
\gpsetdashtype{gp dt solid}
\gpsetlinewidth{1.00}
\draw[gp path] (6.888,2.025)--(6.888,1.845);
\draw[gp path] (6.888,11.075)--(6.888,11.255);
\node[gp node left,rotate=270] at (6.888,1.661) {$100$};
\gpcolor{color=gp lt color axes}
\gpsetlinetype{gp lt axes}
\gpsetdashtype{gp dt axes}
\gpsetlinewidth{0.50}
\draw[gp path] (9.028,2.025)--(9.028,11.075);
\gpcolor{color=gp lt color border}
\gpsetlinetype{gp lt border}
\gpsetdashtype{gp dt solid}
\gpsetlinewidth{1.00}
\draw[gp path] (9.028,2.025)--(9.028,1.845);
\draw[gp path] (9.028,11.075)--(9.028,11.255);
\node[gp node left,rotate=270] at (9.028,1.661) {$150$};
\gpcolor{color=gp lt color axes}
\gpsetlinetype{gp lt axes}
\gpsetdashtype{gp dt axes}
\gpsetlinewidth{0.50}
\draw[gp path] (11.167,2.025)--(11.167,11.075);
\gpcolor{color=gp lt color border}
\gpsetlinetype{gp lt border}
\gpsetdashtype{gp dt solid}
\gpsetlinewidth{1.00}
\draw[gp path] (11.167,2.025)--(11.167,1.845);
\draw[gp path] (11.167,11.075)--(11.167,11.255);
\node[gp node left,rotate=270] at (11.167,1.661) {$200$};
\gpcolor{color=gp lt color axes}
\gpsetlinetype{gp lt axes}
\gpsetdashtype{gp dt axes}
\gpsetlinewidth{0.50}
\draw[gp path] (13.307,2.025)--(13.307,11.075);
\gpcolor{color=gp lt color border}
\gpsetlinetype{gp lt border}
\gpsetdashtype{gp dt solid}
\gpsetlinewidth{1.00}
\draw[gp path] (13.307,2.025)--(13.307,1.845);
\draw[gp path] (13.307,11.075)--(13.307,11.255);
\node[gp node left,rotate=270] at (13.307,1.661) {$250$};
\gpcolor{color=gp lt color axes}
\gpsetlinetype{gp lt axes}
\gpsetdashtype{gp dt axes}
\gpsetlinewidth{0.50}
\draw[gp path] (15.447,2.025)--(15.447,11.075);
\gpcolor{color=gp lt color border}
\gpsetlinetype{gp lt border}
\gpsetdashtype{gp dt solid}
\gpsetlinewidth{1.00}
\draw[gp path] (15.447,2.025)--(15.447,1.845);
\draw[gp path] (15.447,11.075)--(15.447,11.255);
\node[gp node left,rotate=270] at (15.447,1.661) {$300$};
\draw[gp path] (2.608,11.075)--(2.608,2.025)--(15.447,2.025)--(15.447,11.075)--cycle;
\gpcolor{rgb color={0.000,0.000,0.000}}
\gpsetlinewidth{2.00}
\draw[gp path] (2.608,10.368)--(2.608,9.903)--(2.608,9.702)--(2.651,9.624)--(2.651,9.599)%
  --(2.651,9.399)--(2.651,9.116)--(2.651,8.819)--(2.651,8.532)--(2.651,8.116)--(2.651,7.980)%
  --(2.694,7.773)--(2.694,7.681)--(2.694,7.574)--(2.694,7.475)--(2.694,7.438)--(2.694,7.379)%
  --(2.694,7.323)--(2.736,7.242)--(2.736,7.193)--(2.736,7.026)--(2.736,6.967)--(2.736,6.703)%
  --(2.736,6.654)--(2.779,6.520)--(2.779,6.421)--(2.779,6.387)--(2.779,6.306)--(2.779,6.294)%
  --(2.779,6.112)--(2.779,6.108)--(2.822,5.991)--(2.822,5.927)--(2.822,5.780)--(2.822,5.771)%
  --(2.822,5.682)--(2.865,5.587)--(2.865,5.567)--(2.865,5.525)--(2.865,5.505)--(2.865,5.429)%
  --(2.865,5.373)--(2.908,5.369)--(2.908,5.258)--(2.908,5.253)--(2.950,5.160)--(2.950,5.139)%
  --(2.993,4.958)--(2.993,4.855)--(2.993,4.697)--(2.993,4.575)--(3.036,4.520)--(3.036,4.515)%
  --(3.079,4.481)--(3.122,4.423)--(3.122,4.354)--(3.122,4.333)--(3.122,4.326)--(3.122,4.304)%
  --(3.164,4.271)--(3.164,4.246)--(3.164,4.222)--(3.164,4.151)--(3.164,4.090)--(3.207,4.084)%
  --(3.207,4.050)--(3.207,3.965)--(3.250,3.930)--(3.250,3.782)--(3.336,3.701)--(3.336,3.689)%
  --(3.336,3.668)--(3.378,3.576)--(3.421,3.559)--(3.421,3.496)--(3.464,3.495)--(3.464,3.457)%
  --(3.507,3.446)--(3.507,3.436)--(3.635,3.424)--(3.678,3.421)--(3.678,3.411)--(3.721,3.408)%
  --(3.764,3.400)--(3.764,3.389)--(3.806,3.317)--(3.806,3.309)--(3.806,3.296)--(3.806,3.269)%
  --(3.849,3.214)--(3.849,3.200)--(3.892,3.185)--(3.892,3.183)--(3.892,3.168)--(3.892,3.152)%
  --(3.935,3.121)--(3.935,3.097)--(3.935,3.094)--(3.935,3.065)--(3.935,3.040)--(3.977,2.909)%
  --(3.977,2.889)--(3.977,2.536)--(4.020,2.500)--(4.020,2.485)--(4.063,2.477)--(4.063,2.407)%
  --(4.106,2.375)--(4.234,2.356)--(4.234,2.342)--(4.277,2.277)--(4.320,2.267)--(4.448,2.223)%
  --(4.448,2.218)--(4.662,2.148)--(4.662,2.146)--(4.748,2.134)--(4.876,2.103)--(4.876,2.089)%
  --(4.919,2.060)--(5.005,2.056)--(5.047,4.618)--(5.047,4.610)--(5.090,4.584)--(5.219,4.558)%
  --(5.304,4.536)--(5.304,4.516)--(5.390,4.408)--(5.390,4.383)--(5.390,4.376)--(5.433,4.328)%
  --(5.433,4.281)--(5.475,4.274)--(5.475,4.250)--(5.475,4.244)--(5.518,4.211)--(5.518,4.191)%
  --(5.732,4.173)--(5.818,4.156)--(5.903,4.146)--(5.903,4.134)--(5.946,4.117)--(5.989,4.106)%
  --(6.032,4.078)--(6.075,4.072)--(6.203,4.061)--(6.246,4.050)--(6.802,4.048)--(6.802,4.020)%
  --(6.888,4.000)--(7.016,3.993)--(7.102,3.989)--(7.102,3.988)--(7.273,3.974)--(7.358,3.961)%
  --(7.358,3.954)--(7.444,3.919)--(7.487,3.909)--(7.572,3.906)--(7.572,3.914)--(7.615,3.889)%
  --(7.658,3.876)--(7.744,3.864)--(7.786,3.852)--(7.786,3.851)--(7.915,3.837)--(8.172,3.812)%
  --(8.214,3.790)--(8.300,3.778)--(8.428,3.769)--(8.428,3.760)--(8.557,3.746)--(8.685,3.741)%
  --(8.942,3.718)--(9.241,3.701)--(9.284,3.691)--(9.327,3.690)--(9.455,3.672)--(9.541,3.663)%
  --(9.584,3.663)--(9.627,3.651)--(9.669,3.643)--(9.841,3.615)--(9.969,3.588)--(10.012,3.567)%
  --(10.097,3.574)--(10.140,3.567)--(10.183,3.463)--(10.183,3.447)--(10.568,3.446)--(10.654,3.442)%
  --(10.825,3.424)--(10.911,3.415)--(11.125,3.415)--(11.681,3.414)--(11.809,3.393)--(11.980,3.372)%
  --(12.066,3.369)--(12.109,3.032)--(12.109,3.014)--(12.237,2.986)--(12.537,2.974)--(12.622,2.974)%
  --(12.708,2.958)--(12.751,2.954)--(12.836,2.953)--(12.965,2.943)--(13.350,2.932)--(13.350,2.921)%
  --(14.463,2.915)--(14.591,2.906)--(14.591,2.901)--(14.762,2.891)--(14.848,2.879)--(15.019,2.851)%
  --(15.147,2.938);
\gpcolor{color=gp lt color border}
\gpsetlinewidth{1.00}
\draw[gp path] (2.608,11.075)--(2.608,2.025)--(15.447,2.025)--(15.447,11.075)--cycle;
\node[gp node center,rotate=-270] at (-0.812,6.550) {Objective value};
\node[gp node center] at (9.027,0.215) {Relative time [S]};
\node[gp node center] at (9.027,11.537) {Objective value for Weekly Schedule};
%% coordinates of the plot area
\gpdefrectangularnode{gp plot 1}{\pgfpoint{2.608cm}{2.025cm}}{\pgfpoint{15.447cm}{11.075cm}}
\endtikzpicture
%% gnuplot variables

	}
	\caption{Figure Caption}
	\label{fig:responses:exclusion}
\end{figure}

From figure~\ref{fig:responses:inclusion} and figure~\ref{fig:responses:exclusion} it is 
clear that the AbLNS method can handle dynamic entries of work orders. The next section will discuss the effects of dynamically changing the 
resource capacities $\ParStrategicResource$. 

\subsection{Response to Additional Weekly Capacity}\label{sec:increase_week_cap}
Table~\ref{tab:responses:resource-addition} details the pertubations that the
AbLNS will be subject to during its 360 second execution. Pertubing the
resources $\ParStrategicResource$ affects the solution considerably more than
pertubing $\ParStrategicExclude$ and $\ParStrategicInclude$ and therefore 100
second intervals are specified instead of 60 second intervals.

\begin{table}[H]
	\centering
	\begin{tabular}{|c|c|c|c|c|c|}
	\hline
	                                & \begin{tabular}[c]{@{}c@{}}$\VarMetaTime_1 = 100$\end{tabular}  & \begin{tabular}[c]{@{}c@{}}$\VarMetaTime_2 = 200$\end{tabular}  & \begin{tabular}[c]{@{}c@{}}$\VarMetaTime_3 = 300$\end{tabular} \\ \hline
	$\Delta |\SetPeriod|$           & 52                                                             & 52                                                              & 52                                                             \\ \hline
	$\Delta |\SetResource|$         & 16                                                             & 16                                                              & 16                                                             \\ \hline
	$ |\ParStrategicResource|$ (hours)& 61816                                                           & 111268                                                            & 173083                                                          \\ \hline
	\end{tabular}
	\caption{}
	\label{tab:responses:resource-addition}
\end{table}

Figure~\ref{fig:responses:resources-addition} shows the effects of progressively
increasing  available resources. The AbLNS starts with a base load which is
then reduced at time = 100 seconds causing the objective value by increase
as $\ParStrategicPenalty$ in equation~\ref{eqn:strategic:constraint:resource}
absorbes the exceeded capacity. At time = 200 the resources are increased to the
base load again, and at time = 300 seconds the resources are  increased to their
maximum value.

\begin{figure}[H]%% placement specifier
	\centering
	\resizebox{10cm}{!}{
		\tikzpicture[gnuplot]
%% generated with GNUPLOT 6.0p1 (Lua 5.2; terminal rev. Jun 2020, script rev. 118)
%% Wed 27 Nov 2024 09:15:11 AM UTC
\path (0.000,0.000) rectangle (16.000,12.000);
\gpcolor{color=gp lt color border}
\gpsetlinetype{gp lt border}
\gpsetdashtype{gp dt solid}
\gpsetlinewidth{1.00}
\draw[gp path] (2.424,2.697)--(2.604,2.697);
\draw[gp path] (15.447,2.697)--(15.267,2.697);
\node[gp node right] at (2.240,2.697) {$1.76\times10^{11}$};
\draw[gp path] (2.424,3.459)--(2.604,3.459);
\draw[gp path] (15.447,3.459)--(15.267,3.459);
\node[gp node right] at (2.240,3.459) {$1.78\times10^{11}$};
\draw[gp path] (2.424,4.220)--(2.604,4.220);
\draw[gp path] (15.447,4.220)--(15.267,4.220);
\node[gp node right] at (2.240,4.220) {$1.8\times10^{11}$};
\draw[gp path] (2.424,4.982)--(2.604,4.982);
\draw[gp path] (15.447,4.982)--(15.267,4.982);
\node[gp node right] at (2.240,4.982) {$1.82\times10^{11}$};
\draw[gp path] (2.424,5.744)--(2.604,5.744);
\draw[gp path] (15.447,5.744)--(15.267,5.744);
\node[gp node right] at (2.240,5.744) {$1.84\times10^{11}$};
\draw[gp path] (2.424,6.505)--(2.604,6.505);
\draw[gp path] (15.447,6.505)--(15.267,6.505);
\node[gp node right] at (2.240,6.505) {$1.86\times10^{11}$};
\draw[gp path] (2.424,7.267)--(2.604,7.267);
\draw[gp path] (15.447,7.267)--(15.267,7.267);
\node[gp node right] at (2.240,7.267) {$1.88\times10^{11}$};
\draw[gp path] (2.424,8.028)--(2.604,8.028);
\draw[gp path] (15.447,8.028)--(15.267,8.028);
\node[gp node right] at (2.240,8.028) {$1.9\times10^{11}$};
\draw[gp path] (2.424,8.790)--(2.604,8.790);
\draw[gp path] (15.447,8.790)--(15.267,8.790);
\node[gp node right] at (2.240,8.790) {$1.92\times10^{11}$};
\draw[gp path] (2.424,9.552)--(2.604,9.552);
\draw[gp path] (15.447,9.552)--(15.267,9.552);
\node[gp node right] at (2.240,9.552) {$1.94\times10^{11}$};
\draw[gp path] (2.424,10.313)--(2.604,10.313);
\draw[gp path] (15.447,10.313)--(15.267,10.313);
\node[gp node right] at (2.240,10.313) {$1.96\times10^{11}$};
\draw[gp path] (2.424,11.075)--(2.604,11.075);
\draw[gp path] (15.447,11.075)--(15.267,11.075);
\node[gp node right] at (2.240,11.075) {$1.98\times10^{11}$};
\draw[gp path] (2.424,2.697)--(2.424,2.517);
\draw[gp path] (2.424,11.075)--(2.424,11.255);
\node[gp node left,rotate=270] at (2.424,2.333) {$1.7327\times10^{9}$};
\draw[gp path] (3.871,2.697)--(3.871,2.517);
\draw[gp path] (3.871,11.075)--(3.871,11.255);
\node[gp node left,rotate=270] at (3.871,2.333) {$1.7327\times10^{9}$};
\draw[gp path] (5.318,2.697)--(5.318,2.517);
\draw[gp path] (5.318,11.075)--(5.318,11.255);
\node[gp node left,rotate=270] at (5.318,2.333) {$1.7327\times10^{9}$};
\draw[gp path] (6.765,2.697)--(6.765,2.517);
\draw[gp path] (6.765,11.075)--(6.765,11.255);
\node[gp node left,rotate=270] at (6.765,2.333) {$1.7327\times10^{9}$};
\draw[gp path] (8.212,2.697)--(8.212,2.517);
\draw[gp path] (8.212,11.075)--(8.212,11.255);
\node[gp node left,rotate=270] at (8.212,2.333) {$1.7327\times10^{9}$};
\draw[gp path] (9.659,2.697)--(9.659,2.517);
\draw[gp path] (9.659,11.075)--(9.659,11.255);
\node[gp node left,rotate=270] at (9.659,2.333) {$1.7327\times10^{9}$};
\draw[gp path] (11.106,2.697)--(11.106,2.517);
\draw[gp path] (11.106,11.075)--(11.106,11.255);
\node[gp node left,rotate=270] at (11.106,2.333) {$1.7327\times10^{9}$};
\draw[gp path] (12.553,2.697)--(12.553,2.517);
\draw[gp path] (12.553,11.075)--(12.553,11.255);
\node[gp node left,rotate=270] at (12.553,2.333) {$1.7327\times10^{9}$};
\draw[gp path] (14.000,2.697)--(14.000,2.517);
\draw[gp path] (14.000,11.075)--(14.000,11.255);
\node[gp node left,rotate=270] at (14.000,2.333) {$1.7327\times10^{9}$};
\draw[gp path] (15.447,2.697)--(15.447,2.517);
\draw[gp path] (15.447,11.075)--(15.447,11.255);
\node[gp node left,rotate=270] at (15.447,2.333) {$1.7327\times10^{9}$};
\draw[gp path] (2.424,11.075)--(2.424,2.697)--(15.447,2.697)--(15.447,11.075)--cycle;
\gpcolor{rgb color={0.580,0.000,0.827}}
\draw[gp path] (3.799,10.388)--(3.799,10.234)--(3.799,10.197)--(3.799,10.174)--(3.799,10.102)%
  --(3.871,10.082)--(3.871,10.068)--(3.871,10.030)--(3.871,9.979)--(3.871,9.963)--(3.871,9.928)%
  --(3.871,9.891)--(3.943,9.872)--(3.943,9.815)--(3.943,9.801)--(3.943,9.794)--(4.016,9.767)%
  --(4.016,9.758)--(4.016,9.745)--(4.016,9.730)--(4.016,9.728)--(4.016,9.722)--(4.016,9.677)%
  --(4.016,9.607)--(4.016,9.606)--(4.088,9.591)--(4.088,9.571)--(4.088,9.524)--(4.160,9.515)%
  --(4.160,9.511)--(4.160,9.486)--(4.160,9.477)--(4.160,9.458)--(4.160,9.454)--(4.160,9.439)%
  --(4.233,9.125)--(4.233,9.022)--(4.233,9.088)--(4.233,8.970)--(4.233,8.854)--(4.305,8.915)%
  --(4.305,8.905)--(4.305,8.861)--(4.377,8.881)--(4.377,8.781)--(4.377,8.831)--(4.450,8.796)%
  --(4.450,8.758)--(4.450,8.832)--(4.450,8.810)--(4.522,8.762)--(4.522,8.723)--(4.522,8.736)%
  --(4.595,8.715)--(4.595,8.698)--(4.667,8.728)--(4.667,8.713)--(4.667,8.658)--(4.667,8.685)%
  --(4.739,8.719)--(4.739,8.701)--(4.739,8.684)--(4.812,8.589)--(4.812,8.622)--(4.812,8.624)%
  --(4.812,8.601)--(4.884,8.653)--(4.884,8.525)--(4.956,8.533)--(4.956,8.606)--(5.029,8.528)%
  --(5.029,8.584)--(5.029,8.591)--(5.101,8.473)--(5.101,8.564)--(5.101,8.552)--(5.101,8.572)%
  --(5.173,8.588)--(5.173,8.615)--(5.246,8.534)--(5.246,8.578)--(5.246,8.586)--(5.318,8.577)%
  --(5.318,8.538)--(5.318,8.482)--(5.318,8.432)--(5.390,8.476)--(5.390,8.475)--(5.390,8.502)%
  --(5.463,8.471)--(5.535,8.479)--(5.535,8.430)--(5.607,8.474)--(5.607,8.375)--(5.607,10.530)%
  --(5.607,10.579)--(5.607,10.529)--(5.680,10.506)--(5.680,10.517)--(5.752,10.471)--(5.752,10.312)%
  --(5.824,10.343)--(5.824,10.417)--(5.824,10.365)--(5.897,10.310)--(5.897,10.307)--(5.897,10.339)%
  --(5.969,10.257)--(5.969,10.335)--(5.969,10.358)--(6.042,10.324)--(6.186,10.276)--(6.186,10.284)%
  --(6.186,10.283)--(6.331,10.281)--(6.476,10.286)--(6.476,10.296)--(6.548,10.265)--(6.548,10.271)%
  --(6.693,10.238)--(6.765,10.266)--(6.837,10.314)--(6.910,10.323)--(6.910,10.302)--(6.982,10.245)%
  --(6.982,10.305)--(6.982,10.275)--(7.054,10.247)--(7.054,10.256)--(7.054,10.243)--(7.054,7.049)%
  --(7.127,7.001)--(7.127,6.919)--(7.127,7.013)--(7.199,6.915)--(7.199,6.858)--(7.199,6.829)%
  --(7.271,6.697)--(7.271,6.666)--(7.271,6.704)--(7.271,6.660)--(7.344,6.698)--(7.344,6.662)%
  --(7.344,6.621)--(7.416,6.616)--(7.416,6.583)--(7.416,6.617)--(7.416,6.608)--(7.489,6.542)%
  --(7.489,6.522)--(7.489,6.443)--(7.561,6.391)--(7.561,6.392)--(7.561,6.287)--(7.633,6.282)%
  --(7.633,6.302)--(7.633,6.305)--(7.706,6.240)--(7.706,6.265)--(7.706,6.282)--(7.778,6.291)%
  --(7.850,6.306)--(7.850,6.199)--(7.850,6.203)--(7.923,6.275)--(7.923,6.284)--(7.995,6.265)%
  --(7.995,6.247)--(8.067,6.220)--(8.067,6.203)--(8.067,6.206)--(8.140,6.182)--(8.140,6.196)%
  --(8.140,6.177)--(8.212,6.151)--(8.212,6.157)--(8.284,6.188)--(8.284,6.142)--(8.357,6.162)%
  --(8.357,6.163)--(8.429,6.168)--(8.429,6.126)--(8.429,6.160)--(8.429,6.167)--(8.501,6.107)%
  --(8.574,3.847)--(8.574,3.843)--(8.574,3.751)--(8.574,3.737)--(8.646,3.752)--(8.646,3.784)%
  --(8.646,3.790)--(8.718,3.777)--(8.718,3.772)--(8.718,3.783)--(8.718,3.768)--(8.791,3.767)%
  --(8.791,3.768)--(8.863,3.652)--(8.863,3.775)--(8.936,3.759)--(8.936,3.698)--(9.008,3.754)%
  --(9.008,3.610)--(9.008,3.604)--(9.008,3.644)--(9.080,3.684)--(9.080,3.628)--(9.153,3.684)%
  --(9.153,3.655)--(9.225,3.668)--(9.225,3.674)--(9.225,3.620)--(9.297,3.655)--(9.297,3.662)%
  --(9.370,3.635)--(9.370,3.608)--(9.442,3.584)--(9.442,3.501)--(9.514,3.487)--(9.587,3.440)%
  --(9.587,3.457)--(9.587,3.485)--(9.587,3.461)--(9.659,3.456)--(9.659,3.416)--(9.731,3.414)%
  --(9.804,3.472)--(9.804,3.445)--(9.876,3.401)--(9.876,3.396)--(9.876,3.437)--(9.948,3.390)%
  --(9.948,3.451)--(9.948,3.423)--(10.021,3.457)--(10.021,3.429)--(10.093,3.464)--(10.093,3.446)%
  --(10.093,3.408)--(10.093,3.471)--(10.165,3.463)--(10.165,3.419)--(10.165,3.481)--(10.238,3.424)%
  --(10.238,3.472)--(10.310,3.371)--(10.310,3.442)--(10.310,3.434)--(10.383,3.398)--(10.383,3.314)%
  --(10.455,3.372)--(10.455,3.274)--(10.455,3.366)--(10.527,3.353)--(10.527,3.386)--(10.600,3.387)%
  --(10.600,3.350)--(10.600,3.336)--(10.600,3.350)--(10.672,3.276)--(10.672,3.293)--(10.744,3.350)%
  --(10.744,3.342)--(10.817,3.316)--(10.817,3.310)--(10.817,3.309)--(10.817,3.347)--(10.889,3.330)%
  --(10.889,3.338)--(10.961,3.326)--(10.961,3.336)--(11.106,3.315)--(11.106,3.343)--(11.106,3.312)%
  --(11.178,3.321)--(11.178,3.337)--(11.178,3.281)--(11.251,3.323)--(11.251,3.328)--(11.251,3.346)%
  --(11.323,3.224)--(11.323,3.345)--(11.395,3.339)--(11.395,3.335)--(11.395,3.344)--(11.468,3.241)%
  --(11.468,3.326)--(11.540,3.288)--(11.540,3.307)--(11.612,3.326)--(11.612,3.277)--(11.612,3.313)%
  --(11.685,3.310)--(11.685,3.311)--(11.685,3.331)--(11.757,3.250)--(11.757,3.313)--(11.757,3.284)%
  --(11.830,3.337)--(11.830,3.335)--(11.902,3.337)--(11.902,3.299)--(11.974,3.283)--(11.974,3.346)%
  --(11.974,3.353)--(12.047,3.337)--(12.047,3.350)--(12.119,3.336)--(12.119,3.345)--(12.191,3.336)%
  --(12.191,3.306)--(12.264,3.310)--(12.264,3.326)--(12.336,3.257)--(12.336,3.322)--(12.408,3.315)%
  --(12.408,3.311)--(12.481,3.310)--(12.481,3.297)--(12.553,3.288)--(12.553,3.296)--(12.625,3.272)%
  --(12.625,3.287)--(12.625,3.262)--(12.698,3.292)--(12.698,3.257)--(12.770,3.276)--(12.770,3.235)%
  --(12.842,3.283)--(12.842,3.245)--(12.915,3.282)--(12.915,3.259)--(12.915,3.234)--(12.987,3.255)%
  --(12.987,3.291)--(13.059,3.292)--(13.059,3.235)--(13.059,3.294)--(13.132,3.276)--(13.132,3.268)%
  --(13.132,3.273)--(13.204,3.268)--(13.277,3.281)--(13.349,3.272)--(13.349,3.279)--(13.421,3.290)%
  --(13.421,3.248)--(13.494,3.279)--(13.566,3.256)--(13.566,3.275)--(13.638,3.278)--(13.638,3.252)%
  --(13.638,3.285)--(13.711,3.219)--(13.711,3.171)--(13.783,3.197)--(13.783,3.281)--(13.855,3.288)%
  --(13.855,3.287)--(13.928,3.246)--(13.928,3.284)--(14.000,3.285)--(14.072,3.269)--(14.072,3.255)%
  --(14.072,3.253)--(14.072,3.209)--(14.145,3.251)--(14.145,3.198)--(14.145,3.242)--(14.217,3.232)%
  --(14.217,3.254)--(14.217,3.250)--(14.289,3.278)--(14.289,3.292)--(14.362,3.285)--(14.362,3.246)%
  --(14.434,3.294)--(14.434,3.302)--(14.434,3.305)--(14.506,3.286)--(14.506,3.276)--(14.579,3.247)%
  --(14.579,3.305)--(14.579,3.233)--(14.651,3.212)--(14.651,3.312)--(14.723,3.287)--(14.723,3.284)%
  --(14.723,3.300)--(14.796,3.216)--(14.796,3.315)--(14.868,3.308)--(14.941,3.301)--(14.941,3.309)%
  --(14.941,3.274)--(15.013,3.320)--(15.013,3.316)--(15.013,3.208)--(15.085,3.329)--(15.158,3.301)%
  --(15.158,3.318)--(15.158,3.264)--(15.230,3.301)--(15.302,3.314)--(15.302,3.285)--(15.302,3.301);
\gpcolor{color=gp lt color border}
\draw[gp path] (2.424,11.075)--(2.424,2.697)--(15.447,2.697)--(15.447,11.075)--cycle;
\node[gp node center,rotate=-270] at (0.292,6.886) {Objective value};
\node[gp node center] at (8.935,0.215) {Absolute time};
\node[gp node center] at (8.935,11.537) {Objective value for Weekly Schedule};
%% coordinates of the plot area
\gpdefrectangularnode{gp plot 1}{\pgfpoint{2.424cm}{2.697cm}}{\pgfpoint{15.447cm}{11.075cm}}
\endtikzpicture
%% gnuplot variables

	}
	\caption{Figure Caption}
	\label{fig:responses:resources-addition}
\end{figure}

\subsection{Response to Reduced Weekly Capacity}\label{sec:results:reduced_weekly_capacity}
Table~\ref{tab:resources:resource-subtraction} details the pertubations that the
AbLNS will affected by. Starting from a base load of the amount of available
resources are progressively decreased.
\begin{table}[H]
	\centering
	\begin{tabular}{|c|c|c|c|c|c|}
	\hline
	                                & \begin{tabular}[c]{@{}c@{}}$\VarMetaTime_1 = 100$\end{tabular} & \begin{tabular}[c]{@{}c@{}}$\VarMetaTime_2 = 200$\end{tabular}  & \begin{tabular}[c]{@{}c@{}}$\VarMetaTime_3 = 300$\end{tabular} \\ \hline
	$\Delta |\SetPeriod|$           & 52                                                            & 52                                                              & 52                                                             \\ \hline
	$\Delta |\SetResource|$         & 16                                                            & 16                                                              & 16                                                             \\ \hline
	$ |\ParStrategicResource|$ (hours)& 173083                                                         & 111268                                                            & 61816                                                           \\ \hline
	\end{tabular}
	\label{tab:resources:resource-subtraction}
\end{table}

Figure~\ref{fig:responses:resource-subtraction} shows the effects of perturbing
the AbLNS by starting from a base load  and then progressively reducing
capacity.

\begin{figure}[H]%% placement specifier
	\centering
	\resizebox{10cm}{!}{
		\tikzpicture[gnuplot]
%% generated with GNUPLOT 6.0p1 (Lua 5.2; terminal rev. Jun 2020, script rev. 118)
%% Tue 17 Dec 2024 02:07:05 PM UTC
\path (0.000,0.000) rectangle (16.000,12.000);
\gpcolor{color=gp lt color axes}
\gpsetlinetype{gp lt axes}
\gpsetdashtype{gp dt axes}
\gpsetlinewidth{0.50}
\draw[gp path] (2.424,1.845)--(15.447,1.845);
\gpcolor{color=gp lt color border}
\gpsetlinetype{gp lt border}
\gpsetdashtype{gp dt solid}
\gpsetlinewidth{1.00}
\draw[gp path] (2.424,1.845)--(2.604,1.845);
\draw[gp path] (15.447,1.845)--(15.267,1.845);
\node[gp node right] at (2.240,1.845) {$1.8\times10^{11}$};
\gpcolor{color=gp lt color axes}
\gpsetlinetype{gp lt axes}
\gpsetdashtype{gp dt axes}
\gpsetlinewidth{0.50}
\draw[gp path] (2.424,3.383)--(15.447,3.383);
\gpcolor{color=gp lt color border}
\gpsetlinetype{gp lt border}
\gpsetdashtype{gp dt solid}
\gpsetlinewidth{1.00}
\draw[gp path] (2.424,3.383)--(2.604,3.383);
\draw[gp path] (15.447,3.383)--(15.267,3.383);
\node[gp node right] at (2.240,3.383) {$1.85\times10^{11}$};
\gpcolor{color=gp lt color axes}
\gpsetlinetype{gp lt axes}
\gpsetdashtype{gp dt axes}
\gpsetlinewidth{0.50}
\draw[gp path] (2.424,4.922)--(15.447,4.922);
\gpcolor{color=gp lt color border}
\gpsetlinetype{gp lt border}
\gpsetdashtype{gp dt solid}
\gpsetlinewidth{1.00}
\draw[gp path] (2.424,4.922)--(2.604,4.922);
\draw[gp path] (15.447,4.922)--(15.267,4.922);
\node[gp node right] at (2.240,4.922) {$1.9\times10^{11}$};
\gpcolor{color=gp lt color axes}
\gpsetlinetype{gp lt axes}
\gpsetdashtype{gp dt axes}
\gpsetlinewidth{0.50}
\draw[gp path] (2.424,6.460)--(15.447,6.460);
\gpcolor{color=gp lt color border}
\gpsetlinetype{gp lt border}
\gpsetdashtype{gp dt solid}
\gpsetlinewidth{1.00}
\draw[gp path] (2.424,6.460)--(2.604,6.460);
\draw[gp path] (15.447,6.460)--(15.267,6.460);
\node[gp node right] at (2.240,6.460) {$1.95\times10^{11}$};
\gpcolor{color=gp lt color axes}
\gpsetlinetype{gp lt axes}
\gpsetdashtype{gp dt axes}
\gpsetlinewidth{0.50}
\draw[gp path] (2.424,7.998)--(15.447,7.998);
\gpcolor{color=gp lt color border}
\gpsetlinetype{gp lt border}
\gpsetdashtype{gp dt solid}
\gpsetlinewidth{1.00}
\draw[gp path] (2.424,7.998)--(2.604,7.998);
\draw[gp path] (15.447,7.998)--(15.267,7.998);
\node[gp node right] at (2.240,7.998) {$2\times10^{11}$};
\gpcolor{color=gp lt color axes}
\gpsetlinetype{gp lt axes}
\gpsetdashtype{gp dt axes}
\gpsetlinewidth{0.50}
\draw[gp path] (2.424,9.537)--(15.447,9.537);
\gpcolor{color=gp lt color border}
\gpsetlinetype{gp lt border}
\gpsetdashtype{gp dt solid}
\gpsetlinewidth{1.00}
\draw[gp path] (2.424,9.537)--(2.604,9.537);
\draw[gp path] (15.447,9.537)--(15.267,9.537);
\node[gp node right] at (2.240,9.537) {$2.05\times10^{11}$};
\gpcolor{color=gp lt color axes}
\gpsetlinetype{gp lt axes}
\gpsetdashtype{gp dt axes}
\gpsetlinewidth{0.50}
\draw[gp path] (2.424,11.075)--(15.447,11.075);
\gpcolor{color=gp lt color border}
\gpsetlinetype{gp lt border}
\gpsetdashtype{gp dt solid}
\gpsetlinewidth{1.00}
\draw[gp path] (2.424,11.075)--(2.604,11.075);
\draw[gp path] (15.447,11.075)--(15.267,11.075);
\node[gp node right] at (2.240,11.075) {$2.1\times10^{11}$};
\gpcolor{color=gp lt color axes}
\gpsetlinetype{gp lt axes}
\gpsetdashtype{gp dt axes}
\gpsetlinewidth{0.50}
\draw[gp path] (2.424,1.845)--(2.424,11.075);
\gpcolor{color=gp lt color border}
\gpsetlinetype{gp lt border}
\gpsetdashtype{gp dt solid}
\gpsetlinewidth{1.00}
\draw[gp path] (2.424,1.845)--(2.424,2.025);
\draw[gp path] (2.424,11.075)--(2.424,10.895);
\node[gp node left,rotate=270] at (2.424,1.661) {$0$};
\gpcolor{color=gp lt color axes}
\gpsetlinetype{gp lt axes}
\gpsetdashtype{gp dt axes}
\gpsetlinewidth{0.50}
\draw[gp path] (4.052,1.845)--(4.052,11.075);
\gpcolor{color=gp lt color border}
\gpsetlinetype{gp lt border}
\gpsetdashtype{gp dt solid}
\gpsetlinewidth{1.00}
\draw[gp path] (4.052,1.845)--(4.052,2.025);
\draw[gp path] (4.052,11.075)--(4.052,10.895);
\node[gp node left,rotate=270] at (4.052,1.661) {$50$};
\gpcolor{color=gp lt color axes}
\gpsetlinetype{gp lt axes}
\gpsetdashtype{gp dt axes}
\gpsetlinewidth{0.50}
\draw[gp path] (5.680,1.845)--(5.680,11.075);
\gpcolor{color=gp lt color border}
\gpsetlinetype{gp lt border}
\gpsetdashtype{gp dt solid}
\gpsetlinewidth{1.00}
\draw[gp path] (5.680,1.845)--(5.680,2.025);
\draw[gp path] (5.680,11.075)--(5.680,10.895);
\node[gp node left,rotate=270] at (5.680,1.661) {$100$};
\gpcolor{color=gp lt color axes}
\gpsetlinetype{gp lt axes}
\gpsetdashtype{gp dt axes}
\gpsetlinewidth{0.50}
\draw[gp path] (7.308,1.845)--(7.308,11.075);
\gpcolor{color=gp lt color border}
\gpsetlinetype{gp lt border}
\gpsetdashtype{gp dt solid}
\gpsetlinewidth{1.00}
\draw[gp path] (7.308,1.845)--(7.308,2.025);
\draw[gp path] (7.308,11.075)--(7.308,10.895);
\node[gp node left,rotate=270] at (7.308,1.661) {$150$};
\gpcolor{color=gp lt color axes}
\gpsetlinetype{gp lt axes}
\gpsetdashtype{gp dt axes}
\gpsetlinewidth{0.50}
\draw[gp path] (8.936,1.845)--(8.936,11.075);
\gpcolor{color=gp lt color border}
\gpsetlinetype{gp lt border}
\gpsetdashtype{gp dt solid}
\gpsetlinewidth{1.00}
\draw[gp path] (8.936,1.845)--(8.936,2.025);
\draw[gp path] (8.936,11.075)--(8.936,10.895);
\node[gp node left,rotate=270] at (8.936,1.661) {$200$};
\gpcolor{color=gp lt color axes}
\gpsetlinetype{gp lt axes}
\gpsetdashtype{gp dt axes}
\gpsetlinewidth{0.50}
\draw[gp path] (10.563,1.845)--(10.563,11.075);
\gpcolor{color=gp lt color border}
\gpsetlinetype{gp lt border}
\gpsetdashtype{gp dt solid}
\gpsetlinewidth{1.00}
\draw[gp path] (10.563,1.845)--(10.563,2.025);
\draw[gp path] (10.563,11.075)--(10.563,10.895);
\node[gp node left,rotate=270] at (10.563,1.661) {$250$};
\gpcolor{color=gp lt color axes}
\gpsetlinetype{gp lt axes}
\gpsetdashtype{gp dt axes}
\gpsetlinewidth{0.50}
\draw[gp path] (12.191,1.845)--(12.191,11.075);
\gpcolor{color=gp lt color border}
\gpsetlinetype{gp lt border}
\gpsetdashtype{gp dt solid}
\gpsetlinewidth{1.00}
\draw[gp path] (12.191,1.845)--(12.191,2.025);
\draw[gp path] (12.191,11.075)--(12.191,10.895);
\node[gp node left,rotate=270] at (12.191,1.661) {$300$};
\gpcolor{color=gp lt color axes}
\gpsetlinetype{gp lt axes}
\gpsetdashtype{gp dt axes}
\gpsetlinewidth{0.50}
\draw[gp path] (13.819,1.845)--(13.819,11.075);
\gpcolor{color=gp lt color border}
\gpsetlinetype{gp lt border}
\gpsetdashtype{gp dt solid}
\gpsetlinewidth{1.00}
\draw[gp path] (13.819,1.845)--(13.819,2.025);
\draw[gp path] (13.819,11.075)--(13.819,10.895);
\node[gp node left,rotate=270] at (13.819,1.661) {$350$};
\gpcolor{color=gp lt color axes}
\gpsetlinetype{gp lt axes}
\gpsetdashtype{gp dt axes}
\gpsetlinewidth{0.50}
\draw[gp path] (15.447,1.845)--(15.447,11.075);
\gpcolor{color=gp lt color border}
\gpsetlinetype{gp lt border}
\gpsetdashtype{gp dt solid}
\gpsetlinewidth{1.00}
\draw[gp path] (15.447,1.845)--(15.447,2.025);
\draw[gp path] (15.447,11.075)--(15.447,10.895);
\node[gp node left,rotate=270] at (15.447,1.661) {$400$};
\draw[gp path] (2.424,11.075)--(2.424,1.845)--(15.447,1.845)--(15.447,11.075)--cycle;
\gpcolor{rgb color={0.000,0.000,0.000}}
\gpsetlinewidth{2.00}
\draw[gp path] (2.424,7.213)--(2.424,7.073)--(2.457,7.053)--(2.457,7.025)--(2.457,7.010)%
  --(2.489,6.984)--(2.489,6.971)--(2.489,6.962)--(2.522,6.938)--(2.522,6.921)--(2.522,6.915)%
  --(2.554,6.890)--(2.554,6.848)--(2.554,6.826)--(2.587,6.801)--(2.587,6.788)--(2.587,6.739)%
  --(2.619,6.700)--(2.619,6.698)--(2.619,6.690)--(2.619,6.676)--(2.652,6.651)--(2.652,6.607)%
  --(2.652,6.596)--(2.684,6.595)--(2.684,6.535)--(2.684,6.531)--(2.684,6.496)--(2.717,6.492)%
  --(2.717,6.464)--(2.750,6.426)--(2.750,6.408)--(2.750,6.398)--(2.750,6.386)--(2.750,6.381)%
  --(2.782,6.374)--(2.782,6.364)--(2.782,6.345)--(2.815,6.338)--(2.815,6.333)--(2.847,6.304)%
  --(2.880,6.295)--(2.880,6.294)--(2.912,6.275)--(2.912,6.271)--(2.945,6.262)--(2.945,6.261)%
  --(2.977,6.260)--(2.977,6.254)--(3.010,6.226)--(3.043,6.222)--(3.075,6.214)--(3.108,6.189)%
  --(3.140,6.174)--(3.140,6.158)--(3.173,6.155)--(3.173,6.147)--(3.205,6.145)--(3.205,6.132)%
  --(3.238,6.104)--(3.270,6.097)--(3.336,6.072)--(3.336,6.060)--(3.368,6.051)--(3.368,6.049)%
  --(3.401,6.047)--(3.401,6.046)--(3.401,6.042)--(3.433,6.037)--(3.498,6.028)--(3.498,6.022)%
  --(3.531,6.022)--(3.531,6.018)--(3.564,6.017)--(3.564,6.012)--(3.596,6.008)--(3.596,5.998)%
  --(3.629,5.994)--(3.629,5.989)--(3.661,5.986)--(3.661,5.982)--(3.694,5.982)--(3.759,5.975)%
  --(3.759,5.966)--(3.759,6.043)--(3.759,6.018)--(3.791,6.007)--(3.791,5.978)--(3.791,5.963)%
  --(3.791,5.936)--(3.791,5.935)--(3.791,5.919)--(3.824,5.891)--(3.824,5.868)--(3.824,5.866)%
  --(3.824,5.857)--(3.857,5.840)--(3.857,5.836)--(3.857,5.828)--(3.857,5.817)--(3.889,5.816)%
  --(3.889,5.815)--(3.889,5.808)--(3.889,5.805)--(3.922,5.801)--(3.922,5.795)--(3.922,5.792)%
  --(3.922,5.785)--(3.922,5.782)--(3.954,5.780)--(3.954,5.777)--(3.954,5.774)--(3.987,5.773)%
  --(3.987,5.769)--(3.987,5.764)--(3.987,5.750)--(4.019,5.744)--(4.019,5.742)--(4.019,5.738)%
  --(4.019,5.734)--(4.052,5.741)--(4.052,5.739)--(4.084,5.738)--(4.084,5.736)--(4.084,5.729)%
  --(4.117,5.728)--(4.117,5.727)--(4.117,5.721)--(4.182,5.702)--(4.182,5.710)--(4.182,5.707)%
  --(4.215,5.723)--(4.215,5.716)--(4.247,5.723)--(4.247,5.719)--(4.247,5.716)--(4.247,5.713)%
  --(4.280,5.706)--(4.280,5.717)--(4.312,5.719)--(4.312,5.717)--(4.312,5.716)--(4.377,5.715)%
  --(4.410,5.721)--(4.443,5.720)--(4.443,5.719)--(4.475,5.717)--(4.508,5.716)--(4.508,5.713)%
  --(4.540,5.707)--(4.540,5.702)--(4.573,5.703)--(4.573,5.706)--(4.573,5.704)--(4.605,5.702)%
  --(4.605,5.699)--(4.605,5.697)--(4.638,5.698)--(4.703,5.697)--(4.801,5.696)--(4.801,5.692)%
  --(4.866,5.705)--(4.898,5.703)--(4.898,5.679)--(4.963,5.698)--(5.029,5.696)--(5.061,5.694)%
  --(5.094,5.672)--(5.126,5.672)--(5.159,5.669)--(5.289,5.669)--(5.289,5.667)--(5.354,5.666)%
  --(5.354,5.668)--(5.387,5.666)--(5.419,5.664)--(5.452,3.347)--(5.484,3.339)--(5.484,3.336)%
  --(5.484,3.240)--(5.484,3.233)--(5.517,3.252)--(5.517,3.249)--(5.517,3.243)--(5.517,3.240)%
  --(5.550,3.238)--(5.550,3.225)--(5.550,3.219)--(5.550,3.215)--(5.550,3.213)--(5.582,3.211)%
  --(5.582,3.209)--(5.615,3.137)--(5.615,3.029)--(5.615,3.027)--(5.647,3.070)--(5.647,3.067)%
  --(5.647,3.062)--(5.647,3.061)--(5.680,3.056)--(5.680,3.053)--(5.712,3.050)--(5.712,3.048)%
  --(5.712,3.045)--(5.745,3.043)--(5.745,3.042)--(5.745,3.040)--(5.745,3.038)--(5.777,3.037)%
  --(5.810,3.034)--(5.810,3.033)--(5.810,2.972)--(5.810,2.971)--(5.843,2.970)--(5.843,2.969)%
  --(5.875,2.967)--(5.908,2.965)--(5.908,2.964)--(5.940,2.963)--(6.005,2.961)--(6.038,2.954)%
  --(6.038,2.951)--(6.103,2.948)--(6.103,2.947)--(6.103,2.945)--(6.136,2.945)--(6.168,2.944)%
  --(6.168,2.943)--(6.201,2.939)--(6.331,2.928)--(6.363,2.927)--(6.363,2.923)--(6.396,2.921)%
  --(6.396,2.920)--(6.429,2.919)--(6.461,2.917)--(6.494,2.917)--(6.526,2.916)--(6.559,2.914)%
  --(6.559,2.913)--(6.591,2.912)--(6.624,2.911)--(6.656,2.911)--(6.754,2.910)--(6.852,2.910)%
  --(6.982,2.908)--(7.047,2.905)--(7.112,2.905)--(7.112,2.903)--(7.145,2.903)--(7.177,2.903)%
  --(7.275,2.903)--(7.308,2.902)--(7.405,2.902)--(7.470,2.901)--(7.536,2.900)--(7.568,2.899)%
  --(7.601,2.899)--(7.633,2.897)--(7.666,2.896)--(7.731,2.895)--(7.796,2.894)--(7.829,2.894)%
  --(7.926,2.893)--(8.056,2.891)--(8.122,2.899)--(8.154,2.899)--(8.187,2.899)--(8.187,2.898)%
  --(8.349,2.896)--(8.447,2.896)--(8.480,2.895)--(8.610,2.894)--(8.675,2.894)--(8.740,6.342)%
  --(8.740,6.268)--(8.740,6.200)--(8.740,6.166)--(8.773,6.101)--(8.773,6.057)--(8.773,6.008)%
  --(8.773,5.924)--(8.773,5.888)--(8.773,5.875)--(8.805,5.862)--(8.805,5.861)--(8.805,5.828)%
  --(8.805,5.806)--(8.805,5.777)--(8.838,5.755)--(8.838,5.736)--(8.838,5.725)--(8.838,5.717)%
  --(8.838,5.715)--(8.870,5.712)--(8.870,5.665)--(8.903,5.665)--(8.903,5.664)--(8.903,5.660)%
  --(8.936,5.655)--(8.936,5.653)--(8.936,5.647)--(8.968,5.647)--(8.968,5.644)--(8.968,5.643)%
  --(9.001,5.638)--(9.001,5.630)--(9.098,5.623)--(9.196,5.619)--(9.196,5.617)--(9.229,5.615)%
  --(9.229,5.613)--(9.261,5.609)--(9.261,5.608)--(9.294,5.608)--(9.326,5.607)--(9.359,5.604)%
  --(9.424,5.602)--(9.424,5.600)--(9.456,5.600)--(9.456,5.595)--(9.489,5.591)--(9.489,5.589)%
  --(9.522,5.588)--(9.554,5.573)--(9.587,5.569)--(9.619,5.568)--(9.684,5.568)--(9.684,5.565)%
  --(9.717,5.564)--(9.717,5.562)--(9.749,5.562)--(9.749,5.561)--(9.782,5.560)--(9.815,5.559)%
  --(9.815,5.557)--(9.847,5.556)--(9.880,5.553)--(9.945,5.551)--(10.042,5.550)--(10.075,5.547)%
  --(10.108,5.545)--(10.108,5.544)--(10.173,5.543)--(10.173,5.541)--(10.205,5.541)--(10.335,5.537)%
  --(10.563,5.535)--(10.628,5.532)--(10.661,5.531)--(10.661,5.530)--(10.694,5.528)--(10.726,5.526)%
  --(10.759,5.525)--(10.759,5.520)--(10.856,5.520)--(10.954,5.520)--(11.019,5.517)--(11.215,5.512)%
  --(11.247,5.512)--(11.345,5.509)--(11.605,5.508)--(11.638,5.508)--(11.670,5.507)--(11.735,5.507)%
  --(11.735,5.506)--(11.768,5.505)--(11.768,5.504)--(11.801,5.504)--(11.801,5.503)--(11.931,5.502)%
  --(11.931,5.500)--(11.963,5.500)--(11.996,10.368)--(11.996,10.350)--(12.028,10.291)--(12.028,10.272)%
  --(12.028,10.231)--(12.028,10.225)--(12.061,10.197)--(12.094,10.197)--(12.094,10.189)--(12.094,10.188)%
  --(12.126,10.162)--(12.159,10.161)--(12.224,10.159)--(12.224,10.158)--(12.321,10.133)--(12.321,10.123)%
  --(12.354,10.121)--(12.387,10.099)--(12.387,10.094)--(12.484,10.087)--(12.712,10.085)--(12.842,10.082)%
  --(13.363,10.081)--(13.884,10.081);
\gpcolor{color=gp lt color border}
\gpsetlinewidth{1.00}
\draw[gp path] (2.424,11.075)--(2.424,1.845)--(15.447,1.845)--(15.447,11.075)--cycle;
\node[gp node center,rotate=-270] at (-0.812,6.460) {Objective value};
\node[gp node center] at (8.935,0.215) {Relative time ($\tau$) [S]};
\node[gp node center] at (8.935,11.537) {Objective value for Weekly Schedule};
%% coordinates of the plot area
\gpdefrectangularnode{gp plot 1}{\pgfpoint{2.424cm}{1.845cm}}{\pgfpoint{15.447cm}{11.075cm}}
\endtikzpicture
%% gnuplot variables

	}
	\caption{Figure Caption}
	\label{fig:responses:resource-subtraction}
\end{figure}


\subsection{Response to Changes in Work Order Values}\label{sec:results:strategic_value_changes}
The final parameter that will be changed is the work order value $
\ParStrategicValue$. Table~\ref{tab:responses:value_change} details the
pertubations that the AbLNS will by affected by. On each iteration 100 work
orders are having their values changed by the amount  shown in the 4th row of
table~\ref{tab:responses:value_change}.

\begin{table}[H]
	\centering
	\begin{tabular}{|c|c|c|c|c|c|}
	\hline
	                                                                                                                   & \begin{tabular}[c]{@{}c@{}}$\VarMetaTime_1 = 60$\end{tabular} & \begin{tabular}[c]{@{}c@{}}$\VarMetaTime_2 = 120$\end{tabular} & \begin{tabular}[c]{@{}c@{}}$\VarMetaTime_3 = 180$\end{tabular} & \begin{tabular}[c]{@{}c@{}}$\VarMetaTime_4 = 240$\end{tabular} & \begin{tabular}[c]{@{}c@{}}$\VarMetaTime_5 = 300$\end{tabular} \\ \hline
	$\Delta |\SetWorkOrder[\VarMetaTime_{n}]{} \triangle \SetWorkOrder[\VarMetaTime_{n-1}]{}              |$           & 100                                                           & 100                                                            & 100                                                            & 100                                                            & 100                                                      \\ \hline
	$\Delta |\SetPeriod[\VarMetaTime_{n}]{} \triangle \SetPeriod[\VarMetaTime_{n-1}]{}                    |$           & 52                                                            & 52                                                             & 52                                                             & 52                                                             & 52                                                       \\ \hline
	\makecell{$\Delta |\ParStrategicValue[\VarMetaTime_{n}]| -$\\ $|\ParStrategicValue[\VarMetaTime_{n - 1}]|$}        & $3.75 \cdot 10^{7}$                                           & $3.75 \cdot 10^{7}$                                            & $3.75 \cdot 10^{7}$                                            & $3.75 \cdot 10^{7}$                                            & $3.75 \cdot 10^{7}$                                       \\ \hline
	\end{tabular}
	\label{tab:responses:value_change}
\end{table}

Figure~\ref{fig:responses:value_change} shows the effects of
pertubing the AbLNS by changing the $\ParStrategicValue$ parameter in the objective
function~\ref{eqn:objective:strategic} which specifies the value of assigning a
work order to a specific period.

\begin{figure}[H]%% placement specifier
	\centering
	\definecolor{gp lt10cm}{named}{dtu-red}
	\resizebox{\linewidth}{!}{
		\tikzpicture[gnuplot]
%% generated with GNUPLOT 6.0p1 (Lua 5.2; terminal rev. Jun 2020, script rev. 118)
%% Mon 13 Jan 2025 06:26:31 PM UTC
\tikzset{every node/.append style={scale=1.00}}
\path (0.000,0.000) rectangle (16.000,12.000);
\gpcolor{rgb color={0.753,0.753,0.753}}
\gpsetlinetype{gp lt border}
\gpsetdashtype{gp dt solid}
\gpsetlinewidth{2.00}
\draw[gp path] (2.240,1.845)--(13.667,1.845);
\gpcolor{color=gp lt color border}
\gpsetlinewidth{1.00}
\draw[gp path] (2.240,1.845)--(2.420,1.845);
\node[gp node right] at (2.056,2.153) {$9\times10^{10}$};
\gpcolor{rgb color={0.753,0.753,0.753}}
\gpsetlinewidth{2.00}
\draw[gp path] (2.240,2.753)--(13.667,2.753);
\gpcolor{color=gp lt color border}
\gpsetlinewidth{1.00}
\draw[gp path] (2.240,2.753)--(2.420,2.753);
\node[gp node right] at (2.056,3.061) {$1\times10^{11}$};
\gpcolor{rgb color={0.753,0.753,0.753}}
\gpsetlinewidth{2.00}
\draw[gp path] (2.240,3.660)--(13.667,3.660);
\gpcolor{color=gp lt color border}
\gpsetlinewidth{1.00}
\draw[gp path] (2.240,3.660)--(2.420,3.660);
\node[gp node right] at (2.056,3.968) {$1.1\times10^{11}$};
\gpcolor{rgb color={0.753,0.753,0.753}}
\gpsetlinewidth{2.00}
\draw[gp path] (2.240,4.568)--(13.667,4.568);
\gpcolor{color=gp lt color border}
\gpsetlinewidth{1.00}
\draw[gp path] (2.240,4.568)--(2.420,4.568);
\node[gp node right] at (2.056,4.876) {$1.2\times10^{11}$};
\gpcolor{rgb color={0.753,0.753,0.753}}
\gpsetlinewidth{2.00}
\draw[gp path] (2.240,5.475)--(13.667,5.475);
\gpcolor{color=gp lt color border}
\gpsetlinewidth{1.00}
\draw[gp path] (2.240,5.475)--(2.420,5.475);
\node[gp node right] at (2.056,5.783) {$1.3\times10^{11}$};
\gpcolor{rgb color={0.753,0.753,0.753}}
\gpsetlinewidth{2.00}
\draw[gp path] (2.240,6.383)--(13.667,6.383);
\gpcolor{color=gp lt color border}
\gpsetlinewidth{1.00}
\draw[gp path] (2.240,6.383)--(2.420,6.383);
\node[gp node right] at (2.056,6.691) {$1.4\times10^{11}$};
\gpcolor{rgb color={0.753,0.753,0.753}}
\gpsetlinewidth{2.00}
\draw[gp path] (2.240,7.291)--(13.667,7.291);
\gpcolor{color=gp lt color border}
\gpsetlinewidth{1.00}
\draw[gp path] (2.240,7.291)--(2.420,7.291);
\node[gp node right] at (2.056,7.599) {$1.5\times10^{11}$};
\gpcolor{rgb color={0.753,0.753,0.753}}
\gpsetlinewidth{2.00}
\draw[gp path] (2.240,8.198)--(13.667,8.198);
\gpcolor{color=gp lt color border}
\gpsetlinewidth{1.00}
\draw[gp path] (2.240,8.198)--(2.420,8.198);
\node[gp node right] at (2.056,8.506) {$1.6\times10^{11}$};
\gpcolor{rgb color={0.753,0.753,0.753}}
\gpsetlinewidth{2.00}
\draw[gp path] (2.240,9.106)--(13.667,9.106);
\gpcolor{color=gp lt color border}
\gpsetlinewidth{1.00}
\draw[gp path] (2.240,9.106)--(2.420,9.106);
\node[gp node right] at (2.056,9.414) {$1.7\times10^{11}$};
\gpcolor{rgb color={0.753,0.753,0.753}}
\gpsetlinewidth{2.00}
\draw[gp path] (2.240,10.013)--(13.667,10.013);
\gpcolor{color=gp lt color border}
\gpsetlinewidth{1.00}
\draw[gp path] (2.240,10.013)--(2.420,10.013);
\node[gp node right] at (2.056,10.321) {$1.8\times10^{11}$};
\gpcolor{rgb color={0.753,0.753,0.753}}
\gpsetlinewidth{2.00}
\draw[gp path] (2.240,10.921)--(13.667,10.921);
\gpcolor{color=gp lt color border}
\gpsetlinewidth{1.00}
\draw[gp path] (2.240,10.921)--(2.420,10.921);
\node[gp node right] at (2.056,11.229) {$1.9\times10^{11}$};
\gpcolor{rgb color={0.753,0.753,0.753}}
\gpsetlinewidth{2.00}
\draw[gp path] (2.240,1.845)--(2.240,10.921);
\gpcolor{color=gp lt color border}
\gpsetlinewidth{1.00}
\draw[gp path] (2.240,1.845)--(2.240,2.025);
\draw[gp path] (2.240,10.921)--(2.240,10.741);
\node[gp node left,rotate=270] at (2.240,1.661) {$0$};
\gpcolor{rgb color={0.753,0.753,0.753}}
\gpsetlinewidth{2.00}
\draw[gp path] (3.872,1.845)--(3.872,10.921);
\gpcolor{color=gp lt color border}
\gpsetlinewidth{1.00}
\draw[gp path] (3.872,1.845)--(3.872,2.025);
\draw[gp path] (3.872,10.921)--(3.872,10.741);
\node[gp node left,rotate=270] at (3.872,1.661) {$60$};
\gpcolor{rgb color={0.753,0.753,0.753}}
\gpsetlinewidth{2.00}
\draw[gp path] (5.505,1.845)--(5.505,10.921);
\gpcolor{color=gp lt color border}
\gpsetlinewidth{1.00}
\draw[gp path] (5.505,1.845)--(5.505,2.025);
\draw[gp path] (5.505,10.921)--(5.505,10.741);
\node[gp node left,rotate=270] at (5.505,1.661) {$120$};
\gpcolor{rgb color={0.753,0.753,0.753}}
\gpsetlinewidth{2.00}
\draw[gp path] (7.137,1.845)--(7.137,10.921);
\gpcolor{color=gp lt color border}
\gpsetlinewidth{1.00}
\draw[gp path] (7.137,1.845)--(7.137,2.025);
\draw[gp path] (7.137,10.921)--(7.137,10.741);
\node[gp node left,rotate=270] at (7.137,1.661) {$180$};
\gpcolor{rgb color={0.753,0.753,0.753}}
\gpsetlinewidth{2.00}
\draw[gp path] (8.770,1.845)--(8.770,10.921);
\gpcolor{color=gp lt color border}
\gpsetlinewidth{1.00}
\draw[gp path] (8.770,1.845)--(8.770,2.025);
\draw[gp path] (8.770,10.921)--(8.770,10.741);
\node[gp node left,rotate=270] at (8.770,1.661) {$240$};
\gpcolor{rgb color={0.753,0.753,0.753}}
\gpsetlinewidth{2.00}
\draw[gp path] (10.402,1.845)--(10.402,10.921);
\gpcolor{color=gp lt color border}
\gpsetlinewidth{1.00}
\draw[gp path] (10.402,1.845)--(10.402,2.025);
\draw[gp path] (10.402,10.921)--(10.402,10.741);
\node[gp node left,rotate=270] at (10.402,1.661) {$300$};
\gpcolor{rgb color={0.753,0.753,0.753}}
\gpsetlinewidth{2.00}
\draw[gp path] (12.035,1.845)--(12.035,10.921);
\gpcolor{color=gp lt color border}
\gpsetlinewidth{1.00}
\draw[gp path] (12.035,1.845)--(12.035,2.025);
\draw[gp path] (12.035,10.921)--(12.035,10.741);
\node[gp node left,rotate=270] at (12.035,1.661) {$360$};
\gpcolor{rgb color={0.753,0.753,0.753}}
\gpsetlinewidth{2.00}
\draw[gp path] (13.667,1.845)--(13.667,10.921);
\gpcolor{color=gp lt color border}
\gpsetlinewidth{1.00}
\draw[gp path] (13.667,1.845)--(13.667,2.025);
\draw[gp path] (13.667,10.921)--(13.667,10.741);
\node[gp node left,rotate=270] at (13.667,1.661) {$420$};
\draw[gp path] (13.667,1.845)--(13.487,1.845);
\node[gp node left] at (13.851,2.153) {$163000$};
\draw[gp path] (13.667,3.358)--(13.487,3.358);
\node[gp node left] at (13.851,3.666) {$164000$};
\draw[gp path] (13.667,4.870)--(13.487,4.870);
\node[gp node left] at (13.851,5.178) {$165000$};
\draw[gp path] (13.667,6.383)--(13.487,6.383);
\node[gp node left] at (13.851,6.691) {$166000$};
\draw[gp path] (13.667,7.896)--(13.487,7.896);
\node[gp node left] at (13.851,8.204) {$167000$};
\draw[gp path] (13.667,9.408)--(13.487,9.408);
\node[gp node left] at (13.851,9.716) {$168000$};
\draw[gp path] (13.667,10.921)--(13.487,10.921);
\node[gp node left] at (13.851,11.229) {$169000$};
\draw[gp path] (2.240,10.921)--(2.240,1.845)--(13.667,1.845)--(13.667,10.921)--cycle;
\draw[gp path] (1.517,11.820)--(1.517,12.590)--(14.389,12.590)--(14.389,11.820)--cycle;
\node[gp node right] at (6.669,12.205) {Strategic Urgency};
\gpcolor{rgb color={0.000,0.000,0.000}}
\gpsetlinewidth{2.50}
\draw[gp path] (6.853,12.205)--(7.769,12.205);
\draw[gp path] (2.240,2.378)--(2.240,2.376)--(2.240,2.384)--(2.267,2.385)--(2.267,2.392)%
  --(2.267,2.395)--(2.267,2.396)--(2.267,2.414)--(2.294,2.415)--(2.294,2.418)--(2.294,2.417)%
  --(2.294,2.419)--(2.294,2.423)--(2.322,2.428)--(2.322,2.432)--(2.322,2.436)--(2.322,2.437)%
  --(2.322,2.438)--(2.349,2.439)--(2.349,2.443)--(2.349,2.445)--(2.349,2.444)--(2.349,2.445)%
  --(2.376,2.446)--(2.376,2.447)--(2.376,2.449)--(2.376,2.452)--(2.403,2.459)--(2.403,2.460)%
  --(2.403,2.462)--(2.403,2.465)--(2.403,2.468)--(2.403,2.471)--(2.430,2.471)--(2.430,2.487)%
  --(2.430,2.481)--(2.430,2.485)--(2.430,2.487)--(2.430,2.494)--(2.430,2.493)--(2.458,2.497)%
  --(2.458,2.498)--(2.458,2.505)--(2.458,2.510)--(2.458,2.515)--(2.458,2.522)--(2.485,2.519)%
  --(2.485,2.532)--(2.485,2.531)--(2.485,2.527)--(2.512,2.526)--(2.512,2.524)--(2.512,2.528)%
  --(2.512,2.524)--(2.512,2.526)--(2.539,2.526)--(2.539,2.528)--(2.566,2.533)--(2.566,2.535)%
  --(2.566,2.549)--(2.566,2.546)--(2.594,2.545)--(2.594,2.547)--(2.594,2.546)--(2.594,2.544)%
  --(2.594,2.541)--(2.621,2.539)--(2.621,2.538)--(2.621,2.540)--(2.648,2.540)--(2.648,2.539)%
  --(2.648,2.537)--(2.675,2.536)--(2.675,2.537)--(2.675,2.539)--(2.675,2.553)--(2.703,2.552)%
  --(2.703,2.556)--(2.703,2.555)--(2.730,2.554)--(2.730,2.552)--(2.730,2.551)--(2.757,2.550)%
  --(2.784,2.549)--(2.784,2.545)--(2.811,2.551)--(2.839,2.548)--(2.839,2.549)--(2.839,2.545)%
  --(2.866,2.556)--(2.866,2.554)--(2.893,2.553)--(2.893,2.555)--(2.920,2.559)--(2.920,2.561)%
  --(2.920,2.559)--(2.920,2.555)--(2.947,2.553)--(2.947,2.552)--(2.947,2.548)--(2.947,2.549)%
  --(2.975,2.548)--(2.975,2.546)--(3.002,2.542)--(3.002,2.543)--(3.002,2.541)--(3.056,2.536)%
  --(3.083,2.534)--(3.111,2.535)--(3.138,2.535)--(3.192,2.534)--(3.219,2.537)--(3.247,2.534)%
  --(3.247,2.532)--(3.274,2.530)--(3.301,2.529)--(3.301,2.527)--(3.301,2.529)--(3.301,2.530)%
  --(3.328,2.527)--(3.383,2.526)--(3.383,2.525)--(3.410,2.534)--(3.464,2.532)--(3.492,2.523)%
  --(3.519,2.523)--(3.519,2.522)--(3.519,2.521)--(3.573,2.519)--(3.573,2.518)--(3.628,2.517)%
  --(3.655,2.516)--(3.682,2.514)--(3.709,2.513)--(3.736,2.513)--(3.818,2.512)--(3.845,2.515)%
  --(3.845,2.513)--(3.872,2.512)--(3.900,2.511)--(3.954,3.525)--(3.954,3.521)--(3.954,3.517)%
  --(3.954,3.515)--(3.981,3.515)--(3.981,3.507)--(3.981,3.504)--(4.008,3.501)--(4.036,3.501)%
  --(4.036,3.497)--(4.036,3.507)--(4.063,3.507)--(4.063,3.509)--(4.090,3.508)--(4.090,3.506)%
  --(4.090,3.496)--(4.117,3.489)--(4.117,3.487)--(4.117,3.486)--(4.145,3.484)--(4.145,3.482)%
  --(4.172,3.480)--(4.172,3.482)--(4.172,3.471)--(4.199,3.472)--(4.199,3.470)--(4.199,3.475)%
  --(4.199,3.474)--(4.226,3.473)--(4.226,3.475)--(4.226,3.476)--(4.226,3.477)--(4.253,3.476)%
  --(4.253,3.471)--(4.253,3.470)--(4.253,3.471)--(4.281,3.472)--(4.308,3.465)--(4.335,3.464)%
  --(4.362,3.464)--(4.362,3.466)--(4.362,3.467)--(4.389,3.464)--(4.417,3.458)--(4.417,3.460)%
  --(4.444,3.462)--(4.444,3.463)--(4.444,3.461)--(4.498,3.462)--(4.525,3.461)--(4.553,3.460)%
  --(4.553,3.458)--(4.580,3.459)--(4.607,3.459)--(4.607,3.460)--(4.634,3.458)--(4.634,3.454)%
  --(4.661,3.452)--(4.689,3.454)--(4.689,3.452)--(4.716,3.449)--(4.716,3.450)--(4.852,3.449)%
  --(4.906,3.450)--(4.906,3.446)--(4.934,3.446)--(4.988,3.445)--(5.015,3.444)--(5.124,3.440)%
  --(5.151,3.439)--(5.151,3.438)--(5.178,3.436)--(5.233,3.435)--(5.287,3.437)--(5.342,3.437)%
  --(5.369,3.434)--(5.396,3.433)--(5.423,3.432)--(5.478,3.432)--(5.532,3.431)--(5.586,4.769)%
  --(5.586,4.767)--(5.586,4.760)--(5.586,4.758)--(5.614,4.755)--(5.614,4.752)--(5.614,4.751)%
  --(5.614,4.712)--(5.614,4.706)--(5.641,4.708)--(5.641,4.711)--(5.641,4.712)--(5.641,4.703)%
  --(5.668,4.697)--(5.668,4.696)--(5.668,4.699)--(5.668,4.687)--(5.668,4.684)--(5.695,4.683)%
  --(5.695,4.693)--(5.695,4.694)--(5.723,4.693)--(5.723,4.690)--(5.723,4.696)--(5.750,4.697)%
  --(5.750,4.707)--(5.777,4.706)--(5.777,4.699)--(5.804,4.704)--(5.804,4.701)--(5.804,4.699)%
  --(5.831,4.700)--(5.831,4.702)--(5.831,4.699)--(5.859,4.699)--(5.859,4.691)--(5.859,4.693)%
  --(5.859,4.692)--(5.886,4.674)--(5.913,4.674)--(5.913,4.678)--(5.913,4.681)--(5.940,4.687)%
  --(5.940,4.689)--(5.967,4.685)--(5.967,4.684)--(5.995,4.683)--(6.022,4.680)--(6.049,4.680)%
  --(6.049,4.679)--(6.076,4.678)--(6.076,4.677)--(6.103,4.675)--(6.131,4.673)--(6.212,4.670)%
  --(6.212,4.668)--(6.212,4.667)--(6.239,4.663)--(6.267,4.659)--(6.321,4.662)--(6.321,4.664)%
  --(6.348,4.665)--(6.348,4.666)--(6.348,4.665)--(6.375,4.666)--(6.375,4.665)--(6.403,4.663)%
  --(6.430,4.660)--(6.430,4.657)--(6.457,4.656)--(6.484,4.654)--(6.484,4.653)--(6.512,4.656)%
  --(6.512,4.655)--(6.512,4.652)--(6.593,4.651)--(6.648,4.650)--(6.675,4.650)--(6.675,4.649)%
  --(6.784,4.649)--(6.811,4.647)--(6.865,4.643)--(6.892,4.641)--(6.947,4.642)--(7.083,4.640)%
  --(7.110,4.639)--(7.137,4.638)--(7.219,4.636)--(7.219,5.523)--(7.219,5.515)--(7.219,5.513)%
  --(7.246,5.513)--(7.246,5.515)--(7.246,5.511)--(7.273,5.510)--(7.273,5.508)--(7.273,5.504)%
  --(7.273,5.508)--(7.301,5.510)--(7.301,5.508)--(7.328,5.507)--(7.328,5.491)--(7.328,5.490)%
  --(7.355,5.482)--(7.355,5.487)--(7.382,5.487)--(7.382,5.489)--(7.382,5.486)--(7.409,5.483)%
  --(7.409,5.489)--(7.409,5.487)--(7.437,5.473)--(7.464,5.472)--(7.464,5.469)--(7.464,5.468)%
  --(7.491,5.467)--(7.491,5.465)--(7.491,5.460)--(7.518,5.451)--(7.518,5.450)--(7.545,5.459)%
  --(7.545,5.458)--(7.545,5.456)--(7.573,5.458)--(7.573,5.452)--(7.600,5.450)--(7.627,5.450)%
  --(7.654,5.448)--(7.681,5.446)--(7.681,5.447)--(7.681,5.446)--(7.709,5.445)--(7.763,5.443)%
  --(7.790,5.442)--(7.872,5.439)--(7.899,5.442)--(7.899,5.443)--(7.926,5.436)--(8.008,5.439)%
  --(8.008,5.438)--(8.008,5.445)--(8.035,5.434)--(8.090,5.431)--(8.117,5.432)--(8.144,5.430)%
  --(8.226,5.429)--(8.253,5.430)--(8.307,5.427)--(8.362,5.427)--(8.389,5.426)--(8.389,5.424)%
  --(8.389,5.421)--(8.389,5.417)--(8.443,5.416)--(8.443,5.422)--(8.470,5.421)--(8.525,5.421)%
  --(8.552,5.422)--(8.579,5.421)--(8.606,5.427)--(8.606,5.426)--(8.661,5.423)--(8.661,5.421)%
  --(8.770,5.420)--(8.824,5.419)--(8.851,5.419)--(8.879,8.433)--(8.879,8.429)--(8.879,8.425)%
  --(8.879,8.422)--(8.879,8.424)--(8.879,8.423)--(8.906,8.419)--(8.906,8.416)--(8.906,8.417)%
  --(8.906,8.425)--(8.933,8.429)--(8.933,8.425)--(8.933,8.421)--(8.933,8.415)--(8.933,8.411)%
  --(8.960,8.403)--(8.960,8.395)--(8.987,8.394)--(8.987,8.395)--(8.987,8.376)--(8.987,8.388)%
  --(9.015,8.379)--(9.015,8.378)--(9.015,8.372)--(9.042,8.369)--(9.042,8.367)--(9.069,8.364)%
  --(9.069,8.365)--(9.096,8.362)--(9.123,8.360)--(9.178,8.359)--(9.205,8.359)--(9.205,8.361)%
  --(9.232,8.359)--(9.259,8.357)--(9.259,8.354)--(9.314,8.362)--(9.314,8.360)--(9.314,8.359)%
  --(9.341,8.354)--(9.341,8.353)--(9.341,8.354)--(9.368,8.354)--(9.395,8.358)--(9.395,8.359)%
  --(9.423,8.358)--(9.450,8.363)--(9.450,8.364)--(9.477,8.363)--(9.613,8.361)--(9.613,8.362)%
  --(9.722,8.358)--(9.858,8.356)--(9.912,8.352)--(9.912,8.351)--(9.994,8.351)--(9.994,8.349)%
  --(10.021,8.347)--(10.021,8.339)--(10.048,8.336)--(10.130,8.334)--(10.157,8.334)--(10.266,8.340)%
  --(10.266,8.337)--(10.348,8.336)--(10.429,8.342)--(10.484,8.340)--(10.511,10.545)--(10.511,10.536)%
  --(10.511,10.534)--(10.538,10.534)--(10.538,10.531)--(10.538,10.525)--(10.565,10.532)--(10.565,10.528)%
  --(10.593,10.528)--(10.593,10.523)--(10.593,10.520)--(10.620,10.517)--(10.620,10.518)--(10.620,10.513)%
  --(10.647,10.511)--(10.647,10.510)--(10.647,10.513)--(10.674,10.511)--(10.674,10.516)--(10.701,10.516)%
  --(10.701,10.513)--(10.756,10.512)--(10.783,10.516)--(10.783,10.513)--(10.783,10.511)--(10.837,10.507)%
  --(10.837,10.504)--(10.865,10.503)--(10.892,10.504)--(10.919,10.504)--(10.919,10.506)--(10.919,10.504)%
  --(10.919,10.494)--(10.946,10.490)--(10.973,10.488)--(11.055,10.482)--(11.082,10.482)--(11.110,10.481)%
  --(11.246,10.482)--(11.300,10.480)--(11.327,10.478)--(11.354,10.476)--(11.409,10.475)--(11.436,10.475)%
  --(11.463,10.470)--(11.490,10.474)--(11.518,10.474)--(11.545,10.469)--(11.572,10.467)--(11.572,10.469)%
  --(11.626,10.469)--(11.681,10.470)--(11.708,10.470)--(11.844,10.470)--(11.844,10.469)--(11.844,10.467)%
  --(11.926,10.466)--(11.980,10.469)--(12.007,10.468)--(12.062,10.463);
\gpcolor{color=gp lt color border}
\node[gp node right] at (13.105,12.205) {Strategic Resource Penalty};
\gpcolor{rgb color={0.184,0.243,0.918}}
\draw[gp path] (13.289,12.205)--(14.205,12.205);
\draw[gp path] (2.240,10.278)--(2.240,10.122)--(2.240,9.938)--(2.267,9.726)--(2.267,9.653)%
  --(2.267,9.495)--(2.267,9.287)--(2.267,9.154)--(2.294,9.059)--(2.294,8.893)--(2.294,8.717)%
  --(2.294,8.631)--(2.294,8.313)--(2.322,8.292)--(2.322,8.192)--(2.322,8.056)--(2.322,7.933)%
  --(2.322,7.847)--(2.349,7.809)--(2.349,7.643)--(2.349,7.589)--(2.349,7.555)--(2.349,7.454)%
  --(2.376,7.452)--(2.376,7.428)--(2.376,7.244)--(2.376,7.153)--(2.403,7.027)--(2.403,6.974)%
  --(2.403,6.835)--(2.403,6.676)--(2.403,6.611)--(2.403,6.483)--(2.430,6.451)--(2.430,6.359)%
  --(2.430,6.338)--(2.430,6.324)--(2.430,6.280)--(2.430,6.227)--(2.430,6.185)--(2.458,6.162)%
  --(2.458,6.146)--(2.458,6.102)--(2.458,6.058)--(2.458,5.975)--(2.458,5.929)--(2.458,5.885)%
  --(2.485,5.813)--(2.485,5.698)--(2.485,5.680)--(2.485,5.668)--(2.512,5.616)--(2.512,5.605)%
  --(2.512,5.542)--(2.512,5.491)--(2.512,5.451)--(2.539,5.427)--(2.539,5.406)--(2.539,5.341)%
  --(2.566,5.321)--(2.566,5.291)--(2.566,5.289)--(2.566,5.274)--(2.594,5.273)--(2.594,5.217)%
  --(2.594,5.202)--(2.594,5.156)--(2.594,5.137)--(2.621,5.085)--(2.621,5.079)--(2.621,5.059)%
  --(2.648,5.059)--(2.648,5.020)--(2.648,4.993)--(2.675,4.981)--(2.675,4.963)--(2.675,4.946)%
  --(2.675,4.928)--(2.675,4.693)--(2.703,4.678)--(2.703,4.394)--(2.730,4.364)--(2.730,4.355)%
  --(2.757,4.341)--(2.784,4.338)--(2.784,4.333)--(2.811,4.320)--(2.839,4.320)--(2.839,4.296)%
  --(2.866,4.285)--(2.866,4.279)--(2.893,4.279)--(2.893,4.268)--(2.920,4.262)--(2.920,3.948)%
  --(2.920,3.905)--(2.920,3.904)--(2.947,3.904)--(2.947,3.899)--(2.947,3.895)--(2.975,3.895)%
  --(2.975,3.893)--(3.002,3.867)--(3.002,3.864)--(3.002,3.825)--(3.056,3.810)--(3.083,3.810)%
  --(3.083,3.786)--(3.111,3.778)--(3.138,3.768)--(3.192,3.768)--(3.219,3.766)--(3.247,3.765)%
  --(3.274,3.765)--(3.301,3.765)--(3.301,3.749)--(3.301,3.742)--(3.328,3.742)--(3.383,3.742)%
  --(3.410,3.719)--(3.410,3.710)--(3.464,3.710)--(3.492,3.710)--(3.519,3.710)--(3.573,3.698)%
  --(3.628,3.698)--(3.655,3.698)--(3.682,3.698)--(3.709,3.698)--(3.736,3.697)--(3.818,3.697)%
  --(3.845,3.672)--(3.872,3.672)--(3.900,3.662)--(3.954,4.766)--(3.954,4.721)--(3.954,4.569)%
  --(3.954,4.565)--(3.981,4.562)--(3.981,4.533)--(3.981,4.515)--(4.008,4.479)--(4.036,4.479)%
  --(4.036,4.473)--(4.036,4.459)--(4.036,4.451)--(4.063,4.445)--(4.063,4.441)--(4.090,4.429)%
  --(4.090,4.338)--(4.117,4.314)--(4.117,4.306)--(4.117,4.276)--(4.145,4.253)--(4.145,4.227)%
  --(4.172,4.220)--(4.172,4.194)--(4.172,4.165)--(4.199,4.147)--(4.199,4.137)--(4.199,4.123)%
  --(4.226,4.111)--(4.226,4.076)--(4.226,4.067)--(4.226,4.064)--(4.253,3.914)--(4.253,3.902)%
  --(4.253,3.896)--(4.281,3.892)--(4.308,3.892)--(4.335,3.892)--(4.362,3.892)--(4.362,3.884)%
  --(4.362,3.849)--(4.389,3.849)--(4.417,3.825)--(4.417,3.777)--(4.444,3.771)--(4.444,3.734)%
  --(4.498,3.719)--(4.525,3.695)--(4.553,3.677)--(4.580,3.672)--(4.607,3.669)--(4.607,3.668)%
  --(4.634,3.666)--(4.661,3.666)--(4.689,3.662)--(4.716,3.644)--(4.716,3.641)--(4.852,3.628)%
  --(4.906,3.619)--(4.906,3.603)--(4.934,3.603)--(4.988,3.603)--(5.015,3.574)--(5.124,3.574)%
  --(5.151,3.571)--(5.178,3.568)--(5.233,3.568)--(5.287,3.320)--(5.342,3.320)--(5.369,3.320)%
  --(5.396,3.296)--(5.423,3.296)--(5.478,3.281)--(5.532,3.281)--(5.586,4.780)--(5.586,4.774)%
  --(5.586,4.698)--(5.614,4.649)--(5.614,4.612)--(5.614,4.547)--(5.614,4.197)--(5.614,4.122)%
  --(5.641,4.109)--(5.641,4.011)--(5.641,4.004)--(5.641,3.943)--(5.668,3.937)--(5.668,3.884)%
  --(5.668,3.867)--(5.668,3.737)--(5.695,3.737)--(5.695,3.700)--(5.695,3.580)--(5.723,3.580)%
  --(5.723,3.563)--(5.723,3.495)--(5.750,3.448)--(5.750,3.442)--(5.777,3.432)--(5.777,3.371)%
  --(5.804,3.370)--(5.804,3.359)--(5.831,3.358)--(5.831,3.294)--(5.859,3.290)--(5.859,3.282)%
  --(5.859,3.252)--(5.886,3.105)--(5.913,3.105)--(5.913,3.096)--(5.913,3.091)--(5.940,3.084)%
  --(5.940,3.075)--(5.967,3.014)--(5.995,3.004)--(6.022,3.004)--(6.049,3.004)--(6.076,3.004)%
  --(6.076,2.961)--(6.076,2.937)--(6.103,2.937)--(6.131,2.937)--(6.212,2.937)--(6.212,2.934)%
  --(6.239,2.934)--(6.267,2.934)--(6.321,2.928)--(6.321,2.916)--(6.321,2.913)--(6.348,2.905)%
  --(6.348,2.899)--(6.375,2.890)--(6.375,2.878)--(6.403,2.878)--(6.430,2.878)--(6.457,2.875)%
  --(6.484,2.875)--(6.512,2.848)--(6.593,2.848)--(6.648,2.848)--(6.675,2.834)--(6.784,2.827)%
  --(6.811,2.827)--(6.865,2.827)--(6.892,2.821)--(6.947,2.803)--(7.083,2.790)--(7.110,2.790)%
  --(7.137,2.790)--(7.219,2.790)--(7.219,4.165)--(7.219,4.093)--(7.219,4.081)--(7.246,4.066)%
  --(7.246,4.034)--(7.246,4.005)--(7.246,3.889)--(7.273,3.881)--(7.273,3.840)--(7.273,3.819)%
  --(7.273,3.721)--(7.301,3.719)--(7.301,3.713)--(7.328,3.710)--(7.328,3.491)--(7.355,3.418)%
  --(7.355,3.409)--(7.382,3.395)--(7.382,3.386)--(7.409,3.386)--(7.409,3.373)--(7.409,3.344)%
  --(7.409,3.336)--(7.409,3.329)--(7.437,3.184)--(7.464,3.184)--(7.464,3.160)--(7.464,3.153)%
  --(7.491,3.135)--(7.491,3.117)--(7.491,3.111)--(7.518,3.094)--(7.545,3.091)--(7.545,3.069)%
  --(7.573,3.020)--(7.573,3.008)--(7.600,3.005)--(7.627,3.005)--(7.654,2.998)--(7.681,2.998)%
  --(7.681,2.989)--(7.709,2.952)--(7.763,2.952)--(7.790,2.952)--(7.872,2.952)--(7.872,2.936)%
  --(7.872,2.913)--(7.899,2.910)--(7.899,2.907)--(7.926,2.893)--(8.008,2.881)--(8.008,2.875)%
  --(8.035,2.875)--(8.090,2.875)--(8.117,2.828)--(8.144,2.812)--(8.226,2.812)--(8.253,2.796)%
  --(8.307,2.781)--(8.362,2.781)--(8.389,2.780)--(8.389,2.778)--(8.389,2.745)--(8.443,2.745)%
  --(8.443,2.509)--(8.470,2.506)--(8.525,2.505)--(8.552,2.503)--(8.579,2.503)--(8.606,2.480)%
  --(8.661,2.477)--(8.770,2.474)--(8.824,2.456)--(8.851,2.456)--(8.879,3.937)--(8.879,3.932)%
  --(8.879,3.881)--(8.879,3.852)--(8.879,3.724)--(8.879,3.701)--(8.906,3.701)--(8.906,3.659)%
  --(8.906,3.650)--(8.906,3.641)--(8.933,3.477)--(8.933,3.464)--(8.933,3.324)--(8.933,3.261)%
  --(8.933,3.209)--(8.960,3.184)--(8.960,3.107)--(8.987,2.958)--(8.987,2.940)--(8.987,2.922)%
  --(8.987,2.754)--(8.987,2.731)--(9.015,2.719)--(9.015,2.680)--(9.042,2.591)--(9.069,2.591)%
  --(9.069,2.585)--(9.096,2.582)--(9.123,2.554)--(9.178,2.536)--(9.205,2.518)--(9.205,2.517)%
  --(9.232,2.511)--(9.259,2.511)--(9.314,2.505)--(9.314,2.501)--(9.341,2.470)--(9.341,2.452)%
  --(9.368,2.452)--(9.395,2.438)--(9.395,2.432)--(9.423,2.432)--(9.450,2.429)--(9.450,2.411)%
  --(9.477,2.411)--(9.613,2.411)--(9.613,2.387)--(9.722,2.387)--(9.858,2.387)--(9.912,2.387)%
  --(9.994,2.382)--(9.994,2.355)--(10.021,2.355)--(10.048,2.355)--(10.130,2.353)--(10.157,2.346)%
  --(10.266,2.335)--(10.266,2.334)--(10.348,2.334)--(10.429,2.325)--(10.484,2.318)--(10.511,3.365)%
  --(10.511,3.234)--(10.511,3.209)--(10.538,3.203)--(10.538,3.182)--(10.538,3.169)--(10.565,3.167)%
  --(10.565,3.155)--(10.565,3.113)--(10.593,3.110)--(10.593,3.049)--(10.593,2.973)--(10.620,2.955)%
  --(10.620,2.924)--(10.620,2.600)--(10.647,2.585)--(10.647,2.553)--(10.647,2.547)--(10.647,2.545)%
  --(10.674,2.545)--(10.674,2.529)--(10.701,2.529)--(10.701,2.517)--(10.701,2.515)--(10.756,2.515)%
  --(10.783,2.492)--(10.783,2.456)--(10.837,2.397)--(10.837,2.385)--(10.865,2.380)--(10.892,2.371)%
  --(10.919,2.362)--(10.919,2.340)--(10.919,2.223)--(10.946,2.199)--(10.973,2.199)--(11.055,2.196)%
  --(11.082,2.196)--(11.110,2.193)--(11.246,2.170)--(11.300,2.149)--(11.327,2.141)--(11.354,2.129)%
  --(11.409,2.120)--(11.436,2.111)--(11.463,2.095)--(11.490,2.089)--(11.518,2.089)--(11.545,2.082)%
  --(11.572,2.082)--(11.572,2.073)--(11.626,2.061)--(11.681,2.049)--(11.708,2.049)--(11.844,2.042)%
  --(11.844,2.034)--(11.844,2.023)--(11.926,2.023)--(11.980,2.016)--(12.007,2.016)--(12.062,1.990);
\gpcolor{color=gp lt color border}
\gpsetlinewidth{1.00}
\draw[gp path] (2.240,10.921)--(2.240,1.845)--(13.667,1.845)--(13.667,10.921)--cycle;
\node[gp node center,rotate=-270] at (-0.628,6.383) {Strategic Urgency [Weighted Tardiness]};
\node[gp node center,rotate=-270] at (16.213,6.383) {Strategic Resource Penalty [Hours]};
\node[gp node center] at (7.953,0.215) {Relative time ($\tau$) [Seconds]};
%% coordinates of the plot area
\gpdefrectangularnode{gp plot 1}{\pgfpoint{2.240cm}{1.845cm}}{\pgfpoint{13.667cm}{10.921cm}}
\endtikzpicture
%% gnuplot variables

	}
	\caption{Figure Caption}
	\label{fig:responses:value_change}
\end{figure}
 
