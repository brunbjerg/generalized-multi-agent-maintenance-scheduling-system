\section{Results}\label{sec:3-results}
To test the AbLNS algorithm, several simulations are conducted in which the data
is perturbed during the algorithm’s execution. The data from the
company is presented in Section~\ref{sec:data_instance}. Then the effect of
forcing work orders into specific weekly $\ElementPeriod \in \SetPeriod$
schedules is presented in Section~\ref{sec:response_work_orders} and excluding
work orders from periods is presented in Section~\ref{sec:exclusion}. The
effects of reducing the period resource capacities $\ParStrategicResource$ is
tested in Section~\ref{sec:results:reduced_weekly_capacity} and increasing
$\ElementResource$ is tested in  Section~\ref{sec:increase_week_cap}. Finally
the effect of changing the work order values $\ParStrategicUrgency$, is tested
in Section~\ref{sec:results:strategic_value_changes}.

\subsection{Data Instance}\label{sec:data_instance}
The data instance used in this paper is provided by Total
Energies~\citep{total-energies} and extracted from their SAP ERP system. It
pertains to a specific offshore platform and covers a two-year time horizon. The
instance includes 3,487 outstanding work orders ($\SetWorkOrder{}$), 16 distinct
resource skill sets ($\SetResource$) (e.g., mechanics, electricians, turbine
specialists, etc.), and spans 52 bi-weekly periods ($\SetPeriod$), roughly
equivalent to two years.
 
\begin{table}[H]
\centering
\begin{tabular}{lrrrr}
\toprule
           & $|\SetWorkOrder{}|$ & $|\SetResource|$ & $|\SetPeriod|$ \\ \midrule
Instance 1 & 3487                & 16               & 52             \\ \bottomrule
\end{tabular}

\caption{Specific data instances from the case company. Here $\SetWorkOrder$ is the set of work orders, $\SetResource$ is the set of resources, and $\SetPeriod$ is the set of weekly periods.} % \label{fig1}
\end{table}



\subsection{Value of Objective Function Parameters}
The optimization problem has three terms and it could be argued that the
a pareto front should be calculated on the value of the different weightings
between them. To not lose sight of the main contribution of the paper the
value of the $\ParClusteringValue$ and
$\ParStrategicResourcePenalty$ has been set so
that the $\ParStrategicResourcePenalty$ always dominates the $\ParStrategicUrgency$ and
that the $\ParStrategicUrgency$ always dominates the $\ParClusteringValue$.
Furthermore the $\ParClusteringValue$ has been excluded from the results section
to put more focus on the user-input interaction.

\subsection{Response to Inclusion of Work Orders}\label{sec:response_work_orders}
The $\ParStrategicInclude$ parameter specifies whether a work order
should be scheduled into a specific period. As the parameter has the time
variable $\VarMetaTime$ it means that this parameter can change at any time
while the metaheuristic is running. The $\ParStrategicInclude$ parameter
constrains the model in equation~\ref{eqn:constraint:strategic:include}.
Table~\ref{tab:responses:inclusion} shows the responses that the model will be
subject to while it is running for different timepoints $\VarMetaTime$.

\begin{table}[H]
	\centering
	\begin{tabular}{lrrrrr}
	\toprule
	                                & $\VarMetaTime_1 = 60$ & $\VarMetaTime_2 = 120$ & $\VarMetaTime_3 = 180$ & $\VarMetaTime_4 = 240$ & $\VarMetaTime_5 = 300$ \\ \midrule
	$\Delta \ParStrategicInclude$ & 50                    & 50                     & 50                     & 50                     & 50                     \\ \bottomrule
	\end{tabular}
	\caption{The including work orders perturbations that the AbLNS will be affected by. 
		Perturbations occur at 60 second time intervals affecting 50 randomly chosen work orders included into random periods.
	}\label{tab:responses:inclusion}
\end{table}


Figure~\ref{fig:responses:inclusion} shows the effects of changing the $
\ParStrategicInclude$ parameter in real-time. The model quickly converges
and when the system is pertubed by an input response the objective 
value~\ref{eqn:objective:strategic} shows a small spike and then quickly converges to
a new solution.

\begin{figure}[H]
	\centering
	\resizebox{10cm}{!}{
		\tikzpicture[gnuplot]
%% generated with GNUPLOT 6.0p1 (Lua 5.2; terminal rev. Jun 2020, script rev. 118)
%% Wed 04 Dec 2024 06:00:47 PM UTC
\path (0.000,0.000) rectangle (16.000,12.000);
\gpcolor{color=gp lt color axes}
\gpsetlinetype{gp lt axes}
\gpsetdashtype{gp dt axes}
\gpsetlinewidth{0.50}
\draw[gp path] (2.424,2.025)--(15.447,2.025);
\gpcolor{color=gp lt color border}
\gpsetlinetype{gp lt border}
\gpsetdashtype{gp dt solid}
\gpsetlinewidth{1.00}
\draw[gp path] (2.424,2.025)--(2.604,2.025);
\draw[gp path] (15.447,2.025)--(15.267,2.025);
\node[gp node right] at (2.240,2.025) {$1.92\times10^{11}$};
\gpcolor{color=gp lt color axes}
\gpsetlinetype{gp lt axes}
\gpsetdashtype{gp dt axes}
\gpsetlinewidth{0.50}
\draw[gp path] (2.424,3.533)--(15.447,3.533);
\gpcolor{color=gp lt color border}
\gpsetlinetype{gp lt border}
\gpsetdashtype{gp dt solid}
\gpsetlinewidth{1.00}
\draw[gp path] (2.424,3.533)--(2.604,3.533);
\draw[gp path] (15.447,3.533)--(15.267,3.533);
\node[gp node right] at (2.240,3.533) {$1.93\times10^{11}$};
\gpcolor{color=gp lt color axes}
\gpsetlinetype{gp lt axes}
\gpsetdashtype{gp dt axes}
\gpsetlinewidth{0.50}
\draw[gp path] (2.424,5.042)--(15.447,5.042);
\gpcolor{color=gp lt color border}
\gpsetlinetype{gp lt border}
\gpsetdashtype{gp dt solid}
\gpsetlinewidth{1.00}
\draw[gp path] (2.424,5.042)--(2.604,5.042);
\draw[gp path] (15.447,5.042)--(15.267,5.042);
\node[gp node right] at (2.240,5.042) {$1.94\times10^{11}$};
\gpcolor{color=gp lt color axes}
\gpsetlinetype{gp lt axes}
\gpsetdashtype{gp dt axes}
\gpsetlinewidth{0.50}
\draw[gp path] (2.424,6.550)--(15.447,6.550);
\gpcolor{color=gp lt color border}
\gpsetlinetype{gp lt border}
\gpsetdashtype{gp dt solid}
\gpsetlinewidth{1.00}
\draw[gp path] (2.424,6.550)--(2.604,6.550);
\draw[gp path] (15.447,6.550)--(15.267,6.550);
\node[gp node right] at (2.240,6.550) {$1.95\times10^{11}$};
\gpcolor{color=gp lt color axes}
\gpsetlinetype{gp lt axes}
\gpsetdashtype{gp dt axes}
\gpsetlinewidth{0.50}
\draw[gp path] (2.424,8.058)--(15.447,8.058);
\gpcolor{color=gp lt color border}
\gpsetlinetype{gp lt border}
\gpsetdashtype{gp dt solid}
\gpsetlinewidth{1.00}
\draw[gp path] (2.424,8.058)--(2.604,8.058);
\draw[gp path] (15.447,8.058)--(15.267,8.058);
\node[gp node right] at (2.240,8.058) {$1.96\times10^{11}$};
\gpcolor{color=gp lt color axes}
\gpsetlinetype{gp lt axes}
\gpsetdashtype{gp dt axes}
\gpsetlinewidth{0.50}
\draw[gp path] (2.424,9.567)--(15.447,9.567);
\gpcolor{color=gp lt color border}
\gpsetlinetype{gp lt border}
\gpsetdashtype{gp dt solid}
\gpsetlinewidth{1.00}
\draw[gp path] (2.424,9.567)--(2.604,9.567);
\draw[gp path] (15.447,9.567)--(15.267,9.567);
\node[gp node right] at (2.240,9.567) {$1.97\times10^{11}$};
\gpcolor{color=gp lt color axes}
\gpsetlinetype{gp lt axes}
\gpsetdashtype{gp dt axes}
\gpsetlinewidth{0.50}
\draw[gp path] (2.424,11.075)--(15.447,11.075);
\gpcolor{color=gp lt color border}
\gpsetlinetype{gp lt border}
\gpsetdashtype{gp dt solid}
\gpsetlinewidth{1.00}
\draw[gp path] (2.424,11.075)--(2.604,11.075);
\draw[gp path] (15.447,11.075)--(15.267,11.075);
\node[gp node right] at (2.240,11.075) {$1.98\times10^{11}$};
\gpcolor{color=gp lt color axes}
\gpsetlinetype{gp lt axes}
\gpsetdashtype{gp dt axes}
\gpsetlinewidth{0.50}
\draw[gp path] (2.424,2.025)--(2.424,11.075);
\gpcolor{color=gp lt color border}
\gpsetlinetype{gp lt border}
\gpsetdashtype{gp dt solid}
\gpsetlinewidth{1.00}
\draw[gp path] (2.424,2.025)--(2.424,1.845);
\draw[gp path] (2.424,11.075)--(2.424,11.255);
\node[gp node left,rotate=270] at (2.424,1.661) {$0$};
\gpcolor{color=gp lt color axes}
\gpsetlinetype{gp lt axes}
\gpsetdashtype{gp dt axes}
\gpsetlinewidth{0.50}
\draw[gp path] (4.284,2.025)--(4.284,11.075);
\gpcolor{color=gp lt color border}
\gpsetlinetype{gp lt border}
\gpsetdashtype{gp dt solid}
\gpsetlinewidth{1.00}
\draw[gp path] (4.284,2.025)--(4.284,1.845);
\draw[gp path] (4.284,11.075)--(4.284,11.255);
\node[gp node left,rotate=270] at (4.284,1.661) {$50$};
\gpcolor{color=gp lt color axes}
\gpsetlinetype{gp lt axes}
\gpsetdashtype{gp dt axes}
\gpsetlinewidth{0.50}
\draw[gp path] (6.145,2.025)--(6.145,11.075);
\gpcolor{color=gp lt color border}
\gpsetlinetype{gp lt border}
\gpsetdashtype{gp dt solid}
\gpsetlinewidth{1.00}
\draw[gp path] (6.145,2.025)--(6.145,1.845);
\draw[gp path] (6.145,11.075)--(6.145,11.255);
\node[gp node left,rotate=270] at (6.145,1.661) {$100$};
\gpcolor{color=gp lt color axes}
\gpsetlinetype{gp lt axes}
\gpsetdashtype{gp dt axes}
\gpsetlinewidth{0.50}
\draw[gp path] (8.005,2.025)--(8.005,11.075);
\gpcolor{color=gp lt color border}
\gpsetlinetype{gp lt border}
\gpsetdashtype{gp dt solid}
\gpsetlinewidth{1.00}
\draw[gp path] (8.005,2.025)--(8.005,1.845);
\draw[gp path] (8.005,11.075)--(8.005,11.255);
\node[gp node left,rotate=270] at (8.005,1.661) {$150$};
\gpcolor{color=gp lt color axes}
\gpsetlinetype{gp lt axes}
\gpsetdashtype{gp dt axes}
\gpsetlinewidth{0.50}
\draw[gp path] (9.866,2.025)--(9.866,11.075);
\gpcolor{color=gp lt color border}
\gpsetlinetype{gp lt border}
\gpsetdashtype{gp dt solid}
\gpsetlinewidth{1.00}
\draw[gp path] (9.866,2.025)--(9.866,1.845);
\draw[gp path] (9.866,11.075)--(9.866,11.255);
\node[gp node left,rotate=270] at (9.866,1.661) {$200$};
\gpcolor{color=gp lt color axes}
\gpsetlinetype{gp lt axes}
\gpsetdashtype{gp dt axes}
\gpsetlinewidth{0.50}
\draw[gp path] (11.726,2.025)--(11.726,11.075);
\gpcolor{color=gp lt color border}
\gpsetlinetype{gp lt border}
\gpsetdashtype{gp dt solid}
\gpsetlinewidth{1.00}
\draw[gp path] (11.726,2.025)--(11.726,1.845);
\draw[gp path] (11.726,11.075)--(11.726,11.255);
\node[gp node left,rotate=270] at (11.726,1.661) {$250$};
\gpcolor{color=gp lt color axes}
\gpsetlinetype{gp lt axes}
\gpsetdashtype{gp dt axes}
\gpsetlinewidth{0.50}
\draw[gp path] (13.587,2.025)--(13.587,11.075);
\gpcolor{color=gp lt color border}
\gpsetlinetype{gp lt border}
\gpsetdashtype{gp dt solid}
\gpsetlinewidth{1.00}
\draw[gp path] (13.587,2.025)--(13.587,1.845);
\draw[gp path] (13.587,11.075)--(13.587,11.255);
\node[gp node left,rotate=270] at (13.587,1.661) {$300$};
\gpcolor{color=gp lt color axes}
\gpsetlinetype{gp lt axes}
\gpsetdashtype{gp dt axes}
\gpsetlinewidth{0.50}
\draw[gp path] (15.447,2.025)--(15.447,11.075);
\gpcolor{color=gp lt color border}
\gpsetlinetype{gp lt border}
\gpsetdashtype{gp dt solid}
\gpsetlinewidth{1.00}
\draw[gp path] (15.447,2.025)--(15.447,1.845);
\draw[gp path] (15.447,11.075)--(15.447,11.255);
\node[gp node left,rotate=270] at (15.447,1.661) {$350$};
\draw[gp path] (2.424,11.075)--(2.424,2.025)--(15.447,2.025)--(15.447,11.075)--cycle;
\gpcolor{rgb color={0.000,0.000,0.000}}
\gpsetlinewidth{2.00}
\draw[gp path] (2.424,10.451)--(2.424,10.149)--(2.424,9.920)--(2.424,9.836)--(2.461,9.505)%
  --(2.461,9.494)--(2.461,9.404)--(2.461,9.250)--(2.498,9.202)--(2.498,9.011)--(2.498,8.858)%
  --(2.498,8.818)--(2.498,8.675)--(2.498,8.512)--(2.536,8.284)--(2.536,8.209)--(2.536,7.921)%
  --(2.536,7.869)--(2.536,7.819)--(2.573,7.716)--(2.573,7.521)--(2.573,7.394)--(2.573,7.006)%
  --(2.573,6.935)--(2.573,6.766)--(2.610,6.762)--(2.610,6.758)--(2.610,6.753)--(2.647,6.703)%
  --(2.647,6.701)--(2.647,6.623)--(2.684,6.504)--(2.684,6.434)--(2.684,6.412)--(2.684,6.222)%
  --(2.684,6.191)--(2.722,6.090)--(2.722,6.001)--(2.722,5.978)--(2.722,5.920)--(2.759,5.863)%
  --(2.759,5.804)--(2.759,5.777)--(2.759,5.722)--(2.759,5.693)--(2.759,5.673)--(2.759,5.546)%
  --(2.759,5.457)--(2.796,5.444)--(2.796,5.438)--(2.796,5.424)--(2.796,5.400)--(2.833,5.368)%
  --(2.833,5.224)--(2.833,5.190)--(2.871,5.127)--(2.945,5.125)--(2.945,5.096)--(2.945,5.044)%
  --(2.945,5.042)--(2.982,5.000)--(2.982,4.972)--(3.019,4.968)--(3.019,4.910)--(3.057,4.906)%
  --(3.131,4.875)--(3.131,4.853)--(3.131,4.846)--(3.131,4.785)--(3.168,4.585)--(3.205,4.520)%
  --(3.243,4.496)--(3.243,4.492)--(3.243,4.379)--(3.317,4.375)--(3.354,4.360)--(3.354,4.320)%
  --(3.354,4.297)--(3.391,4.283)--(3.391,4.243)--(3.391,4.237)--(3.429,4.180)--(3.429,4.123)%
  --(3.466,4.089)--(3.466,4.079)--(3.577,4.062)--(3.577,4.059)--(3.615,4.050)--(3.615,4.038)%
  --(3.689,4.008)--(3.726,3.973)--(3.726,3.971)--(3.764,3.942)--(3.838,3.923)--(3.838,3.918)%
  --(3.838,3.909)--(3.875,3.895)--(3.912,3.894)--(3.912,3.889)--(3.912,3.851)--(3.950,3.817)%
  --(4.098,3.815)--(4.098,3.813)--(4.136,3.811)--(4.247,3.794)--(4.247,3.793)--(4.247,3.748)%
  --(4.284,3.721)--(4.322,3.717)--(4.433,3.717)--(4.470,3.702)--(4.470,3.690)--(4.508,3.686)%
  --(4.545,3.579)--(4.545,3.420)--(4.545,3.394)--(4.582,3.388)--(4.619,3.341)--(4.619,3.324)%
  --(4.619,3.286)--(4.657,3.259)--(4.657,3.217)--(4.694,3.212)--(4.694,3.037)--(4.768,2.995)%
  --(4.768,2.963)--(4.843,2.952)--(4.880,2.945)--(4.917,2.934)--(4.917,2.926)--(4.917,2.867)%
  --(4.991,2.863)--(5.029,2.863)--(5.066,2.862)--(5.066,2.855)--(5.103,2.823)--(5.103,2.822)%
  --(5.140,2.803)--(5.140,2.777)--(5.177,2.756)--(5.326,2.744)--(5.438,2.703)--(5.512,2.667)%
  --(5.512,2.661)--(5.512,2.657)--(5.587,2.647)--(5.624,2.635)--(5.661,2.632)--(5.736,2.625)%
  --(5.847,2.589)--(5.847,2.572)--(5.884,2.557)--(5.922,2.549)--(6.033,2.536)--(6.219,2.529)%
  --(6.740,2.529)--(6.740,2.858)--(6.815,2.752)--(6.852,2.738)--(6.889,2.737)--(6.926,2.721)%
  --(6.963,2.720)--(6.963,2.706)--(7.001,2.632)--(7.001,2.630)--(7.038,2.629)--(7.038,2.599)%
  --(7.038,2.591)--(7.149,2.527)--(7.149,2.469)--(7.149,2.452)--(7.187,2.434)--(7.224,2.429)%
  --(7.224,2.409)--(7.261,2.396)--(7.336,2.371)--(7.373,2.370)--(7.410,2.340)--(7.708,2.333)%
  --(7.819,2.332)--(7.819,2.327)--(7.931,2.321)--(7.931,2.304)--(7.931,2.294)--(8.042,2.287)%
  --(8.117,2.276)--(8.191,2.259)--(8.489,2.247)--(8.489,2.226)--(8.601,2.217)--(8.712,2.186)%
  --(8.898,2.186)--(8.936,3.043)--(8.936,3.041)--(8.936,3.034)--(9.010,3.031)--(9.047,3.014)%
  --(9.084,3.001)--(9.122,2.978)--(9.196,2.971)--(9.233,2.942)--(9.270,2.936)--(9.308,2.913)%
  --(9.345,2.897)--(9.345,2.864)--(9.456,2.785)--(9.531,2.766)--(9.568,2.763)--(9.568,2.739)%
  --(9.642,2.732)--(9.642,2.730)--(9.754,2.719)--(9.791,2.661)--(9.829,2.661)--(9.866,2.647)%
  --(9.903,2.625)--(10.015,2.624)--(10.052,2.617)--(10.052,2.581)--(10.089,2.579)--(10.089,2.555)%
  --(10.126,2.543)--(10.126,2.531)--(10.163,2.529)--(10.535,2.521)--(10.647,2.519)--(10.647,2.516)%
  --(10.796,2.490)--(10.833,2.462)--(10.833,2.448)--(10.945,2.447)--(11.094,2.447)--(11.094,3.046)%
  --(11.131,3.035)--(11.131,3.023)--(11.168,3.018)--(11.168,2.952)--(11.242,2.949)--(11.317,2.898)%
  --(11.317,2.870)--(11.354,2.817)--(11.354,2.794)--(11.391,2.789)--(11.726,2.762)--(11.763,2.724)%
  --(11.838,2.712)--(11.949,2.694)--(11.987,2.670)--(12.135,2.659)--(12.396,2.642)--(12.396,2.640)%
  --(12.656,2.639)--(12.768,2.622)--(12.768,2.608)--(12.880,2.602)--(12.880,2.590)--(13.028,2.579)%
  --(13.103,2.566)--(13.140,2.559)--(13.214,2.557)--(13.326,2.557)--(13.326,2.969)--(13.326,2.958)%
  --(13.363,2.932)--(13.363,2.895)--(13.438,2.872)--(13.475,2.860)--(13.512,2.840)--(13.549,2.817)%
  --(13.624,2.816)--(13.698,2.773)--(13.698,2.759)--(13.735,2.744)--(13.810,2.608)--(13.847,2.600)%
  --(13.921,2.589)--(13.921,2.543)--(13.996,2.536)--(14.145,2.519)--(14.182,2.474)--(14.368,2.458)%
  --(14.442,2.451)--(14.480,2.446)--(14.480,2.436)--(14.591,2.428)--(14.777,2.406)--(14.889,2.403)%
  --(15.075,2.378)--(15.112,2.377)--(15.112,2.353)--(15.335,2.340);
\gpcolor{color=gp lt color border}
\gpsetlinewidth{1.00}
\draw[gp path] (2.424,11.075)--(2.424,2.025)--(15.447,2.025)--(15.447,11.075)--cycle;
\node[gp node center,rotate=-270] at (-0.812,6.550) {Objective value};
\node[gp node center] at (8.935,0.215) {Relative time [S]};
\node[gp node center] at (8.935,11.537) {Objective value for Weekly Schedule};
%% coordinates of the plot area
\gpdefrectangularnode{gp plot 1}{\pgfpoint{2.424cm}{2.025cm}}{\pgfpoint{15.447cm}{11.075cm}}
\endtikzpicture
%% gnuplot variables

	}
	\caption{At each 60 second time interval the the solution is perturbed by forcing work orders into specific
		periods. Here the strategic urgency fluctuatues as the dominating term of the strategic resource penalty
		is minimized.
	}\label{fig:responses:inclusion}
\end{figure}


Figure~\ref{fig:responses:inclusion}  show 5 perturbations with the first
at $\tau = 60s$ where the objective value slightly increases in response to
the inclusion, the objective value increases due to the inclusion either
causing the capacity to be exceeded or the inclusion resulting in a selected
$ \ParStrategicUrgency$ that has a higher value. The remaining 4 perturbations
all show the same  pattern, an increase in the strategic urgency and
resource penalty  followed by a subsequent convergence.

\subsection{Response to Exclusion}\label{sec:exclusion}
The response to exclusion is associated with the $\ParStrategicExclude$
parameter and is found in equation~\ref{eqn:strategic:constraint:exclude}.
Here specific work orders ($\ElementWorkOrder \in \SetWorkOrder{}$) are
being excluded from specific periods ($\ElementPeriod \in \SetPeriod$).
The perturbations that the AbLNS will be affected by are shown in
table~\ref{tab:responses:exclusion}
with the setup being very similar to the
one in table~\ref{tab:responses:inclusion}.
The main distinction being that the perturbation affects 500 instead of 50 work orders, the higher number 
of affected work orders is chosen as many exclusions of do not affect the assignment of a work order.

\begin{table}[H]
	\centering
	\begin{tabular}{lrrrrr}
	\toprule
	                                & $\VarMetaTime_1 = 60$ & $\VarMetaTime_2 = 120$ & $\VarMetaTime_3 = 180$ & $\VarMetaTime_4 = 240$ & $\VarMetaTime_5 = 300$ \\ \midrule
	$\Delta \ParStrategicExclude$ & 500                   & 500                    & 500                    & 500                    & 500                    \\ \bottomrule
	\end{tabular}
	\caption{The exclusion of work orders perturbations from specific periods on the weekly schedule. 
		Perturbations occur at 60 second time intervals affecting 500 work orders each time.
	}\label{tab:responses:exclusion}
\end{table}


Figure~\ref{fig:responses:exclusion}  show a substantial spike in the strategic urgency
and a substantial decrease in the strategic resource penalty   
after each perturbation as given by table~\ref{tab:responses:exclusion}. 

\begin{figure}[H]
	\centering
	\resizebox{10cm}{!}{
		\tikzpicture[gnuplot]
%% generated with GNUPLOT 6.0p1 (Lua 5.2; terminal rev. Jun 2020, script rev. 118)
%% Wed 08 Jan 2025 01:56:13 PM UTC
\tikzset{every node/.append style={scale=1.00}}
\path (0.000,0.000) rectangle (14.000,18.000);
\gpcolor{color=gp lt color border}
\node[gp node center] at (7.000,17.692) {Strategic Objective Values};
\gpcolor{rgb color={0.753,0.753,0.753}}
\gpsetlinetype{gp lt border}
\gpsetdashtype{gp dt solid}
\gpsetlinewidth{2.00}
\draw[gp path] (1.400,9.000)--(12.599,9.000);
\gpcolor{color=gp lt color border}
\gpsetlinewidth{1.00}
\draw[gp path] (1.400,9.000)--(1.580,9.000);
\node[gp node right] at (1.216,9.308) {$2.24\times10^{13}$};
\gpcolor{rgb color={0.753,0.753,0.753}}
\gpsetlinewidth{2.00}
\draw[gp path] (1.400,10.350)--(12.599,10.350);
\gpcolor{color=gp lt color border}
\gpsetlinewidth{1.00}
\draw[gp path] (1.400,10.350)--(1.580,10.350);
\node[gp node right] at (1.216,10.658) {$2.26\times10^{13}$};
\gpcolor{rgb color={0.753,0.753,0.753}}
\gpsetlinewidth{2.00}
\draw[gp path] (1.400,11.700)--(12.599,11.700);
\gpcolor{color=gp lt color border}
\gpsetlinewidth{1.00}
\draw[gp path] (1.400,11.700)--(1.580,11.700);
\node[gp node right] at (1.216,12.008) {$2.28\times10^{13}$};
\gpcolor{rgb color={0.753,0.753,0.753}}
\gpsetlinewidth{2.00}
\draw[gp path] (1.400,13.050)--(12.599,13.050);
\gpcolor{color=gp lt color border}
\gpsetlinewidth{1.00}
\draw[gp path] (1.400,13.050)--(1.580,13.050);
\node[gp node right] at (1.216,13.358) {$2.3\times10^{13}$};
\gpcolor{rgb color={0.753,0.753,0.753}}
\gpsetlinewidth{2.00}
\draw[gp path] (1.400,14.399)--(12.599,14.399);
\gpcolor{color=gp lt color border}
\gpsetlinewidth{1.00}
\draw[gp path] (1.400,14.399)--(1.580,14.399);
\node[gp node right] at (1.216,14.707) {$2.32\times10^{13}$};
\gpcolor{rgb color={0.753,0.753,0.753}}
\gpsetlinewidth{2.00}
\draw[gp path] (1.400,15.749)--(6.163,15.749);
\draw[gp path] (12.415,15.749)--(12.599,15.749);
\gpcolor{color=gp lt color border}
\gpsetlinewidth{1.00}
\draw[gp path] (1.400,15.749)--(1.580,15.749);
\node[gp node right] at (1.216,16.057) {$2.34\times10^{13}$};
\gpcolor{rgb color={0.753,0.753,0.753}}
\gpsetlinewidth{2.00}
\draw[gp path] (1.400,17.099)--(12.599,17.099);
\gpcolor{color=gp lt color border}
\gpsetlinewidth{1.00}
\draw[gp path] (1.400,17.099)--(1.580,17.099);
\node[gp node right] at (1.216,17.407) {$2.36\times10^{13}$};
\gpcolor{rgb color={0.753,0.753,0.753}}
\gpsetlinewidth{2.00}
\draw[gp path] (1.400,9.000)--(1.400,17.099);
\gpcolor{color=gp lt color border}
\gpsetlinewidth{1.00}
\draw[gp path] (1.400,9.000)--(1.400,9.180);
\draw[gp path] (1.400,17.099)--(1.400,16.919);
\gpcolor{rgb color={0.753,0.753,0.753}}
\gpsetlinewidth{2.00}
\draw[gp path] (2.800,9.000)--(2.800,17.099);
\gpcolor{color=gp lt color border}
\gpsetlinewidth{1.00}
\draw[gp path] (2.800,9.000)--(2.800,9.180);
\draw[gp path] (2.800,17.099)--(2.800,16.919);
\gpcolor{rgb color={0.753,0.753,0.753}}
\gpsetlinewidth{2.00}
\draw[gp path] (4.200,9.000)--(4.200,17.099);
\gpcolor{color=gp lt color border}
\gpsetlinewidth{1.00}
\draw[gp path] (4.200,9.000)--(4.200,9.180);
\draw[gp path] (4.200,17.099)--(4.200,16.919);
\gpcolor{rgb color={0.753,0.753,0.753}}
\gpsetlinewidth{2.00}
\draw[gp path] (5.600,9.000)--(5.600,17.099);
\gpcolor{color=gp lt color border}
\gpsetlinewidth{1.00}
\draw[gp path] (5.600,9.000)--(5.600,9.180);
\draw[gp path] (5.600,17.099)--(5.600,16.919);
\gpcolor{rgb color={0.753,0.753,0.753}}
\gpsetlinewidth{2.00}
\draw[gp path] (7.000,9.000)--(7.000,15.379);
\draw[gp path] (7.000,16.919)--(7.000,17.099);
\gpcolor{color=gp lt color border}
\gpsetlinewidth{1.00}
\draw[gp path] (7.000,9.000)--(7.000,9.180);
\draw[gp path] (7.000,17.099)--(7.000,16.919);
\gpcolor{rgb color={0.753,0.753,0.753}}
\gpsetlinewidth{2.00}
\draw[gp path] (8.399,9.000)--(8.399,15.379);
\draw[gp path] (8.399,16.919)--(8.399,17.099);
\gpcolor{color=gp lt color border}
\gpsetlinewidth{1.00}
\draw[gp path] (8.399,9.000)--(8.399,9.180);
\draw[gp path] (8.399,17.099)--(8.399,16.919);
\gpcolor{rgb color={0.753,0.753,0.753}}
\gpsetlinewidth{2.00}
\draw[gp path] (9.799,9.000)--(9.799,15.379);
\draw[gp path] (9.799,16.919)--(9.799,17.099);
\gpcolor{color=gp lt color border}
\gpsetlinewidth{1.00}
\draw[gp path] (9.799,9.000)--(9.799,9.180);
\draw[gp path] (9.799,17.099)--(9.799,16.919);
\gpcolor{rgb color={0.753,0.753,0.753}}
\gpsetlinewidth{2.00}
\draw[gp path] (11.199,9.000)--(11.199,15.379);
\draw[gp path] (11.199,16.919)--(11.199,17.099);
\gpcolor{color=gp lt color border}
\gpsetlinewidth{1.00}
\draw[gp path] (11.199,9.000)--(11.199,9.180);
\draw[gp path] (11.199,17.099)--(11.199,16.919);
\gpcolor{rgb color={0.753,0.753,0.753}}
\gpsetlinewidth{2.00}
\draw[gp path] (12.599,9.000)--(12.599,17.099);
\gpcolor{color=gp lt color border}
\gpsetlinewidth{1.00}
\draw[gp path] (12.599,9.000)--(12.599,9.180);
\draw[gp path] (12.599,17.099)--(12.599,16.919);
\draw[gp path] (12.599,9.000)--(12.419,9.000);
\node[gp node left] at (12.783,9.308) {$2.22\times10^{13}$};
\draw[gp path] (12.599,10.350)--(12.419,10.350);
\node[gp node left] at (12.783,10.658) {$2.24\times10^{13}$};
\draw[gp path] (12.599,11.700)--(12.419,11.700);
\node[gp node left] at (12.783,12.008) {$2.26\times10^{13}$};
\draw[gp path] (12.599,13.050)--(12.419,13.050);
\node[gp node left] at (12.783,13.358) {$2.28\times10^{13}$};
\draw[gp path] (12.599,14.399)--(12.419,14.399);
\node[gp node left] at (12.783,14.707) {$2.3\times10^{13}$};
\draw[gp path] (12.599,15.749)--(12.419,15.749);
\node[gp node left] at (12.783,16.057) {$2.32\times10^{13}$};
\draw[gp path] (12.599,17.099)--(12.419,17.099);
\node[gp node left] at (12.783,17.407) {$2.34\times10^{13}$};
\draw[gp path] (1.400,17.099)--(1.400,9.000)--(12.599,9.000)--(12.599,17.099)--cycle;
\draw[gp path] (6.163,15.379)--(6.163,16.919)--(12.415,16.919)--(12.415,15.379)--cycle;
\node[gp node right] at (11.131,16.534) {Strategic Objective Value};
\gpcolor{rgb color={0.000,0.000,0.000}}
\gpsetlinewidth{2.50}
\draw[gp path] (11.315,16.534)--(12.231,16.534);
\draw[gp path] (1.400,9.671)--(1.400,9.640)--(1.428,9.629)--(1.428,9.601)--(1.456,9.531)%
  --(1.540,9.529)--(1.540,9.520)--(1.568,9.516)--(1.568,9.510)--(1.624,9.507)--(1.708,9.491)%
  --(1.736,9.474)--(1.932,9.453)--(1.960,9.427)--(2.016,9.422)--(2.100,9.405)--(2.128,9.403)%
  --(2.128,9.394)--(2.212,9.391)--(2.240,9.387)--(2.296,9.386)--(2.324,9.382)--(2.324,9.373)%
  --(2.324,9.358)--(2.380,9.354)--(2.408,9.349)--(2.408,9.335)--(2.436,9.329)--(2.436,9.313)%
  --(2.464,9.287)--(2.520,9.283)--(2.548,9.249)--(2.604,9.235)--(2.828,9.229)--(2.940,9.229)%
  --(3.136,9.213)--(3.136,9.206)--(3.192,13.267)--(3.192,13.183)--(3.192,12.615)--(3.192,12.590)%
  --(3.220,12.579)--(3.220,12.569)--(3.248,12.557)--(3.276,12.526)--(3.332,12.470)--(3.332,12.378)%
  --(3.332,12.364)--(3.360,12.357)--(3.360,12.337)--(3.360,12.253)--(3.360,12.201)--(3.388,12.177)%
  --(3.388,12.156)--(3.388,12.151)--(3.388,12.141)--(3.416,12.119)--(3.416,12.045)--(3.444,12.043)%
  --(3.472,12.019)--(3.472,12.018)--(3.472,12.009)--(3.500,11.993)--(3.500,11.989)--(3.500,11.975)%
  --(3.528,11.831)--(3.528,11.821)--(3.528,11.690)--(3.528,11.678)--(3.528,11.653)--(3.528,11.648)%
  --(3.584,11.625)--(3.584,11.611)--(3.584,11.595)--(3.612,11.511)--(3.612,11.450)--(3.640,11.433)%
  --(3.640,11.416)--(3.696,11.407)--(3.696,11.398)--(3.752,11.396)--(3.780,11.389)--(3.780,11.381)%
  --(3.808,11.373)--(3.836,11.335)--(3.836,11.318)--(3.836,11.309)--(3.864,11.305)--(3.864,11.283)%
  --(3.892,11.249)--(3.948,11.241)--(4.032,11.230)--(4.144,11.223)--(4.200,11.217)--(4.200,11.189)%
  --(4.200,11.188)--(4.200,11.148)--(4.228,11.143)--(4.228,11.141)--(4.228,11.085)--(4.424,11.066)%
  --(4.424,11.060)--(4.452,11.056)--(4.480,11.045)--(4.480,11.034)--(4.676,11.030)--(4.704,11.028)%
  --(4.760,11.028)--(4.788,11.022)--(4.844,11.014)--(4.844,11.005)--(4.844,11.000)--(4.872,14.071)%
  --(4.872,14.054)--(4.900,14.033)--(4.900,13.955)--(4.900,13.740)--(4.900,13.705)--(4.928,13.679)%
  --(4.928,13.663)--(4.928,13.576)--(4.956,13.558)--(4.956,13.556)--(5.040,13.555)--(5.040,13.542)%
  --(5.068,13.518)--(5.068,13.485)--(5.068,13.471)--(5.068,13.465)--(5.068,13.245)--(5.068,13.236)%
  --(5.096,13.154)--(5.124,13.136)--(5.180,13.126)--(5.180,13.117)--(5.180,13.105)--(5.236,13.081)%
  --(5.264,13.059)--(5.264,13.037)--(5.264,12.994)--(5.264,12.980)--(5.292,12.949)--(5.292,12.943)%
  --(5.292,12.922)--(5.292,12.913)--(5.320,12.878)--(5.376,12.862)--(5.432,12.859)--(5.460,12.832)%
  --(5.460,12.831)--(5.460,12.820)--(5.460,12.818)--(5.460,12.817)--(5.488,12.784)--(5.488,12.774)%
  --(5.488,12.747)--(5.544,12.728)--(5.600,12.693)--(5.684,12.670)--(5.684,12.669)--(5.740,12.664)%
  --(5.768,12.640)--(5.796,12.627)--(5.824,12.620)--(5.908,12.604)--(5.908,12.572)--(5.936,12.540)%
  --(5.936,12.535)--(5.964,12.531)--(6.020,12.523)--(6.048,12.503)--(6.048,12.493)--(6.076,12.487)%
  --(6.076,12.469)--(6.104,12.448)--(6.104,12.435)--(6.104,12.431)--(6.132,12.421)--(6.216,12.384)%
  --(6.244,12.383)--(6.328,12.373)--(6.328,12.346)--(6.384,12.324)--(6.384,12.321)--(6.440,12.279)%
  --(6.496,12.268)--(6.580,14.928)--(6.580,14.911)--(6.608,14.882)--(6.608,14.853)--(6.636,14.805)%
  --(6.636,14.779)--(6.664,14.778)--(6.692,14.762)--(6.720,14.463)--(6.720,14.459)--(6.720,14.447)%
  --(6.748,14.435)--(6.748,14.430)--(6.748,14.389)--(6.776,14.311)--(6.776,14.293)--(6.804,14.263)%
  --(6.804,14.237)--(6.832,13.973)--(6.860,13.940)--(6.860,13.891)--(6.860,13.869)--(6.888,13.860)%
  --(6.888,13.853)--(6.888,13.731)--(6.916,13.728)--(6.944,13.706)--(6.944,13.687)--(6.972,13.682)%
  --(6.972,13.628)--(6.972,13.624)--(6.972,13.620)--(6.972,13.615)--(7.027,13.589)--(7.027,13.555)%
  --(7.083,13.547)--(7.111,13.541)--(7.111,13.534)--(7.139,13.525)--(7.139,13.476)--(7.167,13.471)%
  --(7.251,13.380)--(7.251,13.375)--(7.279,13.372)--(7.335,13.359)--(7.335,13.358)--(7.335,13.339)%
  --(7.391,13.334)--(7.391,13.325)--(7.419,13.315)--(7.475,13.307)--(7.475,13.289)--(7.475,13.288)%
  --(7.503,13.276)--(7.531,13.274)--(7.531,13.256)--(7.531,13.233)--(7.559,13.227)--(7.559,13.223)%
  --(7.559,13.216)--(7.559,13.203)--(7.559,13.185)--(7.615,13.095)--(7.615,13.090)--(7.615,13.083)%
  --(7.615,13.080)--(7.727,13.077)--(7.755,13.052)--(7.783,13.029)--(7.783,13.020)--(7.811,13.017)%
  --(7.867,13.016)--(7.867,12.993)--(7.867,12.984)--(7.895,12.984)--(7.923,12.974)--(7.923,12.966)%
  --(7.979,12.965)--(8.063,12.959)--(8.147,12.959)--(8.147,12.958)--(8.147,12.954)--(8.231,12.937)%
  --(8.259,12.928)--(8.287,15.370)--(8.287,15.357)--(8.287,15.000)--(8.287,14.907)--(8.315,14.894)%
  --(8.371,14.882)--(8.371,14.822)--(8.399,14.795)--(8.399,14.482)--(8.399,14.464)--(8.427,14.463)%
  --(8.427,14.455)--(8.427,14.413)--(8.427,14.411)--(8.455,14.359)--(8.483,14.338)--(8.483,14.298)%
  --(8.483,14.290)--(8.539,14.270)--(8.539,14.262)--(8.651,14.250)--(8.679,14.225)--(8.679,14.216)%
  --(8.707,14.206)--(8.707,14.194)--(8.735,14.175)--(8.763,14.164)--(8.791,14.159)--(8.791,14.157)%
  --(8.791,14.135)--(8.819,14.132)--(8.847,14.121)--(8.847,14.115)--(8.875,14.071)--(8.903,14.070)%
  --(8.903,14.041)--(8.931,13.993)--(8.959,13.979)--(8.959,13.978)--(8.987,13.974)--(9.043,13.963)%
  --(9.071,13.950)--(9.127,13.939)--(9.127,13.934)--(9.183,13.929)--(9.183,13.926)--(9.239,13.911)%
  --(9.295,13.909)--(9.379,13.904)--(9.407,13.893)--(9.407,13.840)--(9.463,13.840)--(9.463,13.836)%
  --(9.463,13.821)--(9.491,13.799)--(9.519,13.787)--(9.547,13.787)--(9.547,13.777)--(9.575,13.750)%
  --(9.575,13.740)--(9.575,13.737)--(9.631,13.737)--(9.631,13.731)--(9.715,13.729)--(9.771,13.727)%
  --(9.883,13.726)--(9.995,16.250)--(9.995,16.216)--(9.995,16.192)--(9.995,16.133)--(10.023,15.809)%
  --(10.023,15.784)--(10.051,15.773)--(10.107,15.757)--(10.163,15.664)--(10.163,15.643)--(10.163,15.582)%
  --(10.163,15.561)--(10.163,15.553)--(10.191,15.519)--(10.191,15.517)--(10.191,15.491)--(10.247,15.474)%
  --(10.247,15.332)--(10.275,15.309)--(10.275,15.299)--(10.275,15.292)--(10.331,15.291)--(10.331,15.276)%
  --(10.359,15.218)--(10.359,15.186)--(10.359,15.073)--(10.387,15.055)--(10.415,15.052)--(10.415,15.039)%
  --(10.415,15.029)--(10.415,15.022)--(10.443,15.010)--(10.499,15.009)--(10.499,15.005)--(10.555,14.948)%
  --(10.611,14.937)--(10.611,14.932)--(10.611,14.931)--(10.667,14.910)--(10.667,14.909)--(10.667,14.901)%
  --(10.695,14.900)--(10.751,14.868)--(10.807,14.851)--(10.807,14.845)--(10.835,14.817)--(10.947,14.802)%
  --(10.947,14.795)--(10.947,14.789)--(10.975,14.782)--(11.003,14.769)--(11.003,14.767)--(11.031,14.763)%
  --(11.031,14.759)--(11.059,14.758)--(11.059,14.754)--(11.059,14.741)--(11.115,14.737)--(11.143,14.715)%
  --(11.143,14.710)--(11.227,14.705)--(11.255,14.675)--(11.283,14.670)--(11.339,14.668)--(11.367,14.660)%
  --(11.479,14.660)--(11.507,14.650)--(11.535,14.647)--(11.563,14.646)--(11.619,14.642)--(11.647,14.642)%
  --(11.647,14.634)--(11.647,14.604);
\gpcolor{color=gp lt color border}
\node[gp node right] at (11.131,15.764) {Strategic Urgency};
\gpcolor{rgb color={0.184,0.243,0.918}}
\draw[gp path] (11.315,15.764)--(12.231,15.764);
\draw[gp path] (1.400,10.509)--(1.400,10.484)--(1.428,10.477)--(1.428,10.451)--(1.456,10.386)%
  --(1.540,10.388)--(1.540,10.382)--(1.568,10.382)--(1.568,10.377)--(1.624,10.376)--(1.708,10.364)%
  --(1.736,10.349)--(1.932,10.330)--(1.960,10.309)--(1.960,10.311)--(2.016,10.307)--(2.100,10.291)%
  --(2.128,10.291)--(2.128,10.284)--(2.212,10.284)--(2.240,10.283)--(2.296,10.284)--(2.324,10.281)%
  --(2.324,10.274)--(2.324,10.261)--(2.380,10.261)--(2.408,10.259)--(2.408,10.250)--(2.436,10.244)%
  --(2.436,10.229)--(2.464,10.204)--(2.520,10.200)--(2.548,10.168)--(2.604,10.155)--(2.828,10.151)%
  --(2.940,10.150)--(3.136,10.136)--(3.136,10.131)--(3.192,14.055)--(3.192,13.977)--(3.192,13.416)%
  --(3.192,13.401)--(3.220,13.394)--(3.220,13.391)--(3.248,13.385)--(3.276,13.357)--(3.332,13.304)%
  --(3.332,13.217)--(3.332,13.205)--(3.360,13.201)--(3.360,13.185)--(3.360,13.106)--(3.360,13.054)%
  --(3.388,13.033)--(3.388,13.014)--(3.388,13.013)--(3.388,13.008)--(3.416,12.988)--(3.416,12.916)%
  --(3.444,12.916)--(3.472,12.896)--(3.472,12.889)--(3.500,12.875)--(3.500,12.872)--(3.500,12.858)%
  --(3.528,12.718)--(3.528,12.711)--(3.528,12.582)--(3.528,12.568)--(3.528,12.543)--(3.528,12.542)%
  --(3.584,12.521)--(3.584,12.506)--(3.584,12.492)--(3.612,12.411)--(3.612,12.352)--(3.640,12.337)%
  --(3.640,12.322)--(3.696,12.314)--(3.696,12.305)--(3.752,12.303)--(3.780,12.300)--(3.780,12.301)%
  --(3.780,12.293)--(3.808,12.286)--(3.836,12.251)--(3.836,12.234)--(3.836,12.226)--(3.864,12.222)%
  --(3.864,12.201)--(3.892,12.168)--(3.948,12.161)--(4.032,12.149)--(4.144,12.146)--(4.200,12.140)%
  --(4.200,12.113)--(4.200,12.076)--(4.228,12.070)--(4.228,12.067)--(4.228,12.014)--(4.424,11.996)%
  --(4.424,11.989)--(4.452,11.985)--(4.480,11.975)--(4.480,11.966)--(4.676,11.963)--(4.704,11.961)%
  --(4.760,11.961)--(4.788,11.954)--(4.844,11.946)--(4.844,11.938)--(4.844,11.934)--(4.872,14.875)%
  --(4.872,14.860)--(4.900,14.844)--(4.900,14.768)--(4.900,14.555)--(4.900,14.521)--(4.928,14.498)%
  --(4.928,14.485)--(4.928,14.398)--(4.956,14.382)--(4.956,14.383)--(5.040,14.385)--(5.040,14.375)%
  --(5.068,14.353)--(5.068,14.322)--(5.068,14.308)--(5.068,14.304)--(5.068,14.086)--(5.068,14.078)%
  --(5.096,14.000)--(5.124,13.984)--(5.180,13.976)--(5.180,13.969)--(5.180,13.957)--(5.236,13.935)%
  --(5.264,13.914)--(5.264,13.894)--(5.264,13.853)--(5.264,13.842)--(5.292,13.812)--(5.292,13.807)%
  --(5.292,13.788)--(5.292,13.780)--(5.320,13.748)--(5.376,13.734)--(5.432,13.731)--(5.460,13.704)%
  --(5.460,13.703)--(5.460,13.694)--(5.460,13.691)--(5.460,13.689)--(5.488,13.657)--(5.488,13.647)%
  --(5.488,13.620)--(5.544,13.604)--(5.600,13.569)--(5.684,13.547)--(5.684,13.546)--(5.740,13.543)%
  --(5.768,13.519)--(5.796,13.506)--(5.824,13.501)--(5.908,13.485)--(5.908,13.455)--(5.936,13.426)%
  --(5.936,13.422)--(5.964,13.418)--(6.020,13.410)--(6.048,13.389)--(6.048,13.382)--(6.076,13.376)%
  --(6.076,13.358)--(6.104,13.336)--(6.104,13.324)--(6.104,13.321)--(6.132,13.311)--(6.216,13.276)%
  --(6.244,13.275)--(6.328,13.265)--(6.328,13.240)--(6.384,13.220)--(6.384,13.218)--(6.440,13.178)%
  --(6.496,13.170)--(6.580,15.701)--(6.580,15.690)--(6.608,15.663)--(6.608,15.634)--(6.636,15.590)%
  --(6.636,15.568)--(6.664,15.569)--(6.692,15.556)--(6.720,15.256)--(6.720,15.245)--(6.748,15.236)%
  --(6.748,15.231)--(6.748,15.194)--(6.776,15.118)--(6.776,15.102)--(6.804,15.072)--(6.804,15.048)%
  --(6.832,14.786)--(6.860,14.756)--(6.860,14.707)--(6.860,14.688)--(6.888,14.679)--(6.888,14.673)%
  --(6.888,14.556)--(6.916,14.553)--(6.944,14.531)--(6.944,14.513)--(6.972,14.508)--(6.972,14.453)%
  --(6.972,14.450)--(6.972,14.445)--(6.972,14.444)--(7.027,14.417)--(7.027,14.384)--(7.083,14.377)%
  --(7.111,14.373)--(7.111,14.366)--(7.139,14.359)--(7.139,14.311)--(7.167,14.307)--(7.251,14.216)%
  --(7.251,14.212)--(7.279,14.213)--(7.335,14.200)--(7.335,14.198)--(7.335,14.180)--(7.391,14.178)%
  --(7.391,14.168)--(7.419,14.159)--(7.475,14.151)--(7.475,14.133)--(7.503,14.125)--(7.531,14.124)%
  --(7.531,14.107)--(7.531,14.085)--(7.559,14.078)--(7.559,14.074)--(7.559,14.066)--(7.559,14.055)%
  --(7.559,14.038)--(7.615,13.947)--(7.615,13.941)--(7.615,13.935)--(7.615,13.933)--(7.727,13.931)%
  --(7.755,13.908)--(7.783,13.886)--(7.783,13.876)--(7.811,13.876)--(7.867,13.875)--(7.867,13.854)%
  --(7.867,13.845)--(7.895,13.844)--(7.923,13.835)--(7.923,13.829)--(7.979,13.826)--(8.063,13.820)%
  --(8.147,13.819)--(8.147,13.818)--(8.147,13.816)--(8.231,13.797)--(8.259,13.787)--(8.287,16.109)%
  --(8.287,16.098)--(8.287,15.745)--(8.287,15.654)--(8.315,15.645)--(8.371,15.636)--(8.371,15.578)%
  --(8.399,15.551)--(8.399,15.241)--(8.399,15.224)--(8.427,15.227)--(8.427,15.221)--(8.427,15.180)%
  --(8.427,15.181)--(8.455,15.129)--(8.483,15.111)--(8.483,15.074)--(8.483,15.067)--(8.539,15.048)%
  --(8.539,15.042)--(8.651,15.030)--(8.679,15.008)--(8.679,15.002)--(8.707,14.991)--(8.707,14.980)%
  --(8.735,14.964)--(8.763,14.954)--(8.791,14.949)--(8.791,14.927)--(8.819,14.924)--(8.847,14.914)%
  --(8.847,14.909)--(8.875,14.865)--(8.903,14.864)--(8.903,14.837)--(8.931,14.790)--(8.959,14.777)%
  --(8.959,14.776)--(8.987,14.771)--(9.043,14.762)--(9.071,14.751)--(9.127,14.742)--(9.127,14.736)%
  --(9.183,14.731)--(9.183,14.728)--(9.239,14.715)--(9.295,14.713)--(9.379,14.712)--(9.407,14.699)%
  --(9.407,14.649)--(9.463,14.648)--(9.463,14.646)--(9.463,14.632)--(9.491,14.612)--(9.519,14.599)%
  --(9.547,14.598)--(9.547,14.589)--(9.575,14.562)--(9.575,14.551)--(9.575,14.550)--(9.631,14.550)%
  --(9.631,14.544)--(9.715,14.542)--(9.771,14.541)--(9.883,14.539)--(9.995,16.945)--(9.995,16.913)%
  --(9.995,16.891)--(9.995,16.834)--(10.023,16.514)--(10.023,16.490)--(10.051,16.480)--(10.107,16.468)%
  --(10.163,16.377)--(10.163,16.361)--(10.163,16.298)--(10.163,16.278)--(10.163,16.273)--(10.191,16.241)%
  --(10.191,16.239)--(10.191,16.217)--(10.247,16.202)--(10.247,16.060)--(10.275,16.040)--(10.275,16.030)%
  --(10.275,16.027)--(10.331,16.027)--(10.331,16.016)--(10.359,15.960)--(10.359,15.929)--(10.359,15.819)%
  --(10.387,15.802)--(10.415,15.800)--(10.415,15.788)--(10.415,15.777)--(10.415,15.771)--(10.443,15.760)%
  --(10.499,15.761)--(10.555,15.706)--(10.611,15.694)--(10.611,15.689)--(10.611,15.690)--(10.667,15.670)%
  --(10.667,15.663)--(10.695,15.662)--(10.751,15.630)--(10.807,15.612)--(10.807,15.608)--(10.835,15.580)%
  --(10.947,15.566)--(10.947,15.560)--(10.947,15.554)--(10.975,15.548)--(11.003,15.535)--(11.003,15.534)%
  --(11.031,15.530)--(11.031,15.525)--(11.059,15.524)--(11.059,15.519)--(11.059,15.505)--(11.115,15.501)%
  --(11.143,15.481)--(11.143,15.476)--(11.227,15.472)--(11.255,15.443)--(11.283,15.439)--(11.339,15.438)%
  --(11.367,15.432)--(11.479,15.431)--(11.507,15.420)--(11.535,15.416)--(11.563,15.415)--(11.619,15.414)%
  --(11.647,15.415)--(11.647,15.408)--(11.647,15.381);
\gpcolor{color=gp lt color border}
\gpsetlinewidth{1.00}
\draw[gp path] (1.400,17.099)--(1.400,9.000)--(12.599,9.000)--(12.599,17.099)--cycle;
\node[gp node center,rotate=-270] at (-1.652,13.049) {Objective value};
\node[gp node center,rotate=-270] at (15.697,13.049) {Strategic Urgency};
%% coordinates of the plot area
\gpdefrectangularnode{gp plot 1}{\pgfpoint{1.400cm}{9.000cm}}{\pgfpoint{12.599cm}{17.099cm}}
\gpcolor{rgb color={0.753,0.753,0.753}}
\gpsetlinewidth{2.00}
\draw[gp path] (1.400,0.000)--(12.599,0.000);
\gpcolor{color=gp lt color border}
\gpsetlinewidth{1.00}
\draw[gp path] (1.400,0.000)--(1.580,0.000);
\node[gp node right] at (1.216,-0.308) {$159000$};
\gpcolor{rgb color={0.753,0.753,0.753}}
\gpsetlinewidth{2.00}
\draw[gp path] (1.400,0.810)--(12.599,0.810);
\gpcolor{color=gp lt color border}
\gpsetlinewidth{1.00}
\draw[gp path] (1.400,0.810)--(1.580,0.810);
\node[gp node right] at (1.216,0.502) {$160000$};
\gpcolor{rgb color={0.753,0.753,0.753}}
\gpsetlinewidth{2.00}
\draw[gp path] (1.400,1.620)--(12.599,1.620);
\gpcolor{color=gp lt color border}
\gpsetlinewidth{1.00}
\draw[gp path] (1.400,1.620)--(1.580,1.620);
\node[gp node right] at (1.216,1.312) {$161000$};
\gpcolor{rgb color={0.753,0.753,0.753}}
\gpsetlinewidth{2.00}
\draw[gp path] (1.400,2.430)--(12.599,2.430);
\gpcolor{color=gp lt color border}
\gpsetlinewidth{1.00}
\draw[gp path] (1.400,2.430)--(1.580,2.430);
\node[gp node right] at (1.216,2.122) {$162000$};
\gpcolor{rgb color={0.753,0.753,0.753}}
\gpsetlinewidth{2.00}
\draw[gp path] (1.400,3.240)--(12.599,3.240);
\gpcolor{color=gp lt color border}
\gpsetlinewidth{1.00}
\draw[gp path] (1.400,3.240)--(1.580,3.240);
\node[gp node right] at (1.216,2.932) {$163000$};
\gpcolor{rgb color={0.753,0.753,0.753}}
\gpsetlinewidth{2.00}
\draw[gp path] (1.400,4.049)--(12.599,4.049);
\gpcolor{color=gp lt color border}
\gpsetlinewidth{1.00}
\draw[gp path] (1.400,4.049)--(1.580,4.049);
\node[gp node right] at (1.216,3.741) {$164000$};
\gpcolor{rgb color={0.753,0.753,0.753}}
\gpsetlinewidth{2.00}
\draw[gp path] (1.400,4.859)--(12.599,4.859);
\gpcolor{color=gp lt color border}
\gpsetlinewidth{1.00}
\draw[gp path] (1.400,4.859)--(1.580,4.859);
\node[gp node right] at (1.216,4.551) {$165000$};
\gpcolor{rgb color={0.753,0.753,0.753}}
\gpsetlinewidth{2.00}
\draw[gp path] (1.400,5.669)--(12.599,5.669);
\gpcolor{color=gp lt color border}
\gpsetlinewidth{1.00}
\draw[gp path] (1.400,5.669)--(1.580,5.669);
\node[gp node right] at (1.216,5.361) {$166000$};
\gpcolor{rgb color={0.753,0.753,0.753}}
\gpsetlinewidth{2.00}
\draw[gp path] (1.400,6.479)--(5.979,6.479);
\draw[gp path] (12.415,6.479)--(12.599,6.479);
\gpcolor{color=gp lt color border}
\gpsetlinewidth{1.00}
\draw[gp path] (1.400,6.479)--(1.580,6.479);
\node[gp node right] at (1.216,6.171) {$167000$};
\gpcolor{rgb color={0.753,0.753,0.753}}
\gpsetlinewidth{2.00}
\draw[gp path] (1.400,7.289)--(5.979,7.289);
\draw[gp path] (12.415,7.289)--(12.599,7.289);
\gpcolor{color=gp lt color border}
\gpsetlinewidth{1.00}
\draw[gp path] (1.400,7.289)--(1.580,7.289);
\node[gp node right] at (1.216,6.981) {$168000$};
\gpcolor{rgb color={0.753,0.753,0.753}}
\gpsetlinewidth{2.00}
\draw[gp path] (1.400,8.099)--(12.599,8.099);
\gpcolor{color=gp lt color border}
\gpsetlinewidth{1.00}
\draw[gp path] (1.400,8.099)--(1.580,8.099);
\node[gp node right] at (1.216,7.791) {$169000$};
\gpcolor{rgb color={0.753,0.753,0.753}}
\gpsetlinewidth{2.00}
\draw[gp path] (1.400,0.000)--(1.400,8.099);
\gpcolor{color=gp lt color border}
\gpsetlinewidth{1.00}
\draw[gp path] (1.400,0.000)--(1.400,0.180);
\draw[gp path] (1.400,8.099)--(1.400,7.919);
\node[gp node left,rotate=270] at (1.400,-0.184) {0};
\gpcolor{rgb color={0.753,0.753,0.753}}
\gpsetlinewidth{2.00}
\draw[gp path] (2.800,0.000)--(2.800,8.099);
\gpcolor{color=gp lt color border}
\gpsetlinewidth{1.00}
\draw[gp path] (2.800,0.000)--(2.800,0.180);
\draw[gp path] (2.800,8.099)--(2.800,7.919);
\node[gp node left,rotate=270] at (2.800,-0.184) {50};
\gpcolor{rgb color={0.753,0.753,0.753}}
\gpsetlinewidth{2.00}
\draw[gp path] (4.200,0.000)--(4.200,8.099);
\gpcolor{color=gp lt color border}
\gpsetlinewidth{1.00}
\draw[gp path] (4.200,0.000)--(4.200,0.180);
\draw[gp path] (4.200,8.099)--(4.200,7.919);
\node[gp node left,rotate=270] at (4.200,-0.184) {100};
\gpcolor{rgb color={0.753,0.753,0.753}}
\gpsetlinewidth{2.00}
\draw[gp path] (5.600,0.000)--(5.600,8.099);
\gpcolor{color=gp lt color border}
\gpsetlinewidth{1.00}
\draw[gp path] (5.600,0.000)--(5.600,0.180);
\draw[gp path] (5.600,8.099)--(5.600,7.919);
\node[gp node left,rotate=270] at (5.600,-0.184) {150};
\gpcolor{rgb color={0.753,0.753,0.753}}
\gpsetlinewidth{2.00}
\draw[gp path] (7.000,0.000)--(7.000,6.379);
\draw[gp path] (7.000,7.919)--(7.000,8.099);
\gpcolor{color=gp lt color border}
\gpsetlinewidth{1.00}
\draw[gp path] (7.000,0.000)--(7.000,0.180);
\draw[gp path] (7.000,8.099)--(7.000,7.919);
\node[gp node left,rotate=270] at (7.000,-0.184) {200};
\gpcolor{rgb color={0.753,0.753,0.753}}
\gpsetlinewidth{2.00}
\draw[gp path] (8.399,0.000)--(8.399,6.379);
\draw[gp path] (8.399,7.919)--(8.399,8.099);
\gpcolor{color=gp lt color border}
\gpsetlinewidth{1.00}
\draw[gp path] (8.399,0.000)--(8.399,0.180);
\draw[gp path] (8.399,8.099)--(8.399,7.919);
\node[gp node left,rotate=270] at (8.399,-0.184) {250};
\gpcolor{rgb color={0.753,0.753,0.753}}
\gpsetlinewidth{2.00}
\draw[gp path] (9.799,0.000)--(9.799,6.379);
\draw[gp path] (9.799,7.919)--(9.799,8.099);
\gpcolor{color=gp lt color border}
\gpsetlinewidth{1.00}
\draw[gp path] (9.799,0.000)--(9.799,0.180);
\draw[gp path] (9.799,8.099)--(9.799,7.919);
\node[gp node left,rotate=270] at (9.799,-0.184) {300};
\gpcolor{rgb color={0.753,0.753,0.753}}
\gpsetlinewidth{2.00}
\draw[gp path] (11.199,0.000)--(11.199,6.379);
\draw[gp path] (11.199,7.919)--(11.199,8.099);
\gpcolor{color=gp lt color border}
\gpsetlinewidth{1.00}
\draw[gp path] (11.199,0.000)--(11.199,0.180);
\draw[gp path] (11.199,8.099)--(11.199,7.919);
\node[gp node left,rotate=270] at (11.199,-0.184) {350};
\gpcolor{rgb color={0.753,0.753,0.753}}
\gpsetlinewidth{2.00}
\draw[gp path] (12.599,0.000)--(12.599,8.099);
\gpcolor{color=gp lt color border}
\gpsetlinewidth{1.00}
\draw[gp path] (12.599,0.000)--(12.599,0.180);
\draw[gp path] (12.599,8.099)--(12.599,7.919);
\node[gp node left,rotate=270] at (12.599,-0.184) {400};
\draw[gp path] (12.599,0.000)--(12.419,0.000);
\node[gp node left] at (12.783,-0.308) {$4.5\times10^{6}$};
\draw[gp path] (12.599,1.012)--(12.419,1.012);
\node[gp node left] at (12.783,0.704) {$5\times10^{6}$};
\draw[gp path] (12.599,2.025)--(12.419,2.025);
\node[gp node left] at (12.783,1.717) {$5.5\times10^{6}$};
\draw[gp path] (12.599,3.037)--(12.419,3.037);
\node[gp node left] at (12.783,2.729) {$6\times10^{6}$};
\draw[gp path] (12.599,4.050)--(12.419,4.050);
\node[gp node left] at (12.783,3.742) {$6.5\times10^{6}$};
\draw[gp path] (12.599,5.062)--(12.419,5.062);
\node[gp node left] at (12.783,4.754) {$7\times10^{6}$};
\draw[gp path] (12.599,6.074)--(12.419,6.074);
\node[gp node left] at (12.783,5.766) {$7.5\times10^{6}$};
\draw[gp path] (12.599,7.087)--(12.419,7.087);
\node[gp node left] at (12.783,6.779) {$8\times10^{6}$};
\draw[gp path] (12.599,8.099)--(12.419,8.099);
\node[gp node left] at (12.783,7.791) {$8.5\times10^{6}$};
\draw[gp path] (1.400,8.099)--(1.400,0.000)--(12.599,0.000)--(12.599,8.099)--cycle;
\draw[gp path] (5.979,6.379)--(5.979,7.919)--(12.415,7.919)--(12.415,6.379)--cycle;
\node[gp node right] at (11.131,7.534) {Strategic Resource Penalty};
\gpcolor{rgb color={0.000,0.533,0.208}}
\gpsetlinewidth{2.50}
\draw[gp path] (11.315,7.534)--(12.231,7.534);
\draw[gp path] (1.400,7.398)--(1.400,7.234)--(1.428,7.184)--(1.428,7.163)--(1.456,7.048)%
  --(1.540,6.919)--(1.540,6.844)--(1.568,6.734)--(1.568,6.699)--(1.624,6.650)--(1.708,6.627)%
  --(1.736,6.580)--(1.932,6.555)--(1.960,6.454)--(1.960,6.354)--(2.016,6.427)--(2.100,6.402)%
  --(2.128,6.344)--(2.128,6.296)--(2.212,6.273)--(2.240,6.265)--(2.296,6.235)--(2.324,6.233)%
  --(2.324,6.192)--(2.324,6.181)--(2.380,6.136)--(2.408,6.090)--(2.408,6.033)--(2.436,6.009)%
  --(2.464,6.005)--(2.520,6.005)--(2.548,5.983)--(2.604,5.975)--(2.828,5.969)--(2.940,5.964)%
  --(3.136,5.925)--(3.136,5.881)--(3.192,5.604)--(3.192,5.419)--(3.192,5.281)--(3.192,5.179)%
  --(3.220,5.072)--(3.220,4.912)--(3.248,4.856)--(3.276,4.779)--(3.332,4.723)--(3.332,4.667)%
  --(3.332,4.586)--(3.360,4.520)--(3.360,4.474)--(3.360,4.402)--(3.360,4.389)--(3.388,4.314)%
  --(3.388,4.286)--(3.388,4.189)--(3.388,4.137)--(3.416,4.118)--(3.416,4.075)--(3.444,4.003)%
  --(3.472,3.884)--(3.472,3.880)--(3.472,3.872)--(3.500,3.843)--(3.500,3.829)--(3.500,3.922)%
  --(3.528,3.790)--(3.528,3.704)--(3.528,3.536)--(3.528,3.551)--(3.528,3.547)--(3.528,3.481)%
  --(3.584,3.329)--(3.584,3.326)--(3.584,3.350)--(3.612,3.266)--(3.612,3.151)--(3.640,3.105)%
  --(3.640,3.108)--(3.696,3.094)--(3.752,3.097)--(3.780,3.013)--(3.780,2.998)--(3.780,2.994)%
  --(3.808,2.993)--(3.836,2.965)--(3.836,2.944)--(3.836,2.929)--(3.864,2.925)--(3.864,2.918)%
  --(3.892,2.953)--(3.948,2.955)--(4.032,2.946)--(4.144,2.929)--(4.200,2.888)--(4.200,2.882)%
  --(4.200,2.861)--(4.200,2.837)--(4.228,2.831)--(4.228,2.844)--(4.228,2.819)--(4.424,2.800)%
  --(4.424,2.856)--(4.452,2.765)--(4.480,2.740)--(4.480,2.719)--(4.676,2.580)--(4.704,2.591)%
  --(4.760,2.594)--(4.788,2.618)--(4.844,2.623)--(4.844,2.612)--(4.844,2.608)--(4.872,2.656)%
  --(4.872,2.618)--(4.900,2.563)--(4.900,2.467)--(4.900,2.413)--(4.900,2.397)--(4.928,2.320)%
  --(4.928,2.291)--(4.928,2.267)--(4.956,2.315)--(4.956,2.296)--(5.040,2.149)--(5.040,2.104)%
  --(5.068,2.101)--(5.068,2.087)--(5.068,2.089)--(5.068,2.059)--(5.068,1.968)--(5.068,1.958)%
  --(5.096,1.902)--(5.124,1.861)--(5.180,1.834)--(5.180,1.841)--(5.180,1.848)--(5.236,1.764)%
  --(5.264,1.739)--(5.264,1.851)--(5.264,1.794)--(5.264,1.773)--(5.292,1.702)--(5.292,1.694)%
  --(5.292,1.686)--(5.292,1.510)--(5.320,1.498)--(5.376,1.485)--(5.432,1.485)--(5.460,1.412)%
  --(5.460,1.401)--(5.460,1.388)--(5.460,1.389)--(5.460,1.410)--(5.488,1.404)--(5.488,1.385)%
  --(5.488,1.390)--(5.544,1.370)--(5.600,1.365)--(5.684,1.376)--(5.684,1.223)--(5.740,1.204)%
  --(5.768,1.318)--(5.796,1.321)--(5.824,1.354)--(5.908,1.426)--(5.908,1.384)--(5.936,1.212)%
  --(5.936,1.197)--(5.964,1.323)--(6.020,1.314)--(6.048,1.311)--(6.048,1.266)--(6.076,1.277)%
  --(6.076,1.301)--(6.104,1.246)--(6.104,1.244)--(6.104,1.191)--(6.132,1.251)--(6.216,1.109)%
  --(6.244,1.112)--(6.328,1.118)--(6.328,1.103)--(6.384,1.083)--(6.384,1.008)--(6.440,0.990)%
  --(6.496,0.922)--(6.580,1.273)--(6.580,1.257)--(6.608,1.233)--(6.608,1.218)--(6.636,1.274)%
  --(6.636,1.259)--(6.664,1.302)--(6.692,1.317)--(6.720,1.344)--(6.720,1.336)--(6.720,1.331)%
  --(6.748,1.238)--(6.748,1.242)--(6.748,1.225)--(6.776,1.178)--(6.776,1.172)--(6.804,1.086)%
  --(6.804,1.047)--(6.832,1.026)--(6.860,1.001)--(6.860,0.998)--(6.860,0.858)--(6.888,0.856)%
  --(6.888,0.841)--(6.888,0.692)--(6.916,0.684)--(6.944,0.624)--(6.944,0.627)--(6.972,0.643)%
  --(6.972,0.654)--(6.972,0.658)--(6.972,0.663)--(6.972,0.616)--(7.027,0.630)--(7.027,0.641)%
  --(7.083,0.639)--(7.111,0.611)--(7.111,0.554)--(7.139,0.543)--(7.139,0.484)--(7.167,0.484)%
  --(7.251,0.441)--(7.251,0.432)--(7.279,0.414)--(7.335,0.425)--(7.335,0.448)--(7.335,0.441)%
  --(7.391,0.384)--(7.391,0.390)--(7.419,0.437)--(7.475,0.439)--(7.475,0.441)--(7.475,0.467)%
  --(7.503,0.458)--(7.531,0.512)--(7.531,0.498)--(7.531,0.445)--(7.559,0.444)--(7.559,0.451)%
  --(7.559,0.449)--(7.559,0.451)--(7.559,0.446)--(7.615,0.310)--(7.615,0.331)--(7.615,0.324)%
  --(7.727,0.314)--(7.755,0.281)--(7.783,0.279)--(7.783,0.305)--(7.811,0.283)--(7.867,0.297)%
  --(7.867,0.212)--(7.867,0.213)--(7.895,0.219)--(7.923,0.157)--(7.923,0.131)--(7.979,0.144)%
  --(8.063,0.144)--(8.147,0.153)--(8.147,0.152)--(8.147,0.143)--(8.231,0.183)--(8.259,0.190)%
  --(8.287,0.931)--(8.287,0.825)--(8.287,0.771)--(8.287,0.608)--(8.315,0.569)--(8.371,0.563)%
  --(8.371,0.535)--(8.399,0.505)--(8.399,0.512)--(8.399,0.506)--(8.427,0.489)--(8.427,0.491)%
  --(8.427,0.462)--(8.427,0.472)--(8.455,0.490)--(8.483,0.425)--(8.483,0.434)--(8.483,0.432)%
  --(8.539,0.454)--(8.539,0.448)--(8.651,0.437)--(8.679,0.433)--(8.679,0.440)--(8.707,0.444)%
  --(8.707,0.448)--(8.735,0.431)--(8.763,0.437)--(8.791,0.441)--(8.791,0.446)--(8.791,0.453)%
  --(8.819,0.466)--(8.847,0.460)--(8.875,0.467)--(8.903,0.435)--(8.903,0.418)--(8.931,0.420)%
  --(8.959,0.433)--(8.959,0.446)--(8.987,0.451)--(9.043,0.436)--(9.071,0.432)--(9.127,0.423)%
  --(9.127,0.418)--(9.183,0.415)--(9.183,0.419)--(9.239,0.411)--(9.295,0.421)--(9.379,0.417)%
  --(9.379,0.415)--(9.407,0.421)--(9.407,0.411)--(9.463,0.364)--(9.463,0.353)--(9.463,0.347)%
  --(9.491,0.342)--(9.519,0.337)--(9.547,0.341)--(9.547,0.335)--(9.575,0.335)--(9.575,0.331)%
  --(9.575,0.327)--(9.631,0.334)--(9.631,0.294)--(9.715,0.294)--(9.771,0.299)--(9.883,0.310)%
  --(9.995,1.178)--(9.995,1.067)--(9.995,1.058)--(9.995,1.038)--(10.023,0.961)--(10.051,0.839)%
  --(10.107,0.705)--(10.163,0.694)--(10.163,0.680)--(10.163,0.660)--(10.163,0.649)--(10.191,0.699)%
  --(10.191,0.692)--(10.191,0.668)--(10.247,0.655)--(10.247,0.625)--(10.275,0.623)--(10.275,0.625)%
  --(10.275,0.621)--(10.331,0.624)--(10.331,0.616)--(10.359,0.607)--(10.359,0.604)--(10.387,0.600)%
  --(10.415,0.600)--(10.415,0.589)--(10.415,0.601)--(10.415,0.600)--(10.443,0.598)--(10.499,0.522)%
  --(10.499,0.506)--(10.555,0.509)--(10.611,0.506)--(10.611,0.628)--(10.611,0.612)--(10.667,0.607)%
  --(10.667,0.605)--(10.667,0.606)--(10.695,0.605)--(10.751,0.603)--(10.807,0.620)--(10.807,0.610)%
  --(10.835,0.610)--(10.947,0.607)--(10.947,0.604)--(10.947,0.610)--(10.975,0.602)--(11.003,0.601)%
  --(11.003,0.594)--(11.031,0.593)--(11.031,0.588)--(11.059,0.590)--(11.059,0.597)--(11.059,0.607)%
  --(11.115,0.608)--(11.143,0.596)--(11.143,0.611)--(11.227,0.515)--(11.255,0.510)--(11.283,0.506)%
  --(11.339,0.501)--(11.367,0.501)--(11.479,0.650)--(11.507,0.663)--(11.535,0.677)--(11.563,0.683)%
  --(11.619,0.673)--(11.647,0.668)--(11.647,0.673)--(11.647,0.646);
\gpcolor{color=gp lt color border}
\node[gp node right] at (11.131,6.764) {Strategic Clustering Value};
\gpcolor{rgb color={0.475,0.137,0.557}}
\draw[gp path] (11.315,6.764)--(12.231,6.764);
\draw[gp path] (1.400,2.835)--(1.400,2.688)--(1.428,2.560)--(1.428,2.494)--(1.456,2.349)%
  --(1.540,2.228)--(1.540,2.143)--(1.568,2.021)--(1.568,1.979)--(1.624,1.920)--(1.708,1.801)%
  --(1.736,1.741)--(1.932,1.705)--(1.960,1.564)--(1.960,1.467)--(2.016,1.445)--(2.100,1.422)%
  --(2.128,1.345)--(2.128,1.303)--(2.212,1.204)--(2.240,1.124)--(2.296,1.059)--(2.324,1.044)%
  --(2.324,0.969)--(2.324,0.899)--(2.380,0.823)--(2.408,0.715)--(2.408,0.567)--(2.436,0.552)%
  --(2.436,0.555)--(2.464,0.519)--(2.520,0.500)--(2.548,0.428)--(2.604,0.412)--(2.828,0.374)%
  --(2.940,0.371)--(3.136,0.331)--(3.136,0.271)--(3.192,4.388)--(3.192,4.204)--(3.192,3.988)%
  --(3.192,3.707)--(3.220,3.574)--(3.220,3.358)--(3.248,3.206)--(3.276,3.098)--(3.332,3.013)%
  --(3.332,2.893)--(3.332,2.832)--(3.360,2.706)--(3.360,2.605)--(3.360,2.465)--(3.360,2.463)%
  --(3.388,2.399)--(3.388,2.322)--(3.388,2.198)--(3.388,2.063)--(3.416,2.018)--(3.416,1.924)%
  --(3.444,1.863)--(3.472,1.766)--(3.472,1.725)--(3.472,1.664)--(3.500,1.627)--(3.500,1.569)%
  --(3.500,1.565)--(3.528,1.470)--(3.528,1.376)--(3.528,1.337)--(3.528,1.359)--(3.528,1.349)%
  --(3.528,1.267)--(3.584,1.193)--(3.584,1.212)--(3.584,1.173)--(3.612,1.082)--(3.612,1.005)%
  --(3.640,0.968)--(3.640,0.913)--(3.696,0.874)--(3.696,0.877)--(3.752,0.858)--(3.780,0.754)%
  --(3.780,0.714)--(3.780,0.713)--(3.808,0.691)--(3.836,0.616)--(3.836,0.596)--(3.836,0.580)%
  --(3.864,0.560)--(3.864,0.542)--(3.892,0.543)--(3.948,0.494)--(4.032,0.494)--(4.144,0.421)%
  --(4.200,0.404)--(4.200,0.388)--(4.200,0.337)--(4.200,0.279)--(4.228,0.279)--(4.228,0.312)%
  --(4.228,0.242)--(4.424,0.202)--(4.424,0.224)--(4.452,0.210)--(4.480,0.192)--(4.480,0.158)%
  --(4.676,0.108)--(4.704,0.108)--(4.760,0.123)--(4.788,0.124)--(4.844,0.143)--(4.844,0.109)%
  --(4.844,0.090)--(4.872,3.965)--(4.872,3.922)--(4.900,3.786)--(4.900,3.724)--(4.900,3.644)%
  --(4.900,3.621)--(4.928,3.560)--(4.928,3.447)--(4.928,3.426)--(4.956,3.389)--(4.956,3.303)%
  --(5.040,3.193)--(5.040,3.112)--(5.068,3.048)--(5.068,3.006)--(5.068,2.945)--(5.068,2.884)%
  --(5.068,2.864)--(5.096,2.744)--(5.124,2.680)--(5.180,2.625)--(5.180,2.562)--(5.180,2.563)%
  --(5.236,2.490)--(5.264,2.475)--(5.264,2.416)--(5.264,2.338)--(5.264,2.280)--(5.292,2.234)%
  --(5.292,2.196)--(5.292,2.157)--(5.292,2.119)--(5.320,2.041)--(5.376,1.980)--(5.432,1.964)%
  --(5.460,1.971)--(5.460,1.955)--(5.460,1.919)--(5.460,1.956)--(5.488,1.940)--(5.488,1.922)%
  --(5.488,1.944)--(5.544,1.864)--(5.600,1.842)--(5.684,1.828)--(5.684,1.811)--(5.740,1.773)%
  --(5.768,1.756)--(5.796,1.758)--(5.824,1.698)--(5.908,1.697)--(5.908,1.634)--(5.936,1.540)%
  --(5.936,1.523)--(5.964,1.523)--(6.020,1.539)--(6.048,1.542)--(6.048,1.467)--(6.076,1.453)%
  --(6.076,1.452)--(6.104,1.476)--(6.104,1.462)--(6.104,1.424)--(6.132,1.430)--(6.216,1.368)%
  --(6.244,1.381)--(6.328,1.363)--(6.328,1.321)--(6.384,1.287)--(6.384,1.207)--(6.440,1.170)%
  --(6.496,1.077)--(6.580,4.929)--(6.580,4.767)--(6.608,4.701)--(6.608,4.696)--(6.636,4.603)%
  --(6.636,4.475)--(6.664,4.411)--(6.692,4.323)--(6.720,4.322)--(6.720,4.230)--(6.720,4.189)%
  --(6.748,4.125)--(6.748,4.099)--(6.748,4.008)--(6.776,3.902)--(6.776,3.863)--(6.804,3.862)%
  --(6.804,3.815)--(6.832,3.747)--(6.860,3.663)--(6.860,3.666)--(6.860,3.557)--(6.888,3.554)%
  --(6.888,3.534)--(6.888,3.393)--(6.916,3.397)--(6.944,3.399)--(6.944,3.379)--(6.972,3.396)%
  --(6.972,3.393)--(6.972,3.366)--(6.972,3.369)--(6.972,3.282)--(7.027,3.308)--(7.027,3.269)%
  --(7.083,3.250)--(7.111,3.209)--(7.111,3.162)--(7.139,3.143)--(7.139,3.105)--(7.167,3.085)%
  --(7.251,3.049)--(7.251,3.027)--(7.279,2.942)--(7.335,2.922)--(7.335,2.945)--(7.335,2.923)%
  --(7.391,2.860)--(7.391,2.857)--(7.419,2.858)--(7.475,2.839)--(7.475,2.822)--(7.475,2.780)%
  --(7.503,2.699)--(7.531,2.654)--(7.531,2.612)--(7.531,2.616)--(7.559,2.611)--(7.559,2.615)%
  --(7.559,2.639)--(7.559,2.621)--(7.559,2.582)--(7.615,2.583)--(7.615,2.622)--(7.615,2.598)%
  --(7.615,2.571)--(7.727,2.530)--(7.755,2.492)--(7.783,2.428)--(7.783,2.468)--(7.811,2.384)%
  --(7.867,2.392)--(7.867,2.309)--(7.867,2.308)--(7.895,2.350)--(7.923,2.330)--(7.923,2.266)%
  --(7.979,2.326)--(8.063,2.328)--(8.147,2.354)--(8.147,2.352)--(8.147,2.291)--(8.231,2.353)%
  --(8.259,2.386)--(8.287,5.981)--(8.287,5.910)--(8.287,5.815)--(8.287,5.743)--(8.315,5.602)%
  --(8.371,5.532)--(8.371,5.449)--(8.399,5.450)--(8.399,5.356)--(8.399,5.336)--(8.427,5.221)%
  --(8.427,5.180)--(8.427,5.158)--(8.427,5.065)--(8.455,5.051)--(8.483,4.957)--(8.483,4.866)%
  --(8.483,4.844)--(8.539,4.801)--(8.539,4.751)--(8.651,4.733)--(8.679,4.668)--(8.679,4.580)%
  --(8.707,4.597)--(8.707,4.557)--(8.735,4.487)--(8.763,4.444)--(8.791,4.443)--(8.791,4.379)%
  --(8.791,4.385)--(8.819,4.390)--(8.847,4.365)--(8.847,4.347)--(8.875,4.326)--(8.903,4.329)%
  --(8.903,4.286)--(8.931,4.240)--(8.959,4.222)--(8.987,4.223)--(9.043,4.161)--(9.071,4.115)%
  --(9.127,4.072)--(9.127,4.096)--(9.183,4.099)--(9.183,4.080)--(9.239,4.011)--(9.295,4.016)%
  --(9.379,3.923)--(9.379,3.922)--(9.407,3.971)--(9.407,3.906)--(9.463,3.904)--(9.463,3.853)%
  --(9.463,3.836)--(9.491,3.774)--(9.519,3.801)--(9.547,3.800)--(9.547,3.796)--(9.575,3.798)%
  --(9.575,3.803)--(9.575,3.782)--(9.631,3.761)--(9.631,3.758)--(9.715,3.764)--(9.771,3.763)%
  --(9.883,3.758)--(9.995,7.298)--(9.995,7.229)--(9.995,7.149)--(9.995,7.107)--(10.023,7.008)%
  --(10.023,6.989)--(10.051,6.944)--(10.107,6.823)--(10.163,6.772)--(10.163,6.625)--(10.163,6.668)%
  --(10.163,6.651)--(10.163,6.559)--(10.191,6.485)--(10.191,6.464)--(10.191,6.362)--(10.247,6.298)%
  --(10.247,6.297)--(10.275,6.220)--(10.275,6.198)--(10.275,6.126)--(10.331,6.075)--(10.331,5.953)%
  --(10.359,5.888)--(10.359,5.867)--(10.359,5.772)--(10.387,5.727)--(10.415,5.701)--(10.415,5.677)%
  --(10.415,5.708)--(10.415,5.672)--(10.443,5.647)--(10.499,5.577)--(10.499,5.455)--(10.555,5.409)%
  --(10.611,5.458)--(10.611,5.426)--(10.611,5.375)--(10.667,5.350)--(10.667,5.304)--(10.667,5.307)%
  --(10.695,5.285)--(10.751,5.307)--(10.807,5.329)--(10.807,5.260)--(10.835,5.266)--(10.947,5.238)%
  --(10.947,5.212)--(10.947,5.211)--(10.975,5.186)--(11.003,5.167)--(11.003,5.140)--(11.031,5.144)%
  --(11.031,5.152)--(11.059,5.176)--(11.059,5.203)--(11.059,5.211)--(11.115,5.211)--(11.143,5.186)%
  --(11.143,5.179)--(11.227,5.131)--(11.255,5.109)--(11.283,5.082)--(11.339,5.053)--(11.367,5.001)%
  --(11.479,5.002)--(11.507,5.057)--(11.535,5.064)--(11.563,5.069)--(11.619,4.999)--(11.647,4.948)%
  --(11.647,4.924)--(11.647,4.828);
\gpcolor{color=gp lt color border}
\gpsetlinewidth{1.00}
\draw[gp path] (1.400,8.099)--(1.400,0.000)--(12.599,0.000)--(12.599,8.099)--cycle;
\node[gp node center,rotate=-270] at (-1.100,4.049) {Strategic Resource Penalty [Hours]};
\node[gp node center,rotate=-270] at (15.329,4.049) {Strategic Clustering Value};
\node[gp node center] at (6.999,-1.629) {Relative time ($\tau$) [S]};
%% coordinates of the plot area
\gpdefrectangularnode{gp plot 2}{\pgfpoint{1.400cm}{0.000cm}}{\pgfpoint{12.599cm}{8.099cm}}
\endtikzpicture
%% gnuplot variables

	}
	\caption{At each 60 second interval the strategic urgency experiences a 
		large spike as urgent work order are being forced to be scheduled further out.
		The strategic resource penalty in lowered as $\ParStrategicInclude$ parameters
		and corresponding constraints are removed 
	}\label{fig:responses:exclusion}
\end{figure}

From figure~\ref{fig:responses:inclusion} and figure~\ref{fig:responses:exclusion} it is 
clear that the AbLNS method can handle dynamic entries of work orders. The next section will discuss the effects of dynamically changing the 
resource capacities $\ParStrategicResource$. 

\subsection{Response to Additional Weekly Capacity}\label{sec:increase_week_cap}
Table~\ref{tab:responses:resource-addition} details the perturbations that the
AbLNS will be subject to during its 360 second execution. Perturbing the
$\ParStrategicResource$ affects the solution considerably more than
perturbing $\ParStrategicExclude$ and $\ParStrategicInclude$ and therefore 100
second intervals are specified instead of 60 second intervals.

\begin{table}[H]
	\centering
	\begin{tabular}{lrrrrr}
	\toprule
	                                     & $\VarMetaTime_1 = 0$ & $\VarMetaTime_2 = 100$ & $\VarMetaTime_3 = 200$   \\ \midrule
	$\Delta |\SetPeriod|$                & 52                     & 52                     & 52                     \\ \midrule
	$\Delta |\SetResource|$              & 16                     & 16                     & 16                     \\ \midrule
	$ ||\ParStrategicResource{}||$ (hours) &  61816                 & 111268                 & 173083                 \\ \bottomrule
	\end{tabular}
	\caption{The resource perturbations that the AbLNS will be affected by measured in hours.
		Here all $\ElementPeriod \in \SetPeriod$ and $\ElementResource \in \SetResource$ are 
		affected
	}\label{tab:responses:resource-addition}
\end{table}


Figure~\ref{fig:responses:resources-addition} shows the effects of progressively
increasing  available resources. The AbLNS starts with an initial load which is
then increased at $\tau = 100$ seconds causing the objective value to decrease
as $\VarStrategicExcess$ in equation~\ref{eqn:strategic:constraint:resource}
can achieve a lower value. At $\tau = 200$ the resources are increased
to their final value.

\begin{figure}[H]%% placement specifier
	\centering
	\resizebox{10cm}{!}{
		\tikzpicture[gnuplot]
%% generated with GNUPLOT 6.0p1 (Lua 5.2; terminal rev. Jun 2020, script rev. 118)
%% Tue 07 Jan 2025 09:05:22 AM UTC
\path (0.000,0.000) rectangle (16.000,12.000);
\gpcolor{rgb color={0.753,0.753,0.753}}
\gpsetlinetype{gp lt border}
\gpsetdashtype{gp dt solid}
\gpsetlinewidth{2.00}
\draw[gp path] (2.240,1.845)--(13.667,1.845);
\gpcolor{color=gp lt color border}
\gpsetlinewidth{1.00}
\draw[gp path] (2.240,1.845)--(2.420,1.845);
\node[gp node right] at (2.056,1.845) {$8.7\times10^{10}$};
\gpcolor{rgb color={0.753,0.753,0.753}}
\gpsetlinewidth{2.00}
\draw[gp path] (2.240,2.871)--(13.667,2.871);
\gpcolor{color=gp lt color border}
\gpsetlinewidth{1.00}
\draw[gp path] (2.240,2.871)--(2.420,2.871);
\node[gp node right] at (2.056,2.871) {$8.8\times10^{10}$};
\gpcolor{rgb color={0.753,0.753,0.753}}
\gpsetlinewidth{2.00}
\draw[gp path] (2.240,3.896)--(13.667,3.896);
\gpcolor{color=gp lt color border}
\gpsetlinewidth{1.00}
\draw[gp path] (2.240,3.896)--(2.420,3.896);
\node[gp node right] at (2.056,3.896) {$8.9\times10^{10}$};
\gpcolor{rgb color={0.753,0.753,0.753}}
\gpsetlinewidth{2.00}
\draw[gp path] (2.240,4.922)--(13.667,4.922);
\gpcolor{color=gp lt color border}
\gpsetlinewidth{1.00}
\draw[gp path] (2.240,4.922)--(2.420,4.922);
\node[gp node right] at (2.056,4.922) {$9\times10^{10}$};
\gpcolor{rgb color={0.753,0.753,0.753}}
\gpsetlinewidth{2.00}
\draw[gp path] (2.240,5.947)--(13.667,5.947);
\gpcolor{color=gp lt color border}
\gpsetlinewidth{1.00}
\draw[gp path] (2.240,5.947)--(2.420,5.947);
\node[gp node right] at (2.056,5.947) {$9.1\times10^{10}$};
\gpcolor{rgb color={0.753,0.753,0.753}}
\gpsetlinewidth{2.00}
\draw[gp path] (2.240,6.973)--(13.667,6.973);
\gpcolor{color=gp lt color border}
\gpsetlinewidth{1.00}
\draw[gp path] (2.240,6.973)--(2.420,6.973);
\node[gp node right] at (2.056,6.973) {$9.2\times10^{10}$};
\gpcolor{rgb color={0.753,0.753,0.753}}
\gpsetlinewidth{2.00}
\draw[gp path] (2.240,7.998)--(13.667,7.998);
\gpcolor{color=gp lt color border}
\gpsetlinewidth{1.00}
\draw[gp path] (2.240,7.998)--(2.420,7.998);
\node[gp node right] at (2.056,7.998) {$9.3\times10^{10}$};
\gpcolor{rgb color={0.753,0.753,0.753}}
\gpsetlinewidth{2.00}
\draw[gp path] (2.240,9.024)--(13.667,9.024);
\gpcolor{color=gp lt color border}
\gpsetlinewidth{1.00}
\draw[gp path] (2.240,9.024)--(2.420,9.024);
\node[gp node right] at (2.056,9.024) {$9.4\times10^{10}$};
\gpcolor{rgb color={0.753,0.753,0.753}}
\gpsetlinewidth{2.00}
\draw[gp path] (2.240,10.049)--(7.047,10.049);
\draw[gp path] (13.483,10.049)--(13.667,10.049);
\gpcolor{color=gp lt color border}
\gpsetlinewidth{1.00}
\draw[gp path] (2.240,10.049)--(2.420,10.049);
\node[gp node right] at (2.056,10.049) {$9.5\times10^{10}$};
\gpcolor{rgb color={0.753,0.753,0.753}}
\gpsetlinewidth{2.00}
\draw[gp path] (2.240,11.075)--(13.667,11.075);
\gpcolor{color=gp lt color border}
\gpsetlinewidth{1.00}
\draw[gp path] (2.240,11.075)--(2.420,11.075);
\node[gp node right] at (2.056,11.075) {$9.6\times10^{10}$};
\gpcolor{rgb color={0.753,0.753,0.753}}
\gpsetlinewidth{2.00}
\draw[gp path] (2.240,1.845)--(2.240,11.075);
\gpcolor{color=gp lt color border}
\gpsetlinewidth{1.00}
\draw[gp path] (2.240,1.845)--(2.240,2.025);
\draw[gp path] (2.240,11.075)--(2.240,10.895);
\node[gp node left,rotate=270] at (2.240,1.661) {$0$};
\gpcolor{rgb color={0.753,0.753,0.753}}
\gpsetlinewidth{2.00}
\draw[gp path] (4.145,1.845)--(4.145,11.075);
\gpcolor{color=gp lt color border}
\gpsetlinewidth{1.00}
\draw[gp path] (4.145,1.845)--(4.145,2.025);
\draw[gp path] (4.145,11.075)--(4.145,10.895);
\node[gp node left,rotate=270] at (4.145,1.661) {$50$};
\gpcolor{rgb color={0.753,0.753,0.753}}
\gpsetlinewidth{2.00}
\draw[gp path] (6.049,1.845)--(6.049,11.075);
\gpcolor{color=gp lt color border}
\gpsetlinewidth{1.00}
\draw[gp path] (6.049,1.845)--(6.049,2.025);
\draw[gp path] (6.049,11.075)--(6.049,10.895);
\node[gp node left,rotate=270] at (6.049,1.661) {$100$};
\gpcolor{rgb color={0.753,0.753,0.753}}
\gpsetlinewidth{2.00}
\draw[gp path] (7.954,1.845)--(7.954,9.355);
\draw[gp path] (7.954,10.895)--(7.954,11.075);
\gpcolor{color=gp lt color border}
\gpsetlinewidth{1.00}
\draw[gp path] (7.954,1.845)--(7.954,2.025);
\draw[gp path] (7.954,11.075)--(7.954,10.895);
\node[gp node left,rotate=270] at (7.954,1.661) {$150$};
\gpcolor{rgb color={0.753,0.753,0.753}}
\gpsetlinewidth{2.00}
\draw[gp path] (9.858,1.845)--(9.858,9.355);
\draw[gp path] (9.858,10.895)--(9.858,11.075);
\gpcolor{color=gp lt color border}
\gpsetlinewidth{1.00}
\draw[gp path] (9.858,1.845)--(9.858,2.025);
\draw[gp path] (9.858,11.075)--(9.858,10.895);
\node[gp node left,rotate=270] at (9.858,1.661) {$200$};
\gpcolor{rgb color={0.753,0.753,0.753}}
\gpsetlinewidth{2.00}
\draw[gp path] (11.763,1.845)--(11.763,9.355);
\draw[gp path] (11.763,10.895)--(11.763,11.075);
\gpcolor{color=gp lt color border}
\gpsetlinewidth{1.00}
\draw[gp path] (11.763,1.845)--(11.763,2.025);
\draw[gp path] (11.763,11.075)--(11.763,10.895);
\node[gp node left,rotate=270] at (11.763,1.661) {$250$};
\gpcolor{rgb color={0.753,0.753,0.753}}
\gpsetlinewidth{2.00}
\draw[gp path] (13.667,1.845)--(13.667,11.075);
\gpcolor{color=gp lt color border}
\gpsetlinewidth{1.00}
\draw[gp path] (13.667,1.845)--(13.667,2.025);
\draw[gp path] (13.667,11.075)--(13.667,10.895);
\node[gp node left,rotate=270] at (13.667,1.661) {$300$};
\draw[gp path] (13.667,1.845)--(13.487,1.845);
\node[gp node left] at (13.851,1.845) {$95000$};
\draw[gp path] (13.667,3.383)--(13.487,3.383);
\node[gp node left] at (13.851,3.383) {$100000$};
\draw[gp path] (13.667,4.922)--(13.487,4.922);
\node[gp node left] at (13.851,4.922) {$105000$};
\draw[gp path] (13.667,6.460)--(13.487,6.460);
\node[gp node left] at (13.851,6.460) {$110000$};
\draw[gp path] (13.667,7.998)--(13.487,7.998);
\node[gp node left] at (13.851,7.998) {$115000$};
\draw[gp path] (13.667,9.537)--(13.487,9.537);
\node[gp node left] at (13.851,9.537) {$120000$};
\draw[gp path] (13.667,11.075)--(13.487,11.075);
\node[gp node left] at (13.851,11.075) {$125000$};
\draw[gp path] (2.240,11.075)--(2.240,1.845)--(13.667,1.845)--(13.667,11.075)--cycle;
\draw[gp path] (7.047,9.355)--(7.047,10.895)--(13.483,10.895)--(13.483,9.355)--cycle;
\node[gp node right] at (12.199,10.510) {Strategic urgency};
\gpcolor{rgb color={0.000,0.000,0.000}}
\gpsetlinewidth{2.50}
\draw[gp path] (12.383,10.510)--(13.299,10.510);
\draw[gp path] (2.240,7.517)--(2.240,7.551)--(2.240,7.600)--(2.240,7.903)--(2.240,7.962)%
  --(2.240,8.146)--(2.240,8.334)--(2.240,8.398)--(2.240,8.561)--(2.240,8.642)--(2.240,8.621)%
  --(2.240,8.683)--(2.240,8.680)--(2.240,8.805)--(2.240,8.804)--(2.278,9.038)--(2.278,9.062)%
  --(2.278,9.118)--(2.278,9.101)--(2.278,9.153)--(2.278,9.168)--(2.278,9.240)--(2.278,9.287)%
  --(2.278,9.377)--(2.278,9.421)--(2.278,9.398)--(2.278,9.463)--(2.278,9.505)--(2.278,9.517)%
  --(2.278,9.557)--(2.278,9.587)--(2.278,9.577)--(2.278,9.597)--(2.278,9.704)--(2.278,9.717)%
  --(2.278,9.720)--(2.278,9.732)--(2.278,9.718)--(2.278,9.701)--(2.278,9.725)--(2.278,9.768)%
  --(2.278,9.782)--(2.278,9.847)--(2.278,9.854)--(2.278,9.849)--(2.278,9.852)--(2.278,9.883)%
  --(2.278,9.825)--(2.278,9.859)--(2.278,9.953)--(2.278,9.971)--(2.278,9.960)--(2.278,10.034)%
  --(2.316,10.041)--(2.316,10.053)--(2.316,10.080)--(2.316,10.093)--(2.316,10.098)--(2.316,10.130)%
  --(2.316,10.235)--(2.316,10.266)--(2.316,10.265)--(2.316,10.263)--(2.316,10.252)--(2.316,10.311)%
  --(2.316,10.304)--(2.316,10.313)--(2.316,10.261)--(2.316,10.268)--(2.316,10.256)--(2.316,10.264)%
  --(2.316,10.293)--(2.316,10.261)--(2.316,10.266)--(2.316,10.247)--(2.316,10.270)--(2.354,10.361)%
  --(2.354,10.357)--(2.354,10.332)--(2.354,10.359)--(2.354,10.363)--(2.354,10.333)--(2.392,10.359)%
  --(2.392,10.358)--(2.392,10.398)--(2.392,10.407)--(2.392,10.389)--(2.392,10.371)--(2.392,10.370)%
  --(2.392,10.376)--(2.392,10.401)--(2.392,10.398)--(2.392,10.396)--(2.392,10.381)--(2.392,10.362)%
  --(2.430,10.366)--(2.430,10.391)--(2.430,10.394)--(2.469,10.367)--(2.469,10.358)--(2.469,10.328)%
  --(2.469,10.341)--(2.469,10.308)--(2.469,10.318)--(2.469,10.325)--(2.469,10.328)--(2.507,10.310)%
  --(2.507,10.330)--(2.507,10.335)--(2.507,10.332)--(2.507,10.325)--(2.507,10.316)--(2.545,10.308)%
  --(2.545,10.305)--(2.545,10.302)--(2.545,10.311)--(2.545,10.304)--(2.583,10.298)--(2.583,10.311)%
  --(2.583,10.314)--(2.583,10.309)--(2.621,10.305)--(2.621,10.284)--(2.621,10.279)--(2.621,10.283)%
  --(2.621,10.285)--(2.621,10.274)--(2.659,10.282)--(2.659,10.265)--(2.735,10.250)--(2.773,10.245)%
  --(2.888,10.243)--(2.926,10.223)--(3.002,10.224)--(3.040,10.221)--(3.040,10.204)--(3.154,10.202)%
  --(3.154,10.208)--(3.154,10.202)--(3.154,10.203)--(3.154,10.206)--(3.192,10.206)--(3.192,10.204)%
  --(3.230,10.214)--(3.230,10.220)--(3.268,10.222)--(3.268,10.237)--(3.307,10.234)--(3.307,10.226)%
  --(3.307,10.223)--(3.345,10.222)--(3.383,10.217)--(3.383,10.286)--(3.421,10.276)--(3.459,10.275)%
  --(3.497,10.277)--(3.497,10.285)--(3.535,10.291)--(3.573,10.291)--(3.611,10.277)--(3.611,10.266)%
  --(3.649,10.265)--(3.649,10.272)--(3.649,10.271)--(3.687,10.269)--(3.764,10.265)--(3.954,10.256)%
  --(3.954,10.254)--(3.954,10.255)--(3.954,10.229)--(3.992,10.192)--(3.992,10.195)--(4.030,10.198)%
  --(4.068,10.199)--(4.068,10.221)--(4.106,10.220)--(4.106,10.217)--(4.145,10.211)--(4.145,10.209)%
  --(4.183,10.205)--(4.221,10.212)--(4.259,10.209)--(4.297,10.218)--(4.297,10.212)--(4.297,10.206)%
  --(4.335,10.199)--(4.411,10.198)--(4.449,10.196)--(4.449,10.185)--(4.487,10.185)--(4.487,10.165)%
  --(4.487,10.156)--(4.525,10.147)--(4.602,10.145)--(4.640,10.147)--(4.640,10.148)--(4.640,10.125)%
  --(4.792,10.119)--(4.868,10.115)--(4.944,10.111)--(5.097,10.107)--(5.173,10.106)--(5.211,10.100)%
  --(5.249,10.099)--(5.249,10.119)--(5.325,10.129)--(5.668,10.118)--(5.744,10.094)--(5.782,10.090)%
  --(5.782,10.089)--(5.859,10.088)--(6.011,9.926)--(6.011,9.747)--(6.011,9.663)--(6.011,9.477)%
  --(6.011,9.272)--(6.011,9.169)--(6.011,9.103)--(6.011,9.132)--(6.011,8.943)--(6.011,8.868)%
  --(6.011,8.739)--(6.011,8.654)--(6.011,8.612)--(6.011,8.546)--(6.011,8.480)--(6.011,8.257)%
  --(6.011,8.163)--(6.011,8.153)--(6.011,8.106)--(6.011,7.902)--(6.011,7.798)--(6.011,7.788)%
  --(6.011,7.738)--(6.011,7.672)--(6.011,7.680)--(6.011,7.652)--(6.011,7.651)--(6.011,7.657)%
  --(6.011,7.599)--(6.011,7.598)--(6.011,7.697)--(6.011,7.665)--(6.011,7.633)--(6.011,7.590)%
  --(6.011,7.587)--(6.011,7.571)--(6.011,7.550)--(6.011,7.529)--(6.011,7.520)--(6.011,7.504)%
  --(6.011,7.483)--(6.011,7.446)--(6.011,7.479)--(6.011,7.499)--(6.011,7.463)--(6.011,7.470)%
  --(6.011,7.487)--(6.011,7.426)--(6.011,7.404)--(6.011,7.327)--(6.011,7.310)--(6.011,7.251)%
  --(6.011,7.272)--(6.049,7.249)--(6.049,7.262)--(6.049,7.246)--(6.049,7.219)--(6.049,7.165)%
  --(6.049,7.192)--(6.049,7.222)--(6.049,7.103)--(6.049,7.128)--(6.049,7.116)--(6.049,7.137)%
  --(6.049,7.133)--(6.049,7.127)--(6.049,7.177)--(6.049,7.164)--(6.049,7.147)--(6.049,7.163)%
  --(6.049,7.130)--(6.049,7.099)--(6.049,7.136)--(6.049,7.043)--(6.049,7.045)--(6.049,7.051)%
  --(6.049,7.093)--(6.049,7.131)--(6.049,7.112)--(6.049,7.159)--(6.049,7.115)--(6.049,7.427)%
  --(6.049,7.482)--(6.049,7.494)--(6.049,7.435)--(6.049,7.402)--(6.049,7.318)--(6.049,7.340)%
  --(6.049,7.303)--(6.049,7.300)--(6.049,7.287)--(6.049,7.266)--(6.049,7.368)--(6.049,7.416)%
  --(6.049,7.338)--(6.049,7.302)--(6.049,7.205)--(6.049,7.230)--(6.049,7.287)--(6.049,7.282)%
  --(6.049,7.309)--(6.049,7.228)--(6.049,7.238)--(6.049,7.256)--(6.049,7.209)--(6.049,7.182)%
  --(6.049,7.176)--(6.049,7.162)--(6.049,7.147)--(6.049,7.121)--(6.049,7.113)--(6.049,7.159)%
  --(6.049,7.093)--(6.049,7.016)--(6.049,7.046)--(6.049,7.029)--(6.087,6.993)--(6.087,6.972)%
  --(6.087,7.030)--(6.087,7.045)--(6.087,7.064)--(6.087,7.051)--(6.087,7.088)--(6.087,7.012)%
  --(6.087,6.997)--(6.087,7.059)--(6.087,7.036)--(6.087,6.997)--(6.087,7.042)--(6.087,7.035)%
  --(6.087,7.022)--(6.087,7.034)--(6.087,7.059)--(6.087,7.022)--(6.087,6.982)--(6.087,6.951)%
  --(6.087,6.892)--(6.087,6.855)--(6.087,6.847)--(6.087,6.832)--(6.087,6.818)--(6.087,6.781)%
  --(6.087,6.777)--(6.087,6.758)--(6.087,6.756)--(6.087,6.737)--(6.087,6.711)--(6.125,6.693)%
  --(6.125,6.673)--(6.125,6.670)--(6.125,6.668)--(6.125,6.647)--(6.125,6.641)--(6.125,6.618)%
  --(6.125,6.651)--(6.125,6.643)--(6.125,6.638)--(6.125,6.618)--(6.125,6.604)--(6.125,6.582)%
  --(6.125,6.574)--(6.125,6.569)--(6.125,6.564)--(6.125,6.543)--(6.125,6.556)--(6.125,6.524)%
  --(6.125,6.515)--(6.125,6.514)--(6.125,6.510)--(6.125,6.503)--(6.125,6.470)--(6.125,6.435)%
  --(6.163,6.461)--(6.163,6.446)--(6.163,6.421)--(6.163,6.419)--(6.163,6.415)--(6.163,6.410)%
  --(6.163,6.402)--(6.163,6.389)--(6.163,6.383)--(6.163,6.367)--(6.163,6.352)--(6.163,6.335)%
  --(6.201,6.331)--(6.201,6.317)--(6.201,6.275)--(6.201,6.259)--(6.201,6.250)--(6.201,6.238)%
  --(6.201,6.219)--(6.201,6.210)--(6.201,6.200)--(6.201,6.181)--(6.239,6.159)--(6.239,6.131)%
  --(6.239,6.121)--(6.239,6.112)--(6.239,6.101)--(6.239,6.082)--(6.239,6.064)--(6.239,6.062)%
  --(6.239,6.060)--(6.278,6.058)--(6.278,6.049)--(6.278,6.044)--(6.278,6.034)--(6.316,6.022)%
  --(6.316,6.017)--(6.316,6.016)--(6.316,6.014)--(6.354,6.005)--(6.354,5.991)--(6.392,5.977)%
  --(6.392,5.971)--(6.392,5.986)--(6.430,5.973)--(6.430,5.982)--(6.430,5.975)--(6.430,5.965)%
  --(6.468,5.965)--(6.468,5.946)--(6.468,5.941)--(6.468,5.938)--(6.506,5.938)--(6.544,5.921)%
  --(6.544,5.915)--(6.544,5.907)--(6.544,5.906)--(6.544,5.905)--(6.544,5.892)--(6.582,5.871)%
  --(6.582,5.863)--(6.582,5.861)--(6.620,5.860)--(6.658,5.859)--(6.658,5.855)--(6.658,5.852)%
  --(6.697,5.850)--(6.697,5.848)--(6.735,5.830)--(6.735,5.824)--(6.773,5.815)--(6.773,5.808)%
  --(6.773,5.806)--(6.887,5.796)--(6.887,5.793)--(6.887,5.792)--(6.887,5.781)--(6.887,5.780)%
  --(6.925,5.773)--(7.001,5.768)--(7.001,5.753)--(7.001,5.750)--(7.039,5.749)--(7.039,5.737)%
  --(7.077,5.731)--(7.077,5.723)--(7.077,5.718)--(7.154,5.716)--(7.154,5.706)--(7.154,5.704)%
  --(7.154,5.702)--(7.192,5.699)--(7.192,5.692)--(7.192,5.690)--(7.192,5.688)--(7.230,5.686)%
  --(7.230,5.683)--(7.230,5.676)--(7.230,5.658)--(7.268,5.656)--(7.268,5.645)--(7.268,5.639)%
  --(7.306,5.631)--(7.306,5.618)--(7.420,5.618)--(7.420,5.617)--(7.420,5.611)--(7.458,5.610)%
  --(7.496,5.609)--(7.535,5.605)--(7.535,5.586)--(7.535,5.578)--(7.535,5.576)--(7.535,5.565)%
  --(7.573,5.552)--(7.573,5.546)--(7.573,5.542)--(7.611,5.539)--(7.649,5.538)--(7.687,5.536)%
  --(7.687,5.534)--(7.725,5.530)--(7.763,5.529)--(7.801,5.528)--(7.801,5.522)--(7.801,5.517)%
  --(7.801,5.515)--(7.801,5.503)--(7.801,5.499)--(7.839,5.485)--(7.877,5.482)--(7.954,5.478)%
  --(7.992,5.474)--(8.030,5.468)--(8.068,5.465)--(8.182,5.456)--(8.296,5.456)--(8.334,5.448)%
  --(8.411,5.444)--(8.525,5.443)--(8.563,5.442)--(8.563,5.437)--(8.601,5.436)--(8.677,5.434)%
  --(8.715,5.430)--(8.715,5.424)--(8.753,5.423)--(8.830,5.421)--(8.830,5.415)--(8.868,5.414)%
  --(8.906,5.413)--(9.172,5.407)--(9.172,5.406)--(9.172,5.405)--(9.210,5.400)--(9.249,5.397)%
  --(9.249,5.391)--(9.477,5.389)--(9.515,5.386)--(9.515,5.384)--(9.553,5.384)--(9.629,5.383)%
  --(9.629,5.381)--(9.706,5.379)--(9.782,5.376)--(9.820,5.323)--(9.820,5.156)--(9.820,4.935)%
  --(9.820,4.802)--(9.820,4.721)--(9.820,4.680)--(9.820,4.584)--(9.820,4.533)--(9.820,4.500)%
  --(9.820,4.490)--(9.820,4.429)--(9.820,4.391)--(9.820,4.358)--(9.820,4.319)--(9.820,4.310)%
  --(9.820,4.283)--(9.820,4.214)--(9.820,4.202)--(9.820,4.122)--(9.820,4.110)--(9.820,4.099)%
  --(9.820,4.051)--(9.820,4.040)--(9.820,3.617)--(9.820,3.616)--(9.820,3.613)--(9.858,3.575)%
  --(9.858,3.554)--(9.858,3.544)--(9.858,3.543)--(9.858,3.537)--(9.858,3.526)--(9.858,3.488)%
  --(9.858,3.472)--(9.858,3.469)--(9.858,3.433)--(9.858,3.406)--(9.858,3.379)--(9.858,3.366)%
  --(9.858,3.289)--(9.858,3.327)--(9.858,3.310)--(9.858,3.309)--(9.858,3.296)--(9.858,3.282)%
  --(9.858,3.258)--(9.858,2.919)--(9.858,2.900)--(9.858,2.884)--(9.858,2.865)--(9.858,2.857)%
  --(9.858,2.868)--(9.858,2.860)--(9.858,2.853)--(9.858,2.849)--(9.896,2.843)--(9.896,2.833)%
  --(9.896,2.827)--(9.896,2.816)--(9.896,2.808)--(9.896,2.793)--(9.896,2.776)--(9.896,2.765)%
  --(9.896,2.756)--(9.896,2.753)--(9.934,2.738)--(9.934,2.715)--(9.934,2.711)--(9.934,2.694)%
  --(9.934,2.676)--(9.934,2.675)--(9.934,2.664)--(9.934,2.656)--(9.972,2.646)--(9.972,2.642)%
  --(9.972,2.630)--(10.010,2.628)--(10.010,2.625)--(10.010,2.623)--(10.010,2.621)--(10.048,2.608)%
  --(10.048,2.605)--(10.048,2.592)--(10.048,2.586)--(10.048,2.584)--(10.048,2.575)--(10.087,2.574)%
  --(10.087,2.564)--(10.087,2.557)--(10.087,2.552)--(10.087,2.540)--(10.087,2.539)--(10.087,2.537)%
  --(10.087,2.536)--(10.087,2.523)--(10.125,2.508)--(10.125,2.497)--(10.125,2.468)--(10.125,2.462)%
  --(10.125,2.459)--(10.125,2.455)--(10.125,2.453)--(10.125,2.452)--(10.125,2.441)--(10.125,2.427)%
  --(10.125,2.423)--(10.163,2.411)--(10.163,2.409)--(10.201,2.398)--(10.201,2.392)--(10.201,2.369)%
  --(10.239,2.369)--(10.239,2.368)--(10.239,2.367)--(10.239,2.362)--(10.239,2.360)--(10.277,2.354)%
  --(10.277,2.352)--(10.277,2.447)--(10.277,2.429)--(10.277,2.387)--(10.277,2.381)--(10.277,2.373)%
  --(10.315,2.370)--(10.315,2.369)--(10.315,2.368)--(10.315,2.345)--(10.315,2.343)--(10.315,2.326)%
  --(10.353,2.320)--(10.391,2.318)--(10.391,2.315)--(10.391,2.313)--(10.391,2.307)--(10.391,2.304)%
  --(10.391,2.301)--(10.429,2.295)--(10.429,2.279)--(10.467,2.278)--(10.467,2.268)--(10.467,2.261)%
  --(10.506,2.256)--(10.506,2.249)--(10.506,2.246)--(10.544,2.240)--(10.582,2.234)--(10.582,2.232)%
  --(10.582,2.224)--(10.582,2.217)--(10.620,2.214)--(10.620,2.209)--(10.620,2.207)--(10.658,2.204)%
  --(10.658,2.197)--(10.734,2.195)--(10.734,2.194)--(10.734,2.191)--(10.734,2.188)--(10.772,2.188)%
  --(10.772,2.185)--(10.810,2.180)--(10.848,2.174)--(10.848,2.166)--(10.886,2.162)--(10.886,2.160)%
  --(10.886,2.156)--(10.925,2.156)--(10.925,2.155)--(10.925,2.149)--(10.925,2.146)--(10.925,2.138)%
  --(10.925,2.134)--(11.077,2.131)--(11.115,2.127)--(11.153,2.127)--(11.191,2.123)--(11.229,2.122)%
  --(11.267,2.121)--(11.267,2.113)--(11.305,2.112)--(11.534,2.111)--(11.572,2.111)--(11.572,2.102)%
  --(11.610,2.098)--(11.610,2.096)--(11.648,2.091)--(11.648,2.090)--(11.648,2.086)--(11.648,2.082)%
  --(11.648,2.081)--(11.686,2.078)--(11.686,2.077)--(11.724,2.073)--(11.763,2.073)--(11.801,2.071)%
  --(11.801,2.070)--(11.801,2.055)--(11.801,2.048)--(11.839,2.047)--(11.915,2.046)--(11.915,2.044)%
  --(11.915,2.042)--(11.915,2.040)--(11.953,2.040)--(11.991,2.039)--(11.991,2.037)--(12.029,2.034)%
  --(12.029,2.032)--(12.067,2.030)--(12.067,2.029);
\gpcolor{color=gp lt color border}
\node[gp node right] at (12.199,9.740) {Strategic Resource penalty};
\gpcolor{rgb color={0.184,0.243,0.918}}
\draw[gp path] (12.383,9.740)--(13.299,9.740);
\draw[gp path] (2.240,10.849)--(2.240,10.806)--(2.240,10.731)--(2.240,10.666)--(2.240,10.604)%
  --(2.240,10.560)--(2.240,10.513)--(2.240,10.469)--(2.240,10.419)--(2.240,10.384)--(2.240,10.378)%
  --(2.240,10.340)--(2.240,10.303)--(2.240,10.274)--(2.240,10.267)--(2.278,10.239)--(2.278,10.238)%
  --(2.278,10.224)--(2.278,10.218)--(2.278,10.210)--(2.278,10.198)--(2.278,10.191)--(2.278,10.155)%
  --(2.278,10.152)--(2.278,10.149)--(2.278,10.142)--(2.278,10.138)--(2.278,10.131)--(2.278,10.128)%
  --(2.278,10.127)--(2.278,10.123)--(2.278,10.120)--(2.278,10.102)--(2.278,10.098)--(2.278,10.090)%
  --(2.278,10.087)--(2.278,10.086)--(2.278,10.085)--(2.278,10.070)--(2.278,10.066)--(2.278,10.055)%
  --(2.278,10.043)--(2.278,10.039)--(2.278,10.031)--(2.278,10.025)--(2.278,10.022)--(2.278,10.006)%
  --(2.278,9.999)--(2.278,9.998)--(2.278,9.989)--(2.278,9.974)--(2.278,9.960)--(2.278,9.956)%
  --(2.278,9.948)--(2.316,9.945)--(2.316,9.934)--(2.316,9.925)--(2.316,9.922)--(2.316,9.921)%
  --(2.316,9.911)--(2.316,9.908)--(2.316,9.904)--(2.316,9.902)--(2.316,9.895)--(2.316,9.892)%
  --(2.316,9.890)--(2.316,9.888)--(2.316,9.878)--(2.316,9.872)--(2.316,9.867)--(2.316,9.866)%
  --(2.316,9.864)--(2.316,9.865)--(2.316,9.863)--(2.316,9.862)--(2.316,9.861)--(2.354,9.858)%
  --(2.354,9.857)--(2.354,9.856)--(2.354,9.855)--(2.354,9.852)--(2.392,9.851)--(2.392,9.847)%
  --(2.392,9.846)--(2.392,9.839)--(2.392,9.834)--(2.392,9.833)--(2.392,9.828)--(2.392,9.821)%
  --(2.392,9.820)--(2.392,9.818)--(2.392,9.813)--(2.430,9.812)--(2.430,9.809)--(2.430,9.805)%
  --(2.469,9.806)--(2.469,9.805)--(2.469,9.804)--(2.469,9.800)--(2.469,9.794)--(2.469,9.791)%
  --(2.507,9.788)--(2.507,9.787)--(2.507,9.785)--(2.507,9.783)--(2.507,9.781)--(2.545,9.780)%
  --(2.545,9.778)--(2.545,9.777)--(2.583,9.773)--(2.583,9.772)--(2.583,9.770)--(2.583,9.769)%
  --(2.621,9.769)--(2.621,9.768)--(2.621,9.766)--(2.659,9.758)--(2.735,9.758)--(2.773,9.758)%
  --(2.888,9.758)--(2.926,9.757)--(3.002,9.755)--(3.040,9.752)--(3.154,9.751)--(3.154,9.750)%
  --(3.154,9.748)--(3.154,9.747)--(3.154,9.745)--(3.192,9.744)--(3.230,9.743)--(3.230,9.737)%
  --(3.268,9.730)--(3.307,9.730)--(3.307,9.727)--(3.345,9.727)--(3.383,9.727)--(3.383,9.724)%
  --(3.421,9.724)--(3.459,9.720)--(3.497,9.719)--(3.497,9.717)--(3.535,9.716)--(3.573,9.716)%
  --(3.611,9.716)--(3.611,9.715)--(3.611,9.714)--(3.649,9.714)--(3.687,9.711)--(3.764,9.711)%
  --(3.954,9.710)--(3.992,9.709)--(4.030,9.708)--(4.068,9.707)--(4.068,9.705)--(4.106,9.705)%
  --(4.145,9.705)--(4.183,9.705)--(4.221,9.703)--(4.259,9.703)--(4.297,9.703)--(4.297,9.702)%
  --(4.335,9.702)--(4.411,9.702)--(4.449,9.699)--(4.487,9.699)--(4.487,9.698)--(4.525,9.698)%
  --(4.602,9.697)--(4.640,9.697)--(4.640,9.695)--(4.792,9.695)--(4.868,9.695)--(4.944,9.695)%
  --(5.097,9.695)--(5.173,9.695)--(5.211,9.695)--(5.249,9.695)--(5.249,9.694)--(5.325,9.693)%
  --(5.668,9.694)--(5.744,9.694)--(5.782,9.694)--(5.859,9.694)--(6.011,7.848)--(6.011,7.713)%
  --(6.011,7.613)--(6.011,7.546)--(6.011,7.466)--(6.011,7.396)--(6.011,7.324)--(6.011,7.289)%
  --(6.011,7.219)--(6.011,7.193)--(6.011,7.072)--(6.011,6.990)--(6.011,6.924)--(6.011,6.843)%
  --(6.011,6.822)--(6.011,6.714)--(6.011,6.626)--(6.011,6.593)--(6.011,6.585)--(6.011,6.498)%
  --(6.011,6.433)--(6.011,6.389)--(6.011,6.374)--(6.011,6.362)--(6.011,6.349)--(6.011,6.329)%
  --(6.011,6.257)--(6.011,6.223)--(6.011,6.138)--(6.011,6.133)--(6.011,6.120)--(6.011,6.099)%
  --(6.011,6.094)--(6.011,6.083)--(6.011,6.044)--(6.011,6.015)--(6.011,5.941)--(6.011,5.935)%
  --(6.011,5.881)--(6.011,5.786)--(6.011,5.776)--(6.011,5.753)--(6.011,5.735)--(6.011,5.722)%
  --(6.011,5.717)--(6.011,5.698)--(6.011,5.677)--(6.011,5.599)--(6.011,5.523)--(6.011,5.505)%
  --(6.011,5.491)--(6.049,5.486)--(6.049,5.478)--(6.049,5.421)--(6.049,5.417)--(6.049,5.365)%
  --(6.049,5.356)--(6.049,5.305)--(6.049,5.299)--(6.049,5.264)--(6.049,5.256)--(6.049,5.208)%
  --(6.049,5.203)--(6.049,5.085)--(6.049,5.066)--(6.049,5.055)--(6.049,5.051)--(6.049,4.988)%
  --(6.049,4.986)--(6.049,4.896)--(6.049,4.891)--(6.049,4.886)--(6.049,4.879)--(6.049,4.818)%
  --(6.049,4.811)--(6.049,4.787)--(6.049,4.743)--(6.049,4.739)--(6.049,4.737)--(6.049,4.726)%
  --(6.049,4.719)--(6.049,4.720)--(6.049,4.715)--(6.049,4.676)--(6.049,4.665)--(6.049,4.661)%
  --(6.049,4.657)--(6.049,4.648)--(6.049,4.644)--(6.049,4.631)--(6.049,4.630)--(6.049,4.548)%
  --(6.049,4.536)--(6.049,4.504)--(6.049,4.501)--(6.049,4.479)--(6.049,4.472)--(6.049,4.452)%
  --(6.049,4.429)--(6.049,4.425)--(6.049,4.422)--(6.087,4.422)--(6.087,4.418)--(6.087,4.407)%
  --(6.087,4.393)--(6.087,4.385)--(6.087,4.382)--(6.087,4.381)--(6.087,4.376)--(6.087,4.368)%
  --(6.087,4.364)--(6.087,4.360)--(6.087,4.358)--(6.087,4.355)--(6.125,4.355)--(6.125,4.352)%
  --(6.125,4.349)--(6.125,4.348)--(6.125,4.347)--(6.125,4.344)--(6.163,4.341)--(6.201,4.341)%
  --(6.239,4.341)--(6.278,4.341)--(6.316,4.341)--(6.354,4.341)--(6.392,4.341)--(6.430,4.341)%
  --(6.430,4.340)--(6.468,4.340)--(6.506,4.340)--(6.544,4.340)--(6.582,4.340)--(6.620,4.340)%
  --(6.658,4.340)--(6.697,4.340)--(6.735,4.340)--(6.773,4.340)--(6.887,4.340)--(6.925,4.340)%
  --(7.001,4.340)--(7.039,4.340)--(7.077,4.340)--(7.154,4.340)--(7.192,4.340)--(7.230,4.340)%
  --(7.268,4.340)--(7.306,4.340)--(7.420,4.340)--(7.458,4.340)--(7.496,4.340)--(7.535,4.340)%
  --(7.573,4.340)--(7.611,4.340)--(7.649,4.340)--(7.687,4.340)--(7.725,4.340)--(7.763,4.340)%
  --(7.801,4.340)--(7.839,4.340)--(7.877,4.340)--(7.954,4.340)--(7.992,4.340)--(8.030,4.340)%
  --(8.068,4.340)--(8.182,4.340)--(8.296,4.340)--(8.334,4.340)--(8.411,4.340)--(8.525,4.340)%
  --(8.563,4.340)--(8.601,4.340)--(8.677,4.340)--(8.715,4.340)--(8.753,4.340)--(8.830,4.340)%
  --(8.868,4.340)--(8.906,4.340)--(9.172,4.340)--(9.210,4.340)--(9.249,4.340)--(9.477,4.340)%
  --(9.515,4.340)--(9.553,4.340)--(9.629,4.340)--(9.706,4.340)--(9.782,4.340)--(9.820,3.036)%
  --(9.820,2.968)--(9.820,2.880)--(9.820,2.809)--(9.820,2.746)--(9.820,2.695)--(9.820,2.627)%
  --(9.858,2.627)--(9.858,2.566)--(9.858,2.419)--(9.858,2.366)--(9.858,2.261)--(9.896,2.261)%
  --(9.934,2.261)--(9.972,2.261)--(10.010,2.261)--(10.048,2.261)--(10.087,2.261)--(10.125,2.261)%
  --(10.163,2.261)--(10.201,2.261)--(10.239,2.261)--(10.277,2.261)--(10.277,2.175)--(10.315,2.175)%
  --(10.353,2.175)--(10.391,2.175)--(10.429,2.175)--(10.467,2.175)--(10.506,2.175)--(10.544,2.175)%
  --(10.582,2.175)--(10.620,2.175)--(10.658,2.175)--(10.734,2.175)--(10.772,2.175)--(10.810,2.175)%
  --(10.848,2.175)--(10.886,2.175)--(10.925,2.175)--(11.077,2.175)--(11.115,2.175)--(11.153,2.175)%
  --(11.191,2.175)--(11.229,2.175)--(11.267,2.175)--(11.305,2.175)--(11.534,2.175)--(11.572,2.175)%
  --(11.610,2.175)--(11.648,2.175)--(11.686,2.175)--(11.724,2.175)--(11.763,2.175)--(11.801,2.175)%
  --(11.839,2.175)--(11.915,2.175)--(11.953,2.175)--(11.991,2.175)--(12.029,2.175)--(12.067,2.175);
\gpcolor{color=gp lt color border}
\gpsetlinewidth{1.00}
\draw[gp path] (2.240,11.075)--(2.240,1.845)--(13.667,1.845)--(13.667,11.075)--cycle;
\node[gp node center,rotate=-270] at (-0.812,6.460) {Objective value};
\node[gp node center,rotate=-270] at (16.397,6.460) {Resource Penalty [Hours]};
\node[gp node center] at (7.953,0.215) {Relative time ($\tau$) [S]};
\node[gp node center] at (7.953,11.537) {Objective Value for Weekly Schedule};
%% coordinates of the plot area
\gpdefrectangularnode{gp plot 1}{\pgfpoint{2.240cm}{1.845cm}}{\pgfpoint{13.667cm}{11.075cm}}
\endtikzpicture
%% gnuplot variables

	}
	\caption{Starting from an initial load of 61816 hours. The resources are increased causing a drop in the
		objective value and the AbLNS then optimizes around the perturbation
	}\label{fig:responses:resources-addition}
\end{figure}

\subsection{Response to Reduced Weekly Capacity}\label{sec:results:reduced_weekly_capacity}
Table~\ref{tab:resources:resource-subtraction} details the perturbations that the
AbLNS will affected by. Starting from an initial amount of available
resource, the resources are progressively decreased.

\input{./tables/resource-subtraction-table.tex}

Figure~\ref{fig:responses:resource-subtraction} shows the effects of perturbing
the AbLNS by starting from an initial load and then progressively reducing
capacity.

\begin{figure}[H]%% placement specifier
	\centering
	\resizebox{10cm}{!}{
		\tikzpicture[gnuplot]
%% generated with GNUPLOT 6.0p1 (Lua 5.2; terminal rev. Jun 2020, script rev. 118)
%% fre 03 jan 2025 11:58:20 CET
\path (0.000,0.000) rectangle (16.000,12.000);
\gpcolor{rgb color={0.753,0.753,0.753}}
\gpsetlinetype{gp lt border}
\gpsetdashtype{gp dt solid}
\gpsetlinewidth{2.00}
\draw[gp path] (2.424,1.845)--(15.447,1.845);
\gpcolor{color=gp lt color border}
\gpsetlinewidth{1.00}
\draw[gp path] (2.424,1.845)--(2.604,1.845);
\draw[gp path] (15.447,1.845)--(15.267,1.845);
\node[gp node right] at (2.240,1.845) {$1.8\times10^{11}$};
\gpcolor{rgb color={0.753,0.753,0.753}}
\gpsetlinewidth{2.00}
\draw[gp path] (2.424,3.383)--(15.447,3.383);
\gpcolor{color=gp lt color border}
\gpsetlinewidth{1.00}
\draw[gp path] (2.424,3.383)--(2.604,3.383);
\draw[gp path] (15.447,3.383)--(15.267,3.383);
\node[gp node right] at (2.240,3.383) {$1.85\times10^{11}$};
\gpcolor{rgb color={0.753,0.753,0.753}}
\gpsetlinewidth{2.00}
\draw[gp path] (2.424,4.922)--(15.447,4.922);
\gpcolor{color=gp lt color border}
\gpsetlinewidth{1.00}
\draw[gp path] (2.424,4.922)--(2.604,4.922);
\draw[gp path] (15.447,4.922)--(15.267,4.922);
\node[gp node right] at (2.240,4.922) {$1.9\times10^{11}$};
\gpcolor{rgb color={0.753,0.753,0.753}}
\gpsetlinewidth{2.00}
\draw[gp path] (2.424,6.460)--(15.447,6.460);
\gpcolor{color=gp lt color border}
\gpsetlinewidth{1.00}
\draw[gp path] (2.424,6.460)--(2.604,6.460);
\draw[gp path] (15.447,6.460)--(15.267,6.460);
\node[gp node right] at (2.240,6.460) {$1.95\times10^{11}$};
\gpcolor{rgb color={0.753,0.753,0.753}}
\gpsetlinewidth{2.00}
\draw[gp path] (2.424,7.998)--(15.447,7.998);
\gpcolor{color=gp lt color border}
\gpsetlinewidth{1.00}
\draw[gp path] (2.424,7.998)--(2.604,7.998);
\draw[gp path] (15.447,7.998)--(15.267,7.998);
\node[gp node right] at (2.240,7.998) {$2\times10^{11}$};
\gpcolor{rgb color={0.753,0.753,0.753}}
\gpsetlinewidth{2.00}
\draw[gp path] (2.424,9.537)--(15.447,9.537);
\gpcolor{color=gp lt color border}
\gpsetlinewidth{1.00}
\draw[gp path] (2.424,9.537)--(2.604,9.537);
\draw[gp path] (15.447,9.537)--(15.267,9.537);
\node[gp node right] at (2.240,9.537) {$2.05\times10^{11}$};
\gpcolor{rgb color={0.753,0.753,0.753}}
\gpsetlinewidth{2.00}
\draw[gp path] (2.424,11.075)--(15.447,11.075);
\gpcolor{color=gp lt color border}
\gpsetlinewidth{1.00}
\draw[gp path] (2.424,11.075)--(2.604,11.075);
\draw[gp path] (15.447,11.075)--(15.267,11.075);
\node[gp node right] at (2.240,11.075) {$2.1\times10^{11}$};
\gpcolor{rgb color={0.753,0.753,0.753}}
\gpsetlinewidth{2.00}
\draw[gp path] (2.424,1.845)--(2.424,11.075);
\gpcolor{color=gp lt color border}
\gpsetlinewidth{1.00}
\draw[gp path] (2.424,1.845)--(2.424,2.025);
\draw[gp path] (2.424,11.075)--(2.424,10.895);
\node[gp node left,rotate=270] at (2.424,1.661) {$0$};
\gpcolor{rgb color={0.753,0.753,0.753}}
\gpsetlinewidth{2.00}
\draw[gp path] (4.595,1.845)--(4.595,11.075);
\gpcolor{color=gp lt color border}
\gpsetlinewidth{1.00}
\draw[gp path] (4.595,1.845)--(4.595,2.025);
\draw[gp path] (4.595,11.075)--(4.595,10.895);
\node[gp node left,rotate=270] at (4.595,1.661) {$50$};
\gpcolor{rgb color={0.753,0.753,0.753}}
\gpsetlinewidth{2.00}
\draw[gp path] (6.765,1.845)--(6.765,11.075);
\gpcolor{color=gp lt color border}
\gpsetlinewidth{1.00}
\draw[gp path] (6.765,1.845)--(6.765,2.025);
\draw[gp path] (6.765,11.075)--(6.765,10.895);
\node[gp node left,rotate=270] at (6.765,1.661) {$100$};
\gpcolor{rgb color={0.753,0.753,0.753}}
\gpsetlinewidth{2.00}
\draw[gp path] (8.936,1.845)--(8.936,11.075);
\gpcolor{color=gp lt color border}
\gpsetlinewidth{1.00}
\draw[gp path] (8.936,1.845)--(8.936,2.025);
\draw[gp path] (8.936,11.075)--(8.936,10.895);
\node[gp node left,rotate=270] at (8.936,1.661) {$150$};
\gpcolor{rgb color={0.753,0.753,0.753}}
\gpsetlinewidth{2.00}
\draw[gp path] (11.106,1.845)--(11.106,11.075);
\gpcolor{color=gp lt color border}
\gpsetlinewidth{1.00}
\draw[gp path] (11.106,1.845)--(11.106,2.025);
\draw[gp path] (11.106,11.075)--(11.106,10.895);
\node[gp node left,rotate=270] at (11.106,1.661) {$200$};
\gpcolor{rgb color={0.753,0.753,0.753}}
\gpsetlinewidth{2.00}
\draw[gp path] (13.277,1.845)--(13.277,11.075);
\gpcolor{color=gp lt color border}
\gpsetlinewidth{1.00}
\draw[gp path] (13.277,1.845)--(13.277,2.025);
\draw[gp path] (13.277,11.075)--(13.277,10.895);
\node[gp node left,rotate=270] at (13.277,1.661) {$250$};
\gpcolor{rgb color={0.753,0.753,0.753}}
\gpsetlinewidth{2.00}
\draw[gp path] (15.447,1.845)--(15.447,11.075);
\gpcolor{color=gp lt color border}
\gpsetlinewidth{1.00}
\draw[gp path] (15.447,1.845)--(15.447,2.025);
\draw[gp path] (15.447,11.075)--(15.447,10.895);
\node[gp node left,rotate=270] at (15.447,1.661) {$300$};
\draw[gp path] (2.424,11.075)--(2.424,1.845)--(15.447,1.845)--(15.447,11.075)--cycle;
\gpcolor{rgb color={0.000,0.000,0.000}}
\gpsetlinewidth{2.00}
\draw[gp path] (2.424,4.595)--(2.424,4.544)--(2.424,4.501)--(2.424,4.454)--(2.424,4.379)%
  --(2.424,4.364)--(2.424,4.348)--(2.424,4.319)--(2.424,4.281)--(2.424,4.279)--(2.424,4.243)%
  --(2.424,4.140)--(2.424,4.129)--(2.424,4.124)--(2.424,4.101)--(2.424,4.085)--(2.424,4.054)%
  --(2.424,4.039)--(2.424,3.951)--(2.467,3.947)--(2.467,3.928)--(2.467,3.926)--(2.467,3.913)%
  --(2.467,3.891)--(2.467,3.829)--(2.467,3.741)--(2.467,3.698)--(2.467,3.690)--(2.467,3.678)%
  --(2.467,3.675)--(2.467,3.661)--(2.467,3.648)--(2.467,3.621)--(2.467,3.601)--(2.467,3.594)%
  --(2.467,3.585)--(2.467,3.575)--(2.467,3.572)--(2.467,3.566)--(2.467,3.553)--(2.467,3.551)%
  --(2.467,3.545)--(2.467,3.538)--(2.467,3.532)--(2.467,3.527)--(2.467,3.505)--(2.467,3.503)%
  --(2.467,3.495)--(2.467,3.479)--(2.467,3.475)--(2.467,3.467)--(2.467,3.458)--(2.467,3.453)%
  --(2.467,3.446)--(2.511,3.442)--(2.511,3.440)--(2.511,3.425)--(2.511,3.419)--(2.511,3.413)%
  --(2.511,3.408)--(2.511,3.399)--(2.511,3.378)--(2.511,3.365)--(2.511,3.360)--(2.511,3.355)%
  --(2.511,3.354)--(2.511,3.347)--(2.511,3.344)--(2.511,3.339)--(2.554,3.299)--(2.554,3.296)%
  --(2.554,3.287)--(2.554,3.283)--(2.554,3.282)--(2.554,3.278)--(2.554,3.276)--(2.554,3.272)%
  --(2.554,3.270)--(2.554,3.269)--(2.554,3.268)--(2.554,3.267)--(2.554,3.266)--(2.554,3.261)%
  --(2.554,3.254)--(2.598,3.252)--(2.598,3.247)--(2.598,3.245)--(2.598,3.244)--(2.598,3.242)%
  --(2.598,3.230)--(2.598,3.225)--(2.598,3.222)--(2.598,3.220)--(2.598,3.218)--(2.598,3.214)%
  --(2.598,3.213)--(2.598,3.208)--(2.598,3.205)--(2.641,3.202)--(2.641,3.195)--(2.641,3.194)%
  --(2.641,3.192)--(2.641,3.189)--(2.641,3.186)--(2.641,3.185)--(2.641,3.184)--(2.641,3.177)%
  --(2.641,3.175)--(2.641,3.173)--(2.641,3.172)--(2.641,3.169)--(2.641,3.165)--(2.641,3.163)%
  --(2.641,3.161)--(2.641,3.159)--(2.641,3.157)--(2.684,3.151)--(2.684,3.148)--(2.684,3.147)%
  --(2.684,3.144)--(2.684,3.141)--(2.684,3.138)--(2.728,3.136)--(2.728,3.135)--(2.728,3.133)%
  --(2.728,3.132)--(2.728,3.129)--(2.728,3.127)--(2.728,3.126)--(2.728,3.124)--(2.728,3.122)%
  --(2.728,3.120)--(2.771,3.119)--(2.771,3.116)--(2.771,3.115)--(2.771,3.113)--(2.771,3.110)%
  --(2.771,3.109)--(2.771,3.108)--(2.771,3.102)--(2.771,3.096)--(2.771,3.092)--(2.815,3.092)%
  --(2.815,3.091)--(2.815,3.088)--(2.815,3.086)--(2.815,3.084)--(2.858,3.081)--(2.858,3.079)%
  --(2.902,3.079)--(2.902,3.078)--(2.902,3.077)--(2.902,3.074)--(2.902,3.071)--(2.902,3.069)%
  --(2.902,3.063)--(2.945,3.062)--(2.945,3.060)--(2.945,3.059)--(2.988,3.056)--(2.988,3.053)%
  --(2.988,3.050)--(2.988,3.048)--(2.988,3.044)--(2.988,3.041)--(3.032,3.039)--(3.032,3.038)%
  --(3.032,3.036)--(3.032,3.035)--(3.075,3.032)--(3.119,3.031)--(3.119,3.030)--(3.162,3.030)%
  --(3.162,3.027)--(3.162,3.026)--(3.205,3.026)--(3.205,3.024)--(3.205,3.023)--(3.292,3.023)%
  --(3.292,3.021)--(3.292,3.020)--(3.336,3.019)--(3.379,3.016)--(3.379,3.015)--(3.379,3.012)%
  --(3.422,3.010)--(3.422,3.009)--(3.466,3.005)--(3.466,3.004)--(3.509,3.003)--(3.509,3.002)%
  --(3.553,3.002)--(3.596,3.001)--(3.596,2.999)--(3.596,2.998)--(3.596,2.996)--(3.683,2.993)%
  --(3.813,2.992)--(3.813,2.991)--(3.900,2.989)--(3.987,2.988)--(4.030,2.988)--(4.030,2.987)%
  --(4.117,2.986)--(4.160,2.985)--(4.160,2.982)--(4.291,2.982)--(4.421,2.981)--(4.421,2.979)%
  --(4.464,2.973)--(4.464,2.971)--(4.464,2.970)--(4.508,2.970)--(4.508,2.969)--(4.595,2.969)%
  --(4.725,2.967)--(4.768,2.967)--(4.812,2.966)--(4.855,2.964)--(4.898,2.963)--(4.942,2.963)%
  --(4.942,2.961)--(4.942,2.958)--(4.985,2.958)--(4.985,2.957)--(5.029,2.957)--(5.029,2.956)%
  --(5.115,2.956)--(5.159,2.955)--(5.246,2.954)--(5.246,2.952)--(5.289,2.951)--(5.289,2.949)%
  --(5.332,2.948)--(5.376,2.947)--(5.376,2.946)--(5.376,2.944)--(5.419,2.944)--(5.550,2.941)%
  --(5.550,2.940)--(5.593,2.939)--(5.593,2.938)--(5.636,2.938)--(5.723,2.938)--(5.767,2.938)%
  --(5.767,2.937)--(5.810,2.937)--(5.810,2.936)--(5.853,2.935)--(5.940,2.935)--(5.940,2.933)%
  --(6.027,2.931)--(6.070,2.931)--(6.157,2.930)--(6.201,2.930)--(6.287,2.929)--(6.374,2.927)%
  --(6.418,2.927)--(6.418,2.925)--(6.418,2.924)--(6.461,2.923)--(6.461,2.920)--(6.505,2.919)%
  --(6.548,2.917)--(6.635,2.915)--(6.678,2.915)--(6.678,6.755)--(6.678,6.733)--(6.678,6.669)%
  --(6.678,6.547)--(6.678,6.478)--(6.678,6.414)--(6.678,6.389)--(6.678,6.333)--(6.722,6.322)%
  --(6.722,6.291)--(6.722,6.225)--(6.722,6.184)--(6.722,6.159)--(6.722,6.139)--(6.722,6.124)%
  --(6.722,6.114)--(6.722,6.101)--(6.722,6.072)--(6.722,6.068)--(6.722,6.018)--(6.722,5.999)%
  --(6.722,5.980)--(6.722,5.977)--(6.722,5.949)--(6.722,5.925)--(6.722,5.921)--(6.722,5.903)%
  --(6.722,5.896)--(6.722,5.891)--(6.722,5.876)--(6.722,5.848)--(6.722,5.844)--(6.722,5.843)%
  --(6.765,5.834)--(6.765,5.822)--(6.765,5.816)--(6.765,5.789)--(6.765,5.787)--(6.765,5.781)%
  --(6.765,5.775)--(6.765,5.768)--(6.765,5.767)--(6.765,5.761)--(6.765,5.760)--(6.765,5.759)%
  --(6.765,5.742)--(6.765,5.724)--(6.808,5.703)--(6.808,5.699)--(6.808,5.692)--(6.808,5.687)%
  --(6.808,5.674)--(6.808,5.670)--(6.808,5.666)--(6.808,5.664)--(6.852,5.661)--(6.852,5.657)%
  --(6.852,5.655)--(6.852,5.654)--(6.852,5.640)--(6.852,5.636)--(6.895,5.632)--(6.895,5.631)%
  --(6.895,5.630)--(6.895,5.627)--(6.895,5.616)--(6.895,5.609)--(6.895,5.606)--(6.895,5.605)%
  --(6.895,5.604)--(6.939,5.602)--(6.939,5.600)--(6.939,5.599)--(6.939,5.595)--(6.939,5.583)%
  --(6.982,5.582)--(6.982,5.576)--(6.982,5.574)--(6.982,5.571)--(6.982,5.569)--(6.982,5.562)%
  --(6.982,5.553)--(7.025,5.550)--(7.025,5.549)--(7.025,5.543)--(7.069,5.543)--(7.069,5.542)%
  --(7.112,5.538)--(7.112,5.537)--(7.156,5.536)--(7.156,5.533)--(7.156,5.528)--(7.199,5.520)%
  --(7.199,5.518)--(7.243,5.515)--(7.243,5.513)--(7.243,5.512)--(7.286,5.511)--(7.286,5.510)%
  --(7.329,5.506)--(7.329,5.503)--(7.373,5.503)--(7.373,5.498)--(7.373,5.493)--(7.373,5.490)%
  --(7.416,5.490)--(7.416,5.487)--(7.460,5.484)--(7.460,5.481)--(7.460,5.478)--(7.460,5.477)%
  --(7.503,5.475)--(7.546,5.473)--(7.590,5.468)--(7.633,5.466)--(7.633,5.461)--(7.677,5.454)%
  --(7.677,5.453)--(7.720,5.452)--(7.720,5.451)--(7.720,5.447)--(7.763,5.445)--(7.763,5.436)%
  --(7.763,5.433)--(7.807,5.429)--(7.850,5.428)--(7.850,5.425)--(7.850,5.422)--(7.894,5.419)%
  --(7.894,5.415)--(7.894,5.414)--(7.894,5.409)--(7.937,5.408)--(7.937,5.405)--(8.024,5.404)%
  --(8.024,5.403)--(8.024,5.402)--(8.067,5.402)--(8.067,5.401)--(8.067,5.399)--(8.067,5.395)%
  --(8.067,5.393)--(8.067,5.391)--(8.111,5.390)--(8.111,5.388)--(8.111,5.385)--(8.154,5.384)%
  --(8.154,5.383)--(8.198,5.379)--(8.241,5.373)--(8.284,5.373)--(8.328,5.370)--(8.328,5.369)%
  --(8.328,5.366)--(8.371,5.363)--(8.371,5.359)--(8.415,5.357)--(8.458,5.356)--(8.458,5.355)%
  --(8.501,5.354)--(8.545,5.351)--(8.588,5.350)--(8.632,5.349)--(8.632,5.347)--(8.675,5.346)%
  --(8.675,5.344)--(8.762,5.343)--(8.762,5.341)--(8.936,5.340)--(9.109,5.339)--(9.109,5.337)%
  --(9.109,5.334)--(9.153,5.332)--(9.153,5.330)--(9.153,5.326)--(9.196,5.325)--(9.239,5.323)%
  --(9.326,5.323)--(9.326,5.322)--(9.370,5.321)--(9.370,5.319)--(9.543,5.318)--(9.587,5.318)%
  --(9.587,5.313)--(9.673,5.312)--(9.673,5.311)--(9.717,5.311)--(9.847,5.311)--(9.891,5.309)%
  --(9.977,5.304)--(10.021,5.302)--(10.151,5.302)--(10.194,5.299)--(10.411,5.297)--(10.455,5.296)%
  --(10.542,5.295)--(10.846,5.294)--(10.976,5.293)--(10.976,5.291)--(11.019,10.840)--(11.063,10.812)%
  --(11.106,10.809)--(11.193,10.775)--(11.323,10.747)--(12.104,10.741)--(12.712,10.738)--(13.059,10.726)%
  --(13.580,10.725)--(13.624,10.717);
\gpcolor{color=gp lt color border}
\gpsetlinewidth{1.00}
\draw[gp path] (2.424,11.075)--(2.424,1.845)--(15.447,1.845)--(15.447,11.075)--cycle;
\node[gp node center,rotate=-270] at (-0.812,6.460) {Objective value};
\node[gp node center] at (8.935,0.215) {Relative time ($\tau$) [S]};
\node[gp node center] at (8.935,11.537) {Objective value for Weekly Schedule};
%% coordinates of the plot area
\gpdefrectangularnode{gp plot 1}{\pgfpoint{2.424cm}{1.845cm}}{\pgfpoint{15.447cm}{11.075cm}}
\endtikzpicture
%% gnuplot variables

	}
	\caption{Starting from an initial load of 173083 total hours the AbLNS is 
		progressively affected by decreasing levels of resources. The AbLNS is pertubed by two decreases
		in resources causing an initial spike in the objective value followed 
		by the AbLNS optimizing around the perturbation
	}\label{fig:responses:resource-subtraction}
\end{figure}

\subsection{Response to Changes in Work Order Values}\label{sec:results:strategic_value_changes}
The final parameter that will be changed is the work order urgency parameter
$\ParStrategicUrgency$. Table~\ref{tab:responses:urgency-change} details the
perturbations that the AbLNS will by affected by. On each iteration 100 work
orders are having their values changed by the amount  shown in the 4th row of
table~\ref{tab:responses:urgency-change}.

\begin{table}[H]
	\centering
	\begin{tabular}{lrrrrrr}
	\toprule
	                                                                                                                   & $\VarMetaTime_1 = 60$ & $\VarMetaTime_2 = 120$ & $\VarMetaTime_3 = 180$ & $\VarMetaTime_4 = 240$ & $\VarMetaTime_5 = 300$ \\ \midrule
	$\Delta |\SetWorkOrder[\VarMetaTime_{n}]{} \triangle \SetWorkOrder[\VarMetaTime_{n-1}]{}              |$           & 100                                                           & 100                                                            & 100                                                            & 100                                                            & 100                                                      \\ \midrule
	$\Delta |\SetPeriod[\VarMetaTime_{n}]{} \triangle \SetPeriod[\VarMetaTime_{n-1}]{}                    |$           & 52                                                            & 52                                                             & 52                                                             & 52                                                             & 52                                                       \\ \midrule
	\makecell{$ ||\ParStrategicUrgency[\VarMetaTime_{n}] -$\\ $\ParStrategicUrgency[\VarMetaTime_{n - 1}]||$}        & $3.75 \cdot 10^{7}$                                           & $3.75 \cdot 10^{7}$                                            & $3.75 \cdot 10^{7}$                                            & $3.75 \cdot 10^{7}$                                            & $3.75 \cdot 10^{7}$                                       \\ \bottomrule
	\end{tabular}
	\caption{Perturbations that the $\ParStrategicUrgency$ will be affected by
	}\label{tab:responses:urgency-change}
\end{table}


Figure~\ref{fig:responses:value_change} shows the effects of
perturbing the AbLNS by changing the $\ParStrategicUrgency$ parameter in the objective
function~\ref{eqn:objective:strategic} which specifies the value of assigning a
work order to a specific period.

\begin{figure}[H]%% placement specifier
	\centering
	\definecolor{gp lt10cm}{named}{dtu-red}
	\resizebox{\linewidth}{!}{
		\tikzpicture[gnuplot]
%% generated with GNUPLOT 6.0p1 (Lua 5.2; terminal rev. Jun 2020, script rev. 118)
%% Fri 10 Jan 2025 11:17:13 AM UTC
\tikzset{every node/.append style={scale=1.00}}
\path (0.000,0.000) rectangle (16.000,12.000);
\gpcolor{rgb color={0.753,0.753,0.753}}
\gpsetlinetype{gp lt border}
\gpsetdashtype{gp dt solid}
\gpsetlinewidth{2.00}
\draw[gp path] (2.424,1.845)--(13.667,1.845);
\gpcolor{color=gp lt color border}
\gpsetlinewidth{1.00}
\draw[gp path] (2.424,1.845)--(2.604,1.845);
\node[gp node right] at (2.240,2.153) {$9.5\times10^{10}$};
\gpcolor{rgb color={0.753,0.753,0.753}}
\gpsetlinewidth{2.00}
\draw[gp path] (2.424,2.871)--(13.667,2.871);
\gpcolor{color=gp lt color border}
\gpsetlinewidth{1.00}
\draw[gp path] (2.424,2.871)--(2.604,2.871);
\node[gp node right] at (2.240,3.179) {$1\times10^{11}$};
\gpcolor{rgb color={0.753,0.753,0.753}}
\gpsetlinewidth{2.00}
\draw[gp path] (2.424,3.896)--(13.667,3.896);
\gpcolor{color=gp lt color border}
\gpsetlinewidth{1.00}
\draw[gp path] (2.424,3.896)--(2.604,3.896);
\node[gp node right] at (2.240,4.204) {$1.05\times10^{11}$};
\gpcolor{rgb color={0.753,0.753,0.753}}
\gpsetlinewidth{2.00}
\draw[gp path] (2.424,4.922)--(13.667,4.922);
\gpcolor{color=gp lt color border}
\gpsetlinewidth{1.00}
\draw[gp path] (2.424,4.922)--(2.604,4.922);
\node[gp node right] at (2.240,5.230) {$1.1\times10^{11}$};
\gpcolor{rgb color={0.753,0.753,0.753}}
\gpsetlinewidth{2.00}
\draw[gp path] (2.424,5.947)--(13.667,5.947);
\gpcolor{color=gp lt color border}
\gpsetlinewidth{1.00}
\draw[gp path] (2.424,5.947)--(2.604,5.947);
\node[gp node right] at (2.240,6.255) {$1.15\times10^{11}$};
\gpcolor{rgb color={0.753,0.753,0.753}}
\gpsetlinewidth{2.00}
\draw[gp path] (2.424,6.973)--(13.667,6.973);
\gpcolor{color=gp lt color border}
\gpsetlinewidth{1.00}
\draw[gp path] (2.424,6.973)--(2.604,6.973);
\node[gp node right] at (2.240,7.281) {$1.2\times10^{11}$};
\gpcolor{rgb color={0.753,0.753,0.753}}
\gpsetlinewidth{2.00}
\draw[gp path] (2.424,7.998)--(13.667,7.998);
\gpcolor{color=gp lt color border}
\gpsetlinewidth{1.00}
\draw[gp path] (2.424,7.998)--(2.604,7.998);
\node[gp node right] at (2.240,8.306) {$1.25\times10^{11}$};
\gpcolor{rgb color={0.753,0.753,0.753}}
\gpsetlinewidth{2.00}
\draw[gp path] (2.424,9.024)--(13.667,9.024);
\gpcolor{color=gp lt color border}
\gpsetlinewidth{1.00}
\draw[gp path] (2.424,9.024)--(2.604,9.024);
\node[gp node right] at (2.240,9.332) {$1.3\times10^{11}$};
\gpcolor{rgb color={0.753,0.753,0.753}}
\gpsetlinewidth{2.00}
\draw[gp path] (2.424,10.049)--(7.047,10.049);
\draw[gp path] (13.483,10.049)--(13.667,10.049);
\gpcolor{color=gp lt color border}
\gpsetlinewidth{1.00}
\draw[gp path] (2.424,10.049)--(2.604,10.049);
\node[gp node right] at (2.240,10.357) {$1.35\times10^{11}$};
\gpcolor{rgb color={0.753,0.753,0.753}}
\gpsetlinewidth{2.00}
\draw[gp path] (2.424,11.075)--(13.667,11.075);
\gpcolor{color=gp lt color border}
\gpsetlinewidth{1.00}
\draw[gp path] (2.424,11.075)--(2.604,11.075);
\node[gp node right] at (2.240,11.383) {$1.4\times10^{11}$};
\gpcolor{rgb color={0.753,0.753,0.753}}
\gpsetlinewidth{2.00}
\draw[gp path] (2.424,1.845)--(2.424,11.075);
\gpcolor{color=gp lt color border}
\gpsetlinewidth{1.00}
\draw[gp path] (2.424,1.845)--(2.424,2.025);
\draw[gp path] (2.424,11.075)--(2.424,10.895);
\node[gp node left,rotate=270] at (2.424,1.661) {$0$};
\gpcolor{rgb color={0.753,0.753,0.753}}
\gpsetlinewidth{2.00}
\draw[gp path] (4.030,1.845)--(4.030,11.075);
\gpcolor{color=gp lt color border}
\gpsetlinewidth{1.00}
\draw[gp path] (4.030,1.845)--(4.030,2.025);
\draw[gp path] (4.030,11.075)--(4.030,10.895);
\node[gp node left,rotate=270] at (4.030,1.661) {$60$};
\gpcolor{rgb color={0.753,0.753,0.753}}
\gpsetlinewidth{2.00}
\draw[gp path] (5.636,1.845)--(5.636,11.075);
\gpcolor{color=gp lt color border}
\gpsetlinewidth{1.00}
\draw[gp path] (5.636,1.845)--(5.636,2.025);
\draw[gp path] (5.636,11.075)--(5.636,10.895);
\node[gp node left,rotate=270] at (5.636,1.661) {$120$};
\gpcolor{rgb color={0.753,0.753,0.753}}
\gpsetlinewidth{2.00}
\draw[gp path] (7.242,1.845)--(7.242,9.355);
\draw[gp path] (7.242,10.895)--(7.242,11.075);
\gpcolor{color=gp lt color border}
\gpsetlinewidth{1.00}
\draw[gp path] (7.242,1.845)--(7.242,2.025);
\draw[gp path] (7.242,11.075)--(7.242,10.895);
\node[gp node left,rotate=270] at (7.242,1.661) {$180$};
\gpcolor{rgb color={0.753,0.753,0.753}}
\gpsetlinewidth{2.00}
\draw[gp path] (8.849,1.845)--(8.849,9.355);
\draw[gp path] (8.849,10.895)--(8.849,11.075);
\gpcolor{color=gp lt color border}
\gpsetlinewidth{1.00}
\draw[gp path] (8.849,1.845)--(8.849,2.025);
\draw[gp path] (8.849,11.075)--(8.849,10.895);
\node[gp node left,rotate=270] at (8.849,1.661) {$240$};
\gpcolor{rgb color={0.753,0.753,0.753}}
\gpsetlinewidth{2.00}
\draw[gp path] (10.455,1.845)--(10.455,9.355);
\draw[gp path] (10.455,10.895)--(10.455,11.075);
\gpcolor{color=gp lt color border}
\gpsetlinewidth{1.00}
\draw[gp path] (10.455,1.845)--(10.455,2.025);
\draw[gp path] (10.455,11.075)--(10.455,10.895);
\node[gp node left,rotate=270] at (10.455,1.661) {$300$};
\gpcolor{rgb color={0.753,0.753,0.753}}
\gpsetlinewidth{2.00}
\draw[gp path] (12.061,1.845)--(12.061,9.355);
\draw[gp path] (12.061,10.895)--(12.061,11.075);
\gpcolor{color=gp lt color border}
\gpsetlinewidth{1.00}
\draw[gp path] (12.061,1.845)--(12.061,2.025);
\draw[gp path] (12.061,11.075)--(12.061,10.895);
\node[gp node left,rotate=270] at (12.061,1.661) {$360$};
\gpcolor{rgb color={0.753,0.753,0.753}}
\gpsetlinewidth{2.00}
\draw[gp path] (13.667,1.845)--(13.667,11.075);
\gpcolor{color=gp lt color border}
\gpsetlinewidth{1.00}
\draw[gp path] (13.667,1.845)--(13.667,2.025);
\draw[gp path] (13.667,11.075)--(13.667,10.895);
\node[gp node left,rotate=270] at (13.667,1.661) {$420$};
\draw[gp path] (13.667,1.845)--(13.487,1.845);
\node[gp node left] at (13.851,2.153) {$164000$};
\draw[gp path] (13.667,2.768)--(13.487,2.768);
\node[gp node left] at (13.851,3.076) {$164500$};
\draw[gp path] (13.667,3.691)--(13.487,3.691);
\node[gp node left] at (13.851,3.999) {$165000$};
\draw[gp path] (13.667,4.614)--(13.487,4.614);
\node[gp node left] at (13.851,4.922) {$165500$};
\draw[gp path] (13.667,5.537)--(13.487,5.537);
\node[gp node left] at (13.851,5.845) {$166000$};
\draw[gp path] (13.667,6.460)--(13.487,6.460);
\node[gp node left] at (13.851,6.768) {$166500$};
\draw[gp path] (13.667,7.383)--(13.487,7.383);
\node[gp node left] at (13.851,7.691) {$167000$};
\draw[gp path] (13.667,8.306)--(13.487,8.306);
\node[gp node left] at (13.851,8.614) {$167500$};
\draw[gp path] (13.667,9.229)--(13.487,9.229);
\node[gp node left] at (13.851,9.537) {$168000$};
\draw[gp path] (13.667,10.152)--(13.487,10.152);
\node[gp node left] at (13.851,10.460) {$168500$};
\draw[gp path] (13.667,11.075)--(13.487,11.075);
\node[gp node left] at (13.851,11.383) {$169000$};
\draw[gp path] (2.424,11.075)--(2.424,1.845)--(13.667,1.845)--(13.667,11.075)--cycle;
\draw[gp path] (7.047,9.355)--(7.047,10.895)--(13.483,10.895)--(13.483,9.355)--cycle;
\node[gp node right] at (12.199,10.510) {Strategic Urgency};
\gpcolor{rgb color={0.000,0.000,0.000}}
\gpsetlinewidth{2.50}
\draw[gp path] (12.383,10.510)--(13.299,10.510);
\draw[gp path] (2.424,2.077)--(2.424,2.116)--(2.451,2.119)--(2.451,2.140)--(2.451,2.149)%
  --(2.451,2.167)--(2.451,2.159)--(2.451,2.176)--(2.451,2.201)--(2.478,2.203)--(2.478,2.209)%
  --(2.478,2.211)--(2.478,2.209)--(2.478,2.241)--(2.478,2.239)--(2.504,2.272)--(2.504,2.284)%
  --(2.504,2.281)--(2.504,2.285)--(2.504,2.316)--(2.504,2.327)--(2.504,2.322)--(2.531,2.326)%
  --(2.531,2.319)--(2.531,2.328)--(2.531,2.339)--(2.531,2.350)--(2.558,2.351)--(2.558,2.360)%
  --(2.558,2.354)--(2.558,2.357)--(2.558,2.363)--(2.558,2.358)--(2.585,2.364)--(2.585,2.379)%
  --(2.585,2.380)--(2.585,2.360)--(2.585,2.361)--(2.585,2.354)--(2.585,2.351)--(2.611,2.353)%
  --(2.611,2.358)--(2.611,2.357)--(2.638,2.372)--(2.638,2.373)--(2.638,2.376)--(2.638,2.366)%
  --(2.638,2.356)--(2.638,2.353)--(2.665,2.360)--(2.665,2.366)--(2.665,2.371)--(2.692,2.365)%
  --(2.692,2.367)--(2.692,2.365)--(2.692,2.368)--(2.718,2.357)--(2.718,2.368)--(2.745,2.371)%
  --(2.745,2.361)--(2.745,2.365)--(2.745,2.368)--(2.772,2.370)--(2.772,2.377)--(2.772,2.385)%
  --(2.772,2.380)--(2.772,2.379)--(2.799,2.383)--(2.799,2.378)--(2.799,2.376)--(2.826,2.374)%
  --(2.826,2.376)--(2.826,2.372)--(2.826,2.377)--(2.852,2.371)--(2.852,2.367)--(2.879,2.365)%
  --(2.879,2.355)--(2.879,2.349)--(2.879,2.353)--(2.879,2.347)--(2.879,2.358)--(2.906,2.357)%
  --(2.906,2.361)--(2.906,2.373)--(2.933,2.372)--(2.933,2.368)--(2.933,2.367)--(2.933,2.362)%
  --(2.986,2.366)--(2.986,2.379)--(3.013,2.382)--(3.013,2.387)--(3.040,2.393)--(3.040,2.391)%
  --(3.040,2.395)--(3.040,2.400)--(3.066,2.396)--(3.066,2.391)--(3.093,2.404)--(3.093,2.401)%
  --(3.120,2.408)--(3.120,2.407)--(3.147,2.403)--(3.147,2.402)--(3.147,2.400)--(3.147,2.407)%
  --(3.147,2.410)--(3.174,2.403)--(3.200,2.403)--(3.200,2.425)--(3.200,2.419)--(3.200,2.421)%
  --(3.200,2.419)--(3.281,2.422)--(3.281,2.420)--(3.281,2.413)--(3.307,2.411)--(3.307,2.407)%
  --(3.307,2.409)--(3.334,2.410)--(3.334,2.408)--(3.334,2.403)--(3.334,2.417)--(3.334,2.413)%
  --(3.361,2.420)--(3.388,2.422)--(3.388,2.420)--(3.441,2.413)--(3.522,2.420)--(3.548,2.421)%
  --(3.548,2.416)--(3.736,2.422)--(3.762,2.416)--(3.762,2.405)--(3.816,2.403)--(3.977,2.404)%
  --(3.977,2.402)--(4.003,2.402)--(4.110,3.398)--(4.137,3.395)--(4.137,3.407)--(4.164,3.405)%
  --(4.191,3.402)--(4.244,3.400)--(4.244,3.399)--(4.271,3.398)--(4.298,3.389)--(4.298,3.375)%
  --(4.325,3.387)--(4.351,3.386)--(4.351,3.385)--(4.458,3.384)--(4.458,3.378)--(4.539,3.378)%
  --(4.592,3.380)--(4.646,3.376)--(4.646,3.381)--(4.753,3.381)--(4.806,3.377)--(4.806,3.375)%
  --(4.887,3.376)--(5.074,3.377)--(5.128,3.374)--(5.502,3.371)--(5.556,3.386)--(5.636,3.384)%
  --(5.663,3.385)--(5.690,3.388)--(5.743,4.792)--(5.931,4.786)--(5.931,4.793)--(6.091,4.793)%
  --(6.118,4.791)--(6.172,4.788)--(6.279,4.788)--(6.306,4.786)--(6.466,4.790)--(6.520,4.790)%
  --(6.546,4.785)--(6.546,4.782)--(6.734,4.768)--(6.761,4.764)--(6.841,4.759)--(7.055,4.772)%
  --(7.189,4.770)--(7.216,4.768)--(7.323,4.758)--(7.323,4.757)--(7.376,5.509)--(7.510,5.506)%
  --(7.510,5.503)--(7.778,5.498)--(8.179,5.496)--(8.313,5.493)--(8.634,5.500)--(9.732,8.337)%
  --(9.919,8.341)--(10.026,8.346)--(10.133,8.346)--(10.187,8.345)--(10.348,8.337)--(10.562,10.992)%
  --(10.642,10.988)--(10.990,10.985)--(11.017,10.983)--(11.499,10.979)--(11.579,10.977)--(11.633,10.977)%
  --(11.740,10.975)--(12.141,10.972);
\gpcolor{color=gp lt color border}
\node[gp node right] at (12.199,9.740) {Strategic Resource Penalty};
\gpcolor{rgb color={0.184,0.243,0.918}}
\draw[gp path] (12.383,9.740)--(13.299,9.740);
\draw[gp path] (2.424,10.698)--(2.424,10.386)--(2.451,10.248)--(2.451,9.925)--(2.451,9.742)%
  --(2.451,9.554)--(2.451,9.332)--(2.451,9.172)--(2.451,8.895)--(2.478,8.721)--(2.478,8.594)%
  --(2.478,8.524)--(2.478,8.419)--(2.478,8.240)--(2.478,8.153)--(2.504,8.027)--(2.504,7.857)%
  --(2.504,7.763)--(2.504,7.712)--(2.504,7.566)--(2.504,7.361)--(2.504,7.171)--(2.531,7.106)%
  --(2.531,7.034)--(2.531,6.894)--(2.531,6.634)--(2.531,6.473)--(2.531,6.381)--(2.558,6.325)%
  --(2.558,6.233)--(2.558,6.166)--(2.558,6.050)--(2.558,6.039)--(2.558,6.026)--(2.558,6.008)%
  --(2.585,5.995)--(2.585,5.956)--(2.585,5.897)--(2.585,5.862)--(2.585,5.795)--(2.585,5.685)%
  --(2.585,5.655)--(2.611,5.603)--(2.611,5.583)--(2.611,5.476)--(2.638,5.393)--(2.638,5.354)%
  --(2.638,5.332)--(2.638,5.328)--(2.638,5.312)--(2.638,5.271)--(2.638,5.183)--(2.665,5.070)%
  --(2.665,5.064)--(2.665,5.048)--(2.692,4.948)--(2.692,4.924)--(2.692,4.895)--(2.692,4.808)%
  --(2.718,4.808)--(2.718,4.546)--(2.745,4.538)--(2.745,4.522)--(2.745,4.448)--(2.745,4.444)%
  --(2.772,4.424)--(2.772,4.392)--(2.772,4.387)--(2.772,4.383)--(2.772,4.049)--(2.799,4.042)%
  --(2.799,4.018)--(2.826,4.003)--(2.826,3.985)--(2.826,3.973)--(2.852,3.961)--(2.879,3.961)%
  --(2.879,3.944)--(2.879,3.916)--(2.879,3.901)--(2.879,3.859)--(2.906,3.859)--(2.906,3.837)%
  --(2.906,3.829)--(2.933,3.804)--(2.933,3.800)--(2.933,3.785)--(2.986,3.778)--(2.986,3.767)%
  --(3.013,3.752)--(3.013,3.746)--(3.040,3.691)--(3.040,3.689)--(3.040,3.676)--(3.066,3.660)%
  --(3.066,3.658)--(3.066,3.654)--(3.093,3.650)--(3.120,3.591)--(3.147,3.569)--(3.147,3.545)%
  --(3.147,3.484)--(3.147,3.469)--(3.174,3.466)--(3.200,3.451)--(3.200,3.449)--(3.200,3.434)%
  --(3.200,3.421)--(3.281,3.421)--(3.281,3.409)--(3.307,3.399)--(3.307,3.361)--(3.334,3.361)%
  --(3.334,3.359)--(3.334,3.355)--(3.361,3.006)--(3.388,3.004)--(3.441,3.004)--(3.522,2.986)%
  --(3.548,2.975)--(3.548,2.971)--(3.736,2.960)--(3.762,2.953)--(3.762,2.949)--(3.816,2.949)%
  --(3.977,2.949)--(3.977,2.940)--(4.003,2.940)--(4.110,2.940)--(4.137,2.930)--(4.137,2.929)%
  --(4.164,2.929)--(4.191,2.929)--(4.244,2.910)--(4.271,2.910)--(4.298,2.910)--(4.325,2.908)%
  --(4.351,2.908)--(4.351,2.875)--(4.458,2.875)--(4.458,2.864)--(4.539,2.864)--(4.592,2.860)%
  --(4.646,2.857)--(4.646,2.847)--(4.753,2.847)--(4.806,2.847)--(4.887,2.847)--(5.074,2.825)%
  --(5.128,2.823)--(5.502,2.823)--(5.556,2.810)--(5.636,2.810)--(5.663,2.770)--(5.690,2.766)%
  --(5.743,2.766)--(5.931,2.766)--(5.931,2.751)--(6.091,2.733)--(6.118,2.733)--(6.172,2.733)%
  --(6.279,2.733)--(6.306,2.733)--(6.466,2.724)--(6.520,2.724)--(6.546,2.718)--(6.734,2.718)%
  --(6.761,2.718)--(6.841,2.718)--(7.055,2.713)--(7.189,2.713)--(7.216,2.713)--(7.323,2.713)%
  --(7.376,2.713)--(7.510,2.713)--(7.778,2.713)--(8.179,2.713)--(8.313,2.713)--(8.634,2.709)%
  --(9.732,2.703)--(9.919,2.702)--(10.026,2.698)--(10.133,2.698)--(10.187,2.698)--(10.348,2.698)%
  --(10.562,2.694)--(10.642,2.694)--(10.990,2.694)--(11.017,2.694)--(11.499,2.694)--(11.579,2.694)%
  --(11.633,2.694)--(11.740,2.694)--(12.141,2.694);
\gpcolor{color=gp lt color border}
\gpsetlinewidth{1.00}
\draw[gp path] (2.424,11.075)--(2.424,1.845)--(13.667,1.845)--(13.667,11.075)--cycle;
\node[gp node center,rotate=-270] at (-0.628,6.460) {Strategic Urgency [Weighted Tardiness]};
\node[gp node center,rotate=-270] at (16.213,6.460) {Strategic Resource Penalty [Hours]};
\node[gp node center] at (8.045,0.215) {Relative time ($\tau$) [Seconds]};
\node[gp node center] at (8.045,11.537) {Objective Values for Weekly Schedule};
%% coordinates of the plot area
\gpdefrectangularnode{gp plot 1}{\pgfpoint{2.424cm}{1.845cm}}{\pgfpoint{13.667cm}{11.075cm}}
\endtikzpicture
%% gnuplot variables

	}
	\caption{The effects of perturbing the AbLNS by dynamically changing the
		$\ParStrategicUrgency$ in the objective function. The objective value
		is increasing in response to the higher cost associated with a late
		scheduling, and a higher initial resource consumption
		after which it optimizes around the perturbation
	}\label{fig:responses:value_change}
\end{figure}
 
