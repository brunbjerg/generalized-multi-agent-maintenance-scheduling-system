\section{Solution Method}
\label{sec:2-solution-method}

\subsection{Actor-based Large Neighborhood Search}
A problem formulation that is affected by the end user and requires real-time feedback
inherrently requires an optimization approach that is able to repair infeasible solution
and while also converging quickly. For this the large neighborhood search (LNS)
metaheuristic has been shown satisfy these requirements in the literature
\cite{gendreauHandbookMetaheuristics2019}. The most general form of the LNS metaheuristic
is defined for static problems, meaning that the parameters that make up the problem instance
is not subject to change after the algorithm has been started. To make the
LNS able adapt to changing parameters in real-time a message system have been
implemented into the existing framework. This  extension is shown in algorithm
\ref{alg:actor-based-large-neighborhood-search}.

\subsubsection{Messages And Destructors}
LNS in its basic form has one constructor and one destructor which
repeatedly destroy and rebuild the solution. For the AbLNS we will generalize
on this concept by including messages as destructors of the classic LNS
implementation. This generalization can be seen as being somewhat similar to how
the adaptive LNS (ALNS) is formulated, but where the different constructors and
destructors are also chosen from sources external to the algorithm.

Extending on the classic setup we define the following set of destructors, $\ElementMessage{}{} \in \SetMessageQueue$.
\begin{itemize}
	\item $m_1$: Inclusion destruct message	
	\item $m_2$: Exclusion destruct message	
	\item $m_3$: Capacities destruct message	
	\item $m_4$: Weights destruct message	
	\item $m_5$: Random destruct message
\end{itemize}


Each of these messages affect different parts of the weekly maintenance scheduling 
problem. Notice here that the first four messages destruct the solution by
changing the parameter space and the last message is a random destructor.

For correct implementation it is critical that the that destruct messages are 
handled one by one, and are not allowed to simply write to the solution and
problem instance whenever they appear.
Generalizing the destructors from being static structures into messages allows
the solution to change in real-time to a changing paramenter space. This means
that the algorithm does not need to restart to handle changes in data.

\newcommand{\MessageQueue}{Q}
\newcommand{\Solution}{X}
\newcommand{\ProblemInstance}{P}
\newcommand{\SharedSolution}{S}

\begin{algorithm}[H]
\caption{Actor-based Large Neighborhood Search} 
\begin{algorithmic}[1]
\State \textbf{Input} $\MessageQueue$    = message queue
\State \textbf{Input} $\ProblemInstance$ = problem instance
\State \textbf{Input} $\Solution$        = initial schedule
\State \textbf{Input} $\SharedSolution$  = shared solution
% \State $\Solution^b = \Solution$
\Repeat \label{alg:actor-based-large-neighborhood-search:main-optimization-loop}
	\State $\Solution^t = clone(\Solution)$ \label{alg:actor-based-large-neighborhood-search:clone-solution}
	\While{$\MessageQueue.has\_message()$} \label{alg:actor-based-large-neighborhood-search:message-loop}
        % \State $m = queue.pop()$
        % \State $m.destruct(\Solution^b)$
		\State $\ProblemInstance.update(\SharedSolution, m)$
        \State $\Solution^t.destruct(\SharedSolution, m)$
    \EndWhile
	
    \State $\Solution^t.repair(\SharedSolution)$ \label{alg:actor-based-large-neighborhood-search:repair}

    \If{accept($\Solution^t, \Solution$)} \label{alg:actor-based-large-neighborhood-search:accept}                       
        \State $\Solution.update(\Solution^t)$
    \EndIf                                         

    \If{$c(\Solution^t) < c(\Solution)$} \label{alg:actor-based-large-neighborhood-search:better-solution}                             
        \State $\Solution.update(\Solution^t)$
		\State $\SharedSolution.atomic\_pointer\_swap(\Solution)$
    \EndIf                                           
	\State $\MessageQueue.push(m)$ \label{alg:actor-based-large-neighborhood-search:schedule-message}
\Until
\end{algorithmic}
\label{alg:actor-based-large-neighborhood-search}
\end{algorithm}



The basic LNS setup have here been extended with a `message queue` $\MessageQueue$. The message
queue will be read from on every iteration of the LNS's main iteration loop.
Here we notice that the incoming message is able to change both the solution $\Solution$
but also the problem instance $\ProblemInstance$ itself. Here we see the defining feature
of the LNS metaheuristic, that due to its inherent property of being
able to optimize a solution that have become infeasible. This is something
that is going happen when you change the parameter of the problem
instance $\ProblemInstance$ itself. Another less obvious property of the message queue is 
that the algorithm can run indefinitely and opposed to restarting the algorithm you
instead pass messages to it to allow it be adjust both the solution space and
the parameter space. This property avoid the issue of time consuming initial
convergence as the algorithm will be found in an optimimal state when the
solution is perturbed. Notice that:

\begin{itemize}
    \item The algorithm responds to changes very quickly, in each iteration line~\ref{alg:actor-based-large-neighborhood-search:message-loop}. We call this a fine-grained response algorithm
    \item If an improved solution is found, it is immediately being pushed to the data (base) on line~\ref{alg:actor-based-large-neighborhood-search:better-solution}
    \item That the optimization occurs in the repair function on line~\ref{alg:actor-based-large-neighborhood-search:repair}, which inserts operations in a greedy fashion. 
\end{itemize}


