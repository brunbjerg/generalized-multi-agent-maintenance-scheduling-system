\section{Solution Method}
\label{sec:2-solution-method}

\subsection{Actor-based Large Neighborhood Search}
A metaheuristic that is affected by the end user and requires real-time feedback
inherently requires an optimization approach that is able to repair infeasible solutions
while also converging quickly. The large neighborhood search (LNS)~\cite{shaw1998using}
metaheuristic has been shown satisfy these requirements in the literature
\cite{gendreauHandbookMetaheuristics2019}. The most general form of the LNS metaheuristic
is defined for static problems, meaning that the parameters that make up the problem instance
is not subject to change after the algorithm has started. To make the
LNS adapt to changing parameters in real-time a message system have been
implemented into the existing framework. This  extension is shown in algorithm
\ref{alg:actor-based-large-neighborhood-search}.

\subsubsection{Messages and Destructors}
LNS in its most basic form has one constructor and one destructor which
repeatedly destroy and rebuild the solution. For the AbLNS we will generalize
on this concept by including messages as destructors. This generalization can
be seen as being somewhat similar to how
the adaptive LNS (ALNS)~\cite{pisinger2007general} is formulated, but where the different constructors and
destructors are also chosen from sources external to the algorithm.

Extending on the classic setup we define the following set of 
destructor messages, $\ElementMessage{}{} \in \SetMessageQueue$. Each
message hits the system at time $\VarMetaTime$.

\begin{itemize}
	\item $m_1(\VarMetaTime)$: Inclusion destruct message, modifies $\ParStrategicInclude$	
	\item $m_2(\VarMetaTime)$: Exclusion destruct message, modifies $\ParStrategicExclude$
	\item $m_3(\VarMetaTime)$: Capacities destruct message, modifies $\ParStrategicResource$
	\item $m_4(\VarMetaTime)$: Weights destruct messags, modifies $\ParStrategicValue$
	\item $m_5(\VarMetaTime)$: Random destruct message, modifies $\VarStrategicWorkOrderAssignment{\ElementWorkOrder}{\ElementPeriod}$
\end{itemize}

Each of these messages affect different parts of the weekly maintenance scheduling 
problem. Notice here that the first four messages destruct the solution by
changing the parameter space and the last message is a random destructor. That destroys
the solution space.

For a correct implementation it is critical that the that destruct messages are 
handled one by one, and are not allowed to simply write to the solution and
problem instance whenever they appear.
Generalizing the destructors from being static structures into messages allows
the solution to change in real-time to a changing parameter space. This means
that the algorithm does not need to restart to handle changes in data.

% FIX: Parameterize the pseudo code. It should be generalized to the paper.

\newcommand{\MessageQueue}{Q}
\newcommand{\Solution}{X}
\newcommand{\ProblemInstance}{P}
\newcommand{\SharedSolution}{S}

\begin{algorithm}[H]
\caption{Actor-based Large Neighborhood Search} 
\begin{algorithmic}[1]
\State \textbf{Input} $\MessageQueue$    = message queue
\State \textbf{Input} $\ProblemInstance$ = problem instance
\State \textbf{Input} $\Solution$        = initial schedule
\State \textbf{Input} $\SharedSolution$  = shared solution
% \State $\Solution^b = \Solution$
\Repeat \label{alg:actor-based-large-neighborhood-search:main-optimization-loop}
	\State $\Solution^t = clone(\Solution)$ \label{alg:actor-based-large-neighborhood-search:clone-solution}
	\While{$\MessageQueue.has\_message()$} \label{alg:actor-based-large-neighborhood-search:message-loop}
        % \State $m = queue.pop()$
        % \State $m.destruct(\Solution^b)$
		\State $\ProblemInstance.update(\SharedSolution, m)$
        \State $\Solution^t.destruct(\SharedSolution, m)$
    \EndWhile
	
    \State $\Solution^t.repair(\SharedSolution)$ \label{alg:actor-based-large-neighborhood-search:repair}

    \If{accept($\Solution^t, \Solution$)} \label{alg:actor-based-large-neighborhood-search:accept}                       
        \State $\Solution.update(\Solution^t)$
    \EndIf                                         

    \If{$c(\Solution^t) < c(\Solution)$} \label{alg:actor-based-large-neighborhood-search:better-solution}                             
        \State $\Solution.update(\Solution^t)$
		\State $\SharedSolution.atomic\_pointer\_swap(\Solution)$
    \EndIf                                           
	\State $\MessageQueue.push(m)$ \label{alg:actor-based-large-neighborhood-search:schedule-message}
\Until
\end{algorithmic}
\label{alg:actor-based-large-neighborhood-search}
\end{algorithm}



The basic LNS setup have here been extended with a `message queue` $\MessageQueue$. The message
queue will be read from on every iteration of the LNS's main iteration loop.
Here we notice that the incoming message is able to change both the solution $\Solution$
but also the problem instance $\ProblemInstance$ itself. Due to the inherent property of LNS being
able to optimize a solution that have become infeasible, LNS is well suited to handle changing parameters 
in the problem instance $\ProblemInstance$ itself. Another property of the message queue is 
that the algorithm can run indefinitely and as opposed to restarting the algorithm you
pass messages containing new data and or state. This property avoid the time consuming initial
convergence as the algorithm will be found in an optimized state when the
solution is perturbed. Notice that:

\begin{itemize}
    \item The algorithm responds quickly to changes. In each iteration line~\ref{alg:actor-based-large-neighborhood-search:message-loop}. We call this a fine-grained response algorithm
    \item If an improved solution is found, it is immediately being pushed to the persistence on line~\ref{alg:actor-based-large-neighborhood-search:better-solution}
    \item That the optimization occurs in the repair function on line~\ref{alg:actor-based-large-neighborhood-search:repair}, which inserts operations in a greedy fashion. 
	\item The pointer swapping that is performed on line~\ref{alg:actor-based-large-neighborhood-search:atomic-pointer-swap} when a new best solution is found instantaneously makes the solution available to other actors. The pointer swapping is crucial as it allows non-hierarchical model setups. 
\end{itemize}

The AbLNS algorithm will continuously run either optimizing the current plan or updating the plan 
with new external input. Since the effects of a message is considered in every iteration, changing 
conditions will immediately get acted upon. Since a fast response is paramount, exact optimization algorithms are deemed unworkable.

\subsection{Atomic Pointer Swapping and Software Architecture}
Atomic pointer swapping offers a novel approach for managing the best
solution found by the Actor-based Large Neighborhood Search (AbLNS) algorithm.
Traditionally, in a metaheuristics such as Large Neighborhood Search (LNS) setup, 
the best solution
is cloned before the search continues. However, atomic pointer swapping enables
the best solution to be shared across system components without introducing data
races. This technique is essential in multithreaded environments, as it ensures
the integrity of the solution during swaps. In software architecture, scaling
is often achieved through vertical scaling (lengthening the dataflow) and
horizontal scaling (increasing throughput) via message passing. While effective
in many domains, this approach poses challenges for metaheuristics, which
rely on rapid state mutations and solution improvements. 

Message passing in metaheuristics can create unpredictable behaviors,
especially with bidirectional communication between components. The
limitations of message-passing systems in metaheuristics are well-documented,
particularly in the context of parallel and multi-agent metaheuristics
\citep{talbiMetaheuristicsDesignImplementation2009}. These systems often
suffer from excessive message generation, leading to poor scalability and
performance. Key issues include: 

\begin{itemize} 
	\item The stochastic nature
		of metaheuristics makes message-passing behavior unpredictable. 
	\item Avoiding
		hierarchical, unidirectional message flows often results in cycles. 
	\item The
		impact of increased messaging is difficult to analyze and manage. 
\end{itemize}

Atomic pointer swapping addresses these challenges by eliminating state
duplication, reducing system complexity, and bypassing the inefficiencies of
message passing. This method provides a streamlined, modular way to integrate
metaheuristics, aligning with modern software architecture principles while
avoiding their typical pitfalls \citep{richards_fundamentals_2020}.
