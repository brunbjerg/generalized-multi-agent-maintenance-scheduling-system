\section{Solution Method}\label{sec:2-solution-method}

\subsection{Actor-based Large Neighborhood Search} 
A metaheuristic that is affected by the end user and requires real-time
feedback inherently requires an optimization approach that is able to repair
infeasible solutions while also converging quickly. The large neighborhood
search (LNS)~\citep{shaw1998using} metaheuristic has been shown satisfy these
requirements in the literature \citep{gendreauHandbookMetaheuristics2019lnschapter}.
The most general form of the LNS metaheuristic is defined for static
problems, meaning that the parameters that make up the problem instance
is not subject to change after the algorithm has started. To make the
LNS adapt to changing parameters in real-time a message system have been
implemented into the existing framework. This  extension is shown in 
algorithm~\ref{alg:ablns}.

\subsubsection{Messages and Destroy Functions}
LNS in its most basic form has one repair and one destroy function which
repeatedly destroy and rebuild the solution. For the AbLNS (pseudocode shown in
algorithm~\ref{alg:ablns}) we will
generalize on this concept by including messages as destroy functions. This
generalization can be seen as being somewhat similar to how the adaptive
LNS (ALNS)~\citep{pisinger2007general} is formulated, but where the different
repair and destroy functions are also chosen from sources external to the
algorithm.

Extending on the classic setup we define the following set of 
destroy messages, $\ElementMessage{}{} \in \SetMessageQueue$. Each
message is received by the system at time $\VarMetaTime \in \mathbb{R}$.

\begin{itemize}
	\item $m^{wp}_{1}(\VarMetaTime)$: Inclusion destroy message, modifies $\ParStrategicInclude$	
	\begin{itemize}
		\item Each instance of this message destroys the solution in line~\ref{alg:ablns:solution-update} 
			and updates the problem parameters in line~\ref{alg:ablns:problem-update} of algorithm~\ref{alg:ablns}.
			On line~\ref{alg:ablns:repair} the repair function then reschedules according to the value of $\ParStrategicInclude$.
	\end{itemize}
	\item $m^{wp}_{2}(\VarMetaTime)$: Exclusion destroy message, modifies $\ParStrategicExclude$
	\begin{itemize}
		\item Each instance of this message updates the $\ParStrategicExclude$ on line~\ref{alg:ablns:problem-update}
			and if the message excludes an already scheduled work order the destroy function on line~\ref{alg:ablns:solution-update}
			will update the solution as well. If the work order was already forced into the schedule by $\ParStrategicInclude$ line~\ref{alg:ablns:problem-update}
			and line~\ref{alg:ablns:problem-update} of algorithm~\ref{alg:ablns} will also update $\ParStrategicInclude$.
	\end{itemize}
	\item $m^{pr}_{3}(\VarMetaTime)$: Capacities destroy message, modifies $\ParStrategicResource{}{}$
	\begin{itemize}
		\item Each instance of this message perturbes the value of available hours of a given resource $\ElementResource \in \SetResource$ for
			a given $\ElementPeriod \in \SetPeriod$, this update happens on line~\ref{alg:ablns:problem-update} in algorithm~\ref{alg:ablns}.
	\end{itemize}
	\item $m^{wp}_{4}(\VarMetaTime)$: Strategic urgency destroy message, modifies $\ParStrategicUrgency$
	\begin{itemize}
		\item Each instance of this message perturbes the strategic urgency of a work order, this happens on line~\ref{alg:ablns:problem-update} and
		line~\ref{alg:ablns:solution-update} of algorithm~\ref{alg:ablns}. Here it should be noted that the
			message changes the $\ParStrategicUrgency$ $\forall\ElementPeriod \in \SetPeriod$ for a given $\ElementWorkOrder \in \SetWorkOrder$. 
	\end{itemize}
	\item $m^{wk}_{5}(\VarMetaTime)$: Random destroy message, modifies $\VarStrategicWorkOrderAssignment{\ElementWorkOrder}{\ElementPeriod}$
	\begin{itemize}
		\item Each instance of this message modifies the main decision variable, $\VarStrategicWorkOrderAssignment{\ElementWorkOrder}{\ElementPeriod}$. 
			Removing random work order assignments from $k$ randomly chosen $\ElementWorkOrder \in \SetWorkOrder$, where $k\in\mathbb{N}$ is a parameter of the AbLNS.
	\end{itemize}
\end{itemize}

Each of these messages impacts different aspects of the weekly maintenance
scheduling problem. Notably, the first four messages destroys the solution by
modifying both the parameter/data space and the solution itself, while the final
message serves as a random destruction function that affects only the solution.

For a correct implementation it is critical that the that destroy messages
are  handled one by one, and are not allowed to simply write to the solution and
problem instance whenever they appear. Generalizing the destroy functions from being
static structures into messages allows the solution to change in real-time to a
changing parameter space. This means that the algorithm does not need to restart
to handle changes in data.

\newcommand{\MessageQueue}{Q}
\newcommand{\Solution}{X}
\newcommand{\ProblemInstance}{P}
\newcommand{\SharedSolution}{S}

\begin{algorithm}[H]
\caption{Actor-based Large Neighborhood Search} 
\begin{algorithmic}[1]
\State \textbf{Input} $\MessageQueue$    = message queue
\State \textbf{Input} $\ProblemInstance$ = problem instance
\State \textbf{Input} $\Solution$        = initial schedule
\State \textbf{Input} $\SharedSolution$  = shared solution
% \State $\Solution^b = \Solution$
\Repeat \label{alg:actor-based-large-neighborhood-search:main-optimization-loop}
	\State $\Solution^t = clone(\Solution)$ \label{alg:actor-based-large-neighborhood-search:clone-solution}
	\While{$\MessageQueue.has\_message()$} \label{alg:actor-based-large-neighborhood-search:message-loop}
        % \State $m = queue.pop()$
        % \State $m.destruct(\Solution^b)$
		\State $\ProblemInstance.update(\SharedSolution, m)$
        \State $\Solution^t.destruct(\SharedSolution, m)$
    \EndWhile
	
    \State $\Solution^t.repair(\SharedSolution)$ \label{alg:actor-based-large-neighborhood-search:repair}

    \If{accept($\Solution^t, \Solution$)} \label{alg:actor-based-large-neighborhood-search:accept}                       
        \State $\Solution.update(\Solution^t)$
    \EndIf                                         

    \If{$c(\Solution^t) < c(\Solution)$} \label{alg:actor-based-large-neighborhood-search:better-solution}                             
        \State $\Solution.update(\Solution^t)$
		\State $\SharedSolution.atomic\_pointer\_swap(\Solution)$
    \EndIf                                           
	\State $\MessageQueue.push(m)$ \label{alg:actor-based-large-neighborhood-search:schedule-message}
\Until
\end{algorithmic}
\label{alg:actor-based-large-neighborhood-search}
\end{algorithm}



The basic LNS setup have here been extended with a ``message queue'' $
\MessageQueue$. The message queue will be read from on every iteration of
the LNS's main iteration loop. Here we notice that the incoming message is
able to change both the solution $\Solution$ (on line~\ref{alg:ablns:solution-update}) but also the problem instance $
\ProblemInstance$ (on line~\ref{alg:ablns:problem-update}) itself. Due to the inherent property of LNS being able to
optimize a solution that have become infeasible, LNS is well suited to handle
changing parameters in the problem instance $\ProblemInstance$. Another
property of the message queue is  that the algorithm can run indefinitely and
as opposed to restarting the algorithm you pass messages containing new data
and/or state. This property avoids the time consuming initial convergence as the
algorithm will be found in an optimized state when the solution is perturbed.
Notice that:

\begin{itemize}
    \item The algorithm responds quickly to changes. In each iteration
		line~\ref{alg:ablns:message-loop}. We
		call this a fine-grained response algorithm     
	\item If an improved
		solution is found, it is immediately being pointer swapped
		on line~\ref{alg:ablns:better-solution}
	\item That the optimization occurs in the repair function on
		line~\ref{alg:ablns:repair},
		which inserts work orders using a greedy algorithm.  	
	\item The atomic pointer swapping that is performed on
		line~\ref{alg:ablns:atomic-pointer-swap} when a
		new best solution is found instantaneously makes the solution available to other
		actors/systems. The pointer swapping is crucial as it allows for lock-free
		sharing of state and makes non-hierarchical model setups easier.
\end{itemize}

The AbLNS algorithm will run continuously, either optimizing the current
schedule or updating the schedule with new external input. Since the
effects of a message is considered in every iteration, changing conditions
will immediately get optimized upon. Since a fast response is paramount, exact
optimization algorithms are deemed impractical.

\subsection{Atomic Pointer Swapping and Software Architecture}
Atomic pointer swapping offers a novel approach for managing the best
solution found by the AbLNS and LNS algorithms.
Traditionally, in a metaheuristics such as Large Neighborhood Search (LNS)
setup, the best solution is cloned (deep copy) before the search continues. However, atomic
pointer swapping enables the best solution to be shared across system components
without introducing data races. This approach is essential in multithreaded
environments, as it ensures the integrity of the solution during swaps, this elaborated in 
\citep{herlihy2020art}. Multiprocessor programming has long been considered an art but 
in recent years new tools have emerged to make this type of programming much 
easier most notably \citep{noauthor_rust_nodate}. In
software architecture, scaling is often achieved through vertical scaling
(lengthening the dataflow) and horizontal scaling (increasing throughput)
via message passing. While effective in many domains, this approach poses
challenges for metaheuristics, which rely on rapid state mutations and solution
improvements.

Message passing in metaheuristics can create unpredictable behaviors, especially
with bidirectional communication between components and metaheuristics. The
limitations of message-passing systems in metaheuristics are well-documented,
particularly in the context of parallel and multi-agent metaheuristics
\citep{talbiMetaheuristicsDesignImplementation2009parallelchapter}. These systems often suffer
from excessive message generation, leading to poor scalability, explainability,
extendability, and performance. Key issues include:

\begin{itemize} 
	\item The stochastic nature
		of metaheuristics makes message-passing behavior unpredictable. 
	\item Using message passing often necessitates a
		hierarchical setup, as bidirectional message flows often results in cycles. 
	\item The
		impact of additional types messages is difficult to understand and analyze. 
\end{itemize}

Atomic pointer swapping addresses these challenges by eliminating state
duplication, reducing system complexity, and bypassing the inefficiencies of
message passing. This method provides a streamlined, modular way to integrate
metaheuristics, aligning with modern software architecture principles while
avoiding their typical pitfalls \citep{richards_fundamentals_2020}. It should
be noted that message passing is exceptional at dealing with one-off events that
have varied and sporatic nature such as interfacing with actors, external
systems, or handling constraints that are hard to find the data for or are 
so rarely binding that you would rather handle them through manual intervention.

