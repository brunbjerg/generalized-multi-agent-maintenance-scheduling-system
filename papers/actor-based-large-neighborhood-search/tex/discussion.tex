\section{Discussion}\label{sec:4-discussion}
The maintenance scheduling process effectively solves a complex scheduling
problem by relying in multiple actors. Through the use of actors the scheduling
process handles uncertainty that is difficult to reason about in a single
mathematical model. These  uncertainties are solved through coordination in
time (modelled as $\VarMetaTime$). Each type of actor in the process acts
according to a model each with different levels of aggregation and properties
where each actor has a solid understanding of his own model. In the discussion
interesting aspects of this approach has been divided into three 
sections: \ref{sec:discussion:actors_and_integration}
actors and integration; \ref{sec:discussion:continuous_optimization}
continuous optimization allows asynchronous optimization;  and
\ref{sec:discussion:future_research} future research.

\subsection{Actors \& Integration}\label{sec:discussion:actors_and_integration}
Often in operation research the failure to reliably solve operational
problems in industry are not due to the problems being computationally
intractable \citep{gendreauHandbookMetaheuristics2019}. It is usually a practical
problem of connecting data streams so that the solution approach continually
receives dynamic data to handle changes and then sends the resulting
solutions to the relevant actors (stakeholders), ideally through a relevant interface
\citep{meignanReviewTaxonomyInteractive2015}. The actor-based approach proposed
in this paper makes integration easier by naturally encapsulating a model with a
reliable interface.

\subsection{Continuous Optimization}\label{sec:discussion:continuous_optimization}
With actor-based metaheuristics, the optimization loop can run indefinitely,
optimizing based on the latest available information. This may seem like a
detail as you could argue that you should only ever optimize the schedule
when there is an explicit need for it, but consider the case when you start
adding more than two actors to a scheduling system, then there arises a need
to coordinate people temporally as each will have to run their optimizing
process one after another. This is depicted in figure~\ref{fig:discussion:hierarchical_model_setup}
where the output of one model is used as the input to the next one, leading
to the hierarchical model setup.

\begin{figure}[H]
	\usetikzlibrary{positioning}
\usetikzlibrary{arrows.meta}
\usetikzlibrary{bending}
\definecolor{red}{HTML}{8A3F3A}
\definecolor{yellow}{HTML}{E0BB3C}
\definecolor{blue}{HTML}{4569E0}
\definecolor{green}{HTML}{17E561}
\definecolor{other}{HTML}{6A939E}

% DTU Colors
\definecolor{dtu-corporate-red}{HTML}{990000}
\definecolor{dtu-white}{HTML}{ffffff}
\definecolor{dtu-black}{HTML}{000000}
\definecolor{dtu-blue}{HTML}{2F3EEA}
\definecolor{dtu-bright-green}{HTML}{1FD082}
\definecolor{dtu-navy-blue}{HTML}{030F4F}
\definecolor{dtu-yellow}{HTML}{F6D04D}
\definecolor{dtu-orange}{HTML}{FC7634}
\definecolor{dtu-pink}{HTML}{F7BBB1}
\definecolor{dtu-grey}{HTML}{DADADA}
\definecolor{dtu-red}{HTML}{E83F48}
\definecolor{dtu-green}{HTML}{008835}
\definecolor{dtu-purple}{HTML}{79238E}


\newlength{\basisb}
\setlength{\basisb}{0.4cm}

\centering
\begin{tikzpicture}[line width=0.0\basisb]
    \draw (2.0\basisb,4.0\basisb) 
		node[rotate=90, minimum height=3\basisb,fill=dtu-blue,minimum width=8\basisb,rounded corners=0.1\basisb] 
			(Dynamic Data) {Dynamic Data};

    \draw (8.0\basisb,7.0\basisb) 
		node[minimum height=2\basisb,fill=dtu-red,minimum width=6\basisb,rounded corners=0.1\basisb] 
			(Scheduler) {Scheduler};
    \draw (14.0\basisb,4.0\basisb) 
		node[minimum height=2\basisb,fill=dtu-red,minimum width=6\basisb,rounded corners=0.1\basisb] 
			(Supervisor) {Supervisor};
    \draw (20.0\basisb,1.0\basisb) 
		node[minimum height=2\basisb,fill=dtu-red,minimum width=6\basisb,rounded corners=0.1\basisb] 
			(Technician) {Technician};

    \draw (26.0\basisb,7.0\basisb) 
		node[minimum height=2\basisb,fill=dtu-yellow,minimum width=3\basisb,rounded corners=0.1\basisb] 
			(UserInterface1) {UI};
    \draw (26.0\basisb,4.0\basisb) 
		node[minimum height=2\basisb,fill=dtu-yellow,minimum width=3\basisb,rounded corners=0.1\basisb] 
			(UserInterface2) {UI};
    \draw (26.0\basisb,1.0\basisb) 
		node[minimum height=2\basisb,fill=dtu-yellow,minimum width=3\basisb,rounded corners=0.1\basisb] 
			(UserInterface3) {UI};

	\draw[<->, line width=0.1\basisb,color=dtu-green] (5.0\basisb, -1\basisb) -- (23.0\basisb, -1\basisb);
	\draw (14.0\basisb, -2.0\basisb) node {Significant Amount of Time};

	\draw[->,>=Triangle, thick, line width=0.1\basisb, color=dtu-corporate-red] (Scheduler) to[out=0, in=90,looseness=1.5]  (Supervisor);
	\draw[->,>=Triangle, thick, line width=0.1\basisb, color=dtu-corporate-red] (Supervisor) to[out=0, in=90,looseness=1.5] (Technician);
	\draw[->,>=Triangle, thick, line width=0.1\basisb] (Dynamic Data.south) ++(0\basisb,3.0\basisb) to[out=0, in=180,looseness=1.0] (Scheduler);
	\draw[->,>=Triangle, thick, line width=0.1\basisb] (Dynamic Data.south) to[out=0, in=180,looseness=1.0] (Supervisor);
	\draw[->,>=Triangle, thick, line width=0.1\basisb] (Dynamic Data.south) ++(0\basisb,-3.0\basisb) to[out=0, in=180,looseness=1.0] (Technician.west);
	\draw[<-,>=Triangle, thick, line width=0.1\basisb] (UserInterface1) to[out=180, in=0,looseness=1.0] (Scheduler);
	\draw[<-,>=Triangle, thick, line width=0.1\basisb] (UserInterface2) to[out=180, in=0,looseness=1.0] (Supervisor);
	\draw[<-,>=Triangle, thick, line width=0.1\basisb] (UserInterface3) to[out=180, in=0,looseness=1.0] (Technician);
	% \draw[<->, thick, line width=0.1\basisb] (Scheduler) -- (UserInterface);
	\begin{scope}[shift={(7,0)}] % Adjust shift to position the legend
    % Legend box
    % Legend lines and text
	    \draw[thick, line width=0.1\basisb] (-1.5,2.4) node[right, font=\footnotesize, align=center] {Solution\\Transfer};
	    \draw[thick, line width=0.1\basisb] (1.0,1.2) node[right, font=\footnotesize, align=center] {Solution\\Transfer};
	\end{scope}
\end{tikzpicture}

	\label{fig:discussion:hierarchical_model_setup}
	\caption{Effects of using hierarchical models setups in human-guided search metaheuristics.
	Due to the dependent nature of each metaheuristic it becomes crucial that the running of 
	the metaheuristics are well coordinated between the actors.}
\end{figure}

In practice there are multiple problems with using a hierarchical setup.
Usually the biggest one is that the information and knowledge needed to 
execute a feasible schedule is usually found in the lower levels of the 
hierarchicy. The operational setting, where the
technicians are working, is usually so complex that it not feasible to 
centralize the knowledge that is required to create and execute a 
schedule. Figure~\ref{fig:discussion:asynchronous_setup}
shows the kind of non-hierarchical setup that an actor-based approach 
allows for.

\begin{figure}[H]
	\usetikzlibrary{positioning}
% \usetikzlibrary{arrows.meta}
\usetikzlibrary{bending}
\usetikzlibrary{backgrounds}
\definecolor{red}{HTML}{8A3F3A}
\definecolor{yellow}{HTML}{E0BB3C}
\definecolor{blue}{HTML}{4569E0}
\definecolor{green}{HTML}{17E561}
\definecolor{other}{HTML}{6A939E}

% DTU Colors
\definecolor{dtu-corporate-red}{HTML}{990000}
\definecolor{dtu-white}{HTML}{ffffff}
\definecolor{dtu-black}{HTML}{000000}
\definecolor{dtu-blue}{HTML}{2F3EEA}
\definecolor{dtu-bright-green}{HTML}{1FD082}
\definecolor{dtu-navy-blue}{HTML}{030F4F}
\definecolor{dtu-yellow}{HTML}{F6D04D}
\definecolor{dtu-orange}{HTML}{FC7634}
\definecolor{dtu-pink}{HTML}{F7BBB1}
\definecolor{dtu-grey}{HTML}{DADADA}
\definecolor{dtu-red}{HTML}{E83F48}
\definecolor{dtu-green}{HTML}{008835}
\definecolor{dtu-purple}{HTML}{79238E}


\newlength{\basisc}
\setlength{\basisc}{0.5cm}

\centering
\begin{tikzpicture}[line width=0.0\basisc]
    \draw (4.0\basisc,4.0\basisc) 
		node[rotate=90, minimum height=3\basisc,fill=dtu-blue,minimum width=8\basisc,rounded corners=0.1\basisc] 
			(Dynamic Data) {Dynamic Data};

			

    \draw (8.0\basisc,10.0\basisc) 
		node[minimum height=2\basisc,fill=dtu-yellow,minimum width=3\basisc,rounded corners=0.1\basisc] 
			(UserInterface1) {UI};
    \draw (12.0\basisc,10.0\basisc) 
		node[minimum height=2\basisc,fill=dtu-yellow,minimum width=3\basisc,rounded corners=0.1\basisc] 
			(UserInterface2) {UI};
    \draw (16.0\basisc,10.0\basisc) 
		node[minimum height=2\basisc,fill=dtu-yellow,minimum width=3\basisc,rounded corners=0.1\basisc] 
			(UserInterface3) {UI};
    \draw (12.0\basisc,7.0\basisc) 
		node[minimum height=2\basisc,fill=dtu-red,minimum width=11\basisc,rounded corners=0.1\basisc] 
			(Scheduler) {Scheduler};
    \draw (12.0\basisc,4.0\basisc) 
		node[minimum height=2\basisc,fill=dtu-red,minimum width=11\basisc,rounded corners=0.1\basisc] 
			(Supervisor) {Supervisor};
    \draw (12.0\basisc,1.0\basisc) 
		node[minimum height=2\basisc,fill=dtu-red,minimum width=11\basisc,rounded corners=0.1\basisc] 
			(Technician) {Technician};


	\begin{pgfonlayer}{background}
		\draw[<->, thick, line width=0.1\basisc] (UserInterface1) to[out=-90, in=90,looseness=1.0] ++(0\basisc,-2.0\basisc)(Scheduler);
		\draw[<->, thick, line width=0.1\basisc] (UserInterface2) to[out=-90, in=90,looseness=1.0] (Supervisor);
		\draw[<->, thick, line width=0.1\basisc] (UserInterface3) to[out=-90, in=90,looseness=1.0] ++(0\basisc,-8.0\basisc)(Technician);

	\end{pgfonlayer}

	\draw[->, line width=0.1\basisc,color=dtu-green] (6.5\basisc, -1\basisc) -- (17.5\basisc, -1\basisc);
	\draw (12.0\basisc, -2.0\basisc) node {Running Continuously};

	\draw[<->, thick, line width=0.1\basisc, color=dtu-corporate-red] (Scheduler)++(-3\basisc, -1.0\basisc) to[out=-90, in=90,looseness=1.0]  ++(0\basisc, -1.0\basisc)(Supervisor);
	\draw[<->, thick, line width=0.1\basisc, color=dtu-corporate-red] (Scheduler)++(-2\basisc, -1.0\basisc) to[out=-90, in=90,looseness=1.0]  ++(0\basisc, -1.0\basisc)(Supervisor);
	\draw[<->, thick, line width=0.1\basisc, color=dtu-corporate-red] (Scheduler)++(2\basisc, -1.0\basisc) to[out=-90, in=90,looseness=1.0]  ++(0\basisc, -1.0\basisc)(Supervisor);

	\draw[<->, thick, line width=0.1\basisc, color=dtu-corporate-red] (Supervisor)++(-4\basisc, -1.0\basisc) to[out=-90, in=90,looseness=1.0] ++(0\basisc, -1.0\basisc)(Technician);
	\draw[<->, thick, line width=0.1\basisc, color=dtu-corporate-red] (Supervisor)++(-1\basisc, -1.0\basisc) to[out=-90, in=90,looseness=1.0] ++(0\basisc, -1.0\basisc)(Technician);
	\draw[<->, thick, line width=0.1\basisc, color=dtu-corporate-red] (Supervisor)++(3\basisc, -1.0\basisc) to[out=-90, in=90,looseness=1.0] ++(0\basisc, -1.0\basisc)(Technician);

	\draw[<->, thick, line width=0.1\basisc] (Dynamic Data.south) ++(0\basisc,3.0\basisc) to[out=0, in=180,looseness=1.0] (Scheduler);
	\draw[<->, thick, line width=0.1\basisc] (Dynamic Data.south) to[out=0, in=180,looseness=1.0] (Supervisor);
	\draw[<->, thick, line width=0.1\basisc] (Dynamic Data.south) ++(0\basisc,-3.0\basisc) to[out=0, in=180,looseness=1.0] (Technician.west);

	% \draw[<->, thick, line width=0.1\basisc] (Scheduler) -- (UserInterface);
\end{tikzpicture}

	\caption{Asynchronous model setup where each metaheuristic runs in perpetuity. In this setup
	there is no need to coordinate actors to run the metaheuristics. Each actor in the 
	scheduling process will always have the solutions of the other actors available to 
	him to guide his own search.}
	\label{fig:discussion:asynchronous_setup}
\end{figure}

When the optimization approach optimize continuously it enables tight
integration between the different model implementations. Instead of running
models to completion you simply handle changes in model parameters, model
solutions, user inputs, and in the dynamic data source as they occur opposed to
restarting the metaheuristics.

\subsection{Future Research}
\label{sec:discussion:future_research}
The future research direction is to demonstrate that
the actor-based approach described here can be used to model and optimize 
multi-actor/multi-level scheduling processes. 
Figures~\ref{fig:ordinator-hexagon:persistence},
~\ref{fig:ordinator-hexagon:atomicpointerswap},
~\ref{fig:ordinator-hexagon:metaheuristics},
~\ref{fig:ordinator-hexagon:orchestrator},
~\ref{fig:ordinator-hexagon:userinterfaces}
show a larger scale setup of the actor-based approach which is being developed with Total Energies.
Here figure~\ref{fig:ordinator-hexagon:metaheuristics}
shows the metaheuristics of scheduling system architecture where each of the actors run an AbLNS
and that each metaheuristic will share its solutions with the other
metaheuristics through atomic pointer swapping shown in figure~\ref{fig:ordinator-hexagon:atomicpointerswap}.
Communicate with the end-user
through userinterfaces and message passing as shown in figure~\ref{fig:ordinator-hexagon:userinterfaces}, 
integrate with a persistent storage through mutex
locks as shown in figure~\ref{fig:ordinator-hexagon:persistence}, 
and the lifecycle of each of the metaheuristics will be controlled by
the orchestrator also through message passing as shown in figure~\ref{fig:ordinator-hexagon:orchestrator}. 

% \begin{figure}[H]
% 	\centering
% 	\usetikzlibrary {positioning}

\definecolor{red}{HTML}{8A3F3A}
\definecolor{yellow}{HTML}{E0BB3C}
\definecolor{blue}{HTML}{4569E0}
\definecolor{green}{HTML}{17E561}
\definecolor{other}{HTML}{6A939E}

% DTU Colors
\definecolor{dtu-corporate-red}{HTML}{990000}
\definecolor{dtu-white}{HTML}{ffffff}
\definecolor{dtu-black}{HTML}{000000}
\definecolor{dtu-blue}{HTML}{2F3EEA}
\definecolor{dtu-bright-green}{HTML}{1FD082}
\definecolor{dtu-navy-blue}{HTML}{030F4F}
\definecolor{dtu-yellow}{HTML}{F6D04D}
\definecolor{dtu-orange}{HTML}{FC7634}
\definecolor{dtu-pink}{HTML}{F7BBB1}
\definecolor{dtu-grey}{HTML}{DADADA}
\definecolor{dtu-red}{HTML}{E83F48}
\definecolor{dtu-green}{HTML}{008835}
\definecolor{dtu-purple}{HTML}{79238E}


\newcommand{\ModelColor}{dtu-red}
\newcommand{\UserInterfaceColor}{dtu-yellow}
\newcommand{\PersistenceColor}{dtu-blue}
\newcommand{\PointerSwapColor}{dtu-green}
\newcommand{\OrchestratorColor}{dtu-bright-green}

\newcommand{\basisinput}{4cm}  % Default value if not set by /graph/basis

\pgfkeys{
	/graph/.is family, /graph,
	default/.style = {
		show_shared_pointer = false,
		show_orchestrator = false,
		show_persistence = false,
		show_user_interface = false,
		basisinput/.estore in = \basisinput,
	},
	show_shared_pointer/.estore in = \ShowSharedSolutionCommunication,
	show_orchestrator/.estore in = \ShowOrchestratorCommunication,
	show_persistence/.estore in = \ShowPersistenceCommunication,
	show_user_interface/.estore in = \ShowUserInterfaceCommunication,
	basisinput/.estore in = \basisinput,
}

\newlength{\basis}
\tikzset{
  basis/.code={\setlength{\basis}{\basisinput}}, % TikZ assignment code
  basis/.default=3cm,                   % Provide a default (\b@sis is undefined/unassigned)
  basis,                                % Set initial Value (\b@sis is defined/assigned)
}

\newcommand{\drawOrdinatorArchitecture}[1]{
	\pgfkeys{/graph, default, #1}
	\setlength{\basis}{\basisinput}
	\begin{tikzpicture}[scale=0.75, line width=0.05\basis]

		\ifthenelse{\equal{\ShowOrchestratorCommunication}{true}}{
			\draw[color=other,-, ultra thick] (Strategic) -- (Orchestrator);
			\draw[color=other,-, ultra thick] (Tactical) -- (Orchestrator);
			\draw[color=other,-, ultra thick] (Supervisor) -- (Orchestrator);
			\draw[color=other,-, ultra thick] (Operational_1) -- (Orchestrator);
			\draw[color=other,-, ultra thick] (Operational_2) -- (Orchestrator);
			\draw[color=other,-, ultra thick] (Operational_3) -- (Orchestrator);
		}{}
		% \draw[help lines] (0\basis, 0\basis) grid (10\basis, 8\basis);
		\draw (5\basis,4\basis) node[minimum height=5\basis,minimum width=7.0\basis,rounded corners=0.1\basis] {};

	    \draw[draw=black] (4.125\basis,4.0\basis) node[opacity=0.5, minimum height=3.5\basis,minimum width=6.25\basis,rounded corners=0.1\basis,fill=\PointerSwapColor] {} ;
	    \draw (2.5\basis,5.5\basis) node[minimum height=1\basis,minimum width=1\basis,fill=\ModelColor,rounded corners=0.1\basis] (Strategic) {Stra};
	    \draw (5.0\basis,4.0\basis) node[minimum height=1\basis,minimum width=1\basis,fill=\ModelColor,rounded corners=0.1\basis] (Supervisor) {Sup};
		\draw (7.5\basis,5.5\basis) node[minimum height=1\basis,minimum width=1\basis,fill=\ModelColor,rounded corners=0.1\basis] (Tactical) {Tac};

		\draw (2.5\basis,2.5\basis) node[minimum height=1\basis,minimum width=1\basis,fill=\ModelColor,rounded corners=0.1\basis] (Operational_1) {$O_{1}$};
		\draw (5.0\basis,2.5\basis) node[minimum height=1\basis,minimum width=1\basis,fill=\ModelColor,rounded corners=0.1\basis] (Operational_2) {$O_{2}$};
		\draw (7.5\basis,2.5\basis) node[minimum height=1\basis,minimum width=1\basis,fill=\ModelColor,rounded corners=0.1\basis,rounded corners=0.1\basis] (Operational_3) {$O_{3}$};
	
		\draw (Strategic) edge (Tactical);
		\draw (Strategic) edge (Tactical);
		\draw (5\basis,5.5\basis) edge (Supervisor);
		\draw (Supervisor) -- (2.5\basis,4.0\basis) -- (Operational_1);
		\draw (Supervisor) edge (Operational_2);
		\draw (Supervisor) -- (7.5\basis,4.0\basis) -- (Operational_3);
		\draw (5.0\basis,0.5\basis)   node[minimum height=1\basis,minimum width=5.0\basis,                fill=\PersistenceColor,rounded corners=0.1\basis] {persistence};
		\draw (5.0\basis,7.5\basis)   node[minimum height=1\basis,minimum width=5.0\basis,                fill=\OrchestratorColor,rounded corners=0.1\basis] (Orchestrator) {Orchestrator};
		\draw (0.5\basis,4.0\basis)   node[rotate=90, minimum height=1.0\basis, minimum width=3.5\basis,  fill=\PointerSwapColor,rounded corners=0.1\basis] {decision variables};
		\draw (9.5\basis,5.75\basis)  node[rotate=90, minimum height=1.0\basis, minimum width=1.0\basis,  fill=\UserInterfaceColor,rounded corners=0.1\basis] {UI};
		\draw (9.5\basis,4.0\basis)   node[rotate=90, minimum height=1.0\basis, minimum width=1.0\basis,  fill=\UserInterfaceColor,rounded corners=0.1\basis] {UI};
		\draw (9.5\basis,2.25\basis)  node[rotate=90, minimum height=1.0\basis, minimum width=1.0\basis,  fill=\UserInterfaceColor,rounded corners=0.1\basis] {UI};

		% Legend
		\begin{scope}[shift={(11.0\basis,5.7\basis)}]
			\node at (-0.25\basis,1\basis) [right] {};
			\draw[color=\OrchestratorColor,fill,rounded corners=0.1\basis] (0\basis,0.0\basis)   rectangle (0.5\basis, 0.5\basis);
			\node[anchor=west] at (0.5\basis, 0.25\basis) { Managing metaheuristic lifetimes };
			\draw[color=\PointerSwapColor,fill,rounded corners=0.1\basis] (0\basis,-1.0\basis)   rectangle(0.5\basis, -0.5\basis); 
			\node[anchor=west] at (0.5\basis, -0.75\basis) { Solution sharing (Atomic pointer swaps) };
			\draw[color=\ModelColor,fill,rounded corners=0.1\basis] (0\basis,-2.0\basis)         rectangle(0.5\basis, -1.5\basis); 
			\node[anchor=west] at (0.5\basis, -1.75\basis) { Metaheurics (Mathematical Models) };
			\draw[color=\PersistenceColor,fill,rounded corners=0.1\basis] (0\basis,-3.0\basis)   rectangle(0.5\basis, -2.5\basis); 
			\node[anchor=west] at (0.5\basis, -2.75\basis) { Data storage (Memory locks)};
			\draw[color=\UserInterfaceColor,fill,rounded corners=0.1\basis] (0\basis,-4.0\basis) rectangle(0.5\basis, -3.5\basis); 
			\node[anchor=west] at (0.5\basis, -3.75\basis) { Message passing (Memory channels) };

		\end{scope}
		\ifthenelse{\equal{\ShowSharedSolutionCommunication}{true}}{
			\draw[->, thick] (Strategic) -- (Orchestrator);
		}{}
		\ifthenelse{\equal{\ShowUserInterfaceCommunication}{true}}{
			\draw[->, thick] (Strategic) -- (Orchestrator);
		}{}
		\ifthenelse{\equal{\ShowPersistenceCommunication}{true}}{
			\draw[->, thick] (Strategic) -- (Orchestrator);
		}{}
		

	\end{tikzpicture}
}


% 	\drawOrdinatorArchitecture{basisinput=1cm}
% 	\label{fig:ordinator-architecture}
% \end{figure}
\begin{figure}[H]
	\centering
	\usetikzlibrary {positioning}
\newcommand{\drawHexagon}[6][draw=black]{
	\draw[#1, fill=#4] (#2,#3) ++(30:#6) -- ++(90:#6) -- ++(150:#6) -- ++(210:#6) -- ++(270:#6) -- ++(330:#6) -- cycle;
	\node[align=center] at (#2,#3+2) {#5};
}

\newif\ifpersistencelayer
\newif\ifatomicpointerswaplayer
\newif\ifmetaheuristicslayer
\newif\ifuserinterfacelayer
\newif\iforchestratorlayer
\newif\ifsimplifiedlayer

\pgfkeys{
	/hexagon/.is family, /hexagon,
	default/.style = {
		persistence=false,
		atomicpointerswap=false,
		metaheuristics=false,
		orchestrator=false,
		userinterface=false,
		simplified=false,
	},
	persistence/.is if=persistencelayer,
	atomicpointerswap/.is if=atomicpointerswaplayer,
	metaheuristics/.is if=metaheuristicslayer,
	orchestrator/.is if=orchestratorlayer,
	userinterface/.is if=userinterfacelayer,
	simplified/.is if=simplifiedlayer,
}
\newcommand{\drawModelSetupHexagon}[1][]{
	\pgfkeys{/hexagon, default, #1}

	\begin{tikzpicture}[font=\footnotesize, scale=0.5, line width=1.05]
	

	\ifpersistencelayer
		\drawHexagon[draw=none]{ 2                      }{ 2}{dtu-blue}{}{2}
		\drawHexagon[draw=none]{{6 - 2 * (2 - sqrt(3)) }}{ 2}{dtu-blue}{}{2}
		\drawHexagon[draw=none]{{4 - 1 * (2 - sqrt(3)) }}{-1}{dtu-blue}{Persistence}{2}
		\drawHexagon[draw=none]{{0 + 1 * (2 - sqrt(3)) }}{-1}{dtu-blue}{}{2}
		\drawHexagon[draw=none]{{8 - 3 * (2 - sqrt(3)) }}{-1}{dtu-blue}{}{2}

		\drawHexagon[draw=none]{{2 - 0 * (2 - sqrt(3)) }}{-4}{dtu-blue}{}{2}
		\drawHexagon[draw=none]{{6 - 2 * (2 - sqrt(3)) }}{-4}{dtu-blue}{}{2}

		\drawHexagon[draw=none]{{10 - 4 * (2 - sqrt(3)) }}{-4}{dtu-blue}{}{2}
		\drawHexagon[draw=none]{{-2 + 2 * (2 - sqrt(3)) }}{-4}{dtu-blue}{}{2}

		\drawHexagon[draw=none]{{12 - 5 * (2 - sqrt(3)) }}{-1}{dtu-blue}{}{2}
		\drawHexagon[draw=none]{{-4 + 3 * (2 - sqrt(3)) }}{-1}{dtu-blue}{}{2}
		% Legend for each layer
		\drawHexagon{{14.0  }}{+3.0}{dtu-blue}{}{0.75}
		\node[align=right, anchor=west] at ({15.0}, +3.75) {Persistence};
		\drawHexagon{{14.0  }}{+1.5}{dtu-white}{}{0.75}
		\node[align=right, anchor=west] at ({15.0}, +2.25) {Atomic Pointer};
		\drawHexagon{{14.0  }}{+0.0}{dtu-white}{}{0.75}
		\node[align=right, anchor=west] at ({15.0}, +0.75) {Metaheuristics};
		\drawHexagon{{14.0  }}{-1.5}{dtu-white}{}{0.75}
		\node[align=right, anchor=west] at ({15.0}, -0.75) {Orchestration};
		\drawHexagon{{14.0  }}{-3.0}{dtu-white}{}{0.75}
		\node[align=right, anchor=west] at ({15.0}, -2.25) {User interfaces};
	\fi


	\ifatomicpointerswaplayer
		\drawHexagon[]{ 2                      }{ 2}{dtu-green}{Shared\\solution\\pointer}{2}
		\drawHexagon[]{{6 - 2 * (2 - sqrt(3)) }}{ 2}{dtu-green}{Shared\\solution\\pointer}{2}
		\drawHexagon[]{{4 - 1 * (2 - sqrt(3)) }}{-1}{dtu-green}{Shared\\solution\\pointer}{2}
		\drawHexagon[]{{0 + 1 * (2 - sqrt(3)) }}{-1}{dtu-green}{Shared\\solution\\pointer}{2}
		\drawHexagon[]{{8 - 3 * (2 - sqrt(3)) }}{-1}{dtu-green}{Shared\\solution\\pointer}{2}

		\drawHexagon[]{{2 - 0 * (2 - sqrt(3)) }}{-4}{dtu-green}{Shared\\solution\\pointer}{2}
		\drawHexagon[]{{6 - 2 * (2 - sqrt(3)) }}{-4}{dtu-green}{Shared\\solution\\pointer}{2}

		\drawHexagon[]{{10 - 4 * (2 - sqrt(3)) }}{-4}{dtu-green}{Shared\\solution\\pointer}{2}
		\drawHexagon[]{{-2 + 2 * (2 - sqrt(3)) }}{-4}{dtu-green}{Shared\\solution\\pointer}{2}

		\drawHexagon[]{{12 - 5 * (2 - sqrt(3)) }}{-1}{dtu-green}{Shared\\solution\\pointer}{2}
		\drawHexagon[]{{-4 + 3 * (2 - sqrt(3)) }}{-1}{dtu-green}{Shared\\solution\\pointer}{2}
		% Legend for each layer
		\drawHexagon{{14.0  }}{+3.0}{dtu-white}{}{0.75}
		\node[align=right, anchor=west] at ({15.0}, +3.75) {Persistence};
		\drawHexagon{{14.0  }}{+1.5}{dtu-green}{}{0.75}
		\node[align=right, anchor=west] at ({15.0}, +2.25) {Atomic Pointer};
		\drawHexagon{{14.0  }}{+0.0}{dtu-white}{}{0.75}
		\node[align=right, anchor=west] at ({15.0}, +0.75) {Metaheuristics};
		\drawHexagon{{14.0  }}{-1.5}{dtu-white}{}{0.75}
		\node[align=right, anchor=west] at ({15.0}, -0.75) {Orchestration};
		\drawHexagon{{14.0  }}{-3.0}{dtu-white}{}{0.75}
		\node[align=right, anchor=west] at ({15.0}, -2.25) {User interfaces};
	\fi

	\ifsimplifiedlayer

		\node[align=right, anchor=west] at ({-5.5}, +3.75) {};
		\drawHexagon{{+2 + 0 * (2 - sqrt(3)) }}{ 2}{dtu-green}{Scheduler}{2}
		\drawHexagon{{+4 - 1 * (2 - sqrt(3)) }}{-1}{dtu-red}{Supervisor}{2}
		\drawHexagon{{+0 + 1 * (2 - sqrt(3)) }}{-1}{dtu-red}{Supervisor}{2}
		\drawHexagon{{+2 - 0 * (2 - sqrt(3)) }}{-4}{dtu-corporate-red}{Technician}{2}
		\drawHexagon{{+6 - 2 * (2 - sqrt(3)) }}{-4}{dtu-corporate-red}{Technician}{2}
		\drawHexagon{{-2 + 2 * (2 - sqrt(3)) }}{-4}{dtu-corporate-red}{Technician}{2}
		\drawHexagon{{+8 - 3 * (2 - sqrt(3)) }}{-1}{dtu-corporate-red}{Technician}{2}
		\drawHexagon{{-4 + 3 * (2 - sqrt(3)) }}{-1}{dtu-corporate-red}{Technician}{2}

		% Scheduler
		\draw[thin, fill=dtu-yellow] (2, 5) circle (0.35);
		\draw[thin, fill=dtu-purple] (2, 3) circle (0.35);
		% Supervisor 1
		\draw[thin, fill=dtu-yellow] ({+4 - 1 * (2 - sqrt(3)) }, 02) circle (0.35);
		\draw[thin, fill=dtu-purple] ({+4 - 1 * (2 - sqrt(3)) }, -0) circle (0.35);
		% Supervisor 2
		\draw[thin, fill=dtu-yellow] ({+0 + 1 * (2 - sqrt(3)) }, 02) circle (0.35);
		\draw[thin, fill=dtu-purple] ({+0 + 1 * (2 - sqrt(3)) }, -0) circle (0.35);
		% Technician 1
		\draw[thin, fill=dtu-yellow] ({+2 - 0 * (2 - sqrt(3)) }, -1) circle (0.35);
		\draw[thin, fill=dtu-purple] ({+2 - 0 * (2 - sqrt(3)) }, -3) circle (0.35);
		% Technician 2
		\draw[thin, fill=dtu-yellow] ({+6 - 2 * (2 - sqrt(3)) }, -1) circle (0.35);
		\draw[thin, fill=dtu-purple] ({+6 - 2 * (2 - sqrt(3)) }, -3) circle (0.35);
		% Technician 3
		\draw[thin, fill=dtu-yellow] ({-2 + 2 * (2 - sqrt(3)) }, -1) circle (0.35);
		\draw[thin, fill=dtu-purple] ({-2 + 2 * (2 - sqrt(3)) }, -3) circle (0.35);
		% Technician 4
		\draw[thin, fill=dtu-yellow] ({+8 - 3 * (2 - sqrt(3)) }, 02) circle (0.35);
		\draw[thin, fill=dtu-purple] ({+8 - 3 * (2 - sqrt(3)) }, -0) circle (0.35);
		% Technician 5
		\draw[thin, fill=dtu-yellow] ({-4 + 3 * (2 - sqrt(3)) }, 02) circle (0.35);
		\draw[thin, fill=dtu-purple] ({-4 + 3 * (2 - sqrt(3)) }, -0) circle (0.35);

		% Legend for each layer
		\node[align=right, anchor=west] at ({12.0}, +3.75) {Atomic Pointer};
		\draw[fill=dtu-purple] (11.0,  +3.75) circle (0.5);

		\node[align=right, anchor=west] at ({12.0}, +2.25) {Scheduler Metaheuristic};
		\drawHexagon{{11.0  }}{+1.75}{dtu-green}{}{0.5}
		\node[align=right, anchor=west] at ({12.0}, +0.75) {Supervisor Metaheuristic};
		\drawHexagon{{11.0  }}{+0.25}{dtu-red}{}{0.5}
		\node[align=right, anchor=west] at ({12.0}, -0.75) {Technician Metaheuristic};
		\drawHexagon{{11.0  }}{-1.25}{dtu-corporate-red}{}{0.5}
		\node[align=right, anchor=west] at ({12.0}, -2.25) {User interfaces (Message Passing)};
		\draw[fill=dtu-yellow] (11.0, -2.25) circle (0.5);
	\fi

	\ifmetaheuristicslayer
		\drawHexagon{ 2                      }{ 2}{dtu-blue}{Strategic}{2}
		\drawHexagon{{6 - 2 * (2 - sqrt(3)) }}{ 2}{dtu-green}{Tactical}{2}
		\drawHexagon{{4 - 1 * (2 - sqrt(3)) }}{-1}{dtu-red}{Supervisor}{2}
		\drawHexagon{{0 + 1 * (2 - sqrt(3)) }}{-1}{dtu-red}{Supervisor}{2}
		\drawHexagon{{8 - 3 * (2 - sqrt(3)) }}{-1}{dtu-red}{Supervisor}{2}

		\drawHexagon{{2 - 0 * (2 - sqrt(3)) }}{-4}{dtu-corporate-red}{Technician}{2}
		\drawHexagon{{6 - 2 * (2 - sqrt(3)) }}{-4}{dtu-corporate-red}{Technician}{2}

		\drawHexagon{{10 - 4 * (2 - sqrt(3)) }}{-4}{dtu-corporate-red}{Technician}{2}
		\drawHexagon{{-2 + 2 * (2 - sqrt(3)) }}{-4}{dtu-corporate-red}{Technician}{2}

		\drawHexagon{{12 - 5 * (2 - sqrt(3)) }}{-1}{dtu-corporate-red}{Technician}{2}
		\drawHexagon{{-4 + 3 * (2 - sqrt(3)) }}{-1}{dtu-corporate-red}{Technician}{2}

		% Legend for each layer
		\drawHexagon{{14.0  }}{+3.0}{dtu-white}{}{0.75}
		\node[align=right, anchor=west] at ({15.0}, +3.75) {Persistence};
		\drawHexagon{{14.0  }}{+1.5}{dtu-white}{}{0.75}
		\node[align=right, anchor=west] at ({15.0}, +2.25) {Atomic Pointer};
		\drawHexagon{{14.0  }}{+0.0}{dtu-corporate-red}{}{0.75}
		\node[align=right, anchor=west] at ({15.0}, +0.75) {Metaheuristics};
		\drawHexagon{{14.0  }}{-1.5}{dtu-white}{}{0.75}
		\node[align=right, anchor=west] at ({15.0}, -0.75) {Orchestration};
		\drawHexagon{{14.0  }}{-3.0}{dtu-white}{}{0.75}
		\node[align=right, anchor=west] at ({15.0}, -2.25) {User interfaces};
	\fi

	\iforchestratorlayer
		\drawHexagon{ 2                      }{ 2}{dtu-orange}{}{2}
		\drawHexagon{{6 - 2 * (2 - sqrt(3)) }}{ 2}{dtu-orange}{}{2}
		\drawHexagon{{4 - 1 * (2 - sqrt(3)) }}{-1}{dtu-orange}{Orche-\\strator}{2}
		\drawHexagon{{0 + 1 * (2 - sqrt(3)) }}{-1}{dtu-orange}{}{2}
		\drawHexagon{{8 - 3 * (2 - sqrt(3)) }}{-1}{dtu-orange}{}{2}

		\drawHexagon{{2 - 0 * (2 - sqrt(3)) }}{-4}{dtu-orange}{}{2}
		\drawHexagon{{6 - 2 * (2 - sqrt(3)) }}{-4}{dtu-orange}{}{2}

		\drawHexagon{{10 - 4 * (2 - sqrt(3)) }}{-4}{dtu-orange}{}{2}
		\drawHexagon{{-2 + 2 * (2 - sqrt(3)) }}{-4}{dtu-orange}{}{2}

		\drawHexagon{{12 - 5 * (2 - sqrt(3)) }}{-1}{dtu-orange}{}{2}
		\drawHexagon{{-4 + 3 * (2 - sqrt(3)) }}{-1}{dtu-orange}{}{2}
		% Legend for each layer
		\drawHexagon{{14.0  }}{+3.0}{dtu-white}{}{0.75}
		\node[align=right, anchor=west] at ({15.0}, +3.75) {Persistence};
		\drawHexagon{{14.0  }}{+1.5}{dtu-white}{}{0.75}
		\node[align=right, anchor=west] at ({15.0}, +2.25) {Atomic Pointer};
		\drawHexagon{{14.0  }}{+0.0}{dtu-white}{}{0.75}
		\node[align=right, anchor=west] at ({15.0}, +0.75) {Metaheuristics};
		\drawHexagon{{14.0  }}{-1.5}{dtu-orange}{}{0.75}
		\node[align=right, anchor=west] at ({15.0}, -0.75) {Orchestration};
		\drawHexagon{{14.0  }}{-3.0}{dtu-white}{}{0.75}
		\node[align=right, anchor=west] at ({15.0}, -2.25) {User interfaces};
	\fi

	
	\ifuserinterfacelayer
		\drawHexagon{ 2                      }{ 2}{dtu-yellow}{UI}{2}
		\drawHexagon{{6 - 2 * (2 - sqrt(3)) }}{ 2}{dtu-yellow}{UI}{2}
		\drawHexagon{{4 - 1 * (2 - sqrt(3)) }}{-1}{dtu-yellow}{UI}{2}
		\drawHexagon{{0 + 1 * (2 - sqrt(3)) }}{-1}{dtu-yellow}{UI}{2}
		\drawHexagon{{8 - 3 * (2 - sqrt(3)) }}{-1}{dtu-yellow}{UI}{2}

		\drawHexagon{{2 - 0 * (2 - sqrt(3)) }}{-4}{dtu-yellow}{UI}{2}
		\drawHexagon{{6 - 2 * (2 - sqrt(3)) }}{-4}{dtu-yellow}{UI}{2}

		\drawHexagon{{10 - 4 * (2 - sqrt(3)) }}{-4}{dtu-yellow}{UI}{2}
		\drawHexagon{{-2 + 2 * (2 - sqrt(3)) }}{-4}{dtu-yellow}{UI}{2}

		\drawHexagon{{12 - 5 * (2 - sqrt(3)) }}{-1}{dtu-yellow}{UI}{2}
		\drawHexagon{{-4 + 3 * (2 - sqrt(3)) }}{-1}{dtu-yellow}{UI}{2}
		% Legend for each layer
		\drawHexagon{{14.0  }}{+3.0}{dtu-white}{}{0.75}
		\node[align=right, anchor=west] at ({15.0}, +3.75) {Persistence};
		\drawHexagon{{14.0  }}{+1.5}{dtu-white}{}{0.75}
		\node[align=right, anchor=west] at ({15.0}, +2.25) {Atomic Pointer};
		\drawHexagon{{14.0  }}{+0.0}{dtu-white}{}{0.75}
		\node[align=right, anchor=west] at ({15.0}, +0.75) {Metaheuristics};
		\drawHexagon{{14.0  }}{-1.5}{dtu-white}{}{0.75}
		\node[align=right, anchor=west] at ({15.0}, -0.75) {Orchestration};
		\drawHexagon{{14.0  }}{-3.0}{dtu-yellow}{}{0.75}
		\node[align=right, anchor=west] at ({15.0}, -2.25) {User interfaces};
	\fi
	
	\end{tikzpicture}
}

	\resizebox{0.7\textwidth}{!}{
		\drawModelSetupHexagon[persistence=true]
	}
	\caption{
		Overview of the scheduling process when modelled as actors. When LNS is encapsulated 
		is an actor it becomes possible to optimize parts of a large process individually instead of 
		optimizing the scheduling problem globally from a single model implementation.
	}
	\label{fig:ordinator-hexagon:persistence}
\end{figure}
\begin{figure}[H]
	\centering
	\usetikzlibrary {positioning}
\newcommand{\drawHexagon}[6][draw=black]{
	\draw[#1, fill=#4] (#2,#3) ++(30:#6) -- ++(90:#6) -- ++(150:#6) -- ++(210:#6) -- ++(270:#6) -- ++(330:#6) -- cycle;
	\node[align=center] at (#2,#3+2) {#5};
}

\newif\ifpersistencelayer
\newif\ifatomicpointerswaplayer
\newif\ifmetaheuristicslayer
\newif\ifuserinterfacelayer
\newif\iforchestratorlayer
\newif\ifsimplifiedlayer

\pgfkeys{
	/hexagon/.is family, /hexagon,
	default/.style = {
		persistence=false,
		atomicpointerswap=false,
		metaheuristics=false,
		orchestrator=false,
		userinterface=false,
		simplified=false,
	},
	persistence/.is if=persistencelayer,
	atomicpointerswap/.is if=atomicpointerswaplayer,
	metaheuristics/.is if=metaheuristicslayer,
	orchestrator/.is if=orchestratorlayer,
	userinterface/.is if=userinterfacelayer,
	simplified/.is if=simplifiedlayer,
}
\newcommand{\drawModelSetupHexagon}[1][]{
	\pgfkeys{/hexagon, default, #1}

	\begin{tikzpicture}[font=\footnotesize, scale=0.5, line width=1.05]
	

	\ifpersistencelayer
		\drawHexagon[draw=none]{ 2                      }{ 2}{dtu-blue}{}{2}
		\drawHexagon[draw=none]{{6 - 2 * (2 - sqrt(3)) }}{ 2}{dtu-blue}{}{2}
		\drawHexagon[draw=none]{{4 - 1 * (2 - sqrt(3)) }}{-1}{dtu-blue}{Persistence}{2}
		\drawHexagon[draw=none]{{0 + 1 * (2 - sqrt(3)) }}{-1}{dtu-blue}{}{2}
		\drawHexagon[draw=none]{{8 - 3 * (2 - sqrt(3)) }}{-1}{dtu-blue}{}{2}

		\drawHexagon[draw=none]{{2 - 0 * (2 - sqrt(3)) }}{-4}{dtu-blue}{}{2}
		\drawHexagon[draw=none]{{6 - 2 * (2 - sqrt(3)) }}{-4}{dtu-blue}{}{2}

		\drawHexagon[draw=none]{{10 - 4 * (2 - sqrt(3)) }}{-4}{dtu-blue}{}{2}
		\drawHexagon[draw=none]{{-2 + 2 * (2 - sqrt(3)) }}{-4}{dtu-blue}{}{2}

		\drawHexagon[draw=none]{{12 - 5 * (2 - sqrt(3)) }}{-1}{dtu-blue}{}{2}
		\drawHexagon[draw=none]{{-4 + 3 * (2 - sqrt(3)) }}{-1}{dtu-blue}{}{2}
		% Legend for each layer
		\drawHexagon{{14.0  }}{+3.0}{dtu-blue}{}{0.75}
		\node[align=right, anchor=west] at ({15.0}, +3.75) {Persistence};
		\drawHexagon{{14.0  }}{+1.5}{dtu-white}{}{0.75}
		\node[align=right, anchor=west] at ({15.0}, +2.25) {Atomic Pointer};
		\drawHexagon{{14.0  }}{+0.0}{dtu-white}{}{0.75}
		\node[align=right, anchor=west] at ({15.0}, +0.75) {Metaheuristics};
		\drawHexagon{{14.0  }}{-1.5}{dtu-white}{}{0.75}
		\node[align=right, anchor=west] at ({15.0}, -0.75) {Orchestration};
		\drawHexagon{{14.0  }}{-3.0}{dtu-white}{}{0.75}
		\node[align=right, anchor=west] at ({15.0}, -2.25) {User interfaces};
	\fi


	\ifatomicpointerswaplayer
		\drawHexagon[]{ 2                      }{ 2}{dtu-green}{Shared\\solution\\pointer}{2}
		\drawHexagon[]{{6 - 2 * (2 - sqrt(3)) }}{ 2}{dtu-green}{Shared\\solution\\pointer}{2}
		\drawHexagon[]{{4 - 1 * (2 - sqrt(3)) }}{-1}{dtu-green}{Shared\\solution\\pointer}{2}
		\drawHexagon[]{{0 + 1 * (2 - sqrt(3)) }}{-1}{dtu-green}{Shared\\solution\\pointer}{2}
		\drawHexagon[]{{8 - 3 * (2 - sqrt(3)) }}{-1}{dtu-green}{Shared\\solution\\pointer}{2}

		\drawHexagon[]{{2 - 0 * (2 - sqrt(3)) }}{-4}{dtu-green}{Shared\\solution\\pointer}{2}
		\drawHexagon[]{{6 - 2 * (2 - sqrt(3)) }}{-4}{dtu-green}{Shared\\solution\\pointer}{2}

		\drawHexagon[]{{10 - 4 * (2 - sqrt(3)) }}{-4}{dtu-green}{Shared\\solution\\pointer}{2}
		\drawHexagon[]{{-2 + 2 * (2 - sqrt(3)) }}{-4}{dtu-green}{Shared\\solution\\pointer}{2}

		\drawHexagon[]{{12 - 5 * (2 - sqrt(3)) }}{-1}{dtu-green}{Shared\\solution\\pointer}{2}
		\drawHexagon[]{{-4 + 3 * (2 - sqrt(3)) }}{-1}{dtu-green}{Shared\\solution\\pointer}{2}
		% Legend for each layer
		\drawHexagon{{14.0  }}{+3.0}{dtu-white}{}{0.75}
		\node[align=right, anchor=west] at ({15.0}, +3.75) {Persistence};
		\drawHexagon{{14.0  }}{+1.5}{dtu-green}{}{0.75}
		\node[align=right, anchor=west] at ({15.0}, +2.25) {Atomic Pointer};
		\drawHexagon{{14.0  }}{+0.0}{dtu-white}{}{0.75}
		\node[align=right, anchor=west] at ({15.0}, +0.75) {Metaheuristics};
		\drawHexagon{{14.0  }}{-1.5}{dtu-white}{}{0.75}
		\node[align=right, anchor=west] at ({15.0}, -0.75) {Orchestration};
		\drawHexagon{{14.0  }}{-3.0}{dtu-white}{}{0.75}
		\node[align=right, anchor=west] at ({15.0}, -2.25) {User interfaces};
	\fi

	\ifsimplifiedlayer

		\node[align=right, anchor=west] at ({-5.5}, +3.75) {};
		\drawHexagon{{+2 + 0 * (2 - sqrt(3)) }}{ 2}{dtu-green}{Scheduler}{2}
		\drawHexagon{{+4 - 1 * (2 - sqrt(3)) }}{-1}{dtu-red}{Supervisor}{2}
		\drawHexagon{{+0 + 1 * (2 - sqrt(3)) }}{-1}{dtu-red}{Supervisor}{2}
		\drawHexagon{{+2 - 0 * (2 - sqrt(3)) }}{-4}{dtu-corporate-red}{Technician}{2}
		\drawHexagon{{+6 - 2 * (2 - sqrt(3)) }}{-4}{dtu-corporate-red}{Technician}{2}
		\drawHexagon{{-2 + 2 * (2 - sqrt(3)) }}{-4}{dtu-corporate-red}{Technician}{2}
		\drawHexagon{{+8 - 3 * (2 - sqrt(3)) }}{-1}{dtu-corporate-red}{Technician}{2}
		\drawHexagon{{-4 + 3 * (2 - sqrt(3)) }}{-1}{dtu-corporate-red}{Technician}{2}

		% Scheduler
		\draw[thin, fill=dtu-yellow] (2, 5) circle (0.35);
		\draw[thin, fill=dtu-purple] (2, 3) circle (0.35);
		% Supervisor 1
		\draw[thin, fill=dtu-yellow] ({+4 - 1 * (2 - sqrt(3)) }, 02) circle (0.35);
		\draw[thin, fill=dtu-purple] ({+4 - 1 * (2 - sqrt(3)) }, -0) circle (0.35);
		% Supervisor 2
		\draw[thin, fill=dtu-yellow] ({+0 + 1 * (2 - sqrt(3)) }, 02) circle (0.35);
		\draw[thin, fill=dtu-purple] ({+0 + 1 * (2 - sqrt(3)) }, -0) circle (0.35);
		% Technician 1
		\draw[thin, fill=dtu-yellow] ({+2 - 0 * (2 - sqrt(3)) }, -1) circle (0.35);
		\draw[thin, fill=dtu-purple] ({+2 - 0 * (2 - sqrt(3)) }, -3) circle (0.35);
		% Technician 2
		\draw[thin, fill=dtu-yellow] ({+6 - 2 * (2 - sqrt(3)) }, -1) circle (0.35);
		\draw[thin, fill=dtu-purple] ({+6 - 2 * (2 - sqrt(3)) }, -3) circle (0.35);
		% Technician 3
		\draw[thin, fill=dtu-yellow] ({-2 + 2 * (2 - sqrt(3)) }, -1) circle (0.35);
		\draw[thin, fill=dtu-purple] ({-2 + 2 * (2 - sqrt(3)) }, -3) circle (0.35);
		% Technician 4
		\draw[thin, fill=dtu-yellow] ({+8 - 3 * (2 - sqrt(3)) }, 02) circle (0.35);
		\draw[thin, fill=dtu-purple] ({+8 - 3 * (2 - sqrt(3)) }, -0) circle (0.35);
		% Technician 5
		\draw[thin, fill=dtu-yellow] ({-4 + 3 * (2 - sqrt(3)) }, 02) circle (0.35);
		\draw[thin, fill=dtu-purple] ({-4 + 3 * (2 - sqrt(3)) }, -0) circle (0.35);

		% Legend for each layer
		\node[align=right, anchor=west] at ({12.0}, +3.75) {Atomic Pointer};
		\draw[fill=dtu-purple] (11.0,  +3.75) circle (0.5);

		\node[align=right, anchor=west] at ({12.0}, +2.25) {Scheduler Metaheuristic};
		\drawHexagon{{11.0  }}{+1.75}{dtu-green}{}{0.5}
		\node[align=right, anchor=west] at ({12.0}, +0.75) {Supervisor Metaheuristic};
		\drawHexagon{{11.0  }}{+0.25}{dtu-red}{}{0.5}
		\node[align=right, anchor=west] at ({12.0}, -0.75) {Technician Metaheuristic};
		\drawHexagon{{11.0  }}{-1.25}{dtu-corporate-red}{}{0.5}
		\node[align=right, anchor=west] at ({12.0}, -2.25) {User interfaces (Message Passing)};
		\draw[fill=dtu-yellow] (11.0, -2.25) circle (0.5);
	\fi

	\ifmetaheuristicslayer
		\drawHexagon{ 2                      }{ 2}{dtu-blue}{Strategic}{2}
		\drawHexagon{{6 - 2 * (2 - sqrt(3)) }}{ 2}{dtu-green}{Tactical}{2}
		\drawHexagon{{4 - 1 * (2 - sqrt(3)) }}{-1}{dtu-red}{Supervisor}{2}
		\drawHexagon{{0 + 1 * (2 - sqrt(3)) }}{-1}{dtu-red}{Supervisor}{2}
		\drawHexagon{{8 - 3 * (2 - sqrt(3)) }}{-1}{dtu-red}{Supervisor}{2}

		\drawHexagon{{2 - 0 * (2 - sqrt(3)) }}{-4}{dtu-corporate-red}{Technician}{2}
		\drawHexagon{{6 - 2 * (2 - sqrt(3)) }}{-4}{dtu-corporate-red}{Technician}{2}

		\drawHexagon{{10 - 4 * (2 - sqrt(3)) }}{-4}{dtu-corporate-red}{Technician}{2}
		\drawHexagon{{-2 + 2 * (2 - sqrt(3)) }}{-4}{dtu-corporate-red}{Technician}{2}

		\drawHexagon{{12 - 5 * (2 - sqrt(3)) }}{-1}{dtu-corporate-red}{Technician}{2}
		\drawHexagon{{-4 + 3 * (2 - sqrt(3)) }}{-1}{dtu-corporate-red}{Technician}{2}

		% Legend for each layer
		\drawHexagon{{14.0  }}{+3.0}{dtu-white}{}{0.75}
		\node[align=right, anchor=west] at ({15.0}, +3.75) {Persistence};
		\drawHexagon{{14.0  }}{+1.5}{dtu-white}{}{0.75}
		\node[align=right, anchor=west] at ({15.0}, +2.25) {Atomic Pointer};
		\drawHexagon{{14.0  }}{+0.0}{dtu-corporate-red}{}{0.75}
		\node[align=right, anchor=west] at ({15.0}, +0.75) {Metaheuristics};
		\drawHexagon{{14.0  }}{-1.5}{dtu-white}{}{0.75}
		\node[align=right, anchor=west] at ({15.0}, -0.75) {Orchestration};
		\drawHexagon{{14.0  }}{-3.0}{dtu-white}{}{0.75}
		\node[align=right, anchor=west] at ({15.0}, -2.25) {User interfaces};
	\fi

	\iforchestratorlayer
		\drawHexagon{ 2                      }{ 2}{dtu-orange}{}{2}
		\drawHexagon{{6 - 2 * (2 - sqrt(3)) }}{ 2}{dtu-orange}{}{2}
		\drawHexagon{{4 - 1 * (2 - sqrt(3)) }}{-1}{dtu-orange}{Orche-\\strator}{2}
		\drawHexagon{{0 + 1 * (2 - sqrt(3)) }}{-1}{dtu-orange}{}{2}
		\drawHexagon{{8 - 3 * (2 - sqrt(3)) }}{-1}{dtu-orange}{}{2}

		\drawHexagon{{2 - 0 * (2 - sqrt(3)) }}{-4}{dtu-orange}{}{2}
		\drawHexagon{{6 - 2 * (2 - sqrt(3)) }}{-4}{dtu-orange}{}{2}

		\drawHexagon{{10 - 4 * (2 - sqrt(3)) }}{-4}{dtu-orange}{}{2}
		\drawHexagon{{-2 + 2 * (2 - sqrt(3)) }}{-4}{dtu-orange}{}{2}

		\drawHexagon{{12 - 5 * (2 - sqrt(3)) }}{-1}{dtu-orange}{}{2}
		\drawHexagon{{-4 + 3 * (2 - sqrt(3)) }}{-1}{dtu-orange}{}{2}
		% Legend for each layer
		\drawHexagon{{14.0  }}{+3.0}{dtu-white}{}{0.75}
		\node[align=right, anchor=west] at ({15.0}, +3.75) {Persistence};
		\drawHexagon{{14.0  }}{+1.5}{dtu-white}{}{0.75}
		\node[align=right, anchor=west] at ({15.0}, +2.25) {Atomic Pointer};
		\drawHexagon{{14.0  }}{+0.0}{dtu-white}{}{0.75}
		\node[align=right, anchor=west] at ({15.0}, +0.75) {Metaheuristics};
		\drawHexagon{{14.0  }}{-1.5}{dtu-orange}{}{0.75}
		\node[align=right, anchor=west] at ({15.0}, -0.75) {Orchestration};
		\drawHexagon{{14.0  }}{-3.0}{dtu-white}{}{0.75}
		\node[align=right, anchor=west] at ({15.0}, -2.25) {User interfaces};
	\fi

	
	\ifuserinterfacelayer
		\drawHexagon{ 2                      }{ 2}{dtu-yellow}{UI}{2}
		\drawHexagon{{6 - 2 * (2 - sqrt(3)) }}{ 2}{dtu-yellow}{UI}{2}
		\drawHexagon{{4 - 1 * (2 - sqrt(3)) }}{-1}{dtu-yellow}{UI}{2}
		\drawHexagon{{0 + 1 * (2 - sqrt(3)) }}{-1}{dtu-yellow}{UI}{2}
		\drawHexagon{{8 - 3 * (2 - sqrt(3)) }}{-1}{dtu-yellow}{UI}{2}

		\drawHexagon{{2 - 0 * (2 - sqrt(3)) }}{-4}{dtu-yellow}{UI}{2}
		\drawHexagon{{6 - 2 * (2 - sqrt(3)) }}{-4}{dtu-yellow}{UI}{2}

		\drawHexagon{{10 - 4 * (2 - sqrt(3)) }}{-4}{dtu-yellow}{UI}{2}
		\drawHexagon{{-2 + 2 * (2 - sqrt(3)) }}{-4}{dtu-yellow}{UI}{2}

		\drawHexagon{{12 - 5 * (2 - sqrt(3)) }}{-1}{dtu-yellow}{UI}{2}
		\drawHexagon{{-4 + 3 * (2 - sqrt(3)) }}{-1}{dtu-yellow}{UI}{2}
		% Legend for each layer
		\drawHexagon{{14.0  }}{+3.0}{dtu-white}{}{0.75}
		\node[align=right, anchor=west] at ({15.0}, +3.75) {Persistence};
		\drawHexagon{{14.0  }}{+1.5}{dtu-white}{}{0.75}
		\node[align=right, anchor=west] at ({15.0}, +2.25) {Atomic Pointer};
		\drawHexagon{{14.0  }}{+0.0}{dtu-white}{}{0.75}
		\node[align=right, anchor=west] at ({15.0}, +0.75) {Metaheuristics};
		\drawHexagon{{14.0  }}{-1.5}{dtu-white}{}{0.75}
		\node[align=right, anchor=west] at ({15.0}, -0.75) {Orchestration};
		\drawHexagon{{14.0  }}{-3.0}{dtu-yellow}{}{0.75}
		\node[align=right, anchor=west] at ({15.0}, -2.25) {User interfaces};
	\fi
	
	\end{tikzpicture}
}

	\resizebox{0.7\textwidth}{!}{
		\drawModelSetupHexagon[atomicpointerswap=true]
	}
	\caption{
		Overview of the scheduling process when modelled as actors. When LNS is encapsulated 
		is an actor it becomes possible to optimize parts of a large process individually instead of 
		optimizing the scheduling problem globally from a single model implementation.
	}
	\label{fig:ordinator-hexagon:atomicpointerswap}
\end{figure}

\begin{figure}[H]
	\centering
	\usetikzlibrary {positioning}
\newcommand{\drawHexagon}[6][draw=black]{
	\draw[#1, fill=#4] (#2,#3) ++(30:#6) -- ++(90:#6) -- ++(150:#6) -- ++(210:#6) -- ++(270:#6) -- ++(330:#6) -- cycle;
	\node[align=center] at (#2,#3+2) {#5};
}

\newif\ifpersistencelayer
\newif\ifatomicpointerswaplayer
\newif\ifmetaheuristicslayer
\newif\ifuserinterfacelayer
\newif\iforchestratorlayer
\newif\ifsimplifiedlayer

\pgfkeys{
	/hexagon/.is family, /hexagon,
	default/.style = {
		persistence=false,
		atomicpointerswap=false,
		metaheuristics=false,
		orchestrator=false,
		userinterface=false,
		simplified=false,
	},
	persistence/.is if=persistencelayer,
	atomicpointerswap/.is if=atomicpointerswaplayer,
	metaheuristics/.is if=metaheuristicslayer,
	orchestrator/.is if=orchestratorlayer,
	userinterface/.is if=userinterfacelayer,
	simplified/.is if=simplifiedlayer,
}
\newcommand{\drawModelSetupHexagon}[1][]{
	\pgfkeys{/hexagon, default, #1}

	\begin{tikzpicture}[font=\footnotesize, scale=0.5, line width=1.05]
	

	\ifpersistencelayer
		\drawHexagon[draw=none]{ 2                      }{ 2}{dtu-blue}{}{2}
		\drawHexagon[draw=none]{{6 - 2 * (2 - sqrt(3)) }}{ 2}{dtu-blue}{}{2}
		\drawHexagon[draw=none]{{4 - 1 * (2 - sqrt(3)) }}{-1}{dtu-blue}{Persistence}{2}
		\drawHexagon[draw=none]{{0 + 1 * (2 - sqrt(3)) }}{-1}{dtu-blue}{}{2}
		\drawHexagon[draw=none]{{8 - 3 * (2 - sqrt(3)) }}{-1}{dtu-blue}{}{2}

		\drawHexagon[draw=none]{{2 - 0 * (2 - sqrt(3)) }}{-4}{dtu-blue}{}{2}
		\drawHexagon[draw=none]{{6 - 2 * (2 - sqrt(3)) }}{-4}{dtu-blue}{}{2}

		\drawHexagon[draw=none]{{10 - 4 * (2 - sqrt(3)) }}{-4}{dtu-blue}{}{2}
		\drawHexagon[draw=none]{{-2 + 2 * (2 - sqrt(3)) }}{-4}{dtu-blue}{}{2}

		\drawHexagon[draw=none]{{12 - 5 * (2 - sqrt(3)) }}{-1}{dtu-blue}{}{2}
		\drawHexagon[draw=none]{{-4 + 3 * (2 - sqrt(3)) }}{-1}{dtu-blue}{}{2}
		% Legend for each layer
		\drawHexagon{{14.0  }}{+3.0}{dtu-blue}{}{0.75}
		\node[align=right, anchor=west] at ({15.0}, +3.75) {Persistence};
		\drawHexagon{{14.0  }}{+1.5}{dtu-white}{}{0.75}
		\node[align=right, anchor=west] at ({15.0}, +2.25) {Atomic Pointer};
		\drawHexagon{{14.0  }}{+0.0}{dtu-white}{}{0.75}
		\node[align=right, anchor=west] at ({15.0}, +0.75) {Metaheuristics};
		\drawHexagon{{14.0  }}{-1.5}{dtu-white}{}{0.75}
		\node[align=right, anchor=west] at ({15.0}, -0.75) {Orchestration};
		\drawHexagon{{14.0  }}{-3.0}{dtu-white}{}{0.75}
		\node[align=right, anchor=west] at ({15.0}, -2.25) {User interfaces};
	\fi


	\ifatomicpointerswaplayer
		\drawHexagon[]{ 2                      }{ 2}{dtu-green}{Shared\\solution\\pointer}{2}
		\drawHexagon[]{{6 - 2 * (2 - sqrt(3)) }}{ 2}{dtu-green}{Shared\\solution\\pointer}{2}
		\drawHexagon[]{{4 - 1 * (2 - sqrt(3)) }}{-1}{dtu-green}{Shared\\solution\\pointer}{2}
		\drawHexagon[]{{0 + 1 * (2 - sqrt(3)) }}{-1}{dtu-green}{Shared\\solution\\pointer}{2}
		\drawHexagon[]{{8 - 3 * (2 - sqrt(3)) }}{-1}{dtu-green}{Shared\\solution\\pointer}{2}

		\drawHexagon[]{{2 - 0 * (2 - sqrt(3)) }}{-4}{dtu-green}{Shared\\solution\\pointer}{2}
		\drawHexagon[]{{6 - 2 * (2 - sqrt(3)) }}{-4}{dtu-green}{Shared\\solution\\pointer}{2}

		\drawHexagon[]{{10 - 4 * (2 - sqrt(3)) }}{-4}{dtu-green}{Shared\\solution\\pointer}{2}
		\drawHexagon[]{{-2 + 2 * (2 - sqrt(3)) }}{-4}{dtu-green}{Shared\\solution\\pointer}{2}

		\drawHexagon[]{{12 - 5 * (2 - sqrt(3)) }}{-1}{dtu-green}{Shared\\solution\\pointer}{2}
		\drawHexagon[]{{-4 + 3 * (2 - sqrt(3)) }}{-1}{dtu-green}{Shared\\solution\\pointer}{2}
		% Legend for each layer
		\drawHexagon{{14.0  }}{+3.0}{dtu-white}{}{0.75}
		\node[align=right, anchor=west] at ({15.0}, +3.75) {Persistence};
		\drawHexagon{{14.0  }}{+1.5}{dtu-green}{}{0.75}
		\node[align=right, anchor=west] at ({15.0}, +2.25) {Atomic Pointer};
		\drawHexagon{{14.0  }}{+0.0}{dtu-white}{}{0.75}
		\node[align=right, anchor=west] at ({15.0}, +0.75) {Metaheuristics};
		\drawHexagon{{14.0  }}{-1.5}{dtu-white}{}{0.75}
		\node[align=right, anchor=west] at ({15.0}, -0.75) {Orchestration};
		\drawHexagon{{14.0  }}{-3.0}{dtu-white}{}{0.75}
		\node[align=right, anchor=west] at ({15.0}, -2.25) {User interfaces};
	\fi

	\ifsimplifiedlayer

		\node[align=right, anchor=west] at ({-5.5}, +3.75) {};
		\drawHexagon{{+2 + 0 * (2 - sqrt(3)) }}{ 2}{dtu-green}{Scheduler}{2}
		\drawHexagon{{+4 - 1 * (2 - sqrt(3)) }}{-1}{dtu-red}{Supervisor}{2}
		\drawHexagon{{+0 + 1 * (2 - sqrt(3)) }}{-1}{dtu-red}{Supervisor}{2}
		\drawHexagon{{+2 - 0 * (2 - sqrt(3)) }}{-4}{dtu-corporate-red}{Technician}{2}
		\drawHexagon{{+6 - 2 * (2 - sqrt(3)) }}{-4}{dtu-corporate-red}{Technician}{2}
		\drawHexagon{{-2 + 2 * (2 - sqrt(3)) }}{-4}{dtu-corporate-red}{Technician}{2}
		\drawHexagon{{+8 - 3 * (2 - sqrt(3)) }}{-1}{dtu-corporate-red}{Technician}{2}
		\drawHexagon{{-4 + 3 * (2 - sqrt(3)) }}{-1}{dtu-corporate-red}{Technician}{2}

		% Scheduler
		\draw[thin, fill=dtu-yellow] (2, 5) circle (0.35);
		\draw[thin, fill=dtu-purple] (2, 3) circle (0.35);
		% Supervisor 1
		\draw[thin, fill=dtu-yellow] ({+4 - 1 * (2 - sqrt(3)) }, 02) circle (0.35);
		\draw[thin, fill=dtu-purple] ({+4 - 1 * (2 - sqrt(3)) }, -0) circle (0.35);
		% Supervisor 2
		\draw[thin, fill=dtu-yellow] ({+0 + 1 * (2 - sqrt(3)) }, 02) circle (0.35);
		\draw[thin, fill=dtu-purple] ({+0 + 1 * (2 - sqrt(3)) }, -0) circle (0.35);
		% Technician 1
		\draw[thin, fill=dtu-yellow] ({+2 - 0 * (2 - sqrt(3)) }, -1) circle (0.35);
		\draw[thin, fill=dtu-purple] ({+2 - 0 * (2 - sqrt(3)) }, -3) circle (0.35);
		% Technician 2
		\draw[thin, fill=dtu-yellow] ({+6 - 2 * (2 - sqrt(3)) }, -1) circle (0.35);
		\draw[thin, fill=dtu-purple] ({+6 - 2 * (2 - sqrt(3)) }, -3) circle (0.35);
		% Technician 3
		\draw[thin, fill=dtu-yellow] ({-2 + 2 * (2 - sqrt(3)) }, -1) circle (0.35);
		\draw[thin, fill=dtu-purple] ({-2 + 2 * (2 - sqrt(3)) }, -3) circle (0.35);
		% Technician 4
		\draw[thin, fill=dtu-yellow] ({+8 - 3 * (2 - sqrt(3)) }, 02) circle (0.35);
		\draw[thin, fill=dtu-purple] ({+8 - 3 * (2 - sqrt(3)) }, -0) circle (0.35);
		% Technician 5
		\draw[thin, fill=dtu-yellow] ({-4 + 3 * (2 - sqrt(3)) }, 02) circle (0.35);
		\draw[thin, fill=dtu-purple] ({-4 + 3 * (2 - sqrt(3)) }, -0) circle (0.35);

		% Legend for each layer
		\node[align=right, anchor=west] at ({12.0}, +3.75) {Atomic Pointer};
		\draw[fill=dtu-purple] (11.0,  +3.75) circle (0.5);

		\node[align=right, anchor=west] at ({12.0}, +2.25) {Scheduler Metaheuristic};
		\drawHexagon{{11.0  }}{+1.75}{dtu-green}{}{0.5}
		\node[align=right, anchor=west] at ({12.0}, +0.75) {Supervisor Metaheuristic};
		\drawHexagon{{11.0  }}{+0.25}{dtu-red}{}{0.5}
		\node[align=right, anchor=west] at ({12.0}, -0.75) {Technician Metaheuristic};
		\drawHexagon{{11.0  }}{-1.25}{dtu-corporate-red}{}{0.5}
		\node[align=right, anchor=west] at ({12.0}, -2.25) {User interfaces (Message Passing)};
		\draw[fill=dtu-yellow] (11.0, -2.25) circle (0.5);
	\fi

	\ifmetaheuristicslayer
		\drawHexagon{ 2                      }{ 2}{dtu-blue}{Strategic}{2}
		\drawHexagon{{6 - 2 * (2 - sqrt(3)) }}{ 2}{dtu-green}{Tactical}{2}
		\drawHexagon{{4 - 1 * (2 - sqrt(3)) }}{-1}{dtu-red}{Supervisor}{2}
		\drawHexagon{{0 + 1 * (2 - sqrt(3)) }}{-1}{dtu-red}{Supervisor}{2}
		\drawHexagon{{8 - 3 * (2 - sqrt(3)) }}{-1}{dtu-red}{Supervisor}{2}

		\drawHexagon{{2 - 0 * (2 - sqrt(3)) }}{-4}{dtu-corporate-red}{Technician}{2}
		\drawHexagon{{6 - 2 * (2 - sqrt(3)) }}{-4}{dtu-corporate-red}{Technician}{2}

		\drawHexagon{{10 - 4 * (2 - sqrt(3)) }}{-4}{dtu-corporate-red}{Technician}{2}
		\drawHexagon{{-2 + 2 * (2 - sqrt(3)) }}{-4}{dtu-corporate-red}{Technician}{2}

		\drawHexagon{{12 - 5 * (2 - sqrt(3)) }}{-1}{dtu-corporate-red}{Technician}{2}
		\drawHexagon{{-4 + 3 * (2 - sqrt(3)) }}{-1}{dtu-corporate-red}{Technician}{2}

		% Legend for each layer
		\drawHexagon{{14.0  }}{+3.0}{dtu-white}{}{0.75}
		\node[align=right, anchor=west] at ({15.0}, +3.75) {Persistence};
		\drawHexagon{{14.0  }}{+1.5}{dtu-white}{}{0.75}
		\node[align=right, anchor=west] at ({15.0}, +2.25) {Atomic Pointer};
		\drawHexagon{{14.0  }}{+0.0}{dtu-corporate-red}{}{0.75}
		\node[align=right, anchor=west] at ({15.0}, +0.75) {Metaheuristics};
		\drawHexagon{{14.0  }}{-1.5}{dtu-white}{}{0.75}
		\node[align=right, anchor=west] at ({15.0}, -0.75) {Orchestration};
		\drawHexagon{{14.0  }}{-3.0}{dtu-white}{}{0.75}
		\node[align=right, anchor=west] at ({15.0}, -2.25) {User interfaces};
	\fi

	\iforchestratorlayer
		\drawHexagon{ 2                      }{ 2}{dtu-orange}{}{2}
		\drawHexagon{{6 - 2 * (2 - sqrt(3)) }}{ 2}{dtu-orange}{}{2}
		\drawHexagon{{4 - 1 * (2 - sqrt(3)) }}{-1}{dtu-orange}{Orche-\\strator}{2}
		\drawHexagon{{0 + 1 * (2 - sqrt(3)) }}{-1}{dtu-orange}{}{2}
		\drawHexagon{{8 - 3 * (2 - sqrt(3)) }}{-1}{dtu-orange}{}{2}

		\drawHexagon{{2 - 0 * (2 - sqrt(3)) }}{-4}{dtu-orange}{}{2}
		\drawHexagon{{6 - 2 * (2 - sqrt(3)) }}{-4}{dtu-orange}{}{2}

		\drawHexagon{{10 - 4 * (2 - sqrt(3)) }}{-4}{dtu-orange}{}{2}
		\drawHexagon{{-2 + 2 * (2 - sqrt(3)) }}{-4}{dtu-orange}{}{2}

		\drawHexagon{{12 - 5 * (2 - sqrt(3)) }}{-1}{dtu-orange}{}{2}
		\drawHexagon{{-4 + 3 * (2 - sqrt(3)) }}{-1}{dtu-orange}{}{2}
		% Legend for each layer
		\drawHexagon{{14.0  }}{+3.0}{dtu-white}{}{0.75}
		\node[align=right, anchor=west] at ({15.0}, +3.75) {Persistence};
		\drawHexagon{{14.0  }}{+1.5}{dtu-white}{}{0.75}
		\node[align=right, anchor=west] at ({15.0}, +2.25) {Atomic Pointer};
		\drawHexagon{{14.0  }}{+0.0}{dtu-white}{}{0.75}
		\node[align=right, anchor=west] at ({15.0}, +0.75) {Metaheuristics};
		\drawHexagon{{14.0  }}{-1.5}{dtu-orange}{}{0.75}
		\node[align=right, anchor=west] at ({15.0}, -0.75) {Orchestration};
		\drawHexagon{{14.0  }}{-3.0}{dtu-white}{}{0.75}
		\node[align=right, anchor=west] at ({15.0}, -2.25) {User interfaces};
	\fi

	
	\ifuserinterfacelayer
		\drawHexagon{ 2                      }{ 2}{dtu-yellow}{UI}{2}
		\drawHexagon{{6 - 2 * (2 - sqrt(3)) }}{ 2}{dtu-yellow}{UI}{2}
		\drawHexagon{{4 - 1 * (2 - sqrt(3)) }}{-1}{dtu-yellow}{UI}{2}
		\drawHexagon{{0 + 1 * (2 - sqrt(3)) }}{-1}{dtu-yellow}{UI}{2}
		\drawHexagon{{8 - 3 * (2 - sqrt(3)) }}{-1}{dtu-yellow}{UI}{2}

		\drawHexagon{{2 - 0 * (2 - sqrt(3)) }}{-4}{dtu-yellow}{UI}{2}
		\drawHexagon{{6 - 2 * (2 - sqrt(3)) }}{-4}{dtu-yellow}{UI}{2}

		\drawHexagon{{10 - 4 * (2 - sqrt(3)) }}{-4}{dtu-yellow}{UI}{2}
		\drawHexagon{{-2 + 2 * (2 - sqrt(3)) }}{-4}{dtu-yellow}{UI}{2}

		\drawHexagon{{12 - 5 * (2 - sqrt(3)) }}{-1}{dtu-yellow}{UI}{2}
		\drawHexagon{{-4 + 3 * (2 - sqrt(3)) }}{-1}{dtu-yellow}{UI}{2}
		% Legend for each layer
		\drawHexagon{{14.0  }}{+3.0}{dtu-white}{}{0.75}
		\node[align=right, anchor=west] at ({15.0}, +3.75) {Persistence};
		\drawHexagon{{14.0  }}{+1.5}{dtu-white}{}{0.75}
		\node[align=right, anchor=west] at ({15.0}, +2.25) {Atomic Pointer};
		\drawHexagon{{14.0  }}{+0.0}{dtu-white}{}{0.75}
		\node[align=right, anchor=west] at ({15.0}, +0.75) {Metaheuristics};
		\drawHexagon{{14.0  }}{-1.5}{dtu-white}{}{0.75}
		\node[align=right, anchor=west] at ({15.0}, -0.75) {Orchestration};
		\drawHexagon{{14.0  }}{-3.0}{dtu-yellow}{}{0.75}
		\node[align=right, anchor=west] at ({15.0}, -2.25) {User interfaces};
	\fi
	
	\end{tikzpicture}
}

	\resizebox{0.7\textwidth}{!}{
		\drawModelSetupHexagon[metaheuristics=true]
	}
	\caption{
		Overview of the scheduling process when modelled as actors. When LNS is encapsulated 
		is an actor it becomes possible to optimize parts of a large process individually instead of 
		optimizing the scheduling problem globally from a single model implementation.
	}
	\label{fig:ordinator-hexagon:metaheuristics}
\end{figure}

\begin{figure}[H]
	\centering
	\usetikzlibrary {positioning}
\newcommand{\drawHexagon}[6][draw=black]{
	\draw[#1, fill=#4] (#2,#3) ++(30:#6) -- ++(90:#6) -- ++(150:#6) -- ++(210:#6) -- ++(270:#6) -- ++(330:#6) -- cycle;
	\node[align=center] at (#2,#3+2) {#5};
}

\newif\ifpersistencelayer
\newif\ifatomicpointerswaplayer
\newif\ifmetaheuristicslayer
\newif\ifuserinterfacelayer
\newif\iforchestratorlayer
\newif\ifsimplifiedlayer

\pgfkeys{
	/hexagon/.is family, /hexagon,
	default/.style = {
		persistence=false,
		atomicpointerswap=false,
		metaheuristics=false,
		orchestrator=false,
		userinterface=false,
		simplified=false,
	},
	persistence/.is if=persistencelayer,
	atomicpointerswap/.is if=atomicpointerswaplayer,
	metaheuristics/.is if=metaheuristicslayer,
	orchestrator/.is if=orchestratorlayer,
	userinterface/.is if=userinterfacelayer,
	simplified/.is if=simplifiedlayer,
}
\newcommand{\drawModelSetupHexagon}[1][]{
	\pgfkeys{/hexagon, default, #1}

	\begin{tikzpicture}[font=\footnotesize, scale=0.5, line width=1.05]
	

	\ifpersistencelayer
		\drawHexagon[draw=none]{ 2                      }{ 2}{dtu-blue}{}{2}
		\drawHexagon[draw=none]{{6 - 2 * (2 - sqrt(3)) }}{ 2}{dtu-blue}{}{2}
		\drawHexagon[draw=none]{{4 - 1 * (2 - sqrt(3)) }}{-1}{dtu-blue}{Persistence}{2}
		\drawHexagon[draw=none]{{0 + 1 * (2 - sqrt(3)) }}{-1}{dtu-blue}{}{2}
		\drawHexagon[draw=none]{{8 - 3 * (2 - sqrt(3)) }}{-1}{dtu-blue}{}{2}

		\drawHexagon[draw=none]{{2 - 0 * (2 - sqrt(3)) }}{-4}{dtu-blue}{}{2}
		\drawHexagon[draw=none]{{6 - 2 * (2 - sqrt(3)) }}{-4}{dtu-blue}{}{2}

		\drawHexagon[draw=none]{{10 - 4 * (2 - sqrt(3)) }}{-4}{dtu-blue}{}{2}
		\drawHexagon[draw=none]{{-2 + 2 * (2 - sqrt(3)) }}{-4}{dtu-blue}{}{2}

		\drawHexagon[draw=none]{{12 - 5 * (2 - sqrt(3)) }}{-1}{dtu-blue}{}{2}
		\drawHexagon[draw=none]{{-4 + 3 * (2 - sqrt(3)) }}{-1}{dtu-blue}{}{2}
		% Legend for each layer
		\drawHexagon{{14.0  }}{+3.0}{dtu-blue}{}{0.75}
		\node[align=right, anchor=west] at ({15.0}, +3.75) {Persistence};
		\drawHexagon{{14.0  }}{+1.5}{dtu-white}{}{0.75}
		\node[align=right, anchor=west] at ({15.0}, +2.25) {Atomic Pointer};
		\drawHexagon{{14.0  }}{+0.0}{dtu-white}{}{0.75}
		\node[align=right, anchor=west] at ({15.0}, +0.75) {Metaheuristics};
		\drawHexagon{{14.0  }}{-1.5}{dtu-white}{}{0.75}
		\node[align=right, anchor=west] at ({15.0}, -0.75) {Orchestration};
		\drawHexagon{{14.0  }}{-3.0}{dtu-white}{}{0.75}
		\node[align=right, anchor=west] at ({15.0}, -2.25) {User interfaces};
	\fi


	\ifatomicpointerswaplayer
		\drawHexagon[]{ 2                      }{ 2}{dtu-green}{Shared\\solution\\pointer}{2}
		\drawHexagon[]{{6 - 2 * (2 - sqrt(3)) }}{ 2}{dtu-green}{Shared\\solution\\pointer}{2}
		\drawHexagon[]{{4 - 1 * (2 - sqrt(3)) }}{-1}{dtu-green}{Shared\\solution\\pointer}{2}
		\drawHexagon[]{{0 + 1 * (2 - sqrt(3)) }}{-1}{dtu-green}{Shared\\solution\\pointer}{2}
		\drawHexagon[]{{8 - 3 * (2 - sqrt(3)) }}{-1}{dtu-green}{Shared\\solution\\pointer}{2}

		\drawHexagon[]{{2 - 0 * (2 - sqrt(3)) }}{-4}{dtu-green}{Shared\\solution\\pointer}{2}
		\drawHexagon[]{{6 - 2 * (2 - sqrt(3)) }}{-4}{dtu-green}{Shared\\solution\\pointer}{2}

		\drawHexagon[]{{10 - 4 * (2 - sqrt(3)) }}{-4}{dtu-green}{Shared\\solution\\pointer}{2}
		\drawHexagon[]{{-2 + 2 * (2 - sqrt(3)) }}{-4}{dtu-green}{Shared\\solution\\pointer}{2}

		\drawHexagon[]{{12 - 5 * (2 - sqrt(3)) }}{-1}{dtu-green}{Shared\\solution\\pointer}{2}
		\drawHexagon[]{{-4 + 3 * (2 - sqrt(3)) }}{-1}{dtu-green}{Shared\\solution\\pointer}{2}
		% Legend for each layer
		\drawHexagon{{14.0  }}{+3.0}{dtu-white}{}{0.75}
		\node[align=right, anchor=west] at ({15.0}, +3.75) {Persistence};
		\drawHexagon{{14.0  }}{+1.5}{dtu-green}{}{0.75}
		\node[align=right, anchor=west] at ({15.0}, +2.25) {Atomic Pointer};
		\drawHexagon{{14.0  }}{+0.0}{dtu-white}{}{0.75}
		\node[align=right, anchor=west] at ({15.0}, +0.75) {Metaheuristics};
		\drawHexagon{{14.0  }}{-1.5}{dtu-white}{}{0.75}
		\node[align=right, anchor=west] at ({15.0}, -0.75) {Orchestration};
		\drawHexagon{{14.0  }}{-3.0}{dtu-white}{}{0.75}
		\node[align=right, anchor=west] at ({15.0}, -2.25) {User interfaces};
	\fi

	\ifsimplifiedlayer

		\node[align=right, anchor=west] at ({-5.5}, +3.75) {};
		\drawHexagon{{+2 + 0 * (2 - sqrt(3)) }}{ 2}{dtu-green}{Scheduler}{2}
		\drawHexagon{{+4 - 1 * (2 - sqrt(3)) }}{-1}{dtu-red}{Supervisor}{2}
		\drawHexagon{{+0 + 1 * (2 - sqrt(3)) }}{-1}{dtu-red}{Supervisor}{2}
		\drawHexagon{{+2 - 0 * (2 - sqrt(3)) }}{-4}{dtu-corporate-red}{Technician}{2}
		\drawHexagon{{+6 - 2 * (2 - sqrt(3)) }}{-4}{dtu-corporate-red}{Technician}{2}
		\drawHexagon{{-2 + 2 * (2 - sqrt(3)) }}{-4}{dtu-corporate-red}{Technician}{2}
		\drawHexagon{{+8 - 3 * (2 - sqrt(3)) }}{-1}{dtu-corporate-red}{Technician}{2}
		\drawHexagon{{-4 + 3 * (2 - sqrt(3)) }}{-1}{dtu-corporate-red}{Technician}{2}

		% Scheduler
		\draw[thin, fill=dtu-yellow] (2, 5) circle (0.35);
		\draw[thin, fill=dtu-purple] (2, 3) circle (0.35);
		% Supervisor 1
		\draw[thin, fill=dtu-yellow] ({+4 - 1 * (2 - sqrt(3)) }, 02) circle (0.35);
		\draw[thin, fill=dtu-purple] ({+4 - 1 * (2 - sqrt(3)) }, -0) circle (0.35);
		% Supervisor 2
		\draw[thin, fill=dtu-yellow] ({+0 + 1 * (2 - sqrt(3)) }, 02) circle (0.35);
		\draw[thin, fill=dtu-purple] ({+0 + 1 * (2 - sqrt(3)) }, -0) circle (0.35);
		% Technician 1
		\draw[thin, fill=dtu-yellow] ({+2 - 0 * (2 - sqrt(3)) }, -1) circle (0.35);
		\draw[thin, fill=dtu-purple] ({+2 - 0 * (2 - sqrt(3)) }, -3) circle (0.35);
		% Technician 2
		\draw[thin, fill=dtu-yellow] ({+6 - 2 * (2 - sqrt(3)) }, -1) circle (0.35);
		\draw[thin, fill=dtu-purple] ({+6 - 2 * (2 - sqrt(3)) }, -3) circle (0.35);
		% Technician 3
		\draw[thin, fill=dtu-yellow] ({-2 + 2 * (2 - sqrt(3)) }, -1) circle (0.35);
		\draw[thin, fill=dtu-purple] ({-2 + 2 * (2 - sqrt(3)) }, -3) circle (0.35);
		% Technician 4
		\draw[thin, fill=dtu-yellow] ({+8 - 3 * (2 - sqrt(3)) }, 02) circle (0.35);
		\draw[thin, fill=dtu-purple] ({+8 - 3 * (2 - sqrt(3)) }, -0) circle (0.35);
		% Technician 5
		\draw[thin, fill=dtu-yellow] ({-4 + 3 * (2 - sqrt(3)) }, 02) circle (0.35);
		\draw[thin, fill=dtu-purple] ({-4 + 3 * (2 - sqrt(3)) }, -0) circle (0.35);

		% Legend for each layer
		\node[align=right, anchor=west] at ({12.0}, +3.75) {Atomic Pointer};
		\draw[fill=dtu-purple] (11.0,  +3.75) circle (0.5);

		\node[align=right, anchor=west] at ({12.0}, +2.25) {Scheduler Metaheuristic};
		\drawHexagon{{11.0  }}{+1.75}{dtu-green}{}{0.5}
		\node[align=right, anchor=west] at ({12.0}, +0.75) {Supervisor Metaheuristic};
		\drawHexagon{{11.0  }}{+0.25}{dtu-red}{}{0.5}
		\node[align=right, anchor=west] at ({12.0}, -0.75) {Technician Metaheuristic};
		\drawHexagon{{11.0  }}{-1.25}{dtu-corporate-red}{}{0.5}
		\node[align=right, anchor=west] at ({12.0}, -2.25) {User interfaces (Message Passing)};
		\draw[fill=dtu-yellow] (11.0, -2.25) circle (0.5);
	\fi

	\ifmetaheuristicslayer
		\drawHexagon{ 2                      }{ 2}{dtu-blue}{Strategic}{2}
		\drawHexagon{{6 - 2 * (2 - sqrt(3)) }}{ 2}{dtu-green}{Tactical}{2}
		\drawHexagon{{4 - 1 * (2 - sqrt(3)) }}{-1}{dtu-red}{Supervisor}{2}
		\drawHexagon{{0 + 1 * (2 - sqrt(3)) }}{-1}{dtu-red}{Supervisor}{2}
		\drawHexagon{{8 - 3 * (2 - sqrt(3)) }}{-1}{dtu-red}{Supervisor}{2}

		\drawHexagon{{2 - 0 * (2 - sqrt(3)) }}{-4}{dtu-corporate-red}{Technician}{2}
		\drawHexagon{{6 - 2 * (2 - sqrt(3)) }}{-4}{dtu-corporate-red}{Technician}{2}

		\drawHexagon{{10 - 4 * (2 - sqrt(3)) }}{-4}{dtu-corporate-red}{Technician}{2}
		\drawHexagon{{-2 + 2 * (2 - sqrt(3)) }}{-4}{dtu-corporate-red}{Technician}{2}

		\drawHexagon{{12 - 5 * (2 - sqrt(3)) }}{-1}{dtu-corporate-red}{Technician}{2}
		\drawHexagon{{-4 + 3 * (2 - sqrt(3)) }}{-1}{dtu-corporate-red}{Technician}{2}

		% Legend for each layer
		\drawHexagon{{14.0  }}{+3.0}{dtu-white}{}{0.75}
		\node[align=right, anchor=west] at ({15.0}, +3.75) {Persistence};
		\drawHexagon{{14.0  }}{+1.5}{dtu-white}{}{0.75}
		\node[align=right, anchor=west] at ({15.0}, +2.25) {Atomic Pointer};
		\drawHexagon{{14.0  }}{+0.0}{dtu-corporate-red}{}{0.75}
		\node[align=right, anchor=west] at ({15.0}, +0.75) {Metaheuristics};
		\drawHexagon{{14.0  }}{-1.5}{dtu-white}{}{0.75}
		\node[align=right, anchor=west] at ({15.0}, -0.75) {Orchestration};
		\drawHexagon{{14.0  }}{-3.0}{dtu-white}{}{0.75}
		\node[align=right, anchor=west] at ({15.0}, -2.25) {User interfaces};
	\fi

	\iforchestratorlayer
		\drawHexagon{ 2                      }{ 2}{dtu-orange}{}{2}
		\drawHexagon{{6 - 2 * (2 - sqrt(3)) }}{ 2}{dtu-orange}{}{2}
		\drawHexagon{{4 - 1 * (2 - sqrt(3)) }}{-1}{dtu-orange}{Orche-\\strator}{2}
		\drawHexagon{{0 + 1 * (2 - sqrt(3)) }}{-1}{dtu-orange}{}{2}
		\drawHexagon{{8 - 3 * (2 - sqrt(3)) }}{-1}{dtu-orange}{}{2}

		\drawHexagon{{2 - 0 * (2 - sqrt(3)) }}{-4}{dtu-orange}{}{2}
		\drawHexagon{{6 - 2 * (2 - sqrt(3)) }}{-4}{dtu-orange}{}{2}

		\drawHexagon{{10 - 4 * (2 - sqrt(3)) }}{-4}{dtu-orange}{}{2}
		\drawHexagon{{-2 + 2 * (2 - sqrt(3)) }}{-4}{dtu-orange}{}{2}

		\drawHexagon{{12 - 5 * (2 - sqrt(3)) }}{-1}{dtu-orange}{}{2}
		\drawHexagon{{-4 + 3 * (2 - sqrt(3)) }}{-1}{dtu-orange}{}{2}
		% Legend for each layer
		\drawHexagon{{14.0  }}{+3.0}{dtu-white}{}{0.75}
		\node[align=right, anchor=west] at ({15.0}, +3.75) {Persistence};
		\drawHexagon{{14.0  }}{+1.5}{dtu-white}{}{0.75}
		\node[align=right, anchor=west] at ({15.0}, +2.25) {Atomic Pointer};
		\drawHexagon{{14.0  }}{+0.0}{dtu-white}{}{0.75}
		\node[align=right, anchor=west] at ({15.0}, +0.75) {Metaheuristics};
		\drawHexagon{{14.0  }}{-1.5}{dtu-orange}{}{0.75}
		\node[align=right, anchor=west] at ({15.0}, -0.75) {Orchestration};
		\drawHexagon{{14.0  }}{-3.0}{dtu-white}{}{0.75}
		\node[align=right, anchor=west] at ({15.0}, -2.25) {User interfaces};
	\fi

	
	\ifuserinterfacelayer
		\drawHexagon{ 2                      }{ 2}{dtu-yellow}{UI}{2}
		\drawHexagon{{6 - 2 * (2 - sqrt(3)) }}{ 2}{dtu-yellow}{UI}{2}
		\drawHexagon{{4 - 1 * (2 - sqrt(3)) }}{-1}{dtu-yellow}{UI}{2}
		\drawHexagon{{0 + 1 * (2 - sqrt(3)) }}{-1}{dtu-yellow}{UI}{2}
		\drawHexagon{{8 - 3 * (2 - sqrt(3)) }}{-1}{dtu-yellow}{UI}{2}

		\drawHexagon{{2 - 0 * (2 - sqrt(3)) }}{-4}{dtu-yellow}{UI}{2}
		\drawHexagon{{6 - 2 * (2 - sqrt(3)) }}{-4}{dtu-yellow}{UI}{2}

		\drawHexagon{{10 - 4 * (2 - sqrt(3)) }}{-4}{dtu-yellow}{UI}{2}
		\drawHexagon{{-2 + 2 * (2 - sqrt(3)) }}{-4}{dtu-yellow}{UI}{2}

		\drawHexagon{{12 - 5 * (2 - sqrt(3)) }}{-1}{dtu-yellow}{UI}{2}
		\drawHexagon{{-4 + 3 * (2 - sqrt(3)) }}{-1}{dtu-yellow}{UI}{2}
		% Legend for each layer
		\drawHexagon{{14.0  }}{+3.0}{dtu-white}{}{0.75}
		\node[align=right, anchor=west] at ({15.0}, +3.75) {Persistence};
		\drawHexagon{{14.0  }}{+1.5}{dtu-white}{}{0.75}
		\node[align=right, anchor=west] at ({15.0}, +2.25) {Atomic Pointer};
		\drawHexagon{{14.0  }}{+0.0}{dtu-white}{}{0.75}
		\node[align=right, anchor=west] at ({15.0}, +0.75) {Metaheuristics};
		\drawHexagon{{14.0  }}{-1.5}{dtu-white}{}{0.75}
		\node[align=right, anchor=west] at ({15.0}, -0.75) {Orchestration};
		\drawHexagon{{14.0  }}{-3.0}{dtu-yellow}{}{0.75}
		\node[align=right, anchor=west] at ({15.0}, -2.25) {User interfaces};
	\fi
	
	\end{tikzpicture}
}

	\resizebox{0.7\textwidth}{!}{
		\drawModelSetupHexagon[orchestrator=true]
	}
	\caption{
		Overview of the scheduling process when modelled as actors. When LNS is encapsulated 
		is an actor it becomes possible to optimize parts of a large process individually instead of 
		optimizing the scheduling problem globally from a single model implementation.
	}
	\label{fig:ordinator-hexagon:orchestrator}
\end{figure}
\begin{figure}[H]
	\centering
	\usetikzlibrary {positioning}
\newcommand{\drawHexagon}[6][draw=black]{
	\draw[#1, fill=#4] (#2,#3) ++(30:#6) -- ++(90:#6) -- ++(150:#6) -- ++(210:#6) -- ++(270:#6) -- ++(330:#6) -- cycle;
	\node[align=center] at (#2,#3+2) {#5};
}

\newif\ifpersistencelayer
\newif\ifatomicpointerswaplayer
\newif\ifmetaheuristicslayer
\newif\ifuserinterfacelayer
\newif\iforchestratorlayer
\newif\ifsimplifiedlayer

\pgfkeys{
	/hexagon/.is family, /hexagon,
	default/.style = {
		persistence=false,
		atomicpointerswap=false,
		metaheuristics=false,
		orchestrator=false,
		userinterface=false,
		simplified=false,
	},
	persistence/.is if=persistencelayer,
	atomicpointerswap/.is if=atomicpointerswaplayer,
	metaheuristics/.is if=metaheuristicslayer,
	orchestrator/.is if=orchestratorlayer,
	userinterface/.is if=userinterfacelayer,
	simplified/.is if=simplifiedlayer,
}
\newcommand{\drawModelSetupHexagon}[1][]{
	\pgfkeys{/hexagon, default, #1}

	\begin{tikzpicture}[font=\footnotesize, scale=0.5, line width=1.05]
	

	\ifpersistencelayer
		\drawHexagon[draw=none]{ 2                      }{ 2}{dtu-blue}{}{2}
		\drawHexagon[draw=none]{{6 - 2 * (2 - sqrt(3)) }}{ 2}{dtu-blue}{}{2}
		\drawHexagon[draw=none]{{4 - 1 * (2 - sqrt(3)) }}{-1}{dtu-blue}{Persistence}{2}
		\drawHexagon[draw=none]{{0 + 1 * (2 - sqrt(3)) }}{-1}{dtu-blue}{}{2}
		\drawHexagon[draw=none]{{8 - 3 * (2 - sqrt(3)) }}{-1}{dtu-blue}{}{2}

		\drawHexagon[draw=none]{{2 - 0 * (2 - sqrt(3)) }}{-4}{dtu-blue}{}{2}
		\drawHexagon[draw=none]{{6 - 2 * (2 - sqrt(3)) }}{-4}{dtu-blue}{}{2}

		\drawHexagon[draw=none]{{10 - 4 * (2 - sqrt(3)) }}{-4}{dtu-blue}{}{2}
		\drawHexagon[draw=none]{{-2 + 2 * (2 - sqrt(3)) }}{-4}{dtu-blue}{}{2}

		\drawHexagon[draw=none]{{12 - 5 * (2 - sqrt(3)) }}{-1}{dtu-blue}{}{2}
		\drawHexagon[draw=none]{{-4 + 3 * (2 - sqrt(3)) }}{-1}{dtu-blue}{}{2}
		% Legend for each layer
		\drawHexagon{{14.0  }}{+3.0}{dtu-blue}{}{0.75}
		\node[align=right, anchor=west] at ({15.0}, +3.75) {Persistence};
		\drawHexagon{{14.0  }}{+1.5}{dtu-white}{}{0.75}
		\node[align=right, anchor=west] at ({15.0}, +2.25) {Atomic Pointer};
		\drawHexagon{{14.0  }}{+0.0}{dtu-white}{}{0.75}
		\node[align=right, anchor=west] at ({15.0}, +0.75) {Metaheuristics};
		\drawHexagon{{14.0  }}{-1.5}{dtu-white}{}{0.75}
		\node[align=right, anchor=west] at ({15.0}, -0.75) {Orchestration};
		\drawHexagon{{14.0  }}{-3.0}{dtu-white}{}{0.75}
		\node[align=right, anchor=west] at ({15.0}, -2.25) {User interfaces};
	\fi


	\ifatomicpointerswaplayer
		\drawHexagon[]{ 2                      }{ 2}{dtu-green}{Shared\\solution\\pointer}{2}
		\drawHexagon[]{{6 - 2 * (2 - sqrt(3)) }}{ 2}{dtu-green}{Shared\\solution\\pointer}{2}
		\drawHexagon[]{{4 - 1 * (2 - sqrt(3)) }}{-1}{dtu-green}{Shared\\solution\\pointer}{2}
		\drawHexagon[]{{0 + 1 * (2 - sqrt(3)) }}{-1}{dtu-green}{Shared\\solution\\pointer}{2}
		\drawHexagon[]{{8 - 3 * (2 - sqrt(3)) }}{-1}{dtu-green}{Shared\\solution\\pointer}{2}

		\drawHexagon[]{{2 - 0 * (2 - sqrt(3)) }}{-4}{dtu-green}{Shared\\solution\\pointer}{2}
		\drawHexagon[]{{6 - 2 * (2 - sqrt(3)) }}{-4}{dtu-green}{Shared\\solution\\pointer}{2}

		\drawHexagon[]{{10 - 4 * (2 - sqrt(3)) }}{-4}{dtu-green}{Shared\\solution\\pointer}{2}
		\drawHexagon[]{{-2 + 2 * (2 - sqrt(3)) }}{-4}{dtu-green}{Shared\\solution\\pointer}{2}

		\drawHexagon[]{{12 - 5 * (2 - sqrt(3)) }}{-1}{dtu-green}{Shared\\solution\\pointer}{2}
		\drawHexagon[]{{-4 + 3 * (2 - sqrt(3)) }}{-1}{dtu-green}{Shared\\solution\\pointer}{2}
		% Legend for each layer
		\drawHexagon{{14.0  }}{+3.0}{dtu-white}{}{0.75}
		\node[align=right, anchor=west] at ({15.0}, +3.75) {Persistence};
		\drawHexagon{{14.0  }}{+1.5}{dtu-green}{}{0.75}
		\node[align=right, anchor=west] at ({15.0}, +2.25) {Atomic Pointer};
		\drawHexagon{{14.0  }}{+0.0}{dtu-white}{}{0.75}
		\node[align=right, anchor=west] at ({15.0}, +0.75) {Metaheuristics};
		\drawHexagon{{14.0  }}{-1.5}{dtu-white}{}{0.75}
		\node[align=right, anchor=west] at ({15.0}, -0.75) {Orchestration};
		\drawHexagon{{14.0  }}{-3.0}{dtu-white}{}{0.75}
		\node[align=right, anchor=west] at ({15.0}, -2.25) {User interfaces};
	\fi

	\ifsimplifiedlayer

		\node[align=right, anchor=west] at ({-5.5}, +3.75) {};
		\drawHexagon{{+2 + 0 * (2 - sqrt(3)) }}{ 2}{dtu-green}{Scheduler}{2}
		\drawHexagon{{+4 - 1 * (2 - sqrt(3)) }}{-1}{dtu-red}{Supervisor}{2}
		\drawHexagon{{+0 + 1 * (2 - sqrt(3)) }}{-1}{dtu-red}{Supervisor}{2}
		\drawHexagon{{+2 - 0 * (2 - sqrt(3)) }}{-4}{dtu-corporate-red}{Technician}{2}
		\drawHexagon{{+6 - 2 * (2 - sqrt(3)) }}{-4}{dtu-corporate-red}{Technician}{2}
		\drawHexagon{{-2 + 2 * (2 - sqrt(3)) }}{-4}{dtu-corporate-red}{Technician}{2}
		\drawHexagon{{+8 - 3 * (2 - sqrt(3)) }}{-1}{dtu-corporate-red}{Technician}{2}
		\drawHexagon{{-4 + 3 * (2 - sqrt(3)) }}{-1}{dtu-corporate-red}{Technician}{2}

		% Scheduler
		\draw[thin, fill=dtu-yellow] (2, 5) circle (0.35);
		\draw[thin, fill=dtu-purple] (2, 3) circle (0.35);
		% Supervisor 1
		\draw[thin, fill=dtu-yellow] ({+4 - 1 * (2 - sqrt(3)) }, 02) circle (0.35);
		\draw[thin, fill=dtu-purple] ({+4 - 1 * (2 - sqrt(3)) }, -0) circle (0.35);
		% Supervisor 2
		\draw[thin, fill=dtu-yellow] ({+0 + 1 * (2 - sqrt(3)) }, 02) circle (0.35);
		\draw[thin, fill=dtu-purple] ({+0 + 1 * (2 - sqrt(3)) }, -0) circle (0.35);
		% Technician 1
		\draw[thin, fill=dtu-yellow] ({+2 - 0 * (2 - sqrt(3)) }, -1) circle (0.35);
		\draw[thin, fill=dtu-purple] ({+2 - 0 * (2 - sqrt(3)) }, -3) circle (0.35);
		% Technician 2
		\draw[thin, fill=dtu-yellow] ({+6 - 2 * (2 - sqrt(3)) }, -1) circle (0.35);
		\draw[thin, fill=dtu-purple] ({+6 - 2 * (2 - sqrt(3)) }, -3) circle (0.35);
		% Technician 3
		\draw[thin, fill=dtu-yellow] ({-2 + 2 * (2 - sqrt(3)) }, -1) circle (0.35);
		\draw[thin, fill=dtu-purple] ({-2 + 2 * (2 - sqrt(3)) }, -3) circle (0.35);
		% Technician 4
		\draw[thin, fill=dtu-yellow] ({+8 - 3 * (2 - sqrt(3)) }, 02) circle (0.35);
		\draw[thin, fill=dtu-purple] ({+8 - 3 * (2 - sqrt(3)) }, -0) circle (0.35);
		% Technician 5
		\draw[thin, fill=dtu-yellow] ({-4 + 3 * (2 - sqrt(3)) }, 02) circle (0.35);
		\draw[thin, fill=dtu-purple] ({-4 + 3 * (2 - sqrt(3)) }, -0) circle (0.35);

		% Legend for each layer
		\node[align=right, anchor=west] at ({12.0}, +3.75) {Atomic Pointer};
		\draw[fill=dtu-purple] (11.0,  +3.75) circle (0.5);

		\node[align=right, anchor=west] at ({12.0}, +2.25) {Scheduler Metaheuristic};
		\drawHexagon{{11.0  }}{+1.75}{dtu-green}{}{0.5}
		\node[align=right, anchor=west] at ({12.0}, +0.75) {Supervisor Metaheuristic};
		\drawHexagon{{11.0  }}{+0.25}{dtu-red}{}{0.5}
		\node[align=right, anchor=west] at ({12.0}, -0.75) {Technician Metaheuristic};
		\drawHexagon{{11.0  }}{-1.25}{dtu-corporate-red}{}{0.5}
		\node[align=right, anchor=west] at ({12.0}, -2.25) {User interfaces (Message Passing)};
		\draw[fill=dtu-yellow] (11.0, -2.25) circle (0.5);
	\fi

	\ifmetaheuristicslayer
		\drawHexagon{ 2                      }{ 2}{dtu-blue}{Strategic}{2}
		\drawHexagon{{6 - 2 * (2 - sqrt(3)) }}{ 2}{dtu-green}{Tactical}{2}
		\drawHexagon{{4 - 1 * (2 - sqrt(3)) }}{-1}{dtu-red}{Supervisor}{2}
		\drawHexagon{{0 + 1 * (2 - sqrt(3)) }}{-1}{dtu-red}{Supervisor}{2}
		\drawHexagon{{8 - 3 * (2 - sqrt(3)) }}{-1}{dtu-red}{Supervisor}{2}

		\drawHexagon{{2 - 0 * (2 - sqrt(3)) }}{-4}{dtu-corporate-red}{Technician}{2}
		\drawHexagon{{6 - 2 * (2 - sqrt(3)) }}{-4}{dtu-corporate-red}{Technician}{2}

		\drawHexagon{{10 - 4 * (2 - sqrt(3)) }}{-4}{dtu-corporate-red}{Technician}{2}
		\drawHexagon{{-2 + 2 * (2 - sqrt(3)) }}{-4}{dtu-corporate-red}{Technician}{2}

		\drawHexagon{{12 - 5 * (2 - sqrt(3)) }}{-1}{dtu-corporate-red}{Technician}{2}
		\drawHexagon{{-4 + 3 * (2 - sqrt(3)) }}{-1}{dtu-corporate-red}{Technician}{2}

		% Legend for each layer
		\drawHexagon{{14.0  }}{+3.0}{dtu-white}{}{0.75}
		\node[align=right, anchor=west] at ({15.0}, +3.75) {Persistence};
		\drawHexagon{{14.0  }}{+1.5}{dtu-white}{}{0.75}
		\node[align=right, anchor=west] at ({15.0}, +2.25) {Atomic Pointer};
		\drawHexagon{{14.0  }}{+0.0}{dtu-corporate-red}{}{0.75}
		\node[align=right, anchor=west] at ({15.0}, +0.75) {Metaheuristics};
		\drawHexagon{{14.0  }}{-1.5}{dtu-white}{}{0.75}
		\node[align=right, anchor=west] at ({15.0}, -0.75) {Orchestration};
		\drawHexagon{{14.0  }}{-3.0}{dtu-white}{}{0.75}
		\node[align=right, anchor=west] at ({15.0}, -2.25) {User interfaces};
	\fi

	\iforchestratorlayer
		\drawHexagon{ 2                      }{ 2}{dtu-orange}{}{2}
		\drawHexagon{{6 - 2 * (2 - sqrt(3)) }}{ 2}{dtu-orange}{}{2}
		\drawHexagon{{4 - 1 * (2 - sqrt(3)) }}{-1}{dtu-orange}{Orche-\\strator}{2}
		\drawHexagon{{0 + 1 * (2 - sqrt(3)) }}{-1}{dtu-orange}{}{2}
		\drawHexagon{{8 - 3 * (2 - sqrt(3)) }}{-1}{dtu-orange}{}{2}

		\drawHexagon{{2 - 0 * (2 - sqrt(3)) }}{-4}{dtu-orange}{}{2}
		\drawHexagon{{6 - 2 * (2 - sqrt(3)) }}{-4}{dtu-orange}{}{2}

		\drawHexagon{{10 - 4 * (2 - sqrt(3)) }}{-4}{dtu-orange}{}{2}
		\drawHexagon{{-2 + 2 * (2 - sqrt(3)) }}{-4}{dtu-orange}{}{2}

		\drawHexagon{{12 - 5 * (2 - sqrt(3)) }}{-1}{dtu-orange}{}{2}
		\drawHexagon{{-4 + 3 * (2 - sqrt(3)) }}{-1}{dtu-orange}{}{2}
		% Legend for each layer
		\drawHexagon{{14.0  }}{+3.0}{dtu-white}{}{0.75}
		\node[align=right, anchor=west] at ({15.0}, +3.75) {Persistence};
		\drawHexagon{{14.0  }}{+1.5}{dtu-white}{}{0.75}
		\node[align=right, anchor=west] at ({15.0}, +2.25) {Atomic Pointer};
		\drawHexagon{{14.0  }}{+0.0}{dtu-white}{}{0.75}
		\node[align=right, anchor=west] at ({15.0}, +0.75) {Metaheuristics};
		\drawHexagon{{14.0  }}{-1.5}{dtu-orange}{}{0.75}
		\node[align=right, anchor=west] at ({15.0}, -0.75) {Orchestration};
		\drawHexagon{{14.0  }}{-3.0}{dtu-white}{}{0.75}
		\node[align=right, anchor=west] at ({15.0}, -2.25) {User interfaces};
	\fi

	
	\ifuserinterfacelayer
		\drawHexagon{ 2                      }{ 2}{dtu-yellow}{UI}{2}
		\drawHexagon{{6 - 2 * (2 - sqrt(3)) }}{ 2}{dtu-yellow}{UI}{2}
		\drawHexagon{{4 - 1 * (2 - sqrt(3)) }}{-1}{dtu-yellow}{UI}{2}
		\drawHexagon{{0 + 1 * (2 - sqrt(3)) }}{-1}{dtu-yellow}{UI}{2}
		\drawHexagon{{8 - 3 * (2 - sqrt(3)) }}{-1}{dtu-yellow}{UI}{2}

		\drawHexagon{{2 - 0 * (2 - sqrt(3)) }}{-4}{dtu-yellow}{UI}{2}
		\drawHexagon{{6 - 2 * (2 - sqrt(3)) }}{-4}{dtu-yellow}{UI}{2}

		\drawHexagon{{10 - 4 * (2 - sqrt(3)) }}{-4}{dtu-yellow}{UI}{2}
		\drawHexagon{{-2 + 2 * (2 - sqrt(3)) }}{-4}{dtu-yellow}{UI}{2}

		\drawHexagon{{12 - 5 * (2 - sqrt(3)) }}{-1}{dtu-yellow}{UI}{2}
		\drawHexagon{{-4 + 3 * (2 - sqrt(3)) }}{-1}{dtu-yellow}{UI}{2}
		% Legend for each layer
		\drawHexagon{{14.0  }}{+3.0}{dtu-white}{}{0.75}
		\node[align=right, anchor=west] at ({15.0}, +3.75) {Persistence};
		\drawHexagon{{14.0  }}{+1.5}{dtu-white}{}{0.75}
		\node[align=right, anchor=west] at ({15.0}, +2.25) {Atomic Pointer};
		\drawHexagon{{14.0  }}{+0.0}{dtu-white}{}{0.75}
		\node[align=right, anchor=west] at ({15.0}, +0.75) {Metaheuristics};
		\drawHexagon{{14.0  }}{-1.5}{dtu-white}{}{0.75}
		\node[align=right, anchor=west] at ({15.0}, -0.75) {Orchestration};
		\drawHexagon{{14.0  }}{-3.0}{dtu-yellow}{}{0.75}
		\node[align=right, anchor=west] at ({15.0}, -2.25) {User interfaces};
	\fi
	
	\end{tikzpicture}
}

	\resizebox{0.7\textwidth}{!}{
		\drawModelSetupHexagon[userinterface=true]
	}
	\caption{
		Overview of the scheduling process when modelled as actors. When LNS is encapsulated 
		is an actor it becomes possible to optimize parts of a large process individually instead of 
		optimizing the scheduling problem globally from a single model implementation.
	}
	\label{fig:ordinator-hexagon:userinterfaces}
\end{figure}

The next step in this direction will be to model the remaining stakeholders as their own 
AbLNS metaheuristics, and then make them communicate together through atomic pointer swaps and message
passing. This enables modular concurrency at each layer and ensures real-time
synchronization across multiple optimization levels. Making each metaheuristic expose solutions to each 
other in real-time providing each other with high quality parameters.

\section{Conclusion}
Many current planning problems that industry faces are combinatorial by nature, 
and many combinatorial problems have to be solved continuously to make operations 
run efficiently. For operation research (OR) to be helpful in this process, the solution methods 
should be a minimally invasive in the workflow of the working stakeholders. 
The AbLNS solution approach detailed in this paper aligns
closely with two known problems in operation research: the lack of integration and the issues of 
coordination in multi-actor processes. For these reasons we argue that the
"standard" Operations Research approach of first collecting data, then creating a
model and optimizing it, and then finally providing the solution to the planners
in the company workflow, is not a scalable approach in many situations.

We have here demonstrated that the AbLNS approach works in a practical 
maintenance scheduling setting at Total Energies.

We believe that this
approach of combining a number of smaller planning/optimization problems with
different actors/stakeholders responsible for their part of the overall
solution is the future way of integrating Operation Research techniques in practice.

Modern industrial companies also have the available IT-infrastructure to support
and connect model/metaheuristics together with the relevant actors/stakeholders 
in a way that was not possible just 10 years ago. 

This also align with anacdotal evidence from practical operation research 
that smaller "quick and dirty" often works better in practice. 


The fundamental problem with the "older" paradigm is that optimizing across
actors/stakeholders is very difficult, leading the literature to prefer
integrated models instead of decomposing the model by each of the
processes that make up a business process such as maintenance scheduling.
This paper argues that this is mainly due to an dated understanding of
software architecture that is available today in industry, but not
acknowledged by broader the Operations Research and Metaheuristic
communities \citep{talbiMetaheuristicsDesignImplementation2009},
\citep{gendreauHandbookMetaheuristics2019}.
