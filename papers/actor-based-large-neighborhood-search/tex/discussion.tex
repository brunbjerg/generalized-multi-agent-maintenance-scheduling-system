\section{Discussion}
\label{sec:4-discussion}
The maintenance scheduling effectively solves a complex scheduling problem by
using multiple actors. Through the use of actors the scheduling process handles
uncertainty that is difficult to reason about in a single mathematical model are
easily solved through coordination. Each stakeholder in the process has his own
model each with different levels of aggregation, where each actor understands 
how to exploit his own model.
To further understand the approach
the discussion will be divided into three sections: \ref{sec:discussion:actors_and_integration} 
actors and integration;
\ref{sec:discussion:continuous_optimization} continuous optimization allows asynchronous optimization; 
and \ref{sec:discussion:future_research} future research.

\subsection{Actors \& Integration}
\label{sec:discussion:actors_and_integration}
Often in operation research the failure to reliably solve operational problems in 
industry are not due to the problems being computationally intractable \cite{gendreauHandbookMetaheuristics2019} but a
practical problem of connecting data streams so that the solution approach continually
receives dynamic data to handle changes and then providing the resulting solutions
to the relevant actors (stakeholders) through a relevant interface \cite{meignanReviewTaxonomyInteractive2015}.
The actor-based approach proposed in this paper makes integration easier by
naturally encapsulating a model with a reliable interface. 

\subsection{Continuous Optimization}
\label{sec:discussion:continuous_optimization}
With actor-based metaheuristics, the optimization loop can run indefinitely,
optimizing based on the latest available information. This may seem like a
detail as you could argue that you should only ever optimize the schedule
when there is an explicit need for it, but consider the case when you start
adding more than two actors to a scheduling system, then there arises a need
to coordinate people temporally as each will have to run their optimizing
process one after another. This is depicted in figure~\ref{fig:discussion:hierarchical_model_setup}
where the output of one model is used as the input to the next one, leading
to the hierarchical model setup.

\begin{figure}[H]
	\usetikzlibrary{positioning}
\usetikzlibrary{arrows.meta}
\usetikzlibrary{bending}


\definecolor{red}{HTML}{8A3F3A}
\definecolor{yellow}{HTML}{E0BB3C}
\definecolor{blue}{HTML}{4569E0}
\definecolor{green}{HTML}{17E561}
\definecolor{other}{HTML}{6A939E}


\newlength{\basisb}
\setlength{\basisb}{0.4cm}

\centering
\begin{tikzpicture}[line width=0.0\basisb]
    \draw (2.0\basisb,4.0\basisb) 
		node[rotate=90, minimum height=3\basisb,fill=dtu-blue,minimum width=8\basisb,rounded corners=0.1\basisb] 
			(Dynamic Data) {Dynamic Data};

    \draw (8.0\basisb,7.0\basisb) 
		node[minimum height=2\basisb,fill=dtu-red,minimum width=6\basisb,rounded corners=0.1\basisb] 
			(Scheduler) {Scheduler};
    \draw (14.0\basisb,4.0\basisb) 
		node[minimum height=2\basisb,fill=dtu-red,minimum width=6\basisb,rounded corners=0.1\basisb] 
			(Supervisor) {Supervisor};
    \draw (20.0\basisb,1.0\basisb) 
		node[minimum height=2\basisb,fill=dtu-red,minimum width=6\basisb,rounded corners=0.1\basisb] 
			(Technician) {Technician};

    \draw (26.0\basisb,7.0\basisb) 
		node[minimum height=2\basisb,fill=dtu-yellow,minimum width=3\basisb,rounded corners=0.1\basisb] 
			(UserInterface1) {UI};
    \draw (26.0\basisb,4.0\basisb) 
		node[minimum height=2\basisb,fill=dtu-yellow,minimum width=3\basisb,rounded corners=0.1\basisb] 
			(UserInterface2) {UI};
    \draw (26.0\basisb,1.0\basisb) 
		node[minimum height=2\basisb,fill=dtu-yellow,minimum width=3\basisb,rounded corners=0.1\basisb] 
			(UserInterface3) {UI};

	\draw[<->, line width=0.1\basisb,color=dtu-green] (5.0\basisb, -1\basisb) -- (23.0\basisb, -1\basisb);
	\draw (14.0\basisb, -2.0\basisb) node {Significant Amount of Time};

	\draw[->,>=Triangle, thick, line width=0.1\basisb, color=dtu-corporate-red] (Scheduler) to[out=0, in=90,looseness=1.5]  (Supervisor);
	\draw[->,>=Triangle, thick, line width=0.1\basisb, color=dtu-corporate-red] (Supervisor) to[out=0, in=90,looseness=1.5] (Technician);
	\draw[->,>=Triangle, thick, line width=0.1\basisb] (Dynamic Data.south) ++(0\basisb,3.0\basisb) to[out=0, in=180,looseness=1.0] (Scheduler);
	\draw[->,>=Triangle, thick, line width=0.1\basisb] (Dynamic Data.south) to[out=0, in=180,looseness=1.0] (Supervisor);
	\draw[->,>=Triangle, thick, line width=0.1\basisb] (Dynamic Data.south) ++(0\basisb,-3.0\basisb) to[out=0, in=180,looseness=1.0] (Technician.west);
	\draw[<-,>=Triangle, thick, line width=0.1\basisb] (UserInterface1) to[out=180, in=0,looseness=1.0] (Scheduler);
	\draw[<-,>=Triangle, thick, line width=0.1\basisb] (UserInterface2) to[out=180, in=0,looseness=1.0] (Supervisor);
	\draw[<-,>=Triangle, thick, line width=0.1\basisb] (UserInterface3) to[out=180, in=0,looseness=1.0] (Technician);
	% \draw[<->, thick, line width=0.1\basisb] (Scheduler) -- (UserInterface);
	\begin{scope}[shift={(7,0)}] % Adjust shift to position the legend
    % Legend box
    % Legend lines and text
	    \draw[thick, line width=0.1\basisb] (-1.5,2.4) node[right, font=\footnotesize, align=center] {Solution\\Transfer};
	    \draw[thick, line width=0.1\basisb] (1.0,1.2) node[right, font=\footnotesize, align=center] {Solution\\Transfer};
	\end{scope}
\end{tikzpicture}

	\label{fig:discussion:hierarchical_model_setup}
	\caption{Effects us using hierarchical models setups in human-guided search metaheuristics.
	Due to the dependent nature of each metaheuristics it becomes crucial that the running of 
	the metaheuristics are well coordinated between the metaheuristics.}
\end{figure}

In practice there are multiple problems with using a hierarchical setup.
Usually the largest one is that the information and knowledge needed to 
perform a feasible schedule is usually found in the lower levels of the 
hierarchicy. In maintenance scheduling the operational setting, where the
technicians are working, is usually so complex that it not feasible to 
centralize the knowledge that is required to create and execute a 
schedule. Figure~\ref{fig:discussion:asynchronous_setup}
shows the kind of non-hierarchical setup that an actor-based approach 
allows for.

\begin{figure}[H]
	\usetikzlibrary{positioning}
\usetikzlibrary{arrows.meta}
\usetikzlibrary{bending}
\usetikzlibrary{backgrounds}


\definecolor{red}{HTML}{8A3F3A}
\definecolor{yellow}{HTML}{E0BB3C}
\definecolor{blue}{HTML}{4569E0}
\definecolor{green}{HTML}{17E561}
\definecolor{other}{HTML}{6A939E}


\newlength{\basisc}
\setlength{\basisc}{0.5cm}

\centering
\begin{tikzpicture}[line width=0.0\basisc]
    \draw (4.0\basisc,4.0\basisc) 
		node[rotate=90, minimum height=3\basisc,fill=dtu-blue,minimum width=8\basisc,rounded corners=0.1\basisc] 
			(Dynamic Data) {Dynamic Data};

			

    \draw (8.0\basisc,10.0\basisc) 
		node[minimum height=2\basisc,fill=dtu-yellow,minimum width=3\basisc,rounded corners=0.1\basisc] 
			(UserInterface1) {UI};
    \draw (12.0\basisc,10.0\basisc) 
		node[minimum height=2\basisc,fill=dtu-yellow,minimum width=3\basisc,rounded corners=0.1\basisc] 
			(UserInterface2) {UI};
    \draw (16.0\basisc,10.0\basisc) 
		node[minimum height=2\basisc,fill=dtu-yellow,minimum width=3\basisc,rounded corners=0.1\basisc] 
			(UserInterface3) {UI};
    \draw (12.0\basisc,7.0\basisc) 
		node[minimum height=2\basisc,fill=dtu-red,minimum width=11\basisc,rounded corners=0.1\basisc] 
			(Scheduler) {Scheduler};
    \draw (12.0\basisc,4.0\basisc) 
		node[minimum height=2\basisc,fill=dtu-red,minimum width=11\basisc,rounded corners=0.1\basisc] 
			(Supervisor) {Supervisor};
    \draw (12.0\basisc,1.0\basisc) 
		node[minimum height=2\basisc,fill=dtu-red,minimum width=11\basisc,rounded corners=0.1\basisc] 
			(Technician) {Technician};


	\begin{pgfonlayer}{background}
		\draw[<->, thick, line width=0.1\basisc] (UserInterface1) to[out=-90, in=90,looseness=1.0] ++(0\basisc,-2.0\basisc)(Scheduler);
		\draw[<->, thick, line width=0.1\basisc] (UserInterface2) to[out=-90, in=90,looseness=1.0] (Supervisor);
		\draw[<->, thick, line width=0.1\basisc] (UserInterface3) to[out=-90, in=90,looseness=1.0] ++(0\basisc,-8.0\basisc)(Technician);

	\end{pgfonlayer}

	\draw[->, line width=0.1\basisc,color=dtu-green] (6.5\basisc, -1\basisc) -- (17.5\basisc, -1\basisc);
	\draw (12.0\basisc, -2.0\basisc) node {Running Continuously};

	\draw[<->, thick, line width=0.1\basisc, color=dtu-corporate-red] (Scheduler)++(-3\basisc, -1.0\basisc) to[out=-90, in=90,looseness=1.0]  ++(0\basisc, -1.0\basisc)(Supervisor);
	\draw[<->, thick, line width=0.1\basisc, color=dtu-corporate-red] (Scheduler)++(-2\basisc, -1.0\basisc) to[out=-90, in=90,looseness=1.0]  ++(0\basisc, -1.0\basisc)(Supervisor);
	\draw[<->, thick, line width=0.1\basisc, color=dtu-corporate-red] (Scheduler)++(2\basisc, -1.0\basisc) to[out=-90, in=90,looseness=1.0]  ++(0\basisc, -1.0\basisc)(Supervisor);

	\draw[<->, thick, line width=0.1\basisc, color=dtu-corporate-red] (Supervisor)++(-4\basisc, -1.0\basisc) to[out=-90, in=90,looseness=1.0] ++(0\basisc, -1.0\basisc)(Technician);
	\draw[<->, thick, line width=0.1\basisc, color=dtu-corporate-red] (Supervisor)++(-1\basisc, -1.0\basisc) to[out=-90, in=90,looseness=1.0] ++(0\basisc, -1.0\basisc)(Technician);
	\draw[<->, thick, line width=0.1\basisc, color=dtu-corporate-red] (Supervisor)++(3\basisc, -1.0\basisc) to[out=-90, in=90,looseness=1.0] ++(0\basisc, -1.0\basisc)(Technician);

	\draw[<->, thick, line width=0.1\basisc] (Dynamic Data.south) ++(0\basisc,3.0\basisc) to[out=0, in=180,looseness=1.0] (Scheduler);
	\draw[<->, thick, line width=0.1\basisc] (Dynamic Data.south) to[out=0, in=180,looseness=1.0] (Supervisor);
	\draw[<->, thick, line width=0.1\basisc] (Dynamic Data.south) ++(0\basisc,-3.0\basisc) to[out=0, in=180,looseness=1.0] (Technician.west);

	% \draw[<->, thick, line width=0.1\basisc] (Scheduler) -- (UserInterface);
\end{tikzpicture}

	\caption{Asynchronous model setup where each metaheuristic runs in perpetuity. In this setup
	there is no need to coordinate stakeholders to run the metaheuristics. Each actor in the 
	scheduling process will always have the solutions of the other stakeholder available to 
	him to guide his own search.}
	\label{fig:discussion:asynchronous_setup}
\end{figure}

When the optimization approaches, optimize continuously it becomes possible
to enable tight integration between the different model implementations. Where 
instead of running models to completion you simply handle changes in model 
parameters, model solutions, user inputs, and in the dynamic data source as 
they occur instead of making restarts.

\subsection{Future Research}
\label{sec:discussion:future_research}
The future research of this project is to demonstrate that
the actor-based approach described here can be used to model and optimize 
multi-actor scheduling processes. 
Figure~\ref{fig:ordinator-architecture}
shows a scheduling system architecture where each of the actors run an actor-based metaheuristic
and that each metaheuristic will share its solutions with the other
metaheuristics through atomic pointer swapping. Communicate with the end-user
through message passing, integrate with a persistence storage through mutex
locks, and the lifecycle of each of the metaheuristics will be controlled by
the orchestrator through message passing. 

\begin{figure}[H]
	\centering
	\usetikzlibrary {positioning}


\definecolor{red}{HTML}{8A3F3A}
\definecolor{yellow}{HTML}{E0BB3C}
\definecolor{blue}{HTML}{4569E0}
\definecolor{green}{HTML}{17E561}
\definecolor{other}{HTML}{6A939E}

\newcommand{\ModelColor}{red}
\newcommand{\UserInterfaceColor}{yellow}
\newcommand{\PersistenceColor}{blue}
\newcommand{\PointerSwapColor}{green}
\newcommand{\OrchestratorColor}{other}

\pgfkeys{
	/graph/.is family, /graph,
	default/.style = {
		show_shared_pointer = false,
		show_orchestrator = false,
		show_persistence = false,
		show_user_interface = false,
		basis/.estore in = 2cm,
	},
	show_shared_pointer/.estore in = \ShowSharedSolutionCommunication,
	show_orchestrator/.estore in = \ShowOrchestratorCommunication,
	show_persistence/.estore in = \ShowPersistenceCommunication,
	show_user_interface/.estore in = \ShowUserInterfaceCommunication,
	basis/.estore in = \basisinput,
}
\newlength{\basis}
\tikzset{
  basis/.code={\setlength{\basis}{\basisinput}}, % TikZ assignment code
  basis/.default=1cm,                   % Provide a default (\b@sis is undefined/unassigned)
  basis,                                % Set initial Value (\b@sis is defined/assigned)
 }

\newcommand{\drawGraph}[1]{
	\pgfkeys{/graph, default, #1}
	
	\begin{tikzpicture}[scale=0.75][
		% Define styles and settings
		node distance=2cm,
		block/.style={rectangle, draw, fill=blue!20, text centered, minimum height=3em},
		arrow/.style={-Stealth, thick}
		]


		\ifthenelse{\equal{\ShowOrchestratorCommunication}{true}}{
			\draw[color=other,-, ultra thick] (Strategic) -- (Orchestrator);
			\draw[color=other,-, ultra thick] (Tactical) -- (Orchestrator);
			\draw[color=other,-, ultra thick] (Supervisor) -- (Orchestrator);
			\draw[color=other,-, ultra thick] (Operational_1) -- (Orchestrator);
			\draw[color=other,-, ultra thick] (Operational_2) -- (Orchestrator);
			\draw[color=other,-, ultra thick] (Operational_3) -- (Orchestrator);
		}{}
		% \draw[help lines] (0\basis, 0\basis) grid (10\basis, 8\basis);
		\draw (5\basis,4\basis) node[minimum height=5cm,minimum width=7.0cm,rounded corners=5pt] {};

	    \draw (2.5\basis,5.5\basis) node[minimum height=1cm,minimum width=1cm,fill=\ModelColor,rounded corners=5pt] (Strategic) {Stra};
	    \draw (5.0\basis,4.0\basis) node[minimum height=1cm,minimum width=1\basis,fill=\ModelColor,rounded corners=5pt] (Supervisor) {Sup};
		\draw (7.5\basis,5.5\basis) node[minimum height=1cm,minimum width=1cm,fill=\ModelColor,rounded corners=5pt] (Tactical) {Tac};

		\draw (2.5\basis,2.5\basis) node[minimum height=1cm,minimum width=1cm,fill=\ModelColor,rounded corners=5pt] (Operational_1) {$O_{1}$};
		\draw (5.0\basis,2.5\basis) node[minimum height=1cm,minimum width=1cm,fill=\ModelColor,rounded corners=5pt] (Operational_2) {$O_{2}$};
		\draw (7.5\basis,2.5\basis) node[minimum height=1cm,minimum width=1cm,fill=\ModelColor,rounded corners=5pt,rounded corners=5pt] (Operational_3) {$O_{3}$};
	
		\draw (Strategic) edge (Tactical);
		\draw (Strategic) edge (Tactical);
		\draw (5\basis,5.5\basis) edge (Supervisor);
		\draw (Supervisor) edge (Operational_1);
		\draw (Supervisor) edge (Operational_2);
		\draw (Supervisor) edge (Operational_3);
		\draw (5.0\basis,0.5\basis)   node[minimum height=1cm,minimum width=5.0cm,            fill=\PersistenceColor,rounded corners=5pt] {SchedulingEnvironment};
		\draw (5.0\basis,7.5\basis)   node[minimum height=1cm,minimum width=5.0cm,            fill=\OrchestratorColor,rounded corners=5pt] (Orchestrator) {Orchestrator};
		\draw (0.5\basis,4.0\basis)   node[rotate=90, minimum height=1cm, minimum width=3.25cm,fill=\PointerSwapColor,rounded corners=5pt] {SharedSolution};
		\draw (9.5\basis,5.5\basis) node[rotate=90, minimum height=1cm, minimum width=1cm,fill=\UserInterfaceColor,rounded corners=5pt] {UI};
		\draw (9.5\basis,4.0\basis)   node[rotate=90, minimum height=1cm, minimum width=1cm,fill=\UserInterfaceColor,rounded corners=5pt] {UI};
		\draw (9.5\basis,2.5\basis) node[rotate=90, minimum height=1cm, minimum width=1cm,fill=\UserInterfaceColor,rounded corners=5pt] {UI};

		% Legend
		\begin{scope}[shift={(10.6\basis,5.7\basis)}]
			\node at (-0.25\basis,1\basis) [right] {Communication Type};
			\draw[color=\OrchestratorColor,fill] (0\basis,0.0\basis) rectangle (0.5cm, 0.5cm);
			\node[anchor=west] at (0.5\basis, 0.25\basis) {\scriptsize Channels};
			\draw[color=\PointerSwapColor,fill] (0\basis,-1.0\basis) rectangle(0.5cm, -0.5cm); 
			\node[anchor=west] at (0.5\basis, -0.75\basis) {\scriptsize Atomic Pointer Swap};
			\draw[color=\ModelColor,fill] (0\basis,-2.0\basis) rectangle(0.5cm, -1.5cm); 
			\node[anchor=west] at (0.5\basis, -1.75\basis) {\scriptsize Metaheurics};
			\draw[color=\PersistenceColor,fill] (0\basis,-3.0\basis) rectangle(0.5cm, -2.5cm); 
			\node[anchor=west] at (0.5\basis, -2.75\basis) {\scriptsize Mutex lock};
			\draw[color=\UserInterfaceColor,fill] (0\basis,-4.0\basis) rectangle(0.5cm, -3.5cm); 
			\node[anchor=west] at (0.5\basis, -3.75\basis) {\scriptsize Channels};
		\end{scope}
		\ifthenelse{\equal{\ShowSharedSolutionCommunication}{true}}{
			\draw[->, thick] (Strategic) -- (Orchestrator);
		}{}
		\ifthenelse{\equal{\ShowUserInterfaceCommunication}{true}}{
			\draw[->, thick] (Strategic) -- (Orchestrator);
		}{}
		\ifthenelse{\equal{\ShowPersistenceCommunication}{true}}{
			\draw[->, thick] (Strategic) -- (Orchestrator);
		}{}
		

	\end{tikzpicture}
}


	\drawOrdinatorArchitecture{basisinput=1cm}
	\caption{
		Overview of the scheduling process when modelled as actors. When LNS is encapsulated 
		is an actor it becomes possible to optimize parts of a large process individually instead of 
		optimizing the scheduling problem globally from a single model implementation.
	}
	\label{fig:ordinator-architecture}
\end{figure}
The next step in this research direction will be to model the remaining stakeholder as their own 
AbLNS metaheuristics, and then make them communicate together. Each exposing solutions to each 
other in real-time.
