%% 
%% Copyright 2007-2024 Elsevier Ltd
%% 
%% This file is part of the 'Elsarticle Bundle'.
%% ---------------------------------------------
%%  
%% It may be distributed under the conditions of the LaTeX Project Public
%% License, either version 1.3 of this license or (at your option) any%% later version.  The latest version of this license is in
%%    http://www.latex-project.org/lppl.txt
%% and version 1.3 or later is part of all distributions of LaTeX
%% version 1999/12/01 or later.
%% 
%% The list of all files belonging to the 'Elsarticle Bundle' is
 %% given in the file `manifest.txt'.
%% 
%% Template article for Elsevier's document class `elsarticle'
%% with harvard style bibliographic references

\documentclass[preprint,12pt,authoryear]{elsarticle}

%% Use the option review to obtain double line spacing
%% \documentclass[authoryear,preprint,review,12pt]{elsarticle}

%% Use the options 1p,twocolumn; 3p; 3p,twocolumn; 5p; or 5p,twocolumn
%% for a journal layout:
%% \documentclass[final,1p,times,authoryear]{elsarticle}
%% \documentclass[final,1p,times,twocolumn,authoryear]{elsarticle}
%% \documentclass[final,3p,times,authoryear]{elsarticle}
%% \documentclass[final,3p,times,twocolumn,authoryear]{elsarticle}
%% \documentclass[final,5p,times,authoryear]{elsarticle}
%% \documentclass[final,5p,times,twocolumn,authoryear]{elsarticle}

%% For including figures, graphicx.sty has been loaded in
%% elsarticle.cls. If you prefer to use the old commands
%% please give \usepackage{epsfig}

%% The amssymb package provides various useful mathematical symbols
\usepackage{amssymb}
\usepackage{graphicx}
%% The amsmath package provides various useful equation environments.
\usepackage{amsmath}
\usepackage{algorithm}
\usepackage{algorithmicx}
\usepackage{algpseudocode}
%% The amsthm package provides extended theorem environments
%% \usepackage{amsthm}

% Sets
\newcommand{\SetWorkOrder}[1]{W(\VarMetaTime, \VarStrategicWorkOrderAssignment{}{})}
\newcommand{\SetPeriod}{P(\VarMetaTime)}
\newcommand{\SetResource}{R(\VarMetaTime)} %RECONSIDER | RELEVANT FOR BOTH STRATEGIC AND TACTICAL
\newcommand{\SetOperation}[2]{O_{#1}(\VarMetaTime, #2 )}

\newcommand{\SetDays}[1]{D_{#1}(\VarMetaTime)}
\newcommand{\SetActivity}[2]{A_{#2}(\VarMetaTime, #1)}
\newcommand{\SetTechnician}{T(\VarMetaTime)}
\newcommand{\SetWorkSegment}{K(\VarSupervisorAssignment{}{})}

\newcommand{\SetTimeInstance}{I(\VarMetaTime)}
\newcommand{\SetEvent}{E(\VarMetaTime)}

% Parameters
\newcommand{\ParStrategicValue}{strategic\_value_{wp}(\VarMetaTime)}
\newcommand{\ParStrategicPenalty}{strategic\_penalty}
\newcommand{\ParClusteringValue}{clustering\_value_{w1, w2}}
\newcommand{\ParStrategicResource}{resource_{pr}(\VarMetaTime)}

\newcommand{\ParStrategicWorkOrderWeight}{work\_order\_work_{wr}}
\newcommand{\ParStrategicInclude}{include(\VarMetaTime)}
\newcommand{\ParStrategicExclude}{exclude(\VarMetaTime)}
\newcommand{\ParTacticalValue}{tactical\_value_{do}(\VarMetaTime)}

\newcommand{\ParTacticalPenalty}{tactical\_penalty}
\newcommand{\ParOperationWork}[1]{work_{#1}(\VarMetaTime)}
\newcommand{\ParTacticalResource}{tactical\_resource_{dr}(\VarMetaTime)}
\newcommand{\ParStartStart}{start\_start_{o1, o2}}

\newcommand{\ParFinishStart}{finish\_start_{o1, o2}}
\newcommand{\ParNumberOfPeople}{number_{o}(\VarMetaTime)}
\newcommand{\ParOperatingTime}{operating\_time_{o}}
\newcommand{\ParDurationLower}{duration\_lower_{o}(\VarMetaTime)}

\newcommand{\ParDurationUpper}{duration\_upper_{o}(\VarMetaTime)}
\newcommand{\ParSupervisorValue}{supervisor\_value_{at}(\VarMetaTime)} % We should determine if the supervisor should assign for each operation or for each activity. My guts say that it should be for each activity
\newcommand{\ParFeasible}{feasible_{at}(\VarIncludeActivity{})}
\newcommand{\ParOperationsForWorkOrder}{work\_order\_to\_operations_{w}}

\newcommand{\ParOperationsInWorkOrder}{operations\_in\_work\_order_{w}}
\newcommand{\ParActivitiesForOperation}{activities\_for\_operation_{o}}
\newcommand{\ParLowerActivityWork}{lower\_activity\_work_{a}(\VarMetaTime)}
\newcommand{\ParActivityWork}[1]{activity\_work_{a}(\VarMetaTime, \VarActivityWork{#1})}
\newcommand{\ParPreparation}{preparation_{a1, a2}}

\newcommand{\ParEvent}{event_{ie}}
\newcommand{\ParEventDuration}{duration_{ie}}
\newcommand{\ParConstraintLimit}{constraint\_limit}
\newcommand{\ParTimeWindowStart}{time\_window\_start_{a}(\VarTacticalWork{}{})}

\newcommand{\ParTimeWindowFinish}{time\_window\_finish_{a}(\VarTacticalWork{}{})}
\newcommand{\ParAvailabilityStart}{availability\_start(\VarMetaTime)}
\newcommand{\ParAvailabilityFinish}{availability\_finish(\VarMetaTime)}

% Variables
\newcommand{\VarStrategicWorkOrderAssignment}[2]{\alpha_{#1#2}(\VarMetaTime)}
\newcommand{\VarStrategicExcess}{\epsilon_{pr}(\VarMetaTime)}
\newcommand{\VarTacticalWork}[2]{\beta_{#1#2}(\VarMetaTime)}
\newcommand{\VarTacticalExcess}{\mu_{rd}(\VarMetaTime)} : excess capacity for each day

\newcommand{\VarTacticalWorkBinary}[2]{\sigma_{#1#2}(\VarMetaTime)}
\newcommand{\VarTacticalWorkBinaryConsecutive}{\eta_{do}(\VarMetaTime)}
\newcommand{\VarTacticalOperationDifference}{\Delta_{o}(\VarMetaTime)}
\newcommand{\VarSupervisorAssignment}[2]{\gamma_{#1#2}(\VarMetaTime)}
\newcommand{\VarSupervisorAssignmentWhole}{\phi_{o}(\VarMetaTime)}

\newcommand{\VarActivityWork}[1]{\rho_{#1}(\VarMetaTime)}

\newcommand{\VarProcessingTime}{\delta_{ak}(\VarMetaTime)} 
\newcommand{\VarActiveSegment}[2]{\pi_{#1#2}(\VarMetaTime)}
\newcommand{\VarStartOfSegment}[2]{\lambda_{#1#2}(\VarMetaTime)}
\newcommand{\VarFinishOfSegment}[2]{\Lambda_{#1#2}(\VarMetaTime)}

\newcommand{\VarSegmentInRelation}{\omega_{akie}(\VarMetaTime)}
\newcommand{\VarIncludeActivity}[1]{\theta_{#1}(\VarMetaTime)}

% Meta variables
\newcommand{\VarMetaTime}{\tau}

\usepackage[utf8]{inputenc}
\usepackage[T1]{fontenc}
\usepackage{graphicx}
\usepackage{amsmath}
\usepackage{amssymb}
\usepackage{multicol}
\usepackage{natbib}
\usepackage{hyperref}
\usepackage{booktabs} % For professional quality tables
\usepackage{array}    % For better column alignment
%% The lineno packages adds line numbers. Start line numbering with
%% \begin{linenumbers}, end it with \end{linenumbers}. Or switch it on
%% for the whole article with \linenumbers.
%% \usepackage{lineno}

\journal{Computers and Operation Research}

\begin{document}

\begin{frontmatter}

%% Title, authors and addresses

%% use the tnoteref command within \title for footnotes;
%% use the tnotetext command for theassociated footnote;
%% use the fnref command within \author or \affiliation for footnotes;
%% use the fntext command for theassociated footnote;
%% use the corref command within \author for corresponding author footnotes;
%% use the cortext command for theassociated footnote;
%% use the ead command for the email address,
%% and the form \ead[url] for the home page:
%% \title{Title\tnoteref{label1}}
%% \tnotetext[label1]{}
%% \author{Name\corref{cor1}\fnref{label2}}
%% \ead{email address}
%% \ead[url]{home page}
%% \fntext[label2]{}
%% \cortext[cor1]{}
%% \affiliation{organization={},
%%            addressline={}, 
%%            city={},
%%            postcode={}, 
%%            state={},
%%            country={}}
%% \fntext[label3]{}

\title{Actor-based Large Neighborhood Search for weekly maintenance scheduling} %% Article title

%% use optional labels to link authors explicitly to addresses:
%% \author[label1,label2]{}
%% \affiliation[label1]{organization={},
%%             addressline={},
%%             city={},
%%             postcode={},
%%             state={},
%%             country={}}
%%
%% \affiliation[label2]{organization={},
%%             addressline={},
%%             city={},
%%             postcode={},
%%             state={},
%%             country={}}

\author[DTUconstruct]{Christian Brunbjerg Jespersen} %% Author name
\author[DTUconstruct]{Kristoffer Sigsgaard Wernblad}
\author[DTUmanagement]{Thomas Jacob Riis Stidsen}
\author[DTUconstruct]{Kasper Barslund Hansen}
\author[DTUconstruct]{Jingrui Ge}
\author[DTUconstruct]{Simon Didriksen}
\author[DTUconstruct]{Niels Henrik Mortensen}

%% Author affiliation
\affiliation[DTUconstruct]{organization={DTU Construct, Technical University of Denmark},%Department and Organization
            addressline={Anker Egelundsvej 1}, 
            city={Kongens Lyngby},
            postcode={2800}, 
            state={Hovedstaden},
            country={Denmark}}
\affiliation[DTUmanagement]{organization={DTU Management, Technical University of Denmark},%Department and Organization
            addressline={Anker Egelundsvej 1}, 
            city={Kongens Lyngby},
            postcode={2800}, 
            state={Hovedstaden},
            country={Denmark}}
%\affiliation{organization={Technical University of Denmark},%Department and Organization
%            addressline={Anker Egelundsvej 1}, 
%            city={Kongens Lyngby},
%            postcode={2800}, 
%            state={Hovedstaden},
%            country={Denmark}}
%\affiliation{organization={Technical University of Denmark},%Department and Organization
%            addressline={Anker Egelundsvej 1}, 
%            city={Kongens Lyngby},
%            postcode={2800}, 
%            state={Hovedstaden},
%            country={Denmark}}
%\affiliation{organization={Technical University of Denmark},%Department and Organization
%            addressline={Anker Egelundsvej 1}, 
%            city={Kongens Lyngby},
%            postcode={2800}, 
%            state={Hovedstaden},
%            country={Denmark}}
%\affiliation{organization={Technical University of Denmark},%Department and Organization
%            addressline={Anker Egelundsvej 1}, 
%            city={Kongens Lyngby},
%            postcode={2800}, 
%            state={Hovedstaden},
%            country={Denmark}}
%\affiliation{organization={Technical University of Denmark},%Department and Organization
%            addressline={Anker Egelundsvej 1}, 
%            city={Kongens Lyngby},
%            postcode={2800}, 
%            state={Hovedstaden},
%            country={Denmark}}


%% Abstract
\begin{abstract}
%% Text of abstract
Many planning problems facing the operations research field have proven difficult to solve due to their inherent uncertainty and highly dynamic nature. Stochastic Programming~\citep{birge2011introduction}, fuzzy logic~\citep{zadeh1988fuzzy}, and robust optimization~\citep{ben2009robust} are some of the methods that have been proposed to solve these issues. These methods make an implicit assumption of a static data setting and a static problem setting. Maintenance scheduling is one such problem where the best available information continually updates and then therefore the scheduling continuously needs to be updated. Maintenance scheduling is a complex process often more associated with operation management, but here we will argue that is possible to implement general maintenance scheduling approaches, if the solution approach is designed to be integrated into a business
process of the kind that are usually developed by the principles of operation management. 

This paper proposes a novel optimization method that is capable to optimizing a scheduling problem in the following setting:  Primary data source is changing in real-time; external inputs affects the optimization process; multiple actors are making interdependent decision whose objectives may differ significantly. The proposed solution approach is an actor-based framework including a large neighborhood search metaheuristic implementation. The framework is tested on the real-world problem of maintaining scheduling of oil platforms for Total in the North Sea, but the approach is very general and can be applied to a wide variety of other planning problems.
\end{abstract}

%%Graphical abstract
%\begin{graphicalabstract}
%\includegraphics{grabs}
%\end{graphicalabstract}

%%Research highlights
%\begin{highlights}
%\item How to allow direct and real-time integration into an optimization process?
%\item How to perform optimization in a real-time changing parameter space?
%\end{highlights}

%% Keywords
\begin{keyword}
%% keywords here, in the form: keyword \sep keyword
Large Neighborhood Search \sep Actor Framework \sep Maintenance scheduling \sep Real-time Optimization \sep Human-centered Computing \sep Interactive Systems and Tools \sep Decision Support Systems \sep Interactive Optimization.


%% PACS codes here, in the form: \PACS code \sep code

%% MSC codes here, in the form: \MSC code \sep code
%% or \MSC[2008] code \sep code (2000 is the default)

\end{keyword}

\end{frontmatter}

%% Add \usepackage{lineno} before \begin{document} and uncomment 
%% following line to enable line numbers
%% \linenumbers

%% main text
%%

%% Use \section commands to start a section
\section{Introduction}
\label{sec:1-introduction}
%% Labels are used to cross-reference an item using \ref command.

Maintenance scheduling is an operational problem that have proven hard to solve (NP-hard~\citep{garey1979computers}). Furthermore, for optimization to be utilized the dynamic and uncertain nature of maintenance scheduling requires a tight integration with third party administration software to enable the tacit knowledge of decision makers to influence the planning process easily. Often a number of different decision makers at different company levels take part in the planning process and in this way the industry usually assigns responsibility for decision-making  to an individual representing only a small part of the complete process. 

These multiple smaller planning processes are often difficult to map to a single mathematical model describing the whole system as elaborated by~\citep{barthelemy2002human}. Solving operation research problems that are operational in nature have additional requirements over conventional static problems: they have to be responsive to changing parameters; able to be assimilated into the decision-makers workflow; allow for integration with dynamic data sources such as databases and RESTapi~\citep{meignan_review_2015}. Operational aspects of operation research, as opposed to higher level strategic and tactical aspects, are characterized by extensive amounts negotiation and feedback on proposed schedules. The lack of integration and responsiveness can lead to schedules that are not directly implemented in practice but instead provides initial suggestions~\citep{meignan_review_2015}, which are then iterated else where in the scheduling process. In~\citep{barthelemy2002human} the authors argue that many problems that operation research aim to solve are often composed of a group of individuals whose decisions are consolidated into an "epistemic subject" for which a mathematical model can be formulated and solved, with many scheduling problems being good examples. However often multiple actors have different views on what constitutes an optimal schedule hence resulting in multiple-objectives. Even if multi-objective optimization~\citep{ehrgott2002multiple} is applied to find the Pareto Front~\citep{Pareto1897} a negotiating process still is needed between the actors to select the final schedule.

This paper proposes a solution method that will allow for real-time optimization based on actor/user interaction and connection to a dynamic data source, effectively managing the changes to the parameter space. The proposed solution method will be tested on the multi-compartment multi-knapsack problem (MCMKP) for maintenance scheduling on a large dataset from a company. The MCMKP naturally models what in the practical maintenance is called the weekly schedule, taken form \citep{palmerMaintenancePlanningScheduling2019}. It should be noted that the scientific maintenance scheduling literature deviates significantly from its practical implementation which is detailed in \citep{palmerMaintenancePlanningScheduling2019}. The solution method will by based on the large neighborhood search (LNS) metaheuristic. This meta heuristic was chosen due to its properties of naturally being able to work with and fix infeasible solutions and its state of the art performance on various scheduling problems. 

To understand the need for actor-based methods some background knowledge will be required about the maintenance scheduling process. In figure \ref{fig:simple-maintenance-process} illustrates the general setup of a healthy maintenance planning and scheduling system. The systems actors have the following responsibilities: the planner generates the work orders that are to be scheduled; the scheduler creates weekly schedules based on a knapsack formulation; based on the weekly schedule the supervisor assigns work order activities that the work order is composed of (the assignment problem); the technicians executes the work in sequential pattern (single machine scheduling). A final point on the necessity of actor-based approaches to model should a setup is the idea of ownership of a work order. Throughout the scheduling process a work order is owned by a specific actor and he alone is allow to modify it. This means that a single model approach is very difficult to implement in practice as a work order looks different depending on the actor that currently owns it. This also highlight another an point in maintenance scheduling: that the stochastic nature of the maintenance scheduling process can be handled using a change of model each with different levels of aggregation, opposed to more academic approaches such as fuzzy logic and stochastic optimization.

When the fundamental uncertainties manifest themselves during planning or execution work orders are rescheduled by moving between the different actor (models), meaning that the stochastic elements of maintenance scheduling are handled by dynamic rescheduling between the actors.

\begin{figure}
\includegraphics[width=1.0\textwidth]{figures/Scheduling Process Simple.png}
\caption{Simple overview of the scheduling process with its primary types of actors. The planner, the scheduler, the supervisor(s), and the technicians. The green color highlights the scheduler as it the actor in the maintenance scheduling process that is the foundation for the paper.}
\label{fig:simple-maintenance-process}
\end{figure}

This article describes a number of contributions: 

\begin{itemize}
\item A novel actor/optimization framework
\item A novel specialized (A)LNS metaheuristic, to be utilized in an actor framework
\item Implementation and test on a large realworld maintenance scheduling problem
\end{itemize}

The paper is divided into four different sections. Section \ref{sec:2-solution-method} explains the weekly maintenance scheduling model in detail and forms the fundation of the paper. Section \ref{sec:3-results} shows that results coming from the implemented system where the implementation will be affected by simulated user-interaction. Section \ref{sec:4-discussion} will discuss the implications of the research and possible future research directions.

\subsection{A generic maintenance scheduling model}
\label{sub2sec2}
A large company needs to create a weekly maintenance plan for the next $p \in P$ weeks. The maintenance plan is planned centrally and consists of scheduling the $w \in W$ work orders, i.e. maintenance tasks, such that all are scheduled into different weeks. Each work order requires some resourses to be carried out, e.g. man-power with different qualifications, equipment etc. all of these resourses are available in limited amounts and are called traits $\tau \in T$. To simplify matters, we will assume that the recourse limits are not hard but extra workers can be paid overtime, extra equipment can be rented etc., at the cost (penalty) of $pen_{p\tau}$. The urgency of the different maintenance operations varies and is reflected in a penalty for carrying out a maintenance work order in a certain week $v_{wp}(t)$. Urgent tasks have quickly increasing penalties for the later weeks week $p$. Furthermore, two sets exists which will either will require the work order to be carried out in week $p$, i.e. $(w,p) \in E$ and a set which forbids a work order to carried out in week $p$, i.e. $(w,p) \in I$. The model for the problem is the Multi-compartment Multi-knapsack Problem with capacity penalties MCMKP.

The notation used in the model is based on the notation from the dynamic metaheuristics literature as found in \cite{yangMetaheuristicsDynamicCombinatorial2013}, where $t$ is added as a time variable on all sets, parameters, and variables that are subject to change.
This allows us to be precise in the timing on the messages that are send to the Ab-LNS.  

%\subsection{The Weekly Schedule: Multi-compartment Multi-knapsack Problem with capacity penalties}
%\label{sub2sec2}
%The actor-based large neighborhood search is implemented on the MCMKP which models that weekly schedule in maintenance. The model is comprised of five different sets. $P$ is the number of weekly periods; $W$ is the number of work orders; $\tau$ is the number of different traits; $E$ is a set that defines which work orders should be excluded from a specific weekly period; $I$ is an inclusion set that defines the allocation of specific work orders which should be included in a specific weekly period. The model has four parameters. $v_{pw}$ is the value of work order $w$ in weekly period $p$; $d$ is the penalty for exceeding a specific trait capacity; $c_{w\tau}$ is the capacity requirement for work order $w$ for trait $t$; $cap_{p\tau}$ is the total amount of capacity available in for weekly period p for for trait t. The model has 2 decision variables. $x_{wp}$, is a binary decision variable equal to one if work order w is in weekly period p and zero otherwise; $pen_{p\tau}$ is non-negative decision variable equal to the amount of excess capacity above the $cap_{p\tau}$ in weekly period p for trait $\tau$. The parameters $v$, $cap$, $Q$, and $P$ are functions of time, $\tau$, in this case as they will be subject to change during the solution process.

\section{The Strategic Model}


The Strategic Model have multiple different purposes.
\begin{itemize}
	\item Schedule Work Order out across the weekly periods
	\item Prioritize all the different released work orders
	\item Respect the available weekly hours available for each trait
\end{itemize}

The Strategic model is responsible for grouping work orders into weekly or biweekly periods depending on which kind of maintenance setup that one is running.
This kind of model closely resembles a variant of the multi-compartment multi-knapsack problem. 

\begin{alignat}{2}
	\text{Min} &\sum_{w \in W} \sum_{p \in P} v_{wp}(t) \cdot x_{wp}(t)                                                                                             \\ 
	+ & \sum_{p \in P} \sum_{\tau = 1}^T d \cdot pen_{p\tau}(t)                                                                                                     \\
	+ & \sum_{p \in P} \sum_{w1 \in W} \sum_{w2 \in W} clu_{w1, w2} \cdot x_{w1p} \cdot x_{w2p}                              \label{eqn:objective_function_strategic} \\
    & \text{subject to:} \notag                                                                                                                                       \\
	& \sum_{w \in W} c_{w\tau} \cdot x_{wp}(t) \leq \ cap_{p\tau}(t) + pen_{p\tau}(t) \notag                                                                          \\ 
	& \forall p \in P, \forall \tau \in T                                                                                    \label{eqn:capacity_constraint}          \\
	& \sum_{w \in W} x_{wp}(t) = 1                                              , \quad \forall p \in P                      \label{eqn:single_workorder_constraint}  \\
	& x_{wp}(t) = 0                                                             , \quad \forall (w, p) \in E(t)              \label{eqn:exclusion_constraint}         \\
	& x_{wp}(t) = 1                                                             , \quad \forall (w, p) \in I(t)              \label{eqn:inclusion_constraint}         \\
	& x_{wp}(t) \in \{0, 1\}                                                    , \quad \forall w \in W, \forall p \in P     \label{eqn:x_integrality_constraint}     \\ 
	& pen_{p\tau}(t) \in \mathbb{R}^{+}                                         , \quad \forall p \in P, \forall \tau \in T  \label{eqn:p_non_negativity_constraint}
    & t \in  [0, \infty] 
\end{alignat}

% \begin{alignat}{2}
% 	& \text{Min} \quad \sum_{w = 1}^{W} \sum_{p = 1}^{P} v_{wp}(t) \cdot x_{wp}(t) + \sum_{p = 1}^{P} \sum_{\tau = 1}^T d \cdot pen_{p\tau}(t)   \label{eqn:objective_function_strategic} \\[1em]
%     & \text{subject to:} \notag                                                                                                                                        \\[1em]
% 	& \sum_{w = 1}^W c_{w\tau} \cdot x_{wp}(t) \leq \ cap_{p\tau}(t) + pen_{p\tau}(t)        && \forall p \in P, \forall \tau \in T                      \label{eqn:capacity_constraint}          \\[1em]
% 	& \sum_{w = 1}^{W} x_{wp}(t) = 1                                            && \forall p \in P                                       \label{eqn:single_workorder_constraint}  \\[1em]
% 	& x_{wp}(t) = 0                                                             && \forall (w, p) \in E(t)                               \label{eqn:exclusion_constraint}         \\[1em]
% 	& x_{wp}(t) = 1                                                             && \forall (w, p) \in I(t)                               \label{eqn:inclusion_constraint}         \\[1em]
% 	& x_{wp}(t) \in \{0, 1\}                                                    && \forall w \in W, \forall p \in P                      \label{eqn:x_integrality_constraint}     \\[1em] 
% 	& pen_{p\tau}(t) \in \mathbb{R}^{+}                                         && \forall p \in P, \forall \tau \in T                      \label{eqn:p_non_negativity_constraint}
% \end{alignat}

The objective function \eqref{eqn:objective_function_strategic_value}, \eqref{eqn:objective_function_strategic_penalty}, and
\eqref{eqn:objective_function_strategic_clustering}
minimizes the total weighted delay of all work order assignments together
with the penalty $\ParStrategicPenalty$ for exceeding the resource capacity given in constraint
\eqref{eqn:capacity_constraint}. The thrid term  
of the model contains the $\ParClusteringValue$ which turns the model into a 
quadratic problem. This term optimizes the value of putting two work orders
in the same period, if they have share similarity like close proximity, 
same functional location, etc. 

Constraint \eqref{eqn:capacity_constraint}
ensures that all the weights $\ParOperationWork{wr}$ for each activity in an work
order, given that it has been assigned, is lower than the capacity for each
period and for each trait $\tau$. $pen_{p\tau}$ is the amount of exceeded
capacity that is needed for the current assignment of work order to be
feasible. Constraint \eqref{eqn:single_workorder_constraint} makes sure
that each work order is assigned to at least a single period. Constraint
\eqref{eqn:exclusion_constraint} excludes a work order from a certain period
and constraint \eqref{eqn:inclusion_constraint} forces a specific work order
to be in a specific period. Constraint \eqref{eqn:x_integrality_constraint} and
\eqref{eqn:p_non_negativity_constraint} specify the variable domain for $x_{wp}$
and $pen_{p\tau}$ respectively. The effects of changing $E$, $I$, $cap$, and $v$
in real-time will be examined to determine their effects on the weekly schedules
and objective value.

% \begin{alignat}{4}   
% \texttt{Min}& 
%  \sum_{\ElementWorkOrder \in \SetWorkOrder{}} \sum_{\ElementPeriod \in \SetPeriod} \ParStrategicValue \cdot \VarStrategicWorkOrderAssignment{\ElementWorkOrder}{\ElementPeriod} \notag \\
%  & + \sum_{\ElementPeriod \in \SetPeriod} \sum_{\ElementResource \in \SetResource} \ParStrategicPenalty \cdot \VarStrategicExcess \notag \\
%  & + \sum_{\ElementPeriod \in \SetPeriod} \sum_{\ElementWorkOrder1 \in \SetWorkOrder{}} \sum_{\ElementWorkOrder2 \in \SetWorkOrder{}} \notag \\
%  & \quad \ParClusteringValue \cdot \VarStrategicWorkOrderAssignment{\ElementWorkOrder1}{\ElementPeriod} \cdot \VarStrategicWorkOrderAssignment{\ElementWorkOrder2}{\ElementPeriod}   
%  % \label{eqn:objective_function_strategic} \\
%  %\quad \sum_{l \in L} \sum_{m \in M} v_{l,m} 
%  %&  \label{mod3:o1}\\
%  \text{s.t.}	\notag\\
%  				& \sum_{i \in I} C^{1}_i \cdot x^k_i & = z^k_1 - z^1_1 & \forall k \in K \label{mod3:setz1}\\
% %& \sum_{\ElementWorkOrder \in \SetWorkOrder{}} \ParStrategicWorkOrderWeight \cdot \VarStrategicWorkOrderAssignment{\ElementWorkOrder}{\ElementPeriod} \notag      \\
%  %&				& \sum_{i \in I} C^{2}_i \cdot x^k_i %& = z^k_2 - z^K_2 & \forall k \in K %\label{mod3:setz2}\\
%  %&				& \sum_{i \in I} w_i \cdot x^k_i %&\leq W & \forall \ \forall k \in %K\label{mod3:knapsack_cons}\\
%  %&              & z^k_2 & \geq l \cdot y_{k,l,m} & %\forall k \in K, l \in L, m \in M %\label{mod3:KsquareLimL}\\
%  %&              & z^k_1 & \geq m \cdot y_{k,l,m} & %\quad \forall k \in K, l \in L, m \in M %\label{mod3:KsquareLimM}\\
%  %&              & \sum_{k \in K} y_{k,l,m} & \geq %v_{l,m} & \forall l \in L, m \in M %\label{mod3:VsquareLimL}\\
%  %&              & x^1_i & = x^{UL}_i & \forall i\in %I \label{mod3:fixUL}\\
%  %&              & x^{K}_i & = x^{LR}_i & \forall %i\in I \label{mod3:fixLR}\\
%  %%&              & v_{l,m} & \leq 0 & \forall l \in %L, m \in M | l \leq z^1_1 \label{mod3:limitREF1}\\
%  %%&              & v_{l,m} & \leq 0 & \forall l \in %L, m \in M | l \leq z^K_2 \label{mod3:limitREF2}\\
%  &				& x_i \in \{0,1\},&z_1,z_2 \in Z+& v_{l,m},y_{k,l,m} \in \{0,1\} \label{mod3:domain}
% \end{alignat}


The objective function \eqref{eqn:objective_function_strategic} minimizes the total weight of all work order assignments together with the penalty $d$ for exceeding the capacity given in constraint \eqref{eqn:capacity_constraint}. Constraint \eqref{eqn:capacity_constraint} ensures that all the weights $c_{w\tau}$ for each activity in an work order, given that it has been assigned, is lower than the capacity for each period and for each trait $\tau$. $pen_{p\tau}$ is the amount of exceeded capacity that is needed for the current assignment of work order to be feasible. Constraint \eqref{eqn:single_workorder_constraint} makes sure that each work order is assigned to at least a single period. Constraint \eqref{eqn:exclusion_constraint} excludes a work order from a certain period and constraint \eqref{eqn:inclusion_constraint} forces a specific work order to be in a specific period. Constraint \eqref{eqn:x_integrality_constraint} and \eqref{eqn:p_non_negativity_constraint} specify the variable domain for $x_{wp}$ and $pen_{p\tau}$ respectively. The effects of changing $E$, $I$, $cap$, and $v$ in real-time will be examined to determine their effects on the weekly schedules and objective value.
\section{Solution Method}
\label{sec:2-solution-method}

\subsection{Actor-based Large Neighborhood Search}
A problem which is affected by user-interaction and requires real-time feedback needs an optimization approach that is able to repair infeasible schedules and while also 
converging quickly. For this the large neighborhood search metaheuristic has been shown satisfy these requirements in the literature \cite{gendreauHandbookMetaheuristics2019}. 

The LNS metaheuristic is defined for static problems, meaning that the parameters that make up the problem instance is not subject to change 
after the algorithm has been started. To make the LNS able adapt to changing parameters in real-time a message system have been implemented into the existing framework. This 
extension is shown in algorithm \ref{algo1}.  

\subsubsection{Messages And Destructors}
LNS in its most basic form has one constructor and one destructor which repeatedly destroy and rebuild the schedule. For the AbLNS we will generalize on this concept by 
including messages as destructors of the classic LNS implementation. This generalization can be seen as being somewhat similar to how the adaptive LNS (ALNS) is formulated,
but where the different constructors and destructors are chosen externally as well. 

Extending on the classic setup we define the following set of destructors, $M$:

\begin{itemize}
	\item $m_1$: Inclusion destruct message	
	\item $m_2$: Exclusion destruct message	
	\item $m_3$: Capacities destruct message	
	\item $m_4$: Weights destruct message	
	\item $m_5$: Random destruct message
\end{itemize}

Each of these messages affect different parts of the MCMK problem (weekly schedule). Notice
here that the first four messages destruct the solution by changing the parameter space and the last message is 
a random destructor.

Generalizing the destructors from being static structures into messages
allows the solution to change in real-time to a changing paramenter space meaning
that the algorithm does not need to restart to handle changes in data. 

\newcommand{\MessageQueue}{Q}
\newcommand{\Solution}{X}
\newcommand{\ProblemInstance}{P}
\newcommand{\SharedSolution}{S}

\begin{algorithm}[H]
\caption{Actor-based Large Neighborhood Search} 
\begin{algorithmic}[1]
\State \textbf{Input} $\MessageQueue$    = message queue
\State \textbf{Input} $\ProblemInstance$ = problem instance
\State \textbf{Input} $\Solution$        = initial schedule
\State \textbf{Input} $\SharedSolution$  = SharedSolution
% \State $\Solution^b = \Solution$
\Repeat
	\State $\Solution^t = clone(\Solution)$
	\While{$\MessageQueue.has\_message()$}
        % \State $m = queue.pop()$
        % \State $m.destruct(\Solution^b)$
		\State $\ProblemInstance.update(\SharedSolution, m)$
        \State $\Solution^t.destruct(\SharedSolution, m)$
    \EndWhile
	
    \State $\Solution^t.repair(\SharedSolution)$

    \If{accept($\Solution^t, \Solution$)}                       
        \State $\Solution.update(\Solution^t)$
    \EndIf                                         

    \If{$c(\Solution^t) < c(\Solution)$}                             
        \State $\Solution.update(\Solution^t)$
		\State $\SharedSolution.atomic\_pointer\_swap(\Solution)$
    \EndIf                                           
	\State $\MessageQueue.push(m)$
\Until
\end{algorithmic}
\end{algorithm}



The basic LNS setup have here been extended with a `message queue`. This message queue will be read from on every iteration of the LNS's main iteration loop. Here we notice that the 
incoming message is able to change both the solution but also the problem instance itself. Here we see one of the defining features of the LNS metaheuristic in play, that due to its inherent property of being able to optimize a solution that have become infeasible which is something that is very likely to happen when you change the parameter of the problem instance itself. 

Another less obvious property the message queue allows is for the algorithm to run indefinitely and instead of restarting the algorithm you instead pass 
messages to it to allow it be adjust both the solution space and the parameter space.
This property avoid the issue of time consuming initial convergence as the algorithm will be found in an optimimal state when the solution is perturbed.

Notice that:

\begin{itemize}
    \item The algorithm responds to changes very quickly, in each iteration (line 5-8. We call this a fine-grained response algorithm
    \item If an improved solution is found, it is immediately being pushed to the data (base). (line 12)
    \item That the optimization occurs in the repair function (line 9), which inserts operations, not scheduled yet in a greedy fashion. While not being an optimal insertion, it is fast
\end{itemize}


\section{Results}
\label{sec:3-results}
The results section will: 1. introduce the real-world data instance; 2. show the effect of forcing item set in the specific weekly schedules; 3. show the effect of changing the 
period capacities, and 4. show the effect of dynamically changing the value of the work orders $v$. 

\subsection{Data Instance}
\begin{table}[H]
\centering
\begin{tabular}{|c|c|c|c|c|}
\hline
           & \begin{tabular}[c]{@{}c@{}}Number of\\ Item Sets\end{tabular} & \begin{tabular}[c]{@{}c@{}}Number of\\ Compartments\end{tabular} & \begin{tabular}[c]{@{}c@{}}Number of\\ Knapsacks\end{tabular} \\ \hline
Instance 1 & 3487                                                          & 16                                                               & 52                                                            \\ \hline
\end{tabular}

\caption{Table Caption} % \label{fig1}
\end{table}
% \begin{table}[t]%% placement specifier
% %% Use tabular environment to tag the tabular data.
% %% https://en.wikibooks.org/wiki/LaTeX/Tables#The_tabular_environment
% \centering%% For centre alignment of tabular.
% \begin{tabular}{l c r}%% Table column specifiers
% %% Tabular cells are separated by &
%   1 & 2 & 3 \\ %% A tabular row ends with \\
%   4 & 5 & 6 \\
%   7 & 8 & 9 \\
% \end{tabular}
% %% Use \caption command for table caption and label.
% \end{table}

\subsection{Response to Inclusion}
The response to the inclusion of a work order is given by I parameter of the model which 
is constrained in \ref{eqn:inclusion_constraint} of model given in \ref{sub1sec2}.

The inclusion is made of forcing certain allocations of work orders to be in specific periods. Below a table is provided 
to show what changes will occur and at what and at what point in time.
\begin{table}[H]
\centering
\begin{tabular}{|c|c|c|c|c|c|}
\hline
\begin{tabular}[c]{@{}c@{}}\end{tabular}     & \begin{tabular}[c]{@{}c@{}}At Time:\\ 01:00\end{tabular} & \begin{tabular}[c]{@{}c@{}}At Time:\\ 02:00\end{tabular} & \begin{tabular}[c]{@{}c@{}}At Time:\\ 03:00\end{tabular} & \begin{tabular}[c]{@{}c@{}}At Time:\\ 04:00\end{tabular} & \begin{tabular}[c]{@{}c@{}}At Time:\\ 05:00\end{tabular} \\ \hline
\begin{tabular}[c]{@{}c@{}}$\Delta |P|$\end{tabular} & 10                                                       & 20                                                       & 30                                                       & 40                                                       & 50                                                       \\ \hline
\end{tabular}
\end{table}

With the inputs defined we will explain the main results which are shown in the figure below. 
% Use figure environment to create figures
% Refer following link for more details.
% https://en.wikibooks.org/wiki/LaTeX/Floats,_Figures_and_Captions
\begin{figure}[H]%% placement specifier
%% Use \includegraphics command to insert graphic files. Place graphics files in 
%% working directory.
\centering%% For centre alignment of image.
\includegraphics[width=1.0\textwidth]{figures/objective.png}
%% Use \caption command for figure caption and label.
\caption{Figure Caption}\label{fig:response-to-inclusion}
%% https://en.wikibooks.org/wiki/LaTeX/Importing_Graphics#Importing_external_graphics
\end{figure}

\subsection{Response to Exclusion}
\begin{figure}[H]%% placement specifier
%% Use \includegraphics command to insert graphic files. Place graphics files in 
%% working directory.
\centering%% For centre alignment of image.
\includegraphics[width=1.0\textwidth]{figures/objective-400-exclusions.png}
%% Use \caption command for figure caption and label.
\caption{Figure Caption}\label{fig:objective-exclusion-400}
%% https://en.wikibooks.org/wiki/LaTeX/Importing_Graphics#Importing_external_graphics
\end{figure}

\subsection{Response to Changes in Knapsack Capacities}
The effects of changes to capacities will be illustrated in the same way as it was with the response to inclusion and below we see the table that shows which inputs that the AbLNS will be affected by.

\begin{table}[H]
\centering
\begin{tabular}{|c|c|c|c|c|c|}
\hline
                      & \begin{tabular}[c]{@{}c@{}}At Time:\\ 01:00\end{tabular} & \begin{tabular}[c]{@{}c@{}}At Time:\\ 02:00\end{tabular} & \begin{tabular}[c]{@{}c@{}}At Time:\\ 03:00\end{tabular} & \begin{tabular}[c]{@{}c@{}}At Time:\\ 04:00\end{tabular} & \begin{tabular}[c]{@{}c@{}}At Time:\\ 05:00\end{tabular} \\ \hline
$\Delta |p|$ & 16                                                       & 16                                                       & 16                                                       & 16                                                       & 16                                                       \\ \hline
$\Delta |\tau|$ & 16                                                       & 16                                                       & 16                                                       & 16                                                       & 16                                                       \\ \hline
$\Delta |cap|$& 100                                                      & 200                                                      & 400                                                      & 800                                                      & 1600                                                     \\ \hline
\end{tabular}
\end{table}

\begin{figure}[H]%% placement specifier
%% Use \includegraphics command to insert graphic files. Place graphics files in 
%% working directory.
\centering%% For centre alignment of image.
\includegraphics[width=1.0\textwidth]{figures/objective-resource-increases.png}
%% Use \caption command for figure caption and label.
\caption{Figure Caption}\label{fig:objective-resource-increases}
%% https://en.wikibooks.org/wiki/LaTeX/Importing_Graphics#Importing_external_graphics
\end{figure}

Correspondingly we also have the figure below in which the resources are decreasing.

\subsection{Response to Changes in Item Weights}
The final parameter that will be changed is the work order value $v$. This section will be more elaborate as we have to show how that the
work orders are rearranged due to the changes in their value across the different periods.

\begin{table}[H]
\centering
\begin{tabular}{|c|c|c|c|c|c|}
\hline
             & \begin{tabular}[c]{@{}c@{}}At Time:\\ 01:00\end{tabular} & \begin{tabular}[c]{@{}c@{}}At Time:\\ 02:00\end{tabular} & \begin{tabular}[c]{@{}c@{}}At Time:\\ 03:00\end{tabular} & \begin{tabular}[c]{@{}c@{}}At Time:\\ 04:00\end{tabular} & \begin{tabular}[c]{@{}c@{}}At Time:\\ 05:00\end{tabular} \\ \hline
$\Delta |w|$ & 20                                                       & 40                                                       & 80                                                       & 160                                                      & 320                                                      \\ \hline
$\Delta |p|$ & 26                                                       & 26                                                       & 26                                                       & 26                                                       & 26                                                       \\ \hline
$\Delta |v|$ & $1 \cdot 10^{5}$                                         & $2 \cdot 10^{5}$                                         & $4 \cdot 10^{5}$                                         & $8 \cdot 10^{5}$                                         & $1.6 \cdot 10^{6}$                                       \\ \hline
\end{tabular}
\end{table}


\section{Discussion}
\label{sec:4-discussion}
The maintenance scheduling process effectively solves a complex scheduling problem by
relying in multiple actors. Through the use of actors the scheduling process handles
uncertainty that is difficult to reason about in a single mathematical model. These 
uncertainties are solved through coordination. Each type of stakeholder in the process 
acts according to a  model each with different levels of aggregation and features where
each actor understands how to exploit his own model.
The discussion will be divided into three sections: 
\ref{sec:discussion:actors_and_integration} 
actors and integration;
\ref{sec:discussion:continuous_optimization} 
continuous optimization allows asynchronous optimization; 
and \ref{sec:discussion:future_research} future research.

\subsection{Actors \& Integration}
\label{sec:discussion:actors_and_integration}
Often in operation research the failure to reliably solve operational
problems in  industry are not due to the problems being computationally
intractable \cite{gendreauHandbookMetaheuristics2019} but a practical
problem of connecting data streams so that the solution approach continually
receives dynamic data to handle changes and then providing the resulting
solutions to the relevant actors (stakeholders) through a relevant interface
\cite{meignanReviewTaxonomyInteractive2015}. The actor-based approach proposed
in this paper makes integration easier by naturally encapsulating a model with a
reliable interface.

\subsection{Continuous Optimization}
\label{sec:discussion:continuous_optimization}
With actor-based metaheuristics, the optimization loop can run indefinitely,
optimizing based on the latest available information. This may seem like a
detail as you could argue that you should only ever optimize the schedule
when there is an explicit need for it, but consider the case when you start
adding more than two actors to a scheduling system, then there arises a need
to coordinate people temporally as each will have to run their optimizing
process one after another. This is depicted in figure~\ref{fig:discussion:hierarchical_model_setup}
where the output of one model is used as the input to the next one, leading
to the hierarchical model setup.

\begin{figure}[H]
	\usetikzlibrary{positioning}
\usetikzlibrary{arrows.meta}
\usetikzlibrary{bending}


\definecolor{red}{HTML}{8A3F3A}
\definecolor{yellow}{HTML}{E0BB3C}
\definecolor{blue}{HTML}{4569E0}
\definecolor{green}{HTML}{17E561}
\definecolor{other}{HTML}{6A939E}


\newlength{\basisb}
\setlength{\basisb}{0.4cm}

\centering
\begin{tikzpicture}[line width=0.0\basisb]
    \draw (2.0\basisb,4.0\basisb) 
		node[rotate=90, minimum height=3\basisb,fill=dtu-blue,minimum width=8\basisb,rounded corners=0.1\basisb] 
			(Dynamic Data) {Dynamic Data};

    \draw (8.0\basisb,7.0\basisb) 
		node[minimum height=2\basisb,fill=dtu-red,minimum width=6\basisb,rounded corners=0.1\basisb] 
			(Scheduler) {Scheduler};
    \draw (14.0\basisb,4.0\basisb) 
		node[minimum height=2\basisb,fill=dtu-red,minimum width=6\basisb,rounded corners=0.1\basisb] 
			(Supervisor) {Supervisor};
    \draw (20.0\basisb,1.0\basisb) 
		node[minimum height=2\basisb,fill=dtu-red,minimum width=6\basisb,rounded corners=0.1\basisb] 
			(Technician) {Technician};

    \draw (26.0\basisb,7.0\basisb) 
		node[minimum height=2\basisb,fill=dtu-yellow,minimum width=3\basisb,rounded corners=0.1\basisb] 
			(UserInterface1) {UI};
    \draw (26.0\basisb,4.0\basisb) 
		node[minimum height=2\basisb,fill=dtu-yellow,minimum width=3\basisb,rounded corners=0.1\basisb] 
			(UserInterface2) {UI};
    \draw (26.0\basisb,1.0\basisb) 
		node[minimum height=2\basisb,fill=dtu-yellow,minimum width=3\basisb,rounded corners=0.1\basisb] 
			(UserInterface3) {UI};

	\draw[<->, line width=0.1\basisb,color=dtu-green] (5.0\basisb, -1\basisb) -- (23.0\basisb, -1\basisb);
	\draw (14.0\basisb, -2.0\basisb) node {Significant Amount of Time};

	\draw[->,>=Triangle, thick, line width=0.1\basisb, color=dtu-corporate-red] (Scheduler) to[out=0, in=90,looseness=1.5]  (Supervisor);
	\draw[->,>=Triangle, thick, line width=0.1\basisb, color=dtu-corporate-red] (Supervisor) to[out=0, in=90,looseness=1.5] (Technician);
	\draw[->,>=Triangle, thick, line width=0.1\basisb] (Dynamic Data.south) ++(0\basisb,3.0\basisb) to[out=0, in=180,looseness=1.0] (Scheduler);
	\draw[->,>=Triangle, thick, line width=0.1\basisb] (Dynamic Data.south) to[out=0, in=180,looseness=1.0] (Supervisor);
	\draw[->,>=Triangle, thick, line width=0.1\basisb] (Dynamic Data.south) ++(0\basisb,-3.0\basisb) to[out=0, in=180,looseness=1.0] (Technician.west);
	\draw[<-,>=Triangle, thick, line width=0.1\basisb] (UserInterface1) to[out=180, in=0,looseness=1.0] (Scheduler);
	\draw[<-,>=Triangle, thick, line width=0.1\basisb] (UserInterface2) to[out=180, in=0,looseness=1.0] (Supervisor);
	\draw[<-,>=Triangle, thick, line width=0.1\basisb] (UserInterface3) to[out=180, in=0,looseness=1.0] (Technician);
	% \draw[<->, thick, line width=0.1\basisb] (Scheduler) -- (UserInterface);
	\begin{scope}[shift={(7,0)}] % Adjust shift to position the legend
    % Legend box
    % Legend lines and text
	    \draw[thick, line width=0.1\basisb] (-1.5,2.4) node[right, font=\footnotesize, align=center] {Solution\\Transfer};
	    \draw[thick, line width=0.1\basisb] (1.0,1.2) node[right, font=\footnotesize, align=center] {Solution\\Transfer};
	\end{scope}
\end{tikzpicture}

	\label{fig:discussion:hierarchical_model_setup}
	\caption{Effects of using hierarchical models setups in human-guided search metaheuristics.
	Due to the dependent nature of each metaheuristic it becomes crucial that the running of 
	the metaheuristics are well coordinated between the stakeholders.}
\end{figure}

In practice there are multiple problems with using a hierarchical setup.
Usually the biggest one is that the information and knowledge needed to 
execute a feasible schedule is usually found in the lower levels of the 
hierarchicy. The operational setting, where the
technicians are working, is usually so complex that it not feasible to 
centralize the knowledge that is required to create and execute a 
schedule. Figure~\ref{fig:discussion:asynchronous_setup}
shows the kind of non-hierarchical setup that an actor-based approach 
allows for.

\begin{figure}[H]
	\usetikzlibrary{positioning}
\usetikzlibrary{arrows.meta}
\usetikzlibrary{bending}
\usetikzlibrary{backgrounds}


\definecolor{red}{HTML}{8A3F3A}
\definecolor{yellow}{HTML}{E0BB3C}
\definecolor{blue}{HTML}{4569E0}
\definecolor{green}{HTML}{17E561}
\definecolor{other}{HTML}{6A939E}


\newlength{\basisc}
\setlength{\basisc}{0.5cm}

\centering
\begin{tikzpicture}[line width=0.0\basisc]
    \draw (4.0\basisc,4.0\basisc) 
		node[rotate=90, minimum height=3\basisc,fill=dtu-blue,minimum width=8\basisc,rounded corners=0.1\basisc] 
			(Dynamic Data) {Dynamic Data};

			

    \draw (8.0\basisc,10.0\basisc) 
		node[minimum height=2\basisc,fill=dtu-yellow,minimum width=3\basisc,rounded corners=0.1\basisc] 
			(UserInterface1) {UI};
    \draw (12.0\basisc,10.0\basisc) 
		node[minimum height=2\basisc,fill=dtu-yellow,minimum width=3\basisc,rounded corners=0.1\basisc] 
			(UserInterface2) {UI};
    \draw (16.0\basisc,10.0\basisc) 
		node[minimum height=2\basisc,fill=dtu-yellow,minimum width=3\basisc,rounded corners=0.1\basisc] 
			(UserInterface3) {UI};
    \draw (12.0\basisc,7.0\basisc) 
		node[minimum height=2\basisc,fill=dtu-red,minimum width=11\basisc,rounded corners=0.1\basisc] 
			(Scheduler) {Scheduler};
    \draw (12.0\basisc,4.0\basisc) 
		node[minimum height=2\basisc,fill=dtu-red,minimum width=11\basisc,rounded corners=0.1\basisc] 
			(Supervisor) {Supervisor};
    \draw (12.0\basisc,1.0\basisc) 
		node[minimum height=2\basisc,fill=dtu-red,minimum width=11\basisc,rounded corners=0.1\basisc] 
			(Technician) {Technician};


	\begin{pgfonlayer}{background}
		\draw[<->, thick, line width=0.1\basisc] (UserInterface1) to[out=-90, in=90,looseness=1.0] ++(0\basisc,-2.0\basisc)(Scheduler);
		\draw[<->, thick, line width=0.1\basisc] (UserInterface2) to[out=-90, in=90,looseness=1.0] (Supervisor);
		\draw[<->, thick, line width=0.1\basisc] (UserInterface3) to[out=-90, in=90,looseness=1.0] ++(0\basisc,-8.0\basisc)(Technician);

	\end{pgfonlayer}

	\draw[->, line width=0.1\basisc,color=dtu-green] (6.5\basisc, -1\basisc) -- (17.5\basisc, -1\basisc);
	\draw (12.0\basisc, -2.0\basisc) node {Running Continuously};

	\draw[<->, thick, line width=0.1\basisc, color=dtu-corporate-red] (Scheduler)++(-3\basisc, -1.0\basisc) to[out=-90, in=90,looseness=1.0]  ++(0\basisc, -1.0\basisc)(Supervisor);
	\draw[<->, thick, line width=0.1\basisc, color=dtu-corporate-red] (Scheduler)++(-2\basisc, -1.0\basisc) to[out=-90, in=90,looseness=1.0]  ++(0\basisc, -1.0\basisc)(Supervisor);
	\draw[<->, thick, line width=0.1\basisc, color=dtu-corporate-red] (Scheduler)++(2\basisc, -1.0\basisc) to[out=-90, in=90,looseness=1.0]  ++(0\basisc, -1.0\basisc)(Supervisor);

	\draw[<->, thick, line width=0.1\basisc, color=dtu-corporate-red] (Supervisor)++(-4\basisc, -1.0\basisc) to[out=-90, in=90,looseness=1.0] ++(0\basisc, -1.0\basisc)(Technician);
	\draw[<->, thick, line width=0.1\basisc, color=dtu-corporate-red] (Supervisor)++(-1\basisc, -1.0\basisc) to[out=-90, in=90,looseness=1.0] ++(0\basisc, -1.0\basisc)(Technician);
	\draw[<->, thick, line width=0.1\basisc, color=dtu-corporate-red] (Supervisor)++(3\basisc, -1.0\basisc) to[out=-90, in=90,looseness=1.0] ++(0\basisc, -1.0\basisc)(Technician);

	\draw[<->, thick, line width=0.1\basisc] (Dynamic Data.south) ++(0\basisc,3.0\basisc) to[out=0, in=180,looseness=1.0] (Scheduler);
	\draw[<->, thick, line width=0.1\basisc] (Dynamic Data.south) to[out=0, in=180,looseness=1.0] (Supervisor);
	\draw[<->, thick, line width=0.1\basisc] (Dynamic Data.south) ++(0\basisc,-3.0\basisc) to[out=0, in=180,looseness=1.0] (Technician.west);

	% \draw[<->, thick, line width=0.1\basisc] (Scheduler) -- (UserInterface);
\end{tikzpicture}

	\caption{Asynchronous model setup where each metaheuristic runs in perpetuity. In this setup
	there is no need to coordinate stakeholders to run the metaheuristics. Each actor in the 
	scheduling process will always have the solutions of the other stakeholder available to 
	him to guide his own search.}
	\label{fig:discussion:asynchronous_setup}
\end{figure}

When the optimization approach optimize continuously it enables tight
integration between the different model implementations. Instead of running
models to completion you simply handle changes in model parameters, model
solutions, user inputs, and in the dynamic data source as they occur opposed to
restarting the metaheuristics.

\subsection{Future Research}
\label{sec:discussion:future_research}
A possible future research direction is to demonstrate that
the actor-based approach described here can be used to model and optimize 
multi-actor/multi-level scheduling processes. 
Figures~\ref{fig:ordinator-hexagon:persistence},
~\ref{fig:ordinator-hexagon:atomicpointerswap},
~\ref{fig:ordinator-hexagon:metaheuristics},
~\ref{fig:ordinator-hexagon:orchestrator},
~\ref{fig:ordinator-hexagon:userinterfaces}
show a larger scale setup of the actor-based approach which is being developed with Total Energies.
Here figure~\ref{fig:ordinator-hexagon:metaheuristics}
shows the metaheuristics of scheduling system architecture where each of the actors run an AbLNS
and that each metaheuristic will share its solutions with the other
metaheuristics through atomic pointer swapping shown in figure~\ref{fig:ordinator-hexagon:atomicpointerswap}.
Communicate with the end-user
through userinterfaces and message passing as shown in figure~\ref{fig:ordinator-hexagon:userinterfaces}, 
integrate with a persistent storage through mutex
locks as shown in figure~\ref{fig:ordinator-hexagon:persistence}, 
and the lifecycle of each of the metaheuristics will be controlled by
the orchestrator also through message passing as shown in figure~\ref{fig:ordinator-hexagon:orchestrator}. 

% \begin{figure}[H]
% 	\centering
% 	\usetikzlibrary {positioning}


\definecolor{red}{HTML}{8A3F3A}
\definecolor{yellow}{HTML}{E0BB3C}
\definecolor{blue}{HTML}{4569E0}
\definecolor{green}{HTML}{17E561}
\definecolor{other}{HTML}{6A939E}

\newcommand{\ModelColor}{red}
\newcommand{\UserInterfaceColor}{yellow}
\newcommand{\PersistenceColor}{blue}
\newcommand{\PointerSwapColor}{green}
\newcommand{\OrchestratorColor}{other}

\pgfkeys{
	/graph/.is family, /graph,
	default/.style = {
		show_shared_pointer = false,
		show_orchestrator = false,
		show_persistence = false,
		show_user_interface = false,
		basis/.estore in = 2cm,
	},
	show_shared_pointer/.estore in = \ShowSharedSolutionCommunication,
	show_orchestrator/.estore in = \ShowOrchestratorCommunication,
	show_persistence/.estore in = \ShowPersistenceCommunication,
	show_user_interface/.estore in = \ShowUserInterfaceCommunication,
	basis/.estore in = \basisinput,
}
\newlength{\basis}
\tikzset{
  basis/.code={\setlength{\basis}{\basisinput}}, % TikZ assignment code
  basis/.default=1cm,                   % Provide a default (\b@sis is undefined/unassigned)
  basis,                                % Set initial Value (\b@sis is defined/assigned)
 }

\newcommand{\drawGraph}[1]{
	\pgfkeys{/graph, default, #1}
	
	\begin{tikzpicture}[scale=0.75][
		% Define styles and settings
		node distance=2cm,
		block/.style={rectangle, draw, fill=blue!20, text centered, minimum height=3em},
		arrow/.style={-Stealth, thick}
		]


		\ifthenelse{\equal{\ShowOrchestratorCommunication}{true}}{
			\draw[color=other,-, ultra thick] (Strategic) -- (Orchestrator);
			\draw[color=other,-, ultra thick] (Tactical) -- (Orchestrator);
			\draw[color=other,-, ultra thick] (Supervisor) -- (Orchestrator);
			\draw[color=other,-, ultra thick] (Operational_1) -- (Orchestrator);
			\draw[color=other,-, ultra thick] (Operational_2) -- (Orchestrator);
			\draw[color=other,-, ultra thick] (Operational_3) -- (Orchestrator);
		}{}
		% \draw[help lines] (0\basis, 0\basis) grid (10\basis, 8\basis);
		\draw (5\basis,4\basis) node[minimum height=5cm,minimum width=7.0cm,rounded corners=5pt] {};

	    \draw (2.5\basis,5.5\basis) node[minimum height=1cm,minimum width=1cm,fill=\ModelColor,rounded corners=5pt] (Strategic) {Stra};
	    \draw (5.0\basis,4.0\basis) node[minimum height=1cm,minimum width=1\basis,fill=\ModelColor,rounded corners=5pt] (Supervisor) {Sup};
		\draw (7.5\basis,5.5\basis) node[minimum height=1cm,minimum width=1cm,fill=\ModelColor,rounded corners=5pt] (Tactical) {Tac};

		\draw (2.5\basis,2.5\basis) node[minimum height=1cm,minimum width=1cm,fill=\ModelColor,rounded corners=5pt] (Operational_1) {$O_{1}$};
		\draw (5.0\basis,2.5\basis) node[minimum height=1cm,minimum width=1cm,fill=\ModelColor,rounded corners=5pt] (Operational_2) {$O_{2}$};
		\draw (7.5\basis,2.5\basis) node[minimum height=1cm,minimum width=1cm,fill=\ModelColor,rounded corners=5pt,rounded corners=5pt] (Operational_3) {$O_{3}$};
	
		\draw (Strategic) edge (Tactical);
		\draw (Strategic) edge (Tactical);
		\draw (5\basis,5.5\basis) edge (Supervisor);
		\draw (Supervisor) edge (Operational_1);
		\draw (Supervisor) edge (Operational_2);
		\draw (Supervisor) edge (Operational_3);
		\draw (5.0\basis,0.5\basis)   node[minimum height=1cm,minimum width=5.0cm,            fill=\PersistenceColor,rounded corners=5pt] {SchedulingEnvironment};
		\draw (5.0\basis,7.5\basis)   node[minimum height=1cm,minimum width=5.0cm,            fill=\OrchestratorColor,rounded corners=5pt] (Orchestrator) {Orchestrator};
		\draw (0.5\basis,4.0\basis)   node[rotate=90, minimum height=1cm, minimum width=3.25cm,fill=\PointerSwapColor,rounded corners=5pt] {SharedSolution};
		\draw (9.5\basis,5.5\basis) node[rotate=90, minimum height=1cm, minimum width=1cm,fill=\UserInterfaceColor,rounded corners=5pt] {UI};
		\draw (9.5\basis,4.0\basis)   node[rotate=90, minimum height=1cm, minimum width=1cm,fill=\UserInterfaceColor,rounded corners=5pt] {UI};
		\draw (9.5\basis,2.5\basis) node[rotate=90, minimum height=1cm, minimum width=1cm,fill=\UserInterfaceColor,rounded corners=5pt] {UI};

		% Legend
		\begin{scope}[shift={(10.6\basis,5.7\basis)}]
			\node at (-0.25\basis,1\basis) [right] {Communication Type};
			\draw[color=\OrchestratorColor,fill] (0\basis,0.0\basis) rectangle (0.5cm, 0.5cm);
			\node[anchor=west] at (0.5\basis, 0.25\basis) {\scriptsize Channels};
			\draw[color=\PointerSwapColor,fill] (0\basis,-1.0\basis) rectangle(0.5cm, -0.5cm); 
			\node[anchor=west] at (0.5\basis, -0.75\basis) {\scriptsize Atomic Pointer Swap};
			\draw[color=\ModelColor,fill] (0\basis,-2.0\basis) rectangle(0.5cm, -1.5cm); 
			\node[anchor=west] at (0.5\basis, -1.75\basis) {\scriptsize Metaheurics};
			\draw[color=\PersistenceColor,fill] (0\basis,-3.0\basis) rectangle(0.5cm, -2.5cm); 
			\node[anchor=west] at (0.5\basis, -2.75\basis) {\scriptsize Mutex lock};
			\draw[color=\UserInterfaceColor,fill] (0\basis,-4.0\basis) rectangle(0.5cm, -3.5cm); 
			\node[anchor=west] at (0.5\basis, -3.75\basis) {\scriptsize Channels};
		\end{scope}
		\ifthenelse{\equal{\ShowSharedSolutionCommunication}{true}}{
			\draw[->, thick] (Strategic) -- (Orchestrator);
		}{}
		\ifthenelse{\equal{\ShowUserInterfaceCommunication}{true}}{
			\draw[->, thick] (Strategic) -- (Orchestrator);
		}{}
		\ifthenelse{\equal{\ShowPersistenceCommunication}{true}}{
			\draw[->, thick] (Strategic) -- (Orchestrator);
		}{}
		

	\end{tikzpicture}
}


% 	\drawOrdinatorArchitecture{basisinput=1cm}
% 	\label{fig:ordinator-architecture}
% \end{figure}
\begin{figure}[H]
	\centering
	\documentclass{standalone}
\usepackage{tikz}
\usetikzlibrary {positioning}

<<<<<<< HEAD
\newcommand{\drawHexagon}[5][draw=black]{
	\draw[#1, fill=#4, ] (#2,#3) ++(30:2) -- ++(90:2) -- ++(150:2) -- ++(210:2) -- ++(270:2) -- ++(330:2) -- cycle;
	\node at (#2,#3+2) {#5};
||||||| 2d4327a
\newcommand{\drawHexagon}[4]{
	\draw[fill=#3] (#1,#2) ++(30:2) -- ++(90:2) -- ++(150:2) -- ++(210:2) -- ++(270:2) -- ++(330:2) -- cycle;
	\node at (#1,#2+2) {#4};
=======
\begin{document}
\newcommand{\drawHexagon}[4]{
	\draw[fill=#3] (#1,#2) ++(30:2) -- ++(90:2) -- ++(150:2) -- ++(210:2) -- ++(270:2) -- ++(330:2) -- cycle;
	\node at (#1,#2+2) {#4};
>>>>>>> 023797133fb426c1bb01f920d8f5635c343d11a6
}

\newif\ifuserinterfacelayer
\newif\ifpersistencelayer
\newif\ifmetaheuristicslayer
\newif\ifatomicpointerswaplayer

\pgfkeys{
	/hexagon/.is family, /hexagon,
	default/.style = {
		persistence=false,
		userinterface=false,
		metaheuristics=true,
	},
	persistence/.is if=persistencelayer,
	userinterface/.is if=userinterfacelayer,
	metaheuristics/.is if=metaheuristicslayer,
}
\newcommand{\drawModelSetupHexagon}[1][]{
	\pgfkeys{/hexagon, default, #1}

	\begin{tikzpicture}[scale=0.6, line width=1.05]
	
	\ifuserinterfacelayer
		\drawHexagon{ 2                      }{ 2}{dtu-yellow}{UI}
		\drawHexagon{{6 - 2 * (2 - sqrt(3)) }}{ 2}{dtu-yellow}{UI}
		\drawHexagon{{4 - 1 * (2 - sqrt(3)) }}{-1}{dtu-yellow}{UI}
		\drawHexagon{{0 + 1 * (2 - sqrt(3)) }}{-1}{dtu-yellow}{UI}
		\drawHexagon{{8 - 3 * (2 - sqrt(3)) }}{-1}{dtu-yellow}{UI}

		\drawHexagon{{2 - 0 * (2 - sqrt(3)) }}{-4}{dtu-yellow}{UI}
		\drawHexagon{{6 - 2 * (2 - sqrt(3)) }}{-4}{dtu-yellow}{UI}

		\drawHexagon{{10 - 4 * (2 - sqrt(3)) }}{-4}{dtu-yellow}{UI}
		\drawHexagon{{-2 + 2 * (2 - sqrt(3)) }}{-4}{dtu-yellow}{UI}

		\drawHexagon{{12 - 5 * (2 - sqrt(3)) }}{-1}{dtu-yellow}{UI}
		\drawHexagon{{-4 + 3 * (2 - sqrt(3)) }}{-1}{dtu-yellow}{UI}
	\fi

	\ifpersistencelayer
		\drawHexagon[draw=none]{ 2                      }{ 2}{dtu-blue}{}
		\drawHexagon[draw=none]{{6 - 2 * (2 - sqrt(3)) }}{ 2}{dtu-blue}{}
		\drawHexagon[draw=none]{{4 - 1 * (2 - sqrt(3)) }}{-1}{dtu-blue}{Database}
		\drawHexagon[draw=none]{{0 + 1 * (2 - sqrt(3)) }}{-1}{dtu-blue}{}
		\drawHexagon[draw=none]{{8 - 3 * (2 - sqrt(3)) }}{-1}{dtu-blue}{}

		\drawHexagon[draw=none]{{2 - 0 * (2 - sqrt(3)) }}{-4}{dtu-blue}{}
		\drawHexagon[draw=none]{{6 - 2 * (2 - sqrt(3)) }}{-4}{dtu-blue}{}

		\drawHexagon[draw=none]{{10 - 4 * (2 - sqrt(3)) }}{-4}{dtu-blue}{}
		\drawHexagon[draw=none]{{-2 + 2 * (2 - sqrt(3)) }}{-4}{dtu-blue}{}

		\drawHexagon[draw=none]{{12 - 5 * (2 - sqrt(3)) }}{-1}{dtu-blue}{}
		\drawHexagon[draw=none]{{-4 + 3 * (2 - sqrt(3)) }}{-1}{dtu-blue}{}
	\fi

	\ifmetaheuristicslayer
		\drawHexagon{ 2                      }{ 2}{dtu-blue}{Strategic}
		\drawHexagon{{6 - 2 * (2 - sqrt(3)) }}{ 2}{dtu-green}{Tactical}
		\drawHexagon{{4 - 1 * (2 - sqrt(3)) }}{-1}{dtu-red}{Supervisor}
		\drawHexagon{{0 + 1 * (2 - sqrt(3)) }}{-1}{dtu-red}{Supervisor}
		\drawHexagon{{8 - 3 * (2 - sqrt(3)) }}{-1}{dtu-red}{Supervisor}

		\drawHexagon{{2 - 0 * (2 - sqrt(3)) }}{-4}{dtu-corporate-red}{Technician}
		\drawHexagon{{6 - 2 * (2 - sqrt(3)) }}{-4}{dtu-corporate-red}{Technician}

		\drawHexagon{{10 - 4 * (2 - sqrt(3)) }}{-4}{dtu-corporate-red}{Technician}
		\drawHexagon{{-2 + 2 * (2 - sqrt(3)) }}{-4}{dtu-corporate-red}{Technician}

		\drawHexagon{{12 - 5 * (2 - sqrt(3)) }}{-1}{dtu-corporate-red}{Technician}
		\drawHexagon{{-4 + 3 * (2 - sqrt(3)) }}{-1}{dtu-corporate-red}{Technician}
	\fi

	
	\end{tikzpicture}
}

\ifstandalone
	

\definecolor{red}{HTML}{8A3F3A}
\definecolor{yellow}{HTML}{E0BB3C}
\definecolor{blue}{HTML}{4569E0}
\definecolor{green}{HTML}{17E561}
\definecolor{other}{HTML}{6A939E}

	\drawModelSetupHexagon
\fi

\end{document}

	\resizebox{0.7\textwidth}{!}{
		\drawModelSetupHexagon[persistence=true]
	}
	\caption{
		Overview of the scheduling process when modelled as actors. When LNS is encapsulated 
		is an actor it becomes possible to optimize parts of a large process individually instead of 
		optimizing the scheduling problem globally from a single model implementation.
	}
	\label{fig:ordinator-hexagon:persistence}
\end{figure}
\begin{figure}[H]
	\centering
	\documentclass{standalone}
\usepackage{tikz}
\usetikzlibrary {positioning}

<<<<<<< HEAD
\newcommand{\drawHexagon}[5][draw=black]{
	\draw[#1, fill=#4, ] (#2,#3) ++(30:2) -- ++(90:2) -- ++(150:2) -- ++(210:2) -- ++(270:2) -- ++(330:2) -- cycle;
	\node at (#2,#3+2) {#5};
||||||| 2d4327a
\newcommand{\drawHexagon}[4]{
	\draw[fill=#3] (#1,#2) ++(30:2) -- ++(90:2) -- ++(150:2) -- ++(210:2) -- ++(270:2) -- ++(330:2) -- cycle;
	\node at (#1,#2+2) {#4};
=======
\begin{document}
\newcommand{\drawHexagon}[4]{
	\draw[fill=#3] (#1,#2) ++(30:2) -- ++(90:2) -- ++(150:2) -- ++(210:2) -- ++(270:2) -- ++(330:2) -- cycle;
	\node at (#1,#2+2) {#4};
>>>>>>> 023797133fb426c1bb01f920d8f5635c343d11a6
}

\newif\ifuserinterfacelayer
\newif\ifpersistencelayer
\newif\ifmetaheuristicslayer
\newif\ifatomicpointerswaplayer

\pgfkeys{
	/hexagon/.is family, /hexagon,
	default/.style = {
		persistence=false,
		userinterface=false,
		metaheuristics=true,
	},
	persistence/.is if=persistencelayer,
	userinterface/.is if=userinterfacelayer,
	metaheuristics/.is if=metaheuristicslayer,
}
\newcommand{\drawModelSetupHexagon}[1][]{
	\pgfkeys{/hexagon, default, #1}

	\begin{tikzpicture}[scale=0.6, line width=1.05]
	
	\ifuserinterfacelayer
		\drawHexagon{ 2                      }{ 2}{dtu-yellow}{UI}
		\drawHexagon{{6 - 2 * (2 - sqrt(3)) }}{ 2}{dtu-yellow}{UI}
		\drawHexagon{{4 - 1 * (2 - sqrt(3)) }}{-1}{dtu-yellow}{UI}
		\drawHexagon{{0 + 1 * (2 - sqrt(3)) }}{-1}{dtu-yellow}{UI}
		\drawHexagon{{8 - 3 * (2 - sqrt(3)) }}{-1}{dtu-yellow}{UI}

		\drawHexagon{{2 - 0 * (2 - sqrt(3)) }}{-4}{dtu-yellow}{UI}
		\drawHexagon{{6 - 2 * (2 - sqrt(3)) }}{-4}{dtu-yellow}{UI}

		\drawHexagon{{10 - 4 * (2 - sqrt(3)) }}{-4}{dtu-yellow}{UI}
		\drawHexagon{{-2 + 2 * (2 - sqrt(3)) }}{-4}{dtu-yellow}{UI}

		\drawHexagon{{12 - 5 * (2 - sqrt(3)) }}{-1}{dtu-yellow}{UI}
		\drawHexagon{{-4 + 3 * (2 - sqrt(3)) }}{-1}{dtu-yellow}{UI}
	\fi

	\ifpersistencelayer
		\drawHexagon[draw=none]{ 2                      }{ 2}{dtu-blue}{}
		\drawHexagon[draw=none]{{6 - 2 * (2 - sqrt(3)) }}{ 2}{dtu-blue}{}
		\drawHexagon[draw=none]{{4 - 1 * (2 - sqrt(3)) }}{-1}{dtu-blue}{Database}
		\drawHexagon[draw=none]{{0 + 1 * (2 - sqrt(3)) }}{-1}{dtu-blue}{}
		\drawHexagon[draw=none]{{8 - 3 * (2 - sqrt(3)) }}{-1}{dtu-blue}{}

		\drawHexagon[draw=none]{{2 - 0 * (2 - sqrt(3)) }}{-4}{dtu-blue}{}
		\drawHexagon[draw=none]{{6 - 2 * (2 - sqrt(3)) }}{-4}{dtu-blue}{}

		\drawHexagon[draw=none]{{10 - 4 * (2 - sqrt(3)) }}{-4}{dtu-blue}{}
		\drawHexagon[draw=none]{{-2 + 2 * (2 - sqrt(3)) }}{-4}{dtu-blue}{}

		\drawHexagon[draw=none]{{12 - 5 * (2 - sqrt(3)) }}{-1}{dtu-blue}{}
		\drawHexagon[draw=none]{{-4 + 3 * (2 - sqrt(3)) }}{-1}{dtu-blue}{}
	\fi

	\ifmetaheuristicslayer
		\drawHexagon{ 2                      }{ 2}{dtu-blue}{Strategic}
		\drawHexagon{{6 - 2 * (2 - sqrt(3)) }}{ 2}{dtu-green}{Tactical}
		\drawHexagon{{4 - 1 * (2 - sqrt(3)) }}{-1}{dtu-red}{Supervisor}
		\drawHexagon{{0 + 1 * (2 - sqrt(3)) }}{-1}{dtu-red}{Supervisor}
		\drawHexagon{{8 - 3 * (2 - sqrt(3)) }}{-1}{dtu-red}{Supervisor}

		\drawHexagon{{2 - 0 * (2 - sqrt(3)) }}{-4}{dtu-corporate-red}{Technician}
		\drawHexagon{{6 - 2 * (2 - sqrt(3)) }}{-4}{dtu-corporate-red}{Technician}

		\drawHexagon{{10 - 4 * (2 - sqrt(3)) }}{-4}{dtu-corporate-red}{Technician}
		\drawHexagon{{-2 + 2 * (2 - sqrt(3)) }}{-4}{dtu-corporate-red}{Technician}

		\drawHexagon{{12 - 5 * (2 - sqrt(3)) }}{-1}{dtu-corporate-red}{Technician}
		\drawHexagon{{-4 + 3 * (2 - sqrt(3)) }}{-1}{dtu-corporate-red}{Technician}
	\fi

	
	\end{tikzpicture}
}

\ifstandalone
	

\definecolor{red}{HTML}{8A3F3A}
\definecolor{yellow}{HTML}{E0BB3C}
\definecolor{blue}{HTML}{4569E0}
\definecolor{green}{HTML}{17E561}
\definecolor{other}{HTML}{6A939E}

	\drawModelSetupHexagon
\fi

\end{document}

	\resizebox{0.7\textwidth}{!}{
		\drawModelSetupHexagon[atomicpointerswap=true]
	}
	\caption{
		Overview of the scheduling process when modelled as actors. When LNS is encapsulated 
		is an actor it becomes possible to optimize parts of a large process individually instead of 
		optimizing the scheduling problem globally from a single model implementation.
	}
	\label{fig:ordinator-hexagon:atomicpointerswap}
\end{figure}

\begin{figure}[H]
	\centering
	\documentclass{standalone}
\usepackage{tikz}
\usetikzlibrary {positioning}

<<<<<<< HEAD
\newcommand{\drawHexagon}[5][draw=black]{
	\draw[#1, fill=#4, ] (#2,#3) ++(30:2) -- ++(90:2) -- ++(150:2) -- ++(210:2) -- ++(270:2) -- ++(330:2) -- cycle;
	\node at (#2,#3+2) {#5};
||||||| 2d4327a
\newcommand{\drawHexagon}[4]{
	\draw[fill=#3] (#1,#2) ++(30:2) -- ++(90:2) -- ++(150:2) -- ++(210:2) -- ++(270:2) -- ++(330:2) -- cycle;
	\node at (#1,#2+2) {#4};
=======
\begin{document}
\newcommand{\drawHexagon}[4]{
	\draw[fill=#3] (#1,#2) ++(30:2) -- ++(90:2) -- ++(150:2) -- ++(210:2) -- ++(270:2) -- ++(330:2) -- cycle;
	\node at (#1,#2+2) {#4};
>>>>>>> 023797133fb426c1bb01f920d8f5635c343d11a6
}

\newif\ifuserinterfacelayer
\newif\ifpersistencelayer
\newif\ifmetaheuristicslayer
\newif\ifatomicpointerswaplayer

\pgfkeys{
	/hexagon/.is family, /hexagon,
	default/.style = {
		persistence=false,
		userinterface=false,
		metaheuristics=true,
	},
	persistence/.is if=persistencelayer,
	userinterface/.is if=userinterfacelayer,
	metaheuristics/.is if=metaheuristicslayer,
}
\newcommand{\drawModelSetupHexagon}[1][]{
	\pgfkeys{/hexagon, default, #1}

	\begin{tikzpicture}[scale=0.6, line width=1.05]
	
	\ifuserinterfacelayer
		\drawHexagon{ 2                      }{ 2}{dtu-yellow}{UI}
		\drawHexagon{{6 - 2 * (2 - sqrt(3)) }}{ 2}{dtu-yellow}{UI}
		\drawHexagon{{4 - 1 * (2 - sqrt(3)) }}{-1}{dtu-yellow}{UI}
		\drawHexagon{{0 + 1 * (2 - sqrt(3)) }}{-1}{dtu-yellow}{UI}
		\drawHexagon{{8 - 3 * (2 - sqrt(3)) }}{-1}{dtu-yellow}{UI}

		\drawHexagon{{2 - 0 * (2 - sqrt(3)) }}{-4}{dtu-yellow}{UI}
		\drawHexagon{{6 - 2 * (2 - sqrt(3)) }}{-4}{dtu-yellow}{UI}

		\drawHexagon{{10 - 4 * (2 - sqrt(3)) }}{-4}{dtu-yellow}{UI}
		\drawHexagon{{-2 + 2 * (2 - sqrt(3)) }}{-4}{dtu-yellow}{UI}

		\drawHexagon{{12 - 5 * (2 - sqrt(3)) }}{-1}{dtu-yellow}{UI}
		\drawHexagon{{-4 + 3 * (2 - sqrt(3)) }}{-1}{dtu-yellow}{UI}
	\fi

	\ifpersistencelayer
		\drawHexagon[draw=none]{ 2                      }{ 2}{dtu-blue}{}
		\drawHexagon[draw=none]{{6 - 2 * (2 - sqrt(3)) }}{ 2}{dtu-blue}{}
		\drawHexagon[draw=none]{{4 - 1 * (2 - sqrt(3)) }}{-1}{dtu-blue}{Database}
		\drawHexagon[draw=none]{{0 + 1 * (2 - sqrt(3)) }}{-1}{dtu-blue}{}
		\drawHexagon[draw=none]{{8 - 3 * (2 - sqrt(3)) }}{-1}{dtu-blue}{}

		\drawHexagon[draw=none]{{2 - 0 * (2 - sqrt(3)) }}{-4}{dtu-blue}{}
		\drawHexagon[draw=none]{{6 - 2 * (2 - sqrt(3)) }}{-4}{dtu-blue}{}

		\drawHexagon[draw=none]{{10 - 4 * (2 - sqrt(3)) }}{-4}{dtu-blue}{}
		\drawHexagon[draw=none]{{-2 + 2 * (2 - sqrt(3)) }}{-4}{dtu-blue}{}

		\drawHexagon[draw=none]{{12 - 5 * (2 - sqrt(3)) }}{-1}{dtu-blue}{}
		\drawHexagon[draw=none]{{-4 + 3 * (2 - sqrt(3)) }}{-1}{dtu-blue}{}
	\fi

	\ifmetaheuristicslayer
		\drawHexagon{ 2                      }{ 2}{dtu-blue}{Strategic}
		\drawHexagon{{6 - 2 * (2 - sqrt(3)) }}{ 2}{dtu-green}{Tactical}
		\drawHexagon{{4 - 1 * (2 - sqrt(3)) }}{-1}{dtu-red}{Supervisor}
		\drawHexagon{{0 + 1 * (2 - sqrt(3)) }}{-1}{dtu-red}{Supervisor}
		\drawHexagon{{8 - 3 * (2 - sqrt(3)) }}{-1}{dtu-red}{Supervisor}

		\drawHexagon{{2 - 0 * (2 - sqrt(3)) }}{-4}{dtu-corporate-red}{Technician}
		\drawHexagon{{6 - 2 * (2 - sqrt(3)) }}{-4}{dtu-corporate-red}{Technician}

		\drawHexagon{{10 - 4 * (2 - sqrt(3)) }}{-4}{dtu-corporate-red}{Technician}
		\drawHexagon{{-2 + 2 * (2 - sqrt(3)) }}{-4}{dtu-corporate-red}{Technician}

		\drawHexagon{{12 - 5 * (2 - sqrt(3)) }}{-1}{dtu-corporate-red}{Technician}
		\drawHexagon{{-4 + 3 * (2 - sqrt(3)) }}{-1}{dtu-corporate-red}{Technician}
	\fi

	
	\end{tikzpicture}
}

\ifstandalone
	

\definecolor{red}{HTML}{8A3F3A}
\definecolor{yellow}{HTML}{E0BB3C}
\definecolor{blue}{HTML}{4569E0}
\definecolor{green}{HTML}{17E561}
\definecolor{other}{HTML}{6A939E}

	\drawModelSetupHexagon
\fi

\end{document}

	\resizebox{0.7\textwidth}{!}{
		\drawModelSetupHexagon[metaheuristics=true]
	}
	\caption{
		Overview of the scheduling process when modelled as actors. When LNS is encapsulated 
		is an actor it becomes possible to optimize parts of a large process individually instead of 
		optimizing the scheduling problem globally from a single model implementation.
	}
	\label{fig:ordinator-hexagon:metaheuristics}
\end{figure}

\begin{figure}[H]
	\centering
	\documentclass{standalone}
\usepackage{tikz}
\usetikzlibrary {positioning}

<<<<<<< HEAD
\newcommand{\drawHexagon}[5][draw=black]{
	\draw[#1, fill=#4, ] (#2,#3) ++(30:2) -- ++(90:2) -- ++(150:2) -- ++(210:2) -- ++(270:2) -- ++(330:2) -- cycle;
	\node at (#2,#3+2) {#5};
||||||| 2d4327a
\newcommand{\drawHexagon}[4]{
	\draw[fill=#3] (#1,#2) ++(30:2) -- ++(90:2) -- ++(150:2) -- ++(210:2) -- ++(270:2) -- ++(330:2) -- cycle;
	\node at (#1,#2+2) {#4};
=======
\begin{document}
\newcommand{\drawHexagon}[4]{
	\draw[fill=#3] (#1,#2) ++(30:2) -- ++(90:2) -- ++(150:2) -- ++(210:2) -- ++(270:2) -- ++(330:2) -- cycle;
	\node at (#1,#2+2) {#4};
>>>>>>> 023797133fb426c1bb01f920d8f5635c343d11a6
}

\newif\ifuserinterfacelayer
\newif\ifpersistencelayer
\newif\ifmetaheuristicslayer
\newif\ifatomicpointerswaplayer

\pgfkeys{
	/hexagon/.is family, /hexagon,
	default/.style = {
		persistence=false,
		userinterface=false,
		metaheuristics=true,
	},
	persistence/.is if=persistencelayer,
	userinterface/.is if=userinterfacelayer,
	metaheuristics/.is if=metaheuristicslayer,
}
\newcommand{\drawModelSetupHexagon}[1][]{
	\pgfkeys{/hexagon, default, #1}

	\begin{tikzpicture}[scale=0.6, line width=1.05]
	
	\ifuserinterfacelayer
		\drawHexagon{ 2                      }{ 2}{dtu-yellow}{UI}
		\drawHexagon{{6 - 2 * (2 - sqrt(3)) }}{ 2}{dtu-yellow}{UI}
		\drawHexagon{{4 - 1 * (2 - sqrt(3)) }}{-1}{dtu-yellow}{UI}
		\drawHexagon{{0 + 1 * (2 - sqrt(3)) }}{-1}{dtu-yellow}{UI}
		\drawHexagon{{8 - 3 * (2 - sqrt(3)) }}{-1}{dtu-yellow}{UI}

		\drawHexagon{{2 - 0 * (2 - sqrt(3)) }}{-4}{dtu-yellow}{UI}
		\drawHexagon{{6 - 2 * (2 - sqrt(3)) }}{-4}{dtu-yellow}{UI}

		\drawHexagon{{10 - 4 * (2 - sqrt(3)) }}{-4}{dtu-yellow}{UI}
		\drawHexagon{{-2 + 2 * (2 - sqrt(3)) }}{-4}{dtu-yellow}{UI}

		\drawHexagon{{12 - 5 * (2 - sqrt(3)) }}{-1}{dtu-yellow}{UI}
		\drawHexagon{{-4 + 3 * (2 - sqrt(3)) }}{-1}{dtu-yellow}{UI}
	\fi

	\ifpersistencelayer
		\drawHexagon[draw=none]{ 2                      }{ 2}{dtu-blue}{}
		\drawHexagon[draw=none]{{6 - 2 * (2 - sqrt(3)) }}{ 2}{dtu-blue}{}
		\drawHexagon[draw=none]{{4 - 1 * (2 - sqrt(3)) }}{-1}{dtu-blue}{Database}
		\drawHexagon[draw=none]{{0 + 1 * (2 - sqrt(3)) }}{-1}{dtu-blue}{}
		\drawHexagon[draw=none]{{8 - 3 * (2 - sqrt(3)) }}{-1}{dtu-blue}{}

		\drawHexagon[draw=none]{{2 - 0 * (2 - sqrt(3)) }}{-4}{dtu-blue}{}
		\drawHexagon[draw=none]{{6 - 2 * (2 - sqrt(3)) }}{-4}{dtu-blue}{}

		\drawHexagon[draw=none]{{10 - 4 * (2 - sqrt(3)) }}{-4}{dtu-blue}{}
		\drawHexagon[draw=none]{{-2 + 2 * (2 - sqrt(3)) }}{-4}{dtu-blue}{}

		\drawHexagon[draw=none]{{12 - 5 * (2 - sqrt(3)) }}{-1}{dtu-blue}{}
		\drawHexagon[draw=none]{{-4 + 3 * (2 - sqrt(3)) }}{-1}{dtu-blue}{}
	\fi

	\ifmetaheuristicslayer
		\drawHexagon{ 2                      }{ 2}{dtu-blue}{Strategic}
		\drawHexagon{{6 - 2 * (2 - sqrt(3)) }}{ 2}{dtu-green}{Tactical}
		\drawHexagon{{4 - 1 * (2 - sqrt(3)) }}{-1}{dtu-red}{Supervisor}
		\drawHexagon{{0 + 1 * (2 - sqrt(3)) }}{-1}{dtu-red}{Supervisor}
		\drawHexagon{{8 - 3 * (2 - sqrt(3)) }}{-1}{dtu-red}{Supervisor}

		\drawHexagon{{2 - 0 * (2 - sqrt(3)) }}{-4}{dtu-corporate-red}{Technician}
		\drawHexagon{{6 - 2 * (2 - sqrt(3)) }}{-4}{dtu-corporate-red}{Technician}

		\drawHexagon{{10 - 4 * (2 - sqrt(3)) }}{-4}{dtu-corporate-red}{Technician}
		\drawHexagon{{-2 + 2 * (2 - sqrt(3)) }}{-4}{dtu-corporate-red}{Technician}

		\drawHexagon{{12 - 5 * (2 - sqrt(3)) }}{-1}{dtu-corporate-red}{Technician}
		\drawHexagon{{-4 + 3 * (2 - sqrt(3)) }}{-1}{dtu-corporate-red}{Technician}
	\fi

	
	\end{tikzpicture}
}

\ifstandalone
	

\definecolor{red}{HTML}{8A3F3A}
\definecolor{yellow}{HTML}{E0BB3C}
\definecolor{blue}{HTML}{4569E0}
\definecolor{green}{HTML}{17E561}
\definecolor{other}{HTML}{6A939E}

	\drawModelSetupHexagon
\fi

\end{document}

	\resizebox{0.7\textwidth}{!}{
		\drawModelSetupHexagon[orchestrator=true]
	}
	\caption{
		Overview of the scheduling process when modelled as actors. When LNS is encapsulated 
		is an actor it becomes possible to optimize parts of a large process individually instead of 
		optimizing the scheduling problem globally from a single model implementation.
	}
	\label{fig:ordinator-hexagon:orchestrator}
\end{figure}
\begin{figure}[H]
	\centering
	\documentclass{standalone}
\usepackage{tikz}
\usetikzlibrary {positioning}

<<<<<<< HEAD
\newcommand{\drawHexagon}[5][draw=black]{
	\draw[#1, fill=#4, ] (#2,#3) ++(30:2) -- ++(90:2) -- ++(150:2) -- ++(210:2) -- ++(270:2) -- ++(330:2) -- cycle;
	\node at (#2,#3+2) {#5};
||||||| 2d4327a
\newcommand{\drawHexagon}[4]{
	\draw[fill=#3] (#1,#2) ++(30:2) -- ++(90:2) -- ++(150:2) -- ++(210:2) -- ++(270:2) -- ++(330:2) -- cycle;
	\node at (#1,#2+2) {#4};
=======
\begin{document}
\newcommand{\drawHexagon}[4]{
	\draw[fill=#3] (#1,#2) ++(30:2) -- ++(90:2) -- ++(150:2) -- ++(210:2) -- ++(270:2) -- ++(330:2) -- cycle;
	\node at (#1,#2+2) {#4};
>>>>>>> 023797133fb426c1bb01f920d8f5635c343d11a6
}

\newif\ifuserinterfacelayer
\newif\ifpersistencelayer
\newif\ifmetaheuristicslayer
\newif\ifatomicpointerswaplayer

\pgfkeys{
	/hexagon/.is family, /hexagon,
	default/.style = {
		persistence=false,
		userinterface=false,
		metaheuristics=true,
	},
	persistence/.is if=persistencelayer,
	userinterface/.is if=userinterfacelayer,
	metaheuristics/.is if=metaheuristicslayer,
}
\newcommand{\drawModelSetupHexagon}[1][]{
	\pgfkeys{/hexagon, default, #1}

	\begin{tikzpicture}[scale=0.6, line width=1.05]
	
	\ifuserinterfacelayer
		\drawHexagon{ 2                      }{ 2}{dtu-yellow}{UI}
		\drawHexagon{{6 - 2 * (2 - sqrt(3)) }}{ 2}{dtu-yellow}{UI}
		\drawHexagon{{4 - 1 * (2 - sqrt(3)) }}{-1}{dtu-yellow}{UI}
		\drawHexagon{{0 + 1 * (2 - sqrt(3)) }}{-1}{dtu-yellow}{UI}
		\drawHexagon{{8 - 3 * (2 - sqrt(3)) }}{-1}{dtu-yellow}{UI}

		\drawHexagon{{2 - 0 * (2 - sqrt(3)) }}{-4}{dtu-yellow}{UI}
		\drawHexagon{{6 - 2 * (2 - sqrt(3)) }}{-4}{dtu-yellow}{UI}

		\drawHexagon{{10 - 4 * (2 - sqrt(3)) }}{-4}{dtu-yellow}{UI}
		\drawHexagon{{-2 + 2 * (2 - sqrt(3)) }}{-4}{dtu-yellow}{UI}

		\drawHexagon{{12 - 5 * (2 - sqrt(3)) }}{-1}{dtu-yellow}{UI}
		\drawHexagon{{-4 + 3 * (2 - sqrt(3)) }}{-1}{dtu-yellow}{UI}
	\fi

	\ifpersistencelayer
		\drawHexagon[draw=none]{ 2                      }{ 2}{dtu-blue}{}
		\drawHexagon[draw=none]{{6 - 2 * (2 - sqrt(3)) }}{ 2}{dtu-blue}{}
		\drawHexagon[draw=none]{{4 - 1 * (2 - sqrt(3)) }}{-1}{dtu-blue}{Database}
		\drawHexagon[draw=none]{{0 + 1 * (2 - sqrt(3)) }}{-1}{dtu-blue}{}
		\drawHexagon[draw=none]{{8 - 3 * (2 - sqrt(3)) }}{-1}{dtu-blue}{}

		\drawHexagon[draw=none]{{2 - 0 * (2 - sqrt(3)) }}{-4}{dtu-blue}{}
		\drawHexagon[draw=none]{{6 - 2 * (2 - sqrt(3)) }}{-4}{dtu-blue}{}

		\drawHexagon[draw=none]{{10 - 4 * (2 - sqrt(3)) }}{-4}{dtu-blue}{}
		\drawHexagon[draw=none]{{-2 + 2 * (2 - sqrt(3)) }}{-4}{dtu-blue}{}

		\drawHexagon[draw=none]{{12 - 5 * (2 - sqrt(3)) }}{-1}{dtu-blue}{}
		\drawHexagon[draw=none]{{-4 + 3 * (2 - sqrt(3)) }}{-1}{dtu-blue}{}
	\fi

	\ifmetaheuristicslayer
		\drawHexagon{ 2                      }{ 2}{dtu-blue}{Strategic}
		\drawHexagon{{6 - 2 * (2 - sqrt(3)) }}{ 2}{dtu-green}{Tactical}
		\drawHexagon{{4 - 1 * (2 - sqrt(3)) }}{-1}{dtu-red}{Supervisor}
		\drawHexagon{{0 + 1 * (2 - sqrt(3)) }}{-1}{dtu-red}{Supervisor}
		\drawHexagon{{8 - 3 * (2 - sqrt(3)) }}{-1}{dtu-red}{Supervisor}

		\drawHexagon{{2 - 0 * (2 - sqrt(3)) }}{-4}{dtu-corporate-red}{Technician}
		\drawHexagon{{6 - 2 * (2 - sqrt(3)) }}{-4}{dtu-corporate-red}{Technician}

		\drawHexagon{{10 - 4 * (2 - sqrt(3)) }}{-4}{dtu-corporate-red}{Technician}
		\drawHexagon{{-2 + 2 * (2 - sqrt(3)) }}{-4}{dtu-corporate-red}{Technician}

		\drawHexagon{{12 - 5 * (2 - sqrt(3)) }}{-1}{dtu-corporate-red}{Technician}
		\drawHexagon{{-4 + 3 * (2 - sqrt(3)) }}{-1}{dtu-corporate-red}{Technician}
	\fi

	
	\end{tikzpicture}
}

\ifstandalone
	

\definecolor{red}{HTML}{8A3F3A}
\definecolor{yellow}{HTML}{E0BB3C}
\definecolor{blue}{HTML}{4569E0}
\definecolor{green}{HTML}{17E561}
\definecolor{other}{HTML}{6A939E}

	\drawModelSetupHexagon
\fi

\end{document}

	\resizebox{0.7\textwidth}{!}{
		\drawModelSetupHexagon[userinterface=true]
	}
	\caption{
		Overview of the scheduling process when modelled as actors. When LNS is encapsulated 
		is an actor it becomes possible to optimize parts of a large process individually instead of 
		optimizing the scheduling problem globally from a single model implementation.
	}
	\label{fig:ordinator-hexagon:userinterfaces}
\end{figure}

The next step in this direction will be to model the remaining stakeholders as their own 
AbLNS metaheuristics, and then make them communicate together through atomic pointer swaps and message
passing. This enables modular concurrency at each layer and ensures real-time
synchronization across multiple optimization levels. Making each metaheuristic expose solutions to each 
other in real-time providing each other with high quality parameters.

\section{Conclusion}
Many current planning problems that industry faces are combinatorial by nature, 
and many combinatorial problems have to be solved continuously to make operations 
run efficiently. For operation research (OR) to be helpful in this process, the solution methods 
should be a minimally invasive in the workflow of the working stakeholders. 
The AbLNS solution approach detailed in this paper aligns
closely with both the known problems in operation research of the lack of integration and the issues of 
coordination in multi-stakeholder processes. For these reasons we argue that the
standard Operations Research approach of  first collecting data, then creating a
model and optimizing it and then finally providing the solution to the planners
in the company workflow, is not a sound approach in many situations.

We have here demonstrated AbLNS approach works in a practical setting,
maintenance scheduling at  Total Energies and we are convinced that this
approach, combining a number of smaller planning/optimization problems with
different actors/stakeholders responsible for their part of the overall
solution. This paper argues that making a system of numerous smaller  "quick
and dirty" models in an well thought out software architecture is much more
effective in practice than large integrated models.

The fundamental problem with the "older" paradigm is that optimizing across
actors/stakeholders is very difficult, leading the literature to prefer
integrated models instead of decomposing the model by each of the
processes that make up a business process such as maintenance scheduling.
This paper argues that this is mainly due to an dated understanding of
software architecture that is available today in industry, but not
acknowledged by broader the Operations Research and Metaheuristic
communities \citep{talbiMetaheuristicsDesignImplementation2009},
\citep{gendreauHandbookMetaheuristics2019}.



\bibliography{refs}

\bibliographystyle{elsarticle-harv}
\end{document}

hello
\endinput
